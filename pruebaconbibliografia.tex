\documentclass{article}
\usepackage[spanish]{babel}
\usepackage[style=verbose-ibid, backend=biber]{biblatex}

% Personalización para español
\DefineBibliographyStrings{spanish}{
	idem = {Ídem},
	ibidem = {Ibíd.},
}

% Formato de citas en notas al pie
\DeclareFieldFormat{postnote}{#1} % Evita "p." antes del número de página
\renewcommand*{\ibidem}{\autocap{i}bíd\nopunct} % Sin punto después
\renewcommand*{\idememph}{\emph{Ídem}} % Ídem en cursiva

\addbibresource{bibliografia.bib} % Tu archivo .bib


\begin{document}
	Las ontologías diferenciales han sido exploradas por varios autores\footcite[15]{deleuze2002}.
	Más adelante, el mismo autor profundiza en la noción de repetición\footcite[Véase especialmente el capítulo 2][]{deleuze2002}.

	Cuando se cita exactamente la misma obra consecutivamente:
	\footcite[18]{deleuze2002}.
	\footcite[22]{deleuze2002}. % Se convertirá automáticamente en "Ídem, p. 22"

	Al citar otra obra del mismo autor:
	\footcite[45]{deleuze1996}.
	\footcite[30]{deleuze2002}. % Mostrará el título abreviado

	Cuando se repite la cita inmediatamente después:
	\footcite[101]{heidegger1988}.
	\footcite[103]{heidegger1988}. % Se convertirá en "Ibíd., p. 103"
\end{document}


\ifPDF
\chapter[\hspace{1.5pc}Ontología nodal. Un materialismo del encuentro]{Ontología nodal. Un materialismo del encuentro}
\chaptermark{Ontología nodal...}
\Author{Roque Farrán}
\setcounter{PrimPag}{\theCurrentPage}

% encabezado para autor
\begin{center}
	\nombreautor{Roque Farrán}\\
	\vspace{20mm}
\end{center}
\else
\ifHTMLEPUB
\chapter{Ontología nodal. Un materialismo del encuentro}
\fi
\fi


\section{Introducción}

En el presente capítulo efectuaremos una lectura crítica de distintos posicionamientos teóricos en torno a la posibilidad o imposibilidad que nos plantean de sostener una filosofía materialista en la actualidad. Para ello, trataremos de circunscribir la especificidad de una práctica teórica que dé cuenta de otras prácticas (teóricas o no), simultáneamente, sin reducirlas a un lenguaje homogéneo o explicarlas desde un metalenguaje (idealismo). Por supuesto, dichas prácticas apenas serán mencionadas en su posible articulación más no desarrolladas en extensión. En este recorrido vamos a presentar, entonces, distintas perspectivas conceptuales que nos permitirán, a su vez, desplegar nuestro propio punto de vista en torno a la posibilidad de una práctica filosófica de tal índole: una ontología nodal.

En primer lugar, encontramos convergencias muy productivas entre los planteos filosóficos de Althusser, Lacan, Badiou, Nancy, Foucault y Agamben, sobre todo a partir de ciertos conceptos nodales presentes en sus respectivas elaboraciones teóricas. En consecuencia, nos proponemos rastrear a través de distintas lecturas, directas e indirectas, algunos puntos significativos de encuentro entre ellos, en relación con lo que llamaremos aquí el \enquote{materialismo} propio de una práctica teórica. No pretendemos establecer jerarquías entre dichas lecturas, ni pensamos tampoco que los nombres propios remitan a totalizaciones de ningún tipo (\emph{i.e.}, obras), pues consideramos más interesante seguir puntualmente aquellos movimientos y desplazamientos conceptuales oblicuos, producidos en distintos niveles de complejidad, no siempre reductibles a un solo hilo conductor. Al formular una ontología nodal, en el cruce de distintas perspectivas teórico-prácticas, es justamente a esta idea de \emph{un hilo conductor} a la que nos sustraemos. Preciso es remarcar, antes que nada, que las convergencias halladas no están exentas de tensiones y es eso mismo, de hecho, lo que define y anima una radical \emph{politicidad} de los conceptos formulados, de la cual ninguno de estos autores pareciera pretender excluirse.

En los siguientes apartados veremos que Agamben, por ejemplo, elucida el concepto de paradigma entre Foucault, Kuhn y él mismo, delimitando el estatuto singular de la intervención teórica (ensayaremos allí una aproximación oblicua hacia Marx). Bosteels, por su parte, nos muestra la continuidad entre la filosofía de Badiou y su maestro Althusser, aunque haciendo especial hincapié en el materialismo dialéctico más que en el materialismo aleatorio de este último (esto traerá consecuencias que examinaremos ampliamente). Romé, sobre esta misma vía de pensamiento, nos brinda una lectura atenta de los puntos de convergencia entre Lacan y Althusser (psicoanálisis y marxismo) que nos servirá para resaltar la índole singular del materialismo lacaniano. Nancy, por último, nos presenta una ontología nodal de lo político y discute asimismo con la posición filosófica de Badiou, lo cual nos permitirá introducir más abiertamente la cuestión ontológico-política que el anudamiento sostiene. Por consiguiente, en el mismo movimiento de presentar parcialmente estas diferentes lecturas introduciremos, a través de ellas, nuestro propio aporte para pensar una \emph{ontología nodal} y un \emph{materialismo filosófico del encuentro}; los que se tornan inteligibles, en sus complejas dimensiones, a partir del anudamiento borromeo de los términos y dispositivos (co)implicados.\footnote{Tal como hemos formulado en otros artículos, sostenemos que el concepto filosófico encuentra su modo de articulación paradigmática, entre contingencia y necesidad, en el nudo borromeo. Esta estructura topológica tiene la propiedad distintiva de que basta con que un cordel se corte para que el entramado \rdm{que puede contar con infinitos cordeles} se desanude. Hemos mostrado la pertinencia del nudo en diferentes niveles de análisis y en referencia a distintos autores. \cite[Véase][]{@7131-FARRAN2009}; \cite[][]{@7132-FARRANSIN}; \cite[][]{@7133-FARRAN2010}. En el presente texto, en cambio, hemos decidido trabajar nuestra propia propuesta ontológica \emph{entre} diferentes perspectivas que pertenecen al mismo \emph{filum} teórico; y en este sentido, se distingue del intento de marcar diferencias inconciliables al interior del campo ensayada, por ejemplo, por \cite[][]{@7078-LACLAU2008}. Nuestro método, como se podrá apreciar a continuación, no sigue el silogismo estricto ni la dialéctica totalizante, más bien ensaya la recurrencia, la alternancia y la variación de tópicos; se trata, en fin, de una suerte de urdimbre conceptual, un anudamiento complejo de diferentes consistencias discursivas y niveles de análisis, que pretendemos resulte coherente con la propuesta ológintoca.}

En este sentido, nuestra propuesta se declara deudora de los planteos lacanianos, aunque excediéndolos hacia las formulaciones filosófico-políticas de Badiou (y Nancy en menor medida), tal como mostraremos sobre el final del texto.

Podríamos anticipar algunas distinciones efectuadas al interior de este campo de producción teórica que algunos han denominado \enquote{posfundacionalismo}, \enquote{izquierda heideggeriana}, \enquote{marxismo postestructuralista}, o \enquote{izquierda lacaniana}. Si consideramos que, a grandes rasgos, los autores arriba mencionados parten de la imposibilidad real de auto-clausura de todo orden simbólico, sea por falta o por exceso, lo que de algún modo los reúne, en su dispersión de estilos, es el asentimiento o acuerdo de una ontología de lo real: materialista en sentido amplio.\footnote{Así lo expresa Elías Palti, quizá simplificando excesivamente el estatuto de lo real: \enquote{Lo que determina el carácter materialista de una doctrina, y la distingue del idealismo, es la afirmación de la presencia de un residuo ineliminable de materialidad irreductible a toda lógica o concepto (para decirlo en términos lacanianos, un Real que resiste absolutamente su simbolización)} \footcite[90]{@7008-ELIAS2005}.}

Y por lo tanto, la asunción de una radical y necesaria contingencia de los fundamentos y de la historicidad intrínseca a todo concepto. Las diferencias se encuentran, en cambio, en el modo discursivo y teórico de responder \emph{allí}, respecto a la diferencia \emph{qua} diferencia (ontológica); en tal sentido son movilizados electivamente distintos dispositivos discursivos.\footnote{A diferencia de lo que concluye Oliver Marchart en su libro \emph{El pensamiento político post-fundacional}, nosotros no consideramos que estos autores no hayan logrado extraer todas las consecuencias posibles de sus ontologías políticas y hayan desplazado así la diferencia bien hacia la ética (Badiou), la \emph{hauntologie} (Derrida) u otros filosofismos (Nancy), sino que, sostenemos, utilizan distintos operadores conceptuales para seguir tematizando coyunturalmente la diferencia ontológica \emph{en} situación, lo cual implica un claro posicionamiento político \emph{en} la teoría misma. Véase \cite[31]{@6998-MARCHART2009}. Tampoco creemos que ninguno de ellos caiga, después de Heidegger, en la reducción dicotómica de la triplicidad irreductible que entraña la diferencia ontológica, tal como podría sugerir Emmanuel Biset en su crítica justificada a la lectura reductivista de Marchart \parencite[]{@7074-BISET2010} pues cada quien se las ve con dicha irreductibilidad ontológica bajo sus propios medios conceptuales y términos electivos (\emph{i.e.} brecha, real, paradigma, sujeto, quiasmo, hiancia, etc.).}




\begin{enumerate}
	\def\labelenumi{\arabic{enumi}.}
	\item Ontologías de lo real (materialistas en sentido amplio)
	\begin{enumerate}
		\def\labelenumii{\arabic{enumi}.\arabic{enumii}}
		\item Ontología matemática (Badiou)
		\item Ontología lingüística-retórica (Laclau, Marchart)
		\item Ontologías materialistas en sentido estricto
		\begin{enumerate}
			\def\labelenumiii{\arabic{enumi}.\arabic{enumii}.\arabic{enumiii}}
			\item Dialéctica (1º Althusser, Žižek, Bosteels)
			\item Aleatoria (2º Althusser, Romé)
			\item Nodal (Lacan, Nancy)
			\item Paradigmática (Foucault, Agamben)
		\end{enumerate}
	\end{enumerate}
\end{enumerate}

Evidentemente, esta distinción es tan provisoria como artificial y cuestionable, de eso se trata, puesto que es posible observar que un autor bien podría ingresar en varias de estas categorías según el aspecto que se decida privilegiar de su pensamiento (es lo que intenta hacer, como veremos, Bosteels respecto a Badiou, por ejemplo). También se podría decir que todos ellos son \enquote{materialistas discursivos}, nominación con la cual no estaríamos para nada en desacuerdo. No obstante el asunto es, como hemos dicho, remarcar más bien los dispositivos y términos preferenciales a los que acuden nuestros autores (\emph{i.e.} \enquote{materialismo} en el caso de 1.3). En ese sentido está claro, por ejemplo, que por más que Badiou hable estratégicamente de \enquote{dialéctica materialista} (en sentido político-filosófico) el peso del dispositivo matemático en su filosofía es bastante más contundente que la terminología marxista-althusseriana. Por lo tanto, nuestra idea no es marcar las diferencias rígida o jerárquicamente, ni postular tampoco superación alguna de una perspectiva sobre las otras, sino trazar, en consonancia con ellas, un recorrido discursivo materialista que muestre sus continuidades y discontinuidades, inversiones y subversiones. Ello nos llevará a un desequilibrio extensional en los apartados que, esperamos, resulte compensado en la recurrencia y profundización de los nudos problemáticos tratados. Seguiremos así los discursos en sus múltiples aproximaciones y distanciamientos, mostrando puntos de convergencia y bifurcaciones, tensiones y compatibilidades, al mismo tiempo que inscribimos sobre sus pliegues nuestra (im)propia posición ontológico-política: \emph{nodal}. La orientación general de nuestro pensamiento es el materialismo, y el punto privilegiado de indagación, bajo diversas figuras teóricas, es el concepto de sujeto.

Antes de pasar al acto, valga una breve advertencia de índole metodológica. Trabajar desde el post-estructuralismo implica para nosotros cierta subversión de la lógica de lecto-escritura que no debe confundirse con la simple imitación de un lenguaje teórico. Cada quien, sostenemos, debe encontrar un modo propio, un estilo de escritura, incluso si lo hace dentro de cierta tradición de pensamiento. Por eso, no se trata aquí de mera aplicación externa de categorías y conceptos post-estructuralistas a contenidos ónticos específicos. Esta posición ontológico-política no puede ser asimilada sin más por una lógica pedagógica-académica-comunicacional, porque al incorporársele la trastoca. Se configura, en este sentido, un movimiento de pensamiento que resignifica, a su vez, qué pueden ser la transmisión, la institución, la transferencia cuando no hay lógicas totalizantes programáticas, ni modelos típicos de aplicación, ni referentes ideales prefigurados de antemano. Implica, así, soportar la incertidumbre el \emph{tiempo lógico} necesario, e incluir al sujeto de conocimiento en el mismo proceso de conocer; que, por ese mismo acto, deviene también ético y político. Y como dijimos, antes que todo: ontológico.

Más que a Heidegger y su \emph{diferencia ontológica} podemos recurrir a Lévi-Strauss y su \emph{principio antropogenético} para ejemplificar esto (como lo hace incansablemente Agamben).\footnote{Por ejemplo en \cite[27]{@7073-AGAMBEN2010}.
} Al principio fue el lenguaje \emph{in toto}; más no podía haber conocimiento de ello, solo significantes y significados que progresivamente se irían conectando entre sí, vía cognitiva. El problema es que hay así \emph{ab initio}, estructuralmente, un desfasaje entre significados y significantes. Allí vienen las palabras mágicas, en las culturas llamadas primitivas (\emph{maná, orenda, manitou}), o los significantes vacíos en nuestras democracias seculares, a nombrar un exceso de significación inasignable. Sucede lo mismo en todo acto de lectura: tejemos correspondencias entre significantes y significados conocidos, pero hay momentos donde no hallamos pie (correspondencia) lo cual exige de nuestra parte suspender la necesidad de significación inmediata, quizá promoviendo la invención de un puente significante nuevo. Por supuesto, no hay que irse a los extremos imaginables: i) un texto absolutamente hermético (\emph{i.e.} Mallarmé) nos hace casi imposible habitar la falla del lenguaje en que hemos nacido e inventar el suplemento que ligue contingentemente el desfasaje significante/significado (a no ser que seamos en verdad poetas); ii) un texto llanamente pedagógico, que no hace más que transmitir \emph{bits} de información perfectamente codificables, al contrario, cierra la falla y nos convierte en máquinas o animales (\emph{i.e.} abejas). Por eso lo deseable, pensamos, sería que cada quien, en la modesta medida de sus posibilidades \rdm{en verdad: de su deseo}, escriba/piense suspendiendo y articulando de un modo propio las correspondencias recibidas (de tradiciones) entre significantes y significados. Lo ensayaremos a continuación.

Vamos a comenzar por el concepto de paradigma, para mostrar el borde simbólico-material por el que se asume y de algún modo circunscribe lo real inasimilable a dicho orden (una suerte de método).

\section{Paradigma}

Agamben en uno de sus últimos libros, \emph{Signatura rerum}, analiza el concepto de paradigma marcando las diferencias y convergencias que encuentra entre Foucault, Kuhn, él mismo y por supuesto algunos otros autores pertenecientes a la tradición filosófica (Platón, Aristóteles, Kant, etc.). El paradigma constituye una suerte de singularidad universal, ya que pertenece a la clase de fenómenos que delimita (se cuenta \emph{entre} ellos) pero de la cual, a la vez, se auto-excluye para dar cuenta así de la inteligibilidad de estos mismos (como ejemplo-ejemplar).

Consideramos que este concepto se puede poner en equivalencia directa con el concepto de \enquote{síntoma} en la lectura althusseriana de Marx (\emph{Para leer el Capital}); como así también con el término a \enquote{doble función} en el cruce entre marxismo, psicoanálisis y estructuralismo (\emph{El recomienzo del materialismo dialéctico}); o, en la teoría más reciente de Alain Badiou, con el múltiple singular que define un \enquote{sitio de acontecimiento}.\footnote{O también con la lógica del \enquote{trascendental}, véase por ejemplo: \cite[83]{@7013-BADIOU2010}.} Pues justamente su singularidad reside en propiciar o habilitar un punto de cruce e inversión (\emph{quiasmo}) entre dos dimensiones discursivas irreductibles: sincronía y diacronía, estructura e historia, sensibilidad e inteligibilidad (podríamos agregar: fenómeno y \emph{noúmeno}, ser y ente, goce fálico y otro goce). Permite señalar así que las dimensiones aludidas no son dos, claramente diferenciadas y externas (¿desde qué exterioridad trascendental lo serían?), ni tampoco conforman una mezcla indiferente; son, más bien, una \enquote{y} dos, o \enquote{ni} una \enquote{ni} dos. Al desactivarse y suspenderse de su uso normal o fáctico el paradigma se excluye de la regla al mismo tiempo que sostiene su pertenencia a la serie de fenómenos que torna inteligibles en dicha torsión; se vuelve \enquote{contra-fáctico} podríamos decir. Un ejemplo \enquote{paradigmático} es el panóptico de Foucault, donde podemos leer tanto la dimensión histórica como estructural:

\begin{quote}
	Quien ha leído \emph{Surveiller et punir} sabe bien que, ubicado al final de la sección sobre las disciplinas, el \emph{panopticon} desarrolla una función estratégica decisiva para comprender la modalidad disciplinaria del poder, y como tal se transforma en algo así como la figura epistemológica que, a la vez que define el universo disciplinario de la modernidad, marca también el umbral a través del cual se pasa a la sociedad de control.\footcite[24]{@7070-AGAMBEN2009}
\end{quote}

En Marx, podemos inferir retroactivamente, se trataba , en cambio, del modo de producción capitalista, el cual le permitía leer paradigmáticamente los otros modos de producción y hacerlos inteligibles al encontrar los rasgos comunes (invariantes genéricos) de cualquier modo de producción circunscribiendo la singularidad sintomática expuesta por aquel mismo (\emph{i.e.} la plusvalía o la fuerza de trabajo). La desactivación, recorte o suspensión de un fenómeno histórico singular con respecto a los otros no es meramente caprichosa ni auto-evidente: hay una anticipación que marca (nombra) una problemática y que en retroacción trabaja (indaga) sobre el conjunto de fenómenos indicado. La noción topológica de paradigma es clave entonces para no caer en lo inefable (misticismo), pues el \enquote{junto a} que significa el término \enquote{para} (\emph{pará} en griego) da cuenta de la importancia de la vecindad/contigüidad topológica en la que adviene lo inteligible de un fenómeno sensible \rdm{\emph{entre} los fenómenos}, en lugar de suponer un misterioso \enquote{más allá}: \enquote{Así como en la reminiscencia \rdm{que Platón usa con frecuencia como paradigma del conocimiento}, un fenómeno sensible es puesto en una relación no-sensible consigo mismo y, de esta manera, reconocido en el otro, así también, en el paradigma, no se trata simplemente de constatar una cierta semejanza posible, sino de \emph{producirla a través de una operación}}. \footcite[32]{@7070-AGAMBEN2009}[(las cursivas son nuestras)]


De igual modo, en la \emph{operación-Marx} no se trataba tampoco de ver más allá (o más profundo) que los economistas clásicos. Recordemos que Marx produce el concepto de fuerza de trabajo al formular la pregunta latente en los textos de los economistas clásicos: ¿qué cuesta producir el trabajo? Como hacía notar Althusser, en \emph{Para leer el capital}, no era que estos no vieran un objeto que estaba allí presente (sensible) sino que \enquote{no veían que lo habían visto} (relación no-sensible) pues habían dado una respuesta correcta sin haber planteado siquiera la pregunta. Ello requería una reformulación de la problemática en su conjunto. Al igual que los policías, en \emph{La carta robada} de Poe, simplemente desestimaban que la clave del asunto se encontrara a plena vista, dada la importancia por ellos atribuida a su requisa y a la supuesta exhaustividad de su método. No son Marx o Althusser, entonces, quienes ostentan una mirada superior \enquote{sobre} los otros; al contrario, suelen ser los representantes \emph{normales} de la ciencia (los agentes del orden) los~que, con sus valoraciones rígidas acerca de lo que \emph{debe haber}, no pueden dar-se cuenta de lo más simple (lo in-contado), aquello que se halla en la superficie discursiva, dicho sin saber.

En la misma dirección señalada por Agamben, de \emph{producir una operación},\footnote{Para ver cómo esta operación puede consistir incluso en trazar una \enquote{Ontología de la inoperancia}, el capítulo que lleva ese título, escrito por Manuel Moyano.} resultará esclarecedor \rdm{esperamos} traer a colación una fábula conocida en el ámbito psicoanalítico. La fábula psicológica del niño que dice tener tres hermanos: \enquote{Pedro, Juan y yo}, contándose él mismo entre ellos, nos sirve aquí para pensar la función ontológica de la pertenencia y la necesidad de autoexclusión correlativa a la misma. Pues de esa cuenta errónea, al menos para nuestros oídos adultocéntricos, podemos inferir que se trata más bien de aprehender a descontarse como \emph{término} (\emph{i.e}. yo) allí donde se ha captado una \emph{relación} de pertenencia (\emph{i.e.} fraternidad). \emph{Para darse cuenta como relación hay que descontarse como término}. Es decir, cuando se produce una relación (y la pertenencia quizá sea la mínima posible, por ende diremos \enquote{ontológica}) entre diferentes fenómenos o términos, uno de ellos se excluye de la regla, no se cuenta (es el paradigma), para dar lugar así a la inteligibilidad del conjunto que es lógicamente anterior al efecto de representación e identificación por predicados: es la pura pertenencia (presentación) antepredicativa, el ser-dicho.\footnote{En este sentido, ontológico-político de la pertenencia fraternal, se puede entender la frase poética de nuestro querido Atahualpa Yupanqui \enquote{yo tengo tantos hermanos, que no los puedo contar}; los in-contados devienen así múltiples genéricos más allá de lo familiar.} Por ello, es la relación captada o contada la que desustancializa \emph{ipso facto} la referencia de los términos; operación a ser recomenzada cada vez en cada ámbito del pensamiento. Allí convergen Heidegger, Badiou, Agamben y tantos otros. Debemos anticipar aquí algo que se irá mostrando en el recorrido por los distintos apartados del texto; que el paradigma de la (no) relación es, para nosotros, el anudamiento borromeo de los términos en juego (el \enquote{no} niega la sustancialidad de la misma).

Agamben menciona la dimensión ontológica de dicho proceder: \enquote{La inteligibilidad que está en cuestión en el paradigma tiene un carácter ontológico, no se refiere a la relación cognitiva entre el sujeto y el objeto, sino al ser. Hay una ontología paradigmática} \footcite[43]{@7070-AGAMBEN2009}. Es decir, ya no estamos en el campo puramente gnoseológico, donde cabría preguntarse ¿cómo podemos conocer?, o ¿cuáles son las condiciones de posibilidad del conocimiento?; estamos ahora en el campo del lenguaje, donde se juega la radical historicidad del ser, del decir y el acto.\footnote{Refiriéndose a esto, en una crítica a la reducción cognitiva de Lévi-Strauss, decía Agamben: \enquote{Lo que quisiéramos sugerir aquí es que cuando \rdm{como consecuencia de una transformación cuyo estudio no es tarea de las ciencias humanas} el lenguaje apareció en el hombre, lo problemático no pudo haber sido sólo, según la hipótesis de Lévi-Strauss, el aspecto cognitivo de la inadecuación entre significante y significado que constituye el límite del conocimiento humano. Igualmente y quizás aún más decisivo debe haber sido, para el viviente que se descubrió hablante, el problema de la eficacia y la veracidad de su palabra; es decir qué era lo que podía garantizar el nexo original entre los nombres y las cosas, y entre el sujeto que ha devenido hablante \rdm{y, por lo tanto, capaz de afirmar y prometer} y sus acciones}. \cite[][105]{@7073-AGAMBEN2010}.}

En los términos meta-ontológicos de Badiou: el múltiple singular es el punto de origen (\emph{arjé}) indecidible de la pertenencia antepredicativa de los múltiples hasta ese momento in-contados en la situación; es el sitio de acontecimiento, presentado pero no re-presentado, que permite hacer uno de (conectar) los múltiples genéricos cuya única propiedad es pertenecer a la situación, sin ningún rasgo o atributo que los legitime desde el punto de vista del estado (representación). Ahora bien, tal singularidad no está dada ni es auto-evidente, es necesario \rdm{a partir de un acontecimiento} producir una operación efectiva para circunscribir dicho borde. Tal operación es lo que junto a Badiou denominamos \enquote{sujeto}.

\section{Doble función (sujeto)}


Hay una forma teórica recurrente en la que se suele expresar el sujeto, independientemente que se lo nombre como tal; es el término a \enquote{doble función}. Recordemos una célebre nota al pie del texto \emph{El (re)comienzo del materialismo dialéctico}, en ella Badiou escribía:

\begin{quote}
	El problema fundamental de todo estructuralismo es el término a doble función que determina la pertenencia de los restantes términos a la estructura, término que a su vez se haya excluido por la operación específica que lo hace figurar bajo las especies de su representante {[}\emph{lieu-tenant}{]}, para retomar un concepto de Lacan. El gran mérito de Lévi-Strauss es haber reconocido la verdadera importancia de esta cuestión, aunque fuera bajo la forma del significante-cero. Se trata de una localización del lugar ocupado por el término que indica la exclusión específica, la carencia pertinente, o sea la \emph{determinación} o \enquote{estructuralidad} de la estructura.\footcite[][285]{@7134-BADIOU1969}
\end{quote}

El término a doble función, en la lectura althusseriana de Marx, era la instancia \enquote{económica} como \emph{dominante} y a la vez \emph{determinante} de los procesos sociales. Hay también una alusión más o menos explícita a Lacan y a Spinoza en aquél texto de Badiou. En Spinoza el término a doble función sería Dios, Naturaleza o Sustancia, por un lado, junto a sus atributos y modos por el otro. Un \enquote{doble bucle} conceptual por el cual Dios es definido formalmente a partir de sus atributos indefinidos (definición 6 de la \emph{Ética}).\footnote{Véase la Introducción que hace Vidal Peña a Baruch Spinoza.\footcite[][31]{@7135-SPINOZA2006}.} El cierre y apertura en un solo gesto de invención conceptual. En Lacan es el significante de la falta del Otro (S A/), definido en los \emph{Escritos} como aquel significante impronunciable sin el cual~los demás significantes no representarían nada, por lo tanto, circunscribible mediante una operación singular: la producción de un nombre propio.\footcite[][799]{@7142-LACAN2002} En Althusser es, como adelantamos, la economía como instancia dominante y a la vez determinante de los procesos sociales. El cambio de dominante o de coyuntura depende de la causa invariante, o sea la economía, que aun cuando opere \emph{in absentia} a partir de sus efectos es \emph{una} y determina la \emph{estructuralidad} de la estructura. En Badiou, el término a doble función sería la matemática como ontología, núcleo estable del discurso acerca del ser-en-tanto-ser, y como procedimiento genérico de verdad que cada tanto reestructura el campo ontológico. Pero hay aquí una inversión respecto de Althusser, ya que el cambio no proviene de una causa invariante (la economía) sino de múltiples procedimientos genéricos (ciencia, política, arte, amor) y lo que permanece estable, con su singular historicidad, es el núcleo ontológico-matemático. Es decir que tenemos, por un lado, un cierre que define la \emph{estructuralidad} de la estructura (presenta la presentación-múltiple) y, por otro lado, las aperturas imprevistas que producen las demás prácticas en los saberes establecidos, de nuevo: arte, amor, ciencia y política.

Ahora bien, se podría creer que en Badiou son precisamente las matemáticas, en tanto cumplen esta doble función, las que conforman el núcleo paradigmático de su pensamiento; y sin embargo, estas excluyen la dimensión propia del acto/intervención subjetiva que es tan importante en la obra del filósofo francés. Por ello nuestra hipótesis de trabajo afirma que, justamente, es más bien el concepto de sujeto el que constituye la función paradigmática \emph{par excellence} en su complejo sistema filosófico. El concepto de sujeto cumple así la doble función singular-universal del paradigma mencionada por Agamben: a) es uno más entre otros conceptos (situación, acontecimiento, intervención, verdad) y b) a la par permite la inteligibilidad del conjunto, en un doble sentido; por un lado, es tratado en referencia a la singularidad de distintos procedimientos genéricos de verdad, y, por otro lado, atraviesa continuamente la misma labor filosófica en sus múltiples operaciones conceptuales, pues permanece no dicho (o no elaborado explícitamente) que el sujeto filosófico es justamente la función de captación de las verdades de su tiempo, lo que define la especificidad de su acto. Lo dice \rdm{como al pasar} al menos dos veces Badiou; 1) en \emph{Condiciones}: \enquote{La filosofía no es nunca una interpretación de la experiencia. Es el acto de la Verdad respecto de las verdades. Y tal acto, que según la ley del mundo es improductivo (no produce siquiera una verdad) dispone un sujeto sin objeto solo abierto a las verdades que transitan en su captación}. \footcite[72]{@7072-BADIOU2002} Y 2) en \emph{Elogio del amor}: \enquote{He planteado que el filósofo (y bajo esta palabra, que se entiende en neutro, se encuentra también \emph{la} filósofa) sin duda debe ser un científico advertido, un aficionado a los poemas y un militante político, pero también que debe asumir que el pensamiento jamás es separable de las violentas peripecias del amor. Científico(a), artista, militante y amante, tales son los papeles que la filosofía exige de su sujeto. Y, a eso, es a lo que he llamado las cuatro \emph{condiciones} de la filosofía}. \footcite[4]{@7006-ALAIN2009}

En ambas citas aparece fugazmente enunciado este \emph{sujeto filosófico ficcional}, abierto a la captación de verdades, que Badiou cuando trata particularmente el concepto de sujeto desestima, ya que para él solo existe el sujeto cualificado, es decir, en cada caso artístico, científico, político o amoroso. No obstante, \enquote{ficcional} no debe entenderse en sentido negativo; así lo expone el mismo Badiou al presentar cómo concibe la producción filosófica: \enquote{Ficción de saber, la filosofía imita al matema. Ficción de arte, ella imita al poema. Intensidad de un acto, ella es como un amor sin objeto. Dirigida a todos para que todos estén en la captura de la existencia de las verdades, la filosofía es como una estrategia política sin apuesta de poder}. \footcite[71]{@7072-BADIOU2002} Que de esas cuatro ficciones conceptuales \rdm{en especial de su mutuo anudamiento} emerja contingentemente un sujeto, es algo en verdad sorprendente. El sujeto filosófico resulta así de la operación (implícita) de articulación contingente entre discursos heterogéneos e irreductibles. Así, pasamos de las figuras simbólico-estructurales del límite, del término a doble función, de la falta estructural, al nudo material. El sujeto \emph{qua} operación de~anudamiento.


Por otra parte, la incidencia de la filosofía sobre dichos procedimientos es muy difícil \rdm{sino imposible} de garantizar. Y no obstante, esta cuádruple ficción nos orienta sobre la especificidad de una práctica filosófica nodal materialista: ni simple oposición externa entre ciencia e ideología (o verdad \emph{versus} error) ni tampoco mero desfile de máscaras (una ficción por otra), la articulación rigurosa de ficciones conceptuales heterogéneas, compuestas de matemas, poemas, decisiones, tesis, exige un arduo trabajo singular no reductible a ninguna de sus condiciones. Cabe preguntarse, ¿estaríamos ante un nuevo duplicado empírico-trascendental? No hay nada semejante; el \emph{sujeto-operación} que resulta de esta composición heteróclita en nada se parece a sus condiciones concretas de existencia (no obstante su compatibilidad y composibilidad ensayada, una y otra vez, en torno a los demás procedimientos).

A continuación, la excelente lectura en clave materialista de Badiou que nos ofrece Bosteels, nos permitirá recorrer algunos núcleos conflictivos entre los \enquote{marxistas postestructuralistas} y, a su vez, desplegar el materialismo nodal en torno al concepto de sujeto filosófico que allí se anuda (sin saber). En este apartado es donde nos extenderemos un poco más, a fin de dar cuenta de los múltiples movimientos de cruce (conceptuales y genealógicos) que ligan a nuestros autores materialistas.

\section{Lecturas (post) althusserianas} % 4

\subsection{Materialismo dialéctico} % 4.1

En \emph{Badiou o el recomienzo del materialismo dialéctico}, Bruno Bosteels se propone articular dialécticamente las dos partes principales en que se ha dividido la obra de Alain Badiou: el ser y el acontecimiento. Esta división del trabajo intelectual, que supone comentadores o críticos interesados o bien en la ontología o bien en la teoría del sujeto, podría considerarse una consecuencia casi natural del título de su obra magna: \emph{El ser y el acontecimiento}\footnote{\cite[][]{@7143-BADIOU1999}}. Pese a todos los recaudos tomados, finalmente, la escisión entre dos esferas de pensamiento habría prevalecido sobre las complejas articulaciones conceptuales desplegadas en los 37 capítulos del libro. En este contexto de incipiente recepción badiousiana, la intervención de Bosteels consiste en volver al libro quizá más arduo de Badiou, \emph{Teoría del sujeto},\footcite[][]{@7144-BADIOU2009} para mostrar la articulación dialéctica y materialista de su pensamiento: \enquote{Mi tesis es que esta articulación {[}ser y acontecimiento{]} puede considerarse dialéctica en un sentido que hoy en día resulta más controversial e intempestivo que nunca. Eso es lo que pierde de vista el lector que se centra \emph{o bien} en las tesis ontológicas \emph{o bien} en la teoría del sujeto, poniendo firmemente el ser, por un lado, y el acontecimiento por el otro}.\footcite[][10]{@7022-BOSTEELS2007}

De esta manera, Bosteels se encargará de remarcar sobre todo el modo en que los procedimientos genéricos de verdad se producen \emph{en} situación afectando al ser mismo y forzando al vacío estructural a dar cuenta de lo in-contado (múltiples genéricos). No existe un ámbito de pensamiento separado del otro, la distinción y división articulada entre ambos se da en un proceso dialéctico. Por supuesto, todos estos motivos se pueden encontrar desarrollados exhaustivamente en \emph{El ser y el acontecimiento}; sin embargo, parece que resulta muy difícil para los comentadores y críticos de Badiou pensar en simultaneidad los axiomas matemáticos del ser \emph{en tensiónarticulada} con las intervenciones subjetivas artísticas, políticas, amorosas y científicas que aquél comenta y reelabora. Incluso el mismo Bosteels, en su afán de rescatar del olvido dicha articulación pareciera reducirla al ámbito político-filosófico, en lugar de dar cuenta de ella en el complejo campo del discurso filosófico propiamente dicho que incluye \rdm{\emph{externamente}} multiplicidad de prácticas. En fin, si bien su lectura está lejos de ser incorrecta, habría que situar con rigor el sesgo político que la orienta, al menos, si no se quiere menospreciar (como por momentos trasuntan algunas expresiones de Bosteels) el delicado trabajo filosófico de \emph{composibilitación} entre prácticas que lleva cabo Badiou en \emph{El ser y el acontecimiento}.

Sostenemos por nuestra parte que el materialismo filosófico de Alain Badiou se trama, más bien, en la heterogeneidad de hebras discursivas que componen su sistema, de cuyo complejo entrelazamiento resultan sus conceptos y, especialmente, del modo solidario que encuentra su disposición topológica. A este punto de vista se aproxima Bosteels, sobre todo al comienzo de su libro \rdm{aunque luego se diluya por el énfasis puesto en el desarrollo político} cuando retoma la definición materialista de la filosofía ofrecida por Althusser: \enquote{La filosofía produce una problemática general, es decir, una manera de plantear \rdm{y por tanto de resolver} los problemas que puedan surgir. La filosofía produce, en fin, esquemas teóricos, figuras teóricas que sirven de mediadores para superar las contradicciones y de vínculo para ligar y cimentar los elementos de la ideología}.\footnote{Louis Althusser citado por Bosteels \parencite[][17]{@7022-BOSTEELS2007}.} De este modo se puede circunscribir la diferencia entre materialismo e idealismo filosóficos, en su compleja e irresoluble tensión interna, a partir de una distribución topológica de las distintas prácticas (teóricas y no teóricas): \enquote{(\ldots) podemos ya inferir no solo que la tendencia materialista se encuentra en una relación desigual y asimétrica con el idealismo, sino también que, debido a esta desproporción, la definición impura de las prácticas teóricas y su \emph{relación de exclusión interna} con otras prácticas constituyen, juntas, la sustancia misma de toda filosofía materialista}.\footcite[][18]{@7022-BOSTEELS2007} (cursivas nuestras)

Dicha relación de \enquote{exclusión interna} entre las diversas prácticas (políticas, científicas, artísticas, etc.) señala el principio topológico básico del nudo material que trama la práctica filosófica. Se entiende así que lo que está excluido \emph{internamente} sea lo que impide el cierre absoluto de las distintas prácticas sobre sí mismas (lo que conduce al idealismo) y, a su vez, posibilita la apertura problemática hacia las demás.\footnote{En otro artículo hemos mostrado la diferencia entre \enquote{exclusión interna} (lógica del Todo) e \enquote{inclusión externa} (lógica del no-Todo), recurriendo a los operadores ontológicos de pertenencia e inclusión. \cite[][]{@7145-FARRAN2007}. Cabe decir que, a nuestro modo de ver, es más consistente con la posición materialista aquí esbozada pensar la función de composibilitación de la filosofía desde la lógica del no-Todo, pues no supone \emph{a priori} excepción alguna para garantizar la consistencia, sino el mero engarce nodal \emph{partes extra partes}: múltiples de múltiples que no remiten a ninguna totalidad auto-contenida.} Y esto mismo daría cuenta de lo que define la tarea de la filosofía materialista para Badiou: \emph{composibilitar}. Como también lo es, simultáneamente, impedir suturas o reducciones entre los distintos procedimientos genéricos, es decir, preservar su irreductible singularidad. El materialismo filosófico articula/composibilita, pero además sostiene la apertura de los procedimientos impidiendo cierres y homogeneizaciones. De este modo, podemos entender que lo material o real no sea solo lo que resiste toda lógica o concepto (como decía Palti), sino aquello que produce los mismos desplazamientos y atravesamientos discursivos (lógico-conceptuales); sus tensiones problemáticas, bifurcaciones y resoluciones parciales.

Al marcar las diferencias entre algunos autores marxistas postestructuralistas, Bosteels sitúa tanto a la dupla Laclau-Mouffe como a Žižek del lado de la dialéctica estructural idealista, mientras que Badiou representaría un modo de pensar que reactiva la dialéctica materialista \emph{stricto sensu}. Los primeros solo llegarían a postular la idea (negativa) de imposibilidad de cierre del orden social, la \enquote{causa ausente} que opera o bien como vacío estructural/inconsistencia de lo simbólico \rdm{el caso de Laclau-Mouffe} o bien como resto pulsional de goce que gira en torno de ese vacío \rdm{el caso de Lacan/Žižek}. Una suerte de estática estructural que imposibilitaría cualquier transformación radical de la situación. En cambio, Badiou plantearía un paso más respecto de la dialéctica estructural \enquote{idealista}, esto es: la articulación efectiva (positiva) de una \enquote{nueva consistencia} a partir del despliegue de un proceso subjetivo de verdad.

En realidad, para ser justos con los autores mencionados, todos ellos intentan salir del mero señalamiento del vacío estructural adjudicado (desplazado) ahora al desconstructivismo; el asunto pasa más bien por evaluar qué recursos teórico-conceptuales son movilizados en ello.\footnote{Algo que hemos señalado en otro artículo escogiendo acentuar las convergencias entre ellos. \cite[Véase][]{@7146-FARRAN2009}.} Laclau, por ejemplo, queda en cierto impasse entre su teoría de la hegemonía, por un lado, que describe la simultánea necesidad-imposibilidad de constitución del orden social, y la postulación de una democracia radical, por otro lado, que impida que una articulación hegemónica se eternice en el poder. Sucede que es difícil explicar el cambio. Pues, sabemos bien que la imposibilidad de significación absoluta (la inadecuación significante/significado) remite siempre, es decir estructuralmente, a una significación contingente y parcial, pero quien puntúa, quien determina el significante Amo, esa parte de la sociedad que \emph{contingentemente} ostenta el poder y determina así la \emph{materialidad del sentido}, puede ser siempre la misma (\enquote{puede saberlo muy bien, pero aun así\ldots}, tal es la fórmula de la denegación fetichista a la que es tan afecto Žižek).

Por tanto, lo que falta en dichas elaboraciones conceptuales es, más bien, un tercer componente que abra el juego de las marcaciones significantes y las desestabilice, más acá de su simultánea necesidad/imposibilidad lógicas. He aquí donde Laclau apela, en cuanto recursos teóricos, al \enquote{afecto} y al \enquote{objeto a}; términos exportados desde el psicoanálisis para dar cuenta de cierta corporalidad o consistencia diferencial de las políticas emancipatorias (sobre todo a partir de \emph{La razón populista}.\footcite{@7021-LACLAU2007}) Pensar estas nuevas consistencias más allá de las estructuras lógicas dominantes y sus juegos diferenciales/equivalenciales, que remiten al lenguaje propio de cada situación, requiere disponer de distintos recursos teórico-conceptuales para darles \emph{cuerpo} y \emph{materialidad} (operación que también efectúa Žižek a partir de \emph{Visión de paralaje}.\footcite{@7020-ZIZEK2006}) De allí la riqueza que se encuentra en el pensamiento de Badiou, por caso, al desplegar este múltiples operadores conceptuales y no solo de la lingüística o el psicoanálisis sino también de las matemáticas, la literatura, la música, la arquitectura, el marxismo, etcétera, a fin de darles cuerpo a las verdades que \emph{aparecen} en situación.\footcite{@7019-BADIOU2008} (motivo principal de \emph{Lógicas de los mundos})

No se trataría por lo tanto, en relación con estos autores afines, de evaluar epistemológicamente quién lleva la razón, pues todos ellos comparten en definitiva un marco ontológico-político (en) común, sino de cómo dan cuenta desde su propia posición: i) de la destitución subjetiva; ii) de la de-suposición de un sujeto al saber; iii) del descentramiento correlativo a las verdades; iv) de la des-totalización de los saberes; v) de la producción de articulaciones conceptuales complejas en la dispersión epocal actual; vi) de atender a la singularidad-universal de los acontecimientos.

Y así como Badiou nos propone en \emph{Lógicas de los mundos}\footcite{@7019-BADIOU2008} leer \emph{Teoría del sujeto} a partir de las formulaciones ontológicas de \emph{El ser y el acontecimiento}, Bosteels invierte este orden de lectura al proponerse leer \emph{El ser\ldots} a través de \emph{Teoría\ldots} colocando en un primer plano el legado marxista-althusseriano y su lenguaje político-filosófico. Esta lectura es, en efecto, muy productiva ya que logra establecer una genealogía conceptual rigurosa de los planteamientos actuales en torno a lo político despejando, al mismo tiempo, de las críticas a Badiou algunos malentendidos simplificadores.\footcite{@7022-BOSTEELS2007}

No obstante, consideramos que gran parte de los malentendidos proviene no solo de no entender la especificidad de la dimensión política post-acontecimental del pensamiento badiousiano,\footcite{@7071-FARRAN2009} tal como remarca Bosteels, sino de no atender suficientemente la originalidad de los planteamientos ontológicos, en su compleja imbricación con lo acontecimental, que son desplegados en \emph{El ser\ldots} y redefinen los términos nodales en que suelen ser planteadas las problemáticas heredadas del marxismo (la prevalencia de las categorías económicas por ejemplo), la filosofía (cuestiones ontológicas) y el psicoanálisis (cuestiones subjetivas).

No hay que perder de vista, entonces, la riqueza y singularidad que entrañan las nuevas demarcaciones, tomas de posición y decisiones de pensamiento presentes en \emph{El ser y el acontecimiento} (en fidelidad al estilo de pensamiento althusseriano más que a su lenguaje) que amplían enormemente los recursos conceptuales y desplazan los nudos temáticos prevalentes en los 60s y 70s (subordinados mayormente a la lógica política). Fundamentalmente a partir de dos operaciones: 1) la postulación de las matemáticas como ontología racional (una teoría consistente de la inconsistencia) y 2) la elevación del número de condiciones de la filosofía a cuatro (no solo política o ciencia, como en Althusser, sino también arte y amor).


En esta compleja tensión entre los axiomas matemáticos (ontológicos) y los procedimientos genéricos de verdad (condiciones) se traman multiplicidad de operaciones conceptuales que nos permiten recorrer los problemas clásicos de la filosofía desde un punto de vista original y novedoso, sin desconocer por ello las deudas y herencias para con las distintas corrientes de pensamiento precedentes.\footnote{Así como en Spinoza, por ejemplo, entre la postulación de Dios y la definición formal por sus atributos se desplegaba todo el rigor demostrativo de la \emph{Ética}, en Badiou es necesario recorrer todas sus articulaciones conceptuales en torno a los procedimientos genéricos de verdad para captar lo que implica la \enquote{composibilidad}.} En este sentido, mientras Althusser pareciera pasar abruptamente de la mítica rigidez de las leyes de necesidad histórica (materialismo histórico), o de la \emph{estructuralidad} de la estructura determinada por la causa ausente de un \enquote{todo complejo ya dado} (materialismo dialéctico), a la pura contingencia de los átomos sociales cayendo en el vacío y encontrándose por mera casualidad (materialismo aleatorio); en Badiou, en cambio, el pasaje entre necesidad y contingencia se encuentra matizado a través de múltiples y heterogéneas operaciones conceptuales (le llamamos: \enquote{materialismo nodal}).

Las críticas más torpes a Badiou \rdm{como dice Bosteels} insisten no obstante sobre la división dicotómica entre el ser y el acontecimiento, ignorando así la complejidad temática del sujeto, de la intervención en que este se constituye y, en general, de todo lo atinente a la dimensión post-acontecimental de una verdad \emph{qua} procedimiento genérico. Bosteels rescata justamente \emph{Teoría del sujeto} porque allí el concepto aludido aparece expuesto de manera simple y contundente en la escisión misma entre la determinación del lugar estructural (P) y la fuerza (A). A partir de lo cual el sujeto es definido por la vuelta en torsión sobre la determinación estructural, al desplazar su mismo emplazamiento o determinar su determinación. En lugar de una simple oposición/conjunción externa, A y/o P, se da una compleja dialéctica entre estos términos: Ap (A) es la determinación estructural de la fuerza, y A (Ap) indica el límite que abre (y \emph{fuerza}) la totalización estructural. \enquote{La teoría del sujeto de Badiou gira en torno a cómo, exactamente, entendemos su relación dialéctica de inclusión externa, ya sea como un supuesto estructural o como un proceso dividido}.\footcite[][89]{@7022-BOSTEELS2007} Tal como sostuvimos en otro artículo,\footcite{@7071-FARRAN2009} consideramos que es en la diferencia entre \enquote{inclusión externa} y \enquote{exclusión interna} donde se juega la clave (sexual y material) que articula la disyunción problemática entre la lógica del Todo y la lógica del no-Todo (más que en la diferencia entre \enquote{supuesto estructural} y \enquote{proceso dividido}). Esto es: la posibilidad de pensar \emph{en} y \emph{desde} una totalidad abierta, sin límites definidos, que incluya externamente \emph{partes extra partes} según conexiones contingentes y no según la lógica continente/contenido. Pensar, en fin, cómo se produce un sujeto-\emph{fixión}\footnote{Con el doble sentido que le da Lacan a esta palabra: ficción y fijación (parcial). Es también nuestra forma de responder a las paradojas suscitadas entre lo discursivo y extra-discursivo, lo textual y contextual (\emph{i.e.} Foucault-Derrida); la incompletitud de cada práctica hace de \emph{exterior constitutivo} relativo a otra, pero para captarlo hay que recorrer sus bifurcaciones e impasses.}en el cruce de múltiples sujetos efectivos (reales). Mientras Bosteels enfatiza el papel del forzamiento o torsión de la verdad sobre el saber (o del acontecimiento sobre el ser), sobre todo en política, lo que intentamos remarcar por nuestra parte es la materialidad misma de la práctica filosófica, en las múltiples torsiones conceptuales producidas entre los diversos procedimientos genéricos de verdad (o sea, la composibilidad).

Pero la crítica más fuerte de Bosteels a Badiou (al menos al de \emph{El ser\ldots}) está en una nota al pie donde manifiesta claramente que el malentendido sigue girando en torno a la singularidad de la lógica acontecimental:

\begin{quote}
	En \emph{El ser y el acontecimiento}, Badiou adoptará y reformulará esta definición lacaniana del sujeto como aquello que un significante S\textsubscript{1} representa para otro significante S\textsubscript{2}: un sujeto es, entonces, lo que un acontecimiento E\textsubscript{1} representa para otro acontecimiento E\textsubscript{2}. Esto viene a demostrar la orientación estructural-ontológica, potencialmente equívoca, de esta obra más reciente, cuya inevitable parcialidad debería ser suplementada con la orientación topológica de una (nueva) teoría del sujeto.\footnote{\cite[][91-92]{@7022-BOSTEELS2007} nota al pie 15}
	\end{quote}

	¿Pero implica acaso el \enquote{entredós} acontecimental señalado por Badiou exactamente \emph{lo mismo} que el \enquote{entredós} significante estructural de Lacan? Pues ciertamente no, ya que el primero define al sujeto activo que conecta \emph{lo mismo} en lo real mediante un acto de invención que asume el exceso: trazos supernumerarios de acontecimientos dispares, ilegales; mientras que el sujeto evanescente del significante solo alcanza a afirmar, en eclipse, las marcas estructurales que lo determinan, quedando así en una suerte de melancólica letanía ante la falta de-velada (\emph{lo mismo} en lo simbólico)\footnote{Lo mismo en lo real, lo simbólico y lo imaginario aparece distinguido en \cite[][]{@7147-MILNER1999}. En el contexto de la obra de Alain Badiou interpretamos lo mismo en lo real como la conexión de dos acontecimientos por fuera de la ley simbólica (o en su forzamiento), lo que define una intervención-sujeto; mientras que lo mismo en lo simbólico señala la inercia propia de la ley de cuenta-por-uno.}. Por eso pensamos que es más productivo relacionar el \enquote{entredós} acontecimental badiousiano con el sujeto dividido en el~acto lacaniano (el sujeto que se halla identificado a un significante, separado de la cadena, y no por eso menos dividido). Además, en esto reside la novedad del sujeto introducido por Badiou: su definición es eminentemente \emph{intervalar}. No solo el sujeto, sino también el acontecimiento y la intervención son definidos en \emph{El ser y el acontecimiento} de manera intervalar, en remisiones recíprocas entre ellos que abren y articulan un nudo temporal de naturaleza eminentemente retroactiva.

	Hasta cierto punto, entonces, resulta tan inútil como incierto intentar reducir los planteos de Lacan al paradigma dialéctico estructural, pues él mismo trabajó incansablemente sobre la materialidad de los procesos subjetivos a partir de distintos conceptos psicoanalíticos forjados con aportes de las más diversas disciplinas: filosofía, lingüística, lógica, topología, etcétera. El asunto clave, creemos, pasa más bien por poder distinguir las diferentes prácticas y, a su vez, por el modo de intentar dar cuenta de ellas en otra instancia\footnote{Sobre las lecturas reduccionistas de Badiou que hacen algunos psicoanalistas, inadvertidos de estas diferencias de dispositivos, véase aquí mismo el capítulo de Daniel Groisman \enquote{Ontología del sujeto}; y para una lectura atenta de las posibilidades que abren las categorías psicoanalíticas, fuera del dispositivos clínico, véase el capítulo de Gala Aznares y Mercedes Vargas \enquote{Ontología de la falta}.} \rdm{teórica o de transmisión si se quiere}. Pues Lacan trataba de seguir los procesos de subjetivación que tenían lugar en los análisis individuales, mientras que Badiou intenta hacer otro tanto respecto los procesos subjetivos que tienen lugar en el arte, la ciencia, la política y el amor, en tanto verdades genéricas cuya localización es siempre un múltiple singular que cuestiona el \enquote{lugar}~mismo. En ambos casos, Lacan y Badiou evitan la formulación de un metalenguaje único y estabilizado, tarea a la que se aboca la filosofía idealista, por ello movilizan y reformulan múltiples conceptos a fin de seguir la singularidad de los procesos materiales concretos a partir de los cuales una verdad se produce en situación (en acto). De hecho, Lacan también plantea la posibilidad de una \emph{nueva consistencia} que exceda la estática estructural simbólica a partir de lo real del anudamiento borromeo, como antes lo había formulado en relación con el acto analítico y al pase (final de análisis). Hay que tener en cuenta, entonces, la especificidad de~cada dispositivo de pensamiento para no exigirle a uno lo que le corresponde al otro.

	Incluso sería interesante plantear aquí una inversión respecto a la disposición habitual de los discursos psicoanalítico y filosófico. Algunos autores, partiendo del enorme respeto y consideración que tiene Badiou por el psicoanálisis, han argumentado que este último bien podría inscribirse como un procedimiento genérico más (forzando la afirmación badiousiana de que todo discurso contemporáneo sobre el amor tiene que considerar lo elaborado por el psicoanálisis). No caben dudas que Badiou, por otra parte, recurre continuamente a fórmulas lacanianas en sus conceptualizaciones filosóficas. Sin embargo, hay que tener en cuenta también que ha calificado a Lacan de anti-filósofo o sofista, siguiendo la mención de este último sobre su práctica como anti-filosofía. Proponemos entonces una reformulación polémica de estas disposiciones típicas. ¿Y si la filosofía, tal como la concibe Badiou, pudiera considerarse \emph{in extremis} un psicoanálisis de la cultura, un psicoanálisis radicalmente extendido hacia las producciones genéricas de verdad? Pero, muy por el contrario de reducirse a un análisis patologizante de las formas sociales, o centrado incluso en la búsqueda del origen mítico de éstas, se trataría ahora de pensar lo que tiene lugar cuando se han roto los lazos de sentido prevalentes; es decir, las dinámicas heterotópicas que despliegan los sujetos de verdades materiales anónimas y anómalas a las situaciones establecidas, en lo que Badiou denomina \enquote{procedimientos genéricos de verdad}. Hay algunos elementos fuertes que habilitan justamente esta lectura; por ejemplo la idea de que las verdades agujerean el sentido y que la filosofía, en consecuencia, las piensa \emph{conjuntamente} por fuera de los saberes establecidos, acogiéndolas en su estricta singularidad, en lugar de prescribirles un lugar determinado y un \emph{deber ser} bajo el auspicio de un saber absoluto. En fin, la ética de las verdades de Badiou y la ética del psicoanálisis, resumida apodícticamente en la frase: \enquote{no ceder en el deseo propio}, son fuertemente compatibles. Se trata de un \emph{saber hacer allí} con el impasse estructural puesto en acto, desplazado, y no mera mostración de un vacío.

	Por otra parte, el pasaje entre necesidad y contingencia, así como la ruptura con la circularidad de ciertas formulaciones teóricas y, a la par, con la estabilización de un metalenguaje, pueden ser pensados en filosofía a partir de la articulación borromea de distintas consistencias discursivas propuesta desde el psicoanálisis lacaniano. Bastará con abrir los círculos (conceptuales) y extenderlos al infinito de su propia producción y repensar, simultáneamente, la idea de cruce (\emph{clinamen}) al modo de una trenza. Entonces serán necesarios al menos seis movimientos de cruce alternados entre las consistencias discursivas para articular una consistencia borromea.\footnote{Esta operación topológica, muy simple \rdm{al menos para todo aquél que ha hecho una trenza}, Lacan la muestra en uno de sus seminarios: \enquote{La trenza está en el principio del nudo borromeo. En efecto, por poco que ustedes crucen de manera conveniente esos tres hilos, los reencuentran en orden en la sexta maniobra, y eso es lo que constituye el nudo borromeo}. \cite[Véase][]{@7148-LACAN1976-1977}.} Si se corta un hilo el conjunto entramado no se sostiene más, se desvanece. De este modo, cada discurso (como un hilo) pasará en algún punto por encima de otro de manera alternada \rdm{lo que habilita las estratificaciones locales y las tesis metadiscursivas, \emph{i.e.} \enquote{las matemáticas son la ontología}} pero luego ese \emph{otro} pasará por encima del primero, y así también sucederá en relación con un tercero y a un cuarto (como en un trenzado infinito). Por lo tanto, no hay metalenguaje, como no hay discurso \emph{determinante en última instancia} de la inteligibilidad de los demás (sutura); cada discurso cumple esa función en algún punto de cruce y hace a la consistencia nodal del conjunto \rdm{virtualmente infinito}. No necesitamos de la totalidad auto-contenida de un saber absoluto para pensar la articulación, pues los círculos se encuentran escindidos, abiertos, basta con que se crucen alternadamente entre sí para tejer una consistencia nodal, sin jerarquías estructurales rígidas y, en cambio, con una máxima solidaridad entre ellos. Desde nuestro punto de vista, la filosofía de Badiou puede considerarse materialista en sentido lacaniano, más precisamente, por el modo de anudamiento en que articula los conceptos que por una concepción dialéctica de los mismos, aunque esta última no sea incompatible con la primera.\footnote{En un trabajo reciente distingo tres anudamientos distintos (los dos primeros explícitamente mencionados por Badiou no así el último) pero a su vez entrelazados que atraviesan el dispositivo badiousiano y conforman, desde mi punto de vista, su singular materialidad: a) el corpus/nudo textual anunciado en las Introducciones de \emph{El ser y el acontecimiento} y \emph{Lógicas de los mundos} (capítulos matemáticos, filosóficos y conceptuales); b) el nudo de condiciones o procedimientos genéricos (arte, ciencia, amor y política); c) el nudo de temporalidades heterogéneas (ontológico-matemática, situacional-ideológica, acontecimental-genérica). \cite[][]{@7133-FARRAN2010}.}

	Resulta extraño entonces que Bosteels no tome en cuenta las últimas elaboraciones de Althusser en torno al materialismo aleatorio para pensar el pasaje entre necesidad y contingencia. Sí lo hace Badiou al mostrar cómo este permanece, aun así, atado \rdm{aunque oscilando} a un sistema de doble sutura entre dos condiciones privilegiadas: política y ciencia.\footcite[][59-86]{@7010-BADIOU2009} Aunque, por supuesto, la idea principal de la filosofía entendida como práctica efectiva de delimitación de campos, posicionamiento, decisiones de pensamiento y tesis ya está formulada claramente en Althusser: ser científico \emph{en} política y político \emph{en} ciencia. Se trata así de transferir elementos de ámbitos heterogéneos; de circular y atravesar los dispositivos discursivos a fin de desnaturalizarlos. Condición histórica de extranjería y nomadismo de la filosofía. Por eso afirmamos que el materialismo nodal circunscribe lo real en doble sentido: i) en los procesos de indagación que horadan los saberes y lenguajes propios de cada ámbito de pensamiento, y ii) en la heterogeneidad de estos mismos procedimientos genéricos entre sí, habilitando cruces y transferencias de conceptos en un espacio de \enquote{composibilidad} filosófico no totalizable.

	Althusser diferenciaba la dialéctica marxista de la dialéctica hegeliana precisamente a partir de la consideración del \enquote{todo complejo estructurado ya dado}, en lugar del mítico origen de un principio simple (identidad del ser y la nada, por ejemplo). Es decir que la filosofía surge en el devenir de procesos históricos abiertos y complejos ya dados \rdm{ya comenzados} y no como pensamiento sobre el origen (\enquote{toma el tren en marcha} dirá Althusser). \footcite[Véase][]{@7014-ALTHUSSER2002} El asunto clave, aquí, es que se abren dos vías diferentes de indagación según se coloque el acento sobre lo \enquote{estructurado ya dado} o sobre el \enquote{complejo heterogéneo y plural de prácticas}; o sea, la diferencia entre \enquote{estructura} y \enquote{multiplicidad} Esto es lo que nos permite pensar con cierta rigurosidad el planteamiento filosófico de Badiou. De algún modo Bosteels, al privilegiar, en cambio, el esquema simple del sujeto expuesto en \emph{Teoría del sujeto}, parece caer en la crítica de Althusser a Hegel (la contradicción simple entre lo mismo y lo otro). Pues, al contrario, pensar el concepto de sujeto en la complejidad de múltiples enlaces inter e intra-discursivos \rdm{\emph{entre} matemáticas, política, poesía, psicoanálisis} resulta más consecuente con la concepción dialéctica materialista, ya que no busca aislar un principio simple sino pensar (en) el entramado complejo de relaciones que habilitan la construcción del concepto, es decir, su \emph{sobredeterminación}. Por eso, ahora, le podemos llamar \enquote{nodal} a dicha práctica. Claro que también podríamos ser más comprensivos con Bosteels, e inscribir su aporte crítico dentro de los usos demostrativos o políticos de la perspectiva filosófico-materialista tal como hacían, por ejemplo, Marx y Engels con las categorías hegelianas. \footcite[Véase][165]{@7051-ALTHUSSER1965}
	Aunque no se trataría ya, estrictamente hablando, de una \enquote{práctica teórica} en sentido althusseriano.

	En definitiva, Bosteels intenta separar a la filosofía de Badiou de la identificación simple con la ontología matemática y su comentario meta-ontológico, pues ello \rdm{piensa él} retrotraería el materialismo dialéctico planteado en \emph{Teoría del sujeto} a la dialéctica estructural idealista. En esta última, sabemos, predomina la referencia a lo \emph{real} como causa ausente y no como proceso activo e incesante de división,, por lo tanto, el riesgo reside allí en definir los conceptos exclusivamente en referencia a las estructuras ontológicas (axiomas y teoremas), cuasi negativamente, por lo que permiten o no pensar las matemáticas. Sin embargo, el esfuerzo correlativo que realiza Bosteels por identificar la filosofía con los procesos post-acontecimentales puede incurrir en el error contrapuesto. No se percata que Badiou afirma enfáticamente que la filosofía no se identifica con ninguno de los procedimientos genéricos, ni tampoco, por supuesto, con la ontología; es más bien un espacio topológico de \emph{composibilidad} entre las diferentes consistencias discursivas.\footnote{Bosteels parece desplazarse de una sutura matemática a una política en la obra de Badiou: el mismo movimiento que identifica Badiou en la obra de Althusser.} De allí que Badiou trabaje con heterogéneos operadores conceptuales: poemas, matemas, decisiones de pensamiento, tesis. Y esta modalidad de trabajo conceptual, insistimos, está desplegada a lo largo de todas y cada una de las meditaciones de \emph{El ser y el acontecimiento}. En fin, la originalidad de Badiou reside no solo en pensar lo nuevo: el acontecimiento y la verdad como proceso de reestructuración de la situación y su estado, sino la multiplicidad de procedimientos genéricos en simultaneidad: la amplitud de miras que despliega en su dispositivo filosófico, cuya materialidad se teje de heterogéneas consistencias discursivas (praxis), sin predominancias absolutas de ninguna de ellas sobre las otras (sin suturas). Veremos a continuación cómo esta modalidad de trabajo se puede aproximar al materialismo marxista.

	\subsection{Hacia otras lecturas de Marx (o la filosofía como práctica teórica)} % 4.2.

	¿Qué es el materialismo filosófico? Transvaloración, afirmaremos, esto es: cruces entre distintos sistemas de valoración. No se trata de invertir (valorar lo menos valorado), tampoco de instaurar un nuevo sistema de valores (o de no tener ninguno), sino de valorar fragmentos de un sistema según los valores de otro; con lo cual ambos, en su mutua irreductibilidad, resultan modificados. Así se trabaja con las singularidades sintomáticas de cada discurso efectuando torsiones entre ellos. No es solo un trabajo sobre fragmentos particulares desestimados, ni con otros sistemas generales de valor sustitutivos, sino con los cruces y torsiones efectivos \emph{entre} subsistemas, subconjuntos o partes singulares.

	En este sentido, podemos decir que Badiou aprendió la lección de Althusser: la filosofía no tiene consistencia en sí misma (o no tiene objeto) sino en sus \emph{condiciones}; de allí que se aplique a la lectura sintomática de lapsus, vacíos, excesos e impasses de los distintos discursos (praxis), efectuando torsiones entre ellos. La diferencia que marca Badiou respecto de Althusser, en este aspecto, es la multiplicación de síntomas y terrenos problemáticos a indagar y composibilitar; lo cual no nos parece que sea mera ostentación de una erudición incierta, sino la posibilidad misma de la filosofía como pensamiento activo de su época y no mera repetición estéril de otros discursos (de-suturada de cualquier identificación plena con una de sus condiciones). Por ejemplo, como lo muestra la lectura de la teoría matemática de Paul Cohen según el problema filosófico de los indiscernibles, tal es la operación materialista transvalorativa que realiza Badiou en los últimos capítulos de \emph{El ser y el acontecimiento}.

	Desde la perspectiva nodal que aquí sostenemos, si pensamos que las distintas prácticas en tanto instancias del ordenamiento social pueden ocupar alternativamente un lugar dominante o subordinado en relación con las otras, entonces la determinación en última instancia por la economía podría ser re-pensada como el efecto mismo del entramado \rdm{el nudo \emph{entre} ellas} y no en el sentido de una instancia particular universalizada (contable). Este sería un modo de evitar la remisión circular de un concepto o práctica clave: ni esencia originaria simple (Hegel), ni término a doble función (Althusser-Marx), más bien un nudo borromeo o trenza. Podemos hablar, así, de la existencia en un momento dado, en una coyuntura, de dominancias locales, parciales, de puntos de cruce, pero la determinación global dependerá \emph{en última instancia} del entramado mismo de todas y cada una de dichas instancias (prácticas); es decir que cada una de ellas será a su modo la \enquote{última}.

	Pero volvamos al concepto de \enquote{sobredeterminación} tal como lo empleaba Althusser, para entender mejor este complejo proceso de articulación.\footnote{Para un desarrollo más extenso de este concepto véase en este mismo libro el capítulo de Andrés Daín \enquote{Ontologías de la sobredeterminación}.} Althusser comienza por explicitarlo a partir de la idea de \enquote{condición}: \enquote{{[}Las{]} condiciones no son, en efecto, sino la existencia misma del todo en un \enquote{momento} determinado, o \enquote{momento actual} del hombre político, esto es, la relación compleja de condiciones de existencia recíprocas entre las articulaciones de la estructura de un todo}.\footcite[Véase][171]{@7051-ALTHUSSER1965}

	Entender dicha reciprocidad \emph{entre} condiciones desde la mutua imbricación de un nudo borromeo nos aclara bastante la naturaleza de aquel \enquote{todo estructural complejo} supuesto por Althusser para pensar el orden social (anudamiento complejo no totalizable o contable necesariamente). La propiedad más simple del nudo borromeo es, como se sabe aunque jamás domine, que basta con que uno de sus cordeles se corte (y pueden ser estos infinitos) para que todo el entramado no se sostenga más. Así pues, si las revoluciones han ocurrido allí donde se han dado un conjunto de condiciones/contradicciones específicas, se esclarece también la teoría del \enquote{eslabón débil} (Lenin); ya que si pensamos en una articulación borromea, cualquier término es débil, en el sentido de que si es cortado todo es deshecho (en donde también se puede oír \enquote{desecho} como reverso excluido del todo). Por lo que se podría hipotetizar que, en un momento revolucionario, el nudo ontológico-político por el que se sostiene un ordenamiento social dado se visibiliza y, por ende, puede ser cortado desde cualquier instancia local, en tanto todas y cada una de ellas es débil y remite al conjunto articulado del que forma parte. El asunto clave no es solo cómo cortar el nudo sino cómo rehacerlo de otro modo. Ambas operaciones, corte y sutura, son simultáneas ya que si no fuera así solo habría pura dispersión (locura o muerte).

	Aunque hay que tener presente que Althusser se limita a hablar de \enquote{condiciones} en sentido político, mientras que Badiou habla también de condiciones artísticas, amorosas y científicas. Nosotros ahora pensamos que las \enquote{condiciones} en lugar de \enquote{reflejar} la contradicción principal o el antagonismo, como decía Althusser,\footcite[Recordemos que si bien Althusser complejiza el economicismo marxista vulgar al distinguir en el \enquote{todo-complejo-estructurado} la instancia \emph{dominante} de la \emph{determinante}, no puede evitar que la economía prime en última instancia: \enquote{Debido a que cada contradicción \emph{refleja en sí} (en sus relaciones específicas de desigualdad con las otras contradicciones, y en la relación de desigualdad específica entre sus dos aspectos) la estructura dominante del todo complejo en que ella existe, por lo tanto, la existencia actual de ese todo, y, por lo tanto, sus \enquote{condiciones} actuales, podemos hablar de \enquote{condiciones de existencia} del todo, refiriéndonos a las \enquote{condiciones existentes}}. Véase][172]{@7051-ALTHUSSER1965}[(cursivas nuestras).] presentan en torsión sobre sí \rdm{lo que disloca la estructura de presentación} lo real imposible. Pensar la distribución y colectivización de estas tensiones/contradicciones dispares que constituyen al ser social topológicamente, a partir del nudo borromeo, nos abre así múltiples posibilidades articulatorias donde las subordinaciones y dominancias necesariamente se invierten. En vez de reducirse solo al reflejo imaginario, pensamos en la imbricación simultánea de los tres registros: diferencia y semejanza (imaginario), unicidad (simbólico) e irreductibilidad (real). El ser es nodal.

	Althusser, inclusive, se acerca bastante a la noción de nudo al indagar la \enquote{contradicción principal}: \enquote{Ella constituye el \enquote{eslabón decisivo} que es necesario detectar y atraer hacía sí en la lucha política, como dice Lenin (o en la práctica teórica\ldots), para coger toda la cadena o, para emplear una imagen menos lineal, ella ocupa la posición nodal estratégica que es necesario atacar para \emph{\enquote{desmembrar la unidad}} existente}. \footcite[Véase][175]{@7051-ALTHUSSER1965} Vemos cómo Althusser se aleja de la idea de una cadena causal, lineal o rígida, y se aproxima a la articulación compleja y flexible del punto nodal. Enfatiza también allí el papel de la condensación (metaforización) por sobre el papel del desplazamiento de dominante (metonimia).

	En \emph{Teoría del sujeto} Badiou también suturaba la filosofía a la condición política. Las otras condiciones estaban a modo de \enquote{ilustraciones} como bien señala Bosteels. \footcite[][106]{@7022-BOSTEELS2007} Por eso habría que remarcar que, en \emph{El ser y el acontecimiento}, la revolución de pensamiento que produce Badiou, como ya hemos dicho, no solo reside en nombrar a la matemática \enquote{ciencia del ser-en-tanto-ser} (ontología) sino en trabajar sobre y desde la estricta equivalencia de los distintos procedimientos genéricos. A riesgo de repetir. Bosteels señala el peligro de que con \emph{El ser y el acontecimiento}, y la importancia brindada allí a las matemáticas, se retorne al idealismo propio de la dialéctica estructural algebraica que denunciaba el mismo Badiou en \emph{Teoría del sujeto} (lo que denominaríamos \enquote{predominancia de lo simbólico}). Sin embargo, parece que Badiou sigue la empresa althusseriana incluso más allá de la propia tradición marxista. Además del énfasis puesto en los procesos post-acontecimentales \rdm{que señala repetidamente Bosteels} de lo que se trata es de captar la especificidad de la misma praxis filosófica, al anudar consistencias discursivas tan heterogéneas entre sí con el recurso siempre renovado de diferentes operadores conceptuales; consistencias que justamente, anudadas a partir de sus suplementos sintomáticos, muestran su propia inconsistencia (o vacío inherente) y la posibilidad de articulación de una nueva modalidad de consistencia que no remite al Todo (ni ausente ni presente). Tal tarea de la filosofía acerca a Badiou al pensamiento del \enquote{todo estructurado complejo} que intentaba articular Althusser con el objeto de evitar las simplificaciones y homogeneizaciones producidas por el abuso de los conceptos hegelianos en las lecturas de Marx (un peligro en el que puede caer la lectura de Badiou propiciada por Bosteels, al enfatizar este el esquema simple de subjetivación presentado en \emph{Teoría del sujeto}, pues el asunto no es tanto remarcar el vacío o el exceso en el despliegue de los procesos, sino pensar en la heterogeneidad y complejidad implicadas en las articulaciones conceptuales propuestas filosóficamente). Por lo tanto, si nuestro deseo crítico se inscribe en la estela de cierta fidelidad al pensamiento filosófico materialista, no es tanto la oposición idealismo estructural \emph{vs} materialismo dialéctico lo que deberíamos sostener, sino la composición efectiva de conceptos filosóficos acudiendo a distintos registros (real, simbólico e imaginario) y procedimientos de verdad, en su anudamiento recíproco y alternado, composible. Así, que nos inscribamos en este modo de producción teórica no obedece a captar núcleos ideales inmutables (exportados, \emph{i.e.} de Europa) sino a estar dispuestos a re-anudar los materiales disponibles, y sus posibilidades impensadas, en cada tiempo y espacio singular.

	En este sentido, y para retomar un tópico recurrente en los estudios sobre recepción, podríamos decir que las ideas \emph{están} siempre fuera de lugar (\emph{i.e.} Schwarz). Y lo están \emph{estructuralmente}, pues, si en verdad lo \emph{son}, las ideas cuestionan los lugares comunes de los lenguajes establecidos y sus significaciones estañadas; podríamos afirmar incluso que pertenecen al \enquote{tercer género} de conocimiento en Spinoza (veremos más adelante), es decir, son singularidades universales que marcan el punto límite de una situación y a la vez permiten nombrar lo genérico en ella (son paradigmas). Podríamos hablar también de distopía y distorsión pero, en verdad, quizá sea más justo decir: heterotopía y torsión. Como suele repetir Žižek, metaforizando el pasaje de la teoría especial de la relatividad a la generalizada: lo real no es un agujero en el orden simbólico (de las ideas) sino la torsión misma de dicho espacio. A lo cual deberíamos agregar que hay, más bien, \enquote{torsiones} en distintos puntos del espacio simbólico.

	Las ideas no \enquote{reflejan} \rdm{mal o bien} una realidad previamente constituida, como tampoco la crean espontáneamente; contribuyen, en todo caso, a circunscribir lo \emph{real} imposible de cada situación (su inconsistencia de base) para desplazarlo en un proceso de construcción (de indagaciones aleatorias) que, en parte, determina lo que llamamos habitualmente realidad, pero que también la excede y cuestiona permanentemente. Entonces, más que de una cuestión de adecuación o inadecuación de las ideas a la realidad (o de distorsión o reflejo correcto), lo cual respondería a una concepción especular del conocimiento y la acción, de lo que se trata es de tomar las ideas como \emph{instrumentos de uso} para circunscribir lo que siempre queda por fuera del marco simbólico, desplazado. En ese sentido, las ideas verdaderas son eternas, pero dicha constatación performativa solo puede efectuarse retroactivamente a partir de múltiples históricos singulares e intervenciones concretas (síntomas o paradigmas). Es decir, las verdades son procesos genéricos, no son \emph{a priori} ni \emph{telos}; se hallan más bien en el medio, en intervalos y brechas sobre lo estático (o sobre sus modificaciones regulares). Por otra parte, las verdades son para todos y cualesquiera (\emph{quodlibet}), no hacen distinciones con base a particularidades de ningún tipo. En este sentido, podríamos decir, la distorsión de las ideas no es un problema exclusivo de Latinoamérica; todos los grandes pensadores han elaborado sus teorías recurriendo a materiales simbólicos provenientes de otras latitudes, tiempos y disciplinas, y la singularidad de sus respectivos aportes a la cultura universal ha residido, más bien, en el modo original que han encontrado de anudar dichos materiales en respuesta al malestar real de su época.

	Es de sobra conocida, por ejemplo, la caracterización que hace Engels de la teoría marxista en el \emph{Anti-Dühring}: esta sería el producto resultante de la fusión entre la filosofía alemana, la economía inglesa y la política francesa. Pero, ¿y si el verdadero materialismo de la filosofía marxista fuera, más acá de consideraciones historicistas o epistemológicas, el nudo ontológico-político resultante del entrelazamiento entre estos tres \emph{corpus} discursivos? Por supuesto, seguimos aquí en parte la inspiración althusseriana del \enquote{materialismo aleatorio}. No obstante, en lugar de átomos cayendo en el vacío pensamos más bien en cuerdas (consistencias) trenzándose borromeanamente, tal como podemos apreciar, por ejemplo, en el paciente trabajo teórico de Marx a partir de la elaboración de distintos operadores conceptuales: ni pura necesidad histórica de leyes preestablecidas, ni pura contingencia de encuentros en el vacío. Cuando hablamos de consistencias discursivas no referimos tampoco a totalidades auto-consistentes (\enquote{esferas}) que luego se reunirían necesariamente en una suerte de síntesis absoluta \emph{à la} Hegel, sino a las partes sintomáticas que estas presentan, a veces sin saber: los núcleos problemáticos de las prácticas teóricas y no teóricas. He allí el trabajo material de Marx: encontrar huecos, puntos de falla, horadar los \emph{corpus} de los saberes establecidos en su ficción de completitud, para anudarlos entre sí (requisito indispensable para que se produzca un anudamiento: el agujero). Incluso es necesario romper con la idea de secuencia lineal; no es primero el encuentro de la falla discursiva y luego el enlace conceptual, pues precisamente la falla (lapsus o síntoma) se torna visible a la luz de la consideración simultánea de los otros materiales discursivos que convergen en el anudamiento. No de sus diferentes \enquote{perspectivas} que nada tienen que ver entre sí (¿qué tendría que aportar la mirada política socialista francesa a la economía científica alemana, o viceversa?) sino del uso crítico y creativo que el autor hace de sus respectivas categorías y elementos: proceso de extrapolación y transferencia por el cual estos mismos resultan transformados (desnaturalizados). En este sentido, la ventaja de Marx sobre economistas, filósofos y políticos \enquote{puros} reside en su posibilidad de \enquote{circulación} \rdm{y articulación compleja} entre las diversas producciones teórico-prácticas de su tiempo \rdm{y de otros}.

	Materialismo entonces, el de Marx, no solo en el sentido común de que partía y se ocupaba de la actividad práctico-concreta de los hombres, sino en el sentido \emph{extrañamente} inclusivo en que él mismo era un (teórico) práctico que anudaba las elaboraciones sintomáticas de los discursos que subtendían dichas actividades. Los presupuestos impensados, condiciones de posibilidad, singularidades y puntos sintomáticos que torna visibles una lectura crítica como la de Marx provienen de la capacidad de circulación, atravesamiento y frecuentación de distintos dispositivos discursivos. Los puntos de falla se tornan visibles en los propios términos de cada dispositivo práctico-discursivo a partir de las suplementaciones y forzamientos que habilitan otras prácticas, pero los términos que habilitan estos procesos críticos son esencialmente evanescentes, pues no permanece de ellos marca alguna (\emph{clinamen}). Por ello la sensibilidad de Marx para \emph{volver lo sensible específico de un campo teórico \rdm{como la economía política} en una relación no sensible consigo misma} permite la inteligibilidad de dicho campo, pero tal torsión proviene de la circulación entre distintos dispositivos, heterogéneos entre sí, que le permiten \enquote{ver que los otros no ven que ven} (continuando con las paráfrasis de la célebre frase althusseriana). La materialidad proviene así del cruce y anudamiento de prácticas teóricas y no teóricas; materialidad que activa a su vez la heterogeneidad constitutiva subyacente a la homogeneidad en la que se despliega habitualmente toda práctica normal. En fin, las ideas así concebidas deberían ser evaluadas en acto por su poder transformador y articulador de prácticas discursivas heteróclitas más que por su capacidad reflejante de una supuesta realidad estática.

	Hay que decir más. Lo real de la idea es el nudo mismo. \footcite[Véase en relación con esto:][]{@7149-FARRAN2010} Lo real no es ni una sustancia que permanecería eterna, fija e inmóvil en vaya a saber qué \enquote{reino ideal platónico}, ni tampoco es la evasiva regulación trascendental kantiana que se sustrae eternamente a su propia aprehensión. Estas dos figuras de lo real dan cuenta más bien de su abordaje imaginario y simbólico respectivamente. Lo real en tanto real, como nudo efectivo, \emph{consiste} en la mutua imbricación de lo simbólico, lo imaginario y lo real. De lo real como pura multiplicidad inconsistente que solo \emph{habrá sido} en la suspensión retroactiva de una marca simbólica esencialmente \emph{abierta} a otras marcas, incompleta, que a su vez resulta suturada ficticiamente por un imaginario cuyo \emph{exceso} consiste en regular lo contable. De la horadación respectiva de cada uno de estos registros por los otros resulta una consistencia nodal: el materialismo de la idea.\footnote{Expresión que Badiou empieza a utilizar a partir de \emph{Lógicas de los mundos} para continuar calificando su platonismo por medio del oxímoron; así como antes había sido \enquote{platonismo de lo múltiple} ahora será \enquote{materialismo de la idea}.}

	\subsection{Materialismo aleatorio} % 4.3

	En este sentido, el marxismo y el psicoanálisis devienen doctrinas materialistas rigurosas, en virtud de su capacidad de anudar contingentemente los distintos registros de la experiencia (de la praxis) y no de una supuesta capacidad \enquote{reflejante}. Como dice Natalia Romé:

	\begin{quote}
De lo que se trata, entonces, tanto en marxismo como en psicoanálisis, es de una común concepción del conocimiento; sobre esta se funda, podemos conjeturar, la cientificidad que ambos ejercen por pleno derecho. Tanto en uno como en otro no se trata, advierte Althusser, sino de \enquote{casos} singulares y diferentes. Spinoza nos habla de \emph{intuitio}\textbf, como los médicos hablan de \enquote{intuición crónica} o los analistas hablan de \emph{Einsicht} \rdm{nos recuerda} pero, ¿cómo abstraer cualquier cosa de intuiciones singulares y por tanto, no comparables?\footcite[][]{@7075-ROME2009}
\end{quote}

Con relación a esta última pregunta, aclaremos que no se trataría tanto de comparar en función de rasgos positivos identificables (imaginario) o, por ejemplo, en términos de cadenas de equivalencia (simbólico), sino de afirmar \emph{lo mismo} en lo real por medio de una intervención teórica efectiva que anude lo singular en el cruce de los tres registros: real, simbólico, imaginario. Tal singularidad deviene así universal en virtud de su \emph{genericidad} (y, cabe decir, generosidad). Como lo expresa Althusser (citado por Romé):

\begin{quote}
Spinoza hace caso omiso de la objeción, al igual que Marx o que el análisis (\ldots): en la vida individual y social no hay más que singularidades realmente singulares (pero universales), puesto que esas singularidades están como atravesadas y habitadas por invariantes repetitivas o por constantes; no por generalidades (\ldots) Constantes o invariantes genéricas, como se prefiera, que afloran en la existencia de los \enquote{casos} singulares y que permiten su tratamiento (teórico o práctico, poco importa) Constantes e invariantes genéricas, y no \enquote{universales}, constantes y no leyes.\footcite[][]{@7075-ROME2009}
\end{quote}

Y continúa Romé: \enquote{Nos encontramos entonces con un materialismo de los \enquote{casos}, en el sentido de \enquote{lo que cae}, es decir, de \emph{acontecimientos}. Tal como amplía en una serie de entrevistas realizadas por Fernanda Navarro: \enquote{(\ldots)  no existen en el mundo sino casos, situaciones, (\ldots) lo que \thirdquote{nos sobreviene} sin prevenir. Esta tesis, (\ldots) es la tesis fundamental del nominalismo. (Este) no solo es la antesala, sino que es ya el materialismo}}.\footcite[][]{@7075-ROME2009}

¿Cómo lo que cae imprevistamente, el acontecimiento, daría cuenta de lo invariante? Parece una paradoja, y de hecho lo es: el cambio da cuenta de lo que permanece. Es que, justamente, a partir del cambio imprevisto se aprecia de manera retroactiva lo que permanece (lo que \emph{habrá sido}) luego de la caída de semblantes, de la investidura imaginaria que soporta(ba) la realidad. Así, habría que tener en cuenta dicha inversión: lo que cae efectivamente en el acontecimiento es la imagen ideológicamente sostenida de la realidad-toda (unificada), con sus remisiones fijas entre lo particular y lo universal a través de rasgos y atributos establecidos (leyes); mientras que lo que queda \rdm{el resto} tras la caída, es el invariante genérico propiamente dicho (multiplicidad inconsistente), por tanto, no discernible en el lenguaje de la situación pero \emph{pensable} en términos de \enquote{singularidad} (múltiples al borde del vacío). Como vimos al principio, las singularidades tienen la extraña propiedad topológica de pertenecer a la situación y a la vez encarnar su límite inmanente, al borde tanto del vacío como de las multiplicidades infinitas incontadas. Los invariantes genéricos, las~verdades, son múltiples de múltiples, infinitos, no identificables por rasgos particulares fijos; por eso se trata de procedimientos genéricos no estáticos. De allí que siempre se vislumbre, en el relámpago de una caída fortuita, su recomienzo. Pero no hay que maravillarse con dicho acontecimiento sino, más bien, con lo que autoriza ilegalmente su ocurrencia: el despliegue infinito de las consecuencias, es decir, su implicancia material. De este modo lo imaginario cae y se aprecia así, fugazmente, lo real en su pura inconsistencia o invariancia genérica, mientras que lo simbólico es forzado por la operación misma de nominación supernumeraria que conecta los singulares (\emph{i.e.} invención conceptual). Permanencia y cambio, historicidad y eternidad, necesidad y contingencia son apenas dos modos del pensamiento (de la praxis) que se conectan e invierten mutuamente, sin reducirse uno al otro. Por tanto las ideas no se encuentran fijadas en ningún plano trascendental, sin ser por ello una continua variación contingente; su eternidad deviene más bien de la consideración del tiempo retroactivo: el \enquote{habrá sido para lo que está llegando a ser} en un proceso abierto hacia el futuro y hacia el pasado (futuro anterior). Así podemos entender cómo se da esta compleja dialéctica (nodal) entre lo negativo y lo positivo, la falta y el exceso, que tanto problematizan a la \enquote{izquierda lacaniana}.\footcite[Véase][]{@7003-STAVRAKAKIS2010} Normalmente se piensa que la falta refiere a lo simbólico (el significante de la falta del Otro en Lacan) y el exceso a lo real. Pero en la ontología de Badiou es el estado de la situación o metaestructura (imaginario) la que encarna el exceso \emph{par excellence} y oblitera la falta de la estructura o presentación (simbólico) expuesta en el múltiple singular o sitio de acontecimiento. Es decir que aquí el problema de la falta y el exceso está formulado en los registros simbólico e imaginario respectivamente, y lo real bien podría ser pensado tanto por el lado de la dislocación (imposible encuentro entre dos registros) como por el lado del anudamiento/conexión que produce la intervención/sujeto. No solo la primera nominación del acontecimiento, sino la serie de indagaciones posteriores conllevan esta lógica paradojal de la dislocación y el (re)anudamiento. De este modo no se trata de ninguna espera pasiva del \enquote{milagro del acontecimiento}, como dice Bensaïd, \footcite[][]{@7150-BENSAID2001}
pues hay, por el contrario, un trabajo continuo sobre la falta y el exceso, anudando incesantemente los términos en juego.

En este sentido, eminentemente temporal, Romé acerca Althusser a Lacan no solo en su retorno e intervención simultáneos sobre Marx y Freud respectivamente, sino, en particular, respecto a la separación compartida por ambos con toda filosofía de la conciencia (humanismo, subjetivismo), por una parte, y con todo historicismo, por otra. Esto último lo conduce a Althusser a repensar el concepto de temporalidad de manera no lineal \rdm{en el sentido indicado más arriba}. Dice Romé:

\begin{quote}
Podemos conjeturar, en este punto, que lo que encuentra Althusser en el psicoanálisis lacaniano es la posibilidad de bosquejar en conjunto una lógica para esta temporalidad, una lógica que provisionalmente recibe el nombre de \enquote{dialéctica materialista} y que, en principio atiende a los dos ejes que ya hemos sugerido: una ruptura de toda génesis teleológica, homogénea y continuista y, simultáneamente, una interrogación por la eficacia de la operación genética (ideológica) en la configuración de una identidad (sea subjetiva o social). Es aquí donde puede captarse el valor heurístico de la noción de \emph{sobredeterminación} y es aquí también donde cobra una densidad insospechada la frase de Roudinesco que advierte que Althusser \enquote{soñaba con convertir a Lacan a una filosofía capaz de superar la noción de estructura.} Podríamos decir nosotros que difícilmente podría tratarse de \enquote{convertir} a Lacan porque una tal filosofía ya operaba en su pensamiento, quizás, como gustaría de decir Althusser, \enquote{en estado práctico}. Así podría ser concebido su esfuerzo por desmarcarse de la polémica que atravesaba en aquel entonces al medio intelectual francés y que se conocía como la dicotomía entre \emph{historia y estructura}.\footcite[][]{@7075-ROME2009}
\end{quote}

Por último, en la siguiente cita, donde se menciona el descentramiento correlativo a la formulación de otra lógica temporal para el materialismo, resulta esclarecedor introducir el nudo borromeo para pensar la articulación de círculos a los que alude Althusser, sin necesidad de remitir a un centro único:

\begin{quote}
En esta línea de razonamiento, jugará un papel de gran relevancia la categoría freudiana de \emph{sobredeterminación, \enquote{a la vez como un índice y como un problema}};es decir, para indicar el punto en el que se hace necesaria una nueva teorización, una lógica del \emph{descentramiento}que permita explicar tanto el funcionamiento psíquico como el social, por fuera de las \enquote{Filosofías de la conciencia}, porque: \enquote{círculo de círculos, la conciencia no tiene sino un centro, que es el único que la determina: necesitaría poseer círculos que tuvieran otro centro que el de ella, círculos descentrados para que pudiera ser afectada en su centro por su eficacia, para que su esencia fuera \emph{sobredeterminada}por ellos}.\footcite[][]{@7075-ROME2009}
\end{quote}

En el próximo apartado, en discusión con Jean-Luc Nancy, discutiremos más \emph{puntualmente} nuestra ontología filosófica nodal.

\section{Materialismo nodal} % 5.

Hasta aquí hemos recorrido distintos tipos de materialismo emparentados entre sí (dialéctico, aleatorio) y a la vez hemos introducido de manera más o menos alusiva el materialismo nodal. Ahora lo haremos explícitamente.

J.-L. Nancy extrae \rdm{al igual que Lacan} la lección más simple del pensamiento de Marx: el capitalismo es un régimen donde todo valor está subordinado a la ley estricta de la equivalencia dada por la moneda y la forma-mercancía. Por lo tanto: \enquote{{[}e{]}l destino de la democracia está ligado a la posibilidad de un cambio de paradigma de la equivalencia}. \footcite[44]{@7012-NANCY2009} Pero no se trata de una simple sustitución de valores sino de la puesta en cuestión de todo sistema general de valoración, es decir, de la totalidad misma; como decía Lacan en el seminario XXIV \enquote{el todo no es más que una noción de valor}. \footcite[15]{@7067-LACAN1988} Escribe Nancy:

\begin{quote}
No será cuestión de introducir otro sistema de valores diferenciales: se tratará de encontrar, de conquistar, un sentido de la evaluación, de la afirmación evaluadora que le da a cada gesto evaluador \rdm{decisión de existencia, de obra, de porte} la posibilidad de no ser medido de antemano por un sistema dado, sino, al contrario, ser en cada oportunidad la afirmación de un \enquote{valor} \rdm{o un \enquote{sentido}} único, incomparable, insustituible.\footcite[45]{@7012-NANCY2009}
\end{quote}


Esta autovaloración de lo singular abre un espacio de igualdad no sometido a la regla de un valor universal fijado de antemano. Por otra parte, aunque tales valoraciones sean insustituibles e incomparables entre sí, esto no quiere decir que no sean compartidas en común; de hecho es justamente la singularidad afirmada en su propio valor irreductible la que permite que haya comunidad, al menos, tal como la piensa Nancy (\emph{La comunidad desobrada}). De eso se trata también la democracia, de posibilitar un espacio de encuentro de lo múltiple singular, sin las coerciones totalitarias de valores establecidos:

\begin{quote}
La democracia (re)engendra al hombre, declara Rousseau. Abre con nuevos bríos la destinación del hombre y del mundo con él. La \enquote{política} ya no puede dar la medida ni el lugar de esa destinación o destinerrancia (Derrida). Debe permitir su puesta en juego y asegurar sus lugares múltiples, pero no la asume. La política democrática es, pues, política alejada de la asunción. Pone término a toda especie de \enquote{teología política}, sea teocrática o secularizada. Postula en consecuencia como axioma que no todo (ni el todo) es política. Que todo (o el todo) es múltiple, singular-plural, inscripción en fragmentos finitos de un infinito en acto (\enquote{artes}, \enquote{pensamientos}, \enquote{amores}, \enquote{gestos}, \enquote{pasiones} pueden ser algunos de los nombres de esos fragmentos).\footcite[57]{@7012-NANCY2009}
\end{quote}

El axioma \enquote{no todo es política} resulta aquí fundamental. La democracia definida como lugar donde \emph{tengan lugar justamente} múltiples experiencias finitas (sujetos) del infinito actual (verdades); donde las diversas nominaciones del exceso modulen su propio tiempo; donde inventen y \emph{se} inventen sin subsumir o subsumirse a las otras; sin jerarquías ni \emph{telos} de ningún tipo. Si bien esto puede generar una imagen de cierta dispersión, de caos o eclecticismo, la existencia misma del espacio democrático como lugar de \emph{composibilitación} que impide las suturas abre, así, las vías de cruce y transferencias no prescriptivas, producidas al azar, entre los diversos procedimientos genéricos de invención. Afirmamos que esto mismo que Nancy plantea a nivel del pensamiento de lo político \rdm{la verdad de la democracia} es lo que Badiou hace filosóficamente.\footnote{O incluso Foucault cuando define, en sus últimos cursos y textos, la especificidad de la filosofía como la articulación histórica de prácticas irreductibles entre sí: saber (o \textit{aletheia}), poder (o \textit{politeia}), cuidado de sí (o \textit{ethos}). Véase, \textit{i.e.} \cite{@7069-FOUCAULT2010}; o \cite{@7037-FOUCAULT2002}.}


Entonces quisiéramos ahora (re)comenzar \rdm{antes de concluir} parafraseando a Alain Badiou. Él escribe en \emph{Lógicas de los mundos} que la ideología dominante en la actualidad, el materialismo democrático, se sostiene bajo la premisa fundamental de que \enquote{sólo hay cuerpos y lenguajes}. Es cierto, dirá Badiou, \enquote{excepto que hay verdades}. La excepción, inmanente a la existencia absolutista de cuerpos y lenguajes, es pronunciada en nombre de la reactivación del pensamiento dialéctico materialista. Habrá que decir \enquote{las} excepciones, más bien, pues las múltiples verdades aludidas son, por lo menos, los cuatro procedimientos genéricos que horadan los saberes (poderes) establecidos al evitar las clasificaciones propias de sus lenguajes: arte, ciencia, política y amor. De nuestra parte, para evitar cualquier suspicacia en la que se pretendiera distanciar a Badiou de Lacan \rdm{al menos donde, creemos, más se aproximan}, quisiéramos reafirmar que el materialismo nodal que se desprende de las elaboraciones de este último está en concordancia con la dialéctica materialista de aquel otro. Y lo afirmamos ahora de este modo: \emph{es cierto que solo hay lo imaginario y lo simbólico, de cuyo entrelazamiento recíproco emerge el sentido, excepto que hay lo real}. Y lo real en su doble estatuto: no solo como impasse o causa ausente estructural, sino \emph{nudo efectivo}; no solo aquel registro inasible que pasa fugazmente entre los otros dos (cualificable siempre de manera negativa: imposible, indecidible, indiscernible, etc.), sino \emph{el anudamiento efectivo de los tres}.

En \emph{El sentido del mundo},\footcite{@7011-NANCY1993} J.-L. Nancy nos brinda una perspectiva ontológico-política fuertemente vinculada a esta idea del nudo, del anudamiento, aunque no precisa el estatuto borromeo del mismo. Es curioso que Badiou, quien también recurre a la figura del nudo \rdm{sobre todo en \emph{Manifiesto por la filosofía},\footcite{@7126-BADIOU2007} donde resulta clave para pensar la articulación filosófica entre arte, ciencia, política y amor}, tampoco aluda específicamente al nudo borromeo.\footcite{@7072-BADIOU2002} Escribe Nancy:


\begin{quote}
Política de nudos, de anudamientos singulares, de cada \emph{uno} en tanto anudamiento, en tanto que relevo y relanzamiento del anudamiento y de cada nudo en tanto \emph{uno} (pueblo, país, persona, etc.), pero un \emph{uno} que no es \emph{uno} más que según el encadenamiento: ni el \enquote{uno} de una sustancia, ni el uno de un puro conteo distributivo. ¿En qué consiste un nudo?, ¿cuál es su unicidad, cuál es su unidad?, ¿cuál es su modo de ipseidad?, o bien, ¿en qué cosa toda ipseidad es ella misma, un nudo, una nudosidad?; ¿qué pasaría si en la comparación platónica del arte de lo político con el arte del tejedor ya no se considerara más el tejido en cuanto segundo, en cuanto sobreviniendo a un material dado, sino en cuanto primero, y en cuanto él mismo formador de la \emph{res}?, o aun, y para retomar un término que ya he utilizado, ¿qué pasaría si se considerara que nuestra \emph{comparecencia} precede toda \enquote{aparición}?\footcite[95]{@7012-NANCY2009}
\end{quote}

Como Nancy va a criticar la filosofía de Badiou por su énfasis en la puntuación conceptual, en función de la \emph{nodalidad} expuesta~en esta cita, resulta fundamental traer a colación la mención que hace Lacan respecto de un posible cambio de perspectiva en torno a la misma, al \emph{punto} más específicamente, habilitada por el nudo borromeo. Lacan nos sugiere pensar el punto no como un corte/intersección entre dos rectas sino como un sitio de cruce/engarce entre tres cuerdas, al modo borromeo. De este modo la puntuación, quizá más ligada habitualmente a la marcación significante de lo simbólico (conteo y corte), sería tan solo una de las dimensiones del punto nodal. Podemos pensar así un modo de articulación compleja que no depende solo de la marcación diferencial del significante, o del aislamiento de un rasgo imaginario positivo, y tampoco queda librada a la pura dispersión de lo real (o a su retorno en bruto siempre al mismo lugar). Al menos este ha sido nuestro modo de pensar el concepto en Badiou y es lo que nos habilita hablar de un materialismo nodal más allá del dispositivo clínico psicoanalítico.

Cuando Nancy discute sobre el estilo filosófico, dice que tanto Heidegger como Badiou toman la verdad con respecto a la pura presentación, es decir, en su estatuto fenomenológico. Para Nancy, en cambio, el sentido equivale al mundo, es decir, hay \enquote{encadenamiento} y \enquote{arrastre}, tales son los términos que emplea en consonancia con la modalidad antedicha; mientras que para Badiou la verdad corta el sentido, circunscribe un vacío y \rdm{según el término que utiliza Nancy} \enquote{puntúa} con conceptos. Así lo expresa Nancy: \enquote{el mundo nos invita a no pensar más en el registro del fenómeno, cualquiera sea este (surgimiento, aparición, investidura, brillo, advenimiento, acontecimiento), sino en el de, llamémoslo así por el momento, la disposición (espaciamiento, tacto, contacto, recorrido)} \footcite[21]{@7012-NANCY2009}.

Quizás no resultaría vano aclarar que el concepto de acontecimiento entraña una lógica mucho más compleja que la que se subsume bajo los términos fenomenológicos de brillo, surgimiento, aparición, etc. De hecho, la trivialidad del sentido que Nancy rescata como diferencia clave respecto a las metáforas del brillo (fálicas hay que decir) concuerda con el motivo de la verdad en Badiou como procedimiento genérico de despliegue de múltiples de múltiples (la banalidad del infinito actual).\footcite[22]{@7012-NANCY2009}

Nancy reconoce no obstante la comunidad de pensamiento entre diversos planteos contemporáneos (\emph{i.e.} Badiou, Marion) pese a la diferencia de estilos filosóficos: \enquote{Al ensayar estas distinciones y estos enunciados quisiera decir sin embargo que, si bien, en un sentido, opongo unas \enquote{tesis} a otras, en otro sentido subrayo, a través de oposiciones que dividen también mi propio trabajo, una comunidad de época}. \footcite{@7012-NANCY2009}

Es sabido que para Badiou la presentación equivale a la estructura, ley o cuenta-por-uno, y que, en términos de Lacan, respondería al orden simbólico (al \enquote{hay uno}); por lo tanto, la verdad entendida como proceso genérico que se sustrae a la marcación diferencial significante es indiscernible y, por eso mismo, no se presenta simplemente en situación. Una modalidad posible de entender el estatuto paradójico de la verdad es pensarla a partir de la figura topológica del doble bucle o torsión (ocho interior) que implica el primer movimiento acontecimental, esto es: presentarse en su presentación (la auto-pertenencia). La verdad no es pura presentación fugaz (\emph{aletheia}) ni tampoco coincidencia entre presentación y representación (\emph{adaecuatio}); es, más bien, el nudo contingente \rdm{que devendrá necesario} entre la presentación (borde del vacío de la \enquote{singularidad}), la representación (exceso de la \enquote{excrecencia}) y lo impresentado (multiplicidad genérica). En Badiou la presentación nunca se da sola, ya que toda presentación va de la~mano de la representación, por ello resulta conveniente tener en cuenta la torsión implícita en los dos extremos que traza el arco de su sistema teórico: la \emph{presentación de la presentación} es lo que opera la ontología matemática (sin uno) y, además, es la definición misma del múltiple acontecimental (ultra-uno). Entre el discurso del ser-en-tanto-ser por una parte, y los distintos procedimientos genéricos de verdad por otra, se encuentran múltiples torsiones conceptuales filosóficas que trabajan sobre las paradojas de dicha presentación imposible: dislocación, suplementación, forzamiento, correlatos, son algunas de las figuras teóricas que intentan dar cuenta de la misma. Lo interesante aquí es la variación de los conceptos realizada en el recorrido, tomando materiales matemáticos, poéticos, políticos o psicoanalíticos, según el caso.

Otro punto interesante a discutir con Nancy es la dinámica de \enquote{apertura} del sentido (del mundo), del \enquote{espaciamiento}, que postula como especificidad del trabajo filosófico; porque tal concepción no deja de tener resonancias con la tarea de \emph{composibilitación} en Badiou\footcite{@7072-BADIOU2002,@7126-BADIOU2007} y con lo que interrogaba Lacan respecto al nudo como articulación de agujeros (aberturas) que no se atraviesan mutuamente (no se cooptan o suturan) ¿Daría cuenta el nudo borromeo de esa apertura/articulación que no somete, sin tampoco dejar librados los términos a la pura dispersión?\footcite[23]{@7012-NANCY2009}

Encontramos resonancias de ello cuando Nancy escribe: \enquote{Pero lo \enquote{abierto} no es la cualidad vaga de una hiancia indeterminada, ni de un halo de generosidad sentimental. Lo \enquote{abierto} vuelve apretada, trenzada, estrechamente articulada, la estructura del sentido en tanto sentido del mundo}. \footcite{@7012-NANCY2009}


Cuando Nancy habla de \enquote{estilo}, \footcite[23]{@7012-NANCY2009} no se refiere a un simple ornato del discurso sino a la especificidad propia de la praxis filosófica, habitando la tensión entre ámbitos heterogéneos del pensamiento:

\begin{quote}
Se trata de la recuperación de una tensión interna de toda la filosofía, que le es originaria, y que es la tensión misma entre el sentido y la verdad. Lo que la filosofía por nacimiento o por constitución ha distinguido de sí misma bajo el nombre de mito es lo que caracterizaba como una identidad inmediata del sentido y de la verdad (un camino del sentido presentado, recitado) --identidad inmediata a la cual la filosofía no reconocía ni sentido ni verdad. La dislocación del mito proyecta los dos polos del \enquote{sentido} y de la \enquote{verdad} como los dos extremos de una tensión imposible de aplacar, que se vuelve, a la vez, tensión entre dos extremidades de estilo: la de la \enquote{poesía} y la de la \enquote{ciencia}. \footcite[24]{@7012-NANCY2009}
\end{quote}

Con Badiou agregaríamos dos polos más a la tensión indicada entre poesía (arte en general) y ciencia (matemática en particular), estos serían: política y amor. Su estilo filosófico discurre, entonces, entre estos cuatro tópicos heterogéneos impidiendo la sutura o dominancia de uno de ellos sobre los otros.

Por último, en \emph{La verdad de la democracia},\footcite{@7012-NANCY2009} Jean-Luc Nancy nos permite aproximarnos a esta extraña e inquietante concepción badiousiana sobre la tarea filosófica (luego de su acabamiento), que hemos mencionado tantas veces en este texto, a saber: \enquote{composibilitar} = componer/hacer posible. Si uno se atiene a los dos libros donde Badiou asume explícitamente la elucidación de la tarea filosófica, \emph{Manifiesto por la filosofía} y \emph{Condiciones}, entonces se percatará rápidamente que se constituye allí un espacio de pensamiento diferenciado, donde las distintas prácticas efectivas (procedimientos genéricos) entran en composición sin que ninguna domine a las otras ni ostente poseer la última palabra (el significante Amo). Nancy, como Rancière, habla de \emph{com-partir}; mientras Badiou saca a relucir este neologismo en francés que es traducido al castellano como \enquote{composibilitar} (com-partir justamente no lo intercambiable según la lógica Capital sino lo que carece de valor o, más bien, se auto-valora \enquote{junto a} otros, paradigmáticamente). \enquote{El elemento en el cual lo incalculable puede compartirse lleva por nombre arte o amor, amistad o pensamiento, saber o emoción pero no política}, \footcite[34]{@7012-NANCY2009} dice Nancy, \enquote{Esta se abstiene de aspirar a ese reparto, pero garantiza su ejercicio}. \footcite{@7012-NANCY2009} (política o bien filosofía, ¿política filosófica, acaso?) ¿Cómo mediante la abstención no obstante \enquote{garantizar} cualquier cosa? Es aquí donde comienza a desplegarse el campo de lucha filosófico y político donde apuntamos a desestabilizar cualquier intento de captura (o sutura) de un lenguaje sobre los otros, de un pensamiento sobre los otros, sea este de naturaleza política o estética, científica o afectiva. No hay lugar para las múltiples verdades cuando se ponen en marcha las maquinarias sapientes de discernimiento y clasificación; cuando aparecen los pequeños sacerdotes definiendo qué es arte, qué es política, qué es amor y qué ciencia. La lucha filosófica se despliega entonces contra todos aquellos epistemólogos, curadores, sabihondos y politicastros que siempre creen saber y pretenden así ostentar la última palabra sobre lo que no practican. ¿Queremos decir con esto, al modo wittgensteiniano, que \enquote{de lo que no podemos hablar mejor es callar} ?, ¿qué mejor practiquemos? En parte, solo en parte. Porque queremos decir y decimos, por otra parte, que podemos pensar \enquote{conjuntamente} con lo que pulsa por aquí y por allá (no solo la parte muda de la pulsión: la causa ausente, el vacío, el silencio), con lo que insiste; no hace falta definir fronteras ni áreas regionales ni sub-especialidades, \emph{hace falta} decir y pensar a riesgo propio, y hasta impropio, sin acudir tanto a definiciones \emph{a~priori} o protocolos; más bien a conceptos estelares como constelaciones, o maleables como trenzas; hablamos, en efecto, de anudar las pulsiones. Esta es nuestra práctica.



\section*{Referencias}
\printbibliography[heading=none]   % Sin título automático


%%%%%%%%%%%%%%%%%%%%%%%
\ifPDF
\separata{capitulo8}
\fi

\ifPDF
\chapter{Ontología de la falta}
\Author{Aznarez Carini \textit{y} Mercedes Vargas Gala} % poner nombre del autor
\setcounter{PrimPag}{\theCurrentPage}
\else
\ifHTMLEPUB
\chapter{Ontología de la falta}
\fi
\fi

%%%%%%%%%%%%%%%%%%%%%%%
% Empezar con \section{Introducción al Lorem Ipsum}
\nombreautor{Aznarez Carini \textit{y} Mercedes Vargas Gala}

\section{Introducción}

Este artículo intenta ser un aporte más que funcione como campo de apertura a nuevos anudamientos, a partir de la composibilidad de diversos discursos, para aportar al enriquecimiento de lecturas acerca del sujeto y su realidad, partiendo de reflexionar sobre el fundamento (ausente) desde el cual se piensa a los mismos. Esta apuesta apunta a dar un paso más en las implicancias epistémicas y ético-políticas de inscribir el pensamiento sobre la subjetividad en el campo filosófico-político, así como, pensar lo político y social en y desde las herramientas que aporta el psicoanálisis lacaniano. En función de esto y partiendo de la concepción de sujeto de la \emph{falta}, se intentará desarrollar y articular las principales categorías del discurso psicoanalítico (inconsciente, fantasma, sínthoma, goce, entre otras) que consideramos nos permiten profundizar en el análisis socio-político y composibilitar nuevas lecturas en la teoría política contemporánea. Para ello, nos proponemos primeramente realizar un recorrido por los conceptos psicoanalíticos que nos permiten reconstruir una nueva ontología (de la falta) sobre el sujeto (político). En segundo lugar intentaremos desarrollar el concepto de \emph{lo real}, en tanto posibilita re-pensar una ontología (política) del vacío, de la ausencia de fundamento. A su vez, analizaremos el valor político y conceptual que cobran ciertas categorías que se articulan desde el campo del psicoanálisis y la teoría política para el análisis de ciertos fenómenos y coyunturas políticas. Finalmente, destacamos la importancia ética y política que inscribe esta nueva ontología del vacío, de la imposibilidad, en tanto su apuesta implica el reconocimiento de los límites de toda producción socio-simbólica y discursiva, apuntando así a re-pensar los procesos socio-políticos a través de la articulación de campos de pensamientos heterogéneos que nos permitan abordar, solo parcialmente, cierta singularidad y romper con la ilusión de alcanzar una verdad trascendente y absoluta.

\section{El sujeto de la Falta}

El psicoanálisis como campo disciplinar se ha caracterizado desde su formulación freudiana por proponer supuestos ontológicos y epistemológicos diferentes a los establecidos por el contexto científico de la época; supuestos que posibilitaron la emergencia de nuevos dispositivos y prácticas de intervención en torno al sujeto. Sus formulaciones y desarrollos produjeron una ruptura con los postulados provenientes de la filosofía humanista y fenomenológica, en la que se sostenían los supuestos de la psicología de la conciencia. La corriente filosófica cartesiana, que sostiene que el individuo existe y se conoce a través del pensamiento mismo y su experiencia, es puesta en cuestión a partir de Freud.\footnote{Para un desarrollo exhaustivo sobre los pasajes operados dentro del pensamiento filosófico en torno al concepto de \emph{sujeto}, ver en este mismo libro el texto de Daniel Groisman: \enquote{Ontología del sujeto}.} En sus conceptualizaciones, en cambio, Freud considera la estructura psíquica como configurada y determinada por una instancia inconsciente que implica, como eje central, un desconocimiento, un descentramiento fundamental de dicha estructura. El supuesto de una instancia \emph{Inconsciente} se justifica para Freud por ser un concepto \emph{necesario}. Necesario en tanto la experiencia misma nos revela las limitaciones e imposibilidades de explicar todos los procesos anímicos solo desde los procesos y mecanismos conscientes. Ciertos fenómenos de la constitución subjetiva y la vida cotidiana, como los olvidos, sueños, síntomas, entre otros, se nos presentan como incomprensibles e inconexos si no consideramos para su comprensión los procesos inconscientes como parte de la estructuración psíquica. Al respecto, Freud afirma: \enquote{Es preciso, entonces, adoptar ese punto de vista: No es más que una \emph{presunción insostenible} exigir que todo cuanto sucede en el interior de lo anímico tenga que hacerse notorio también para la conciencia}. \footcite[][163]{@7031-FREUD2008} Destaca además la necesidad de distinguir \emph{lo inconsciente} como aquello que no se define a modo de una conciencia segunda, sino como \emph{actos psíquicos que carecen de conciencia} y, por lo tanto, implican el supuesto de una \emph{escisión (spaltung)} de la conciencia. No se trata entonces de una \emph{doble conciencia}, sino de \enquote{una misma conciencia la que se vuelve alternadamente a un campo o al otro}. \footcite[][167]{@7031-FREUD2008}

Freud conceptualiza esta división, esta escisión, a partir de la noción de \emph{represión primordial} en tanto operación originaria y constitutiva que consiste en que a \enquote{la agencia representante {[}\emph{representanz}{]} psíquica (agencia representante-representación) de la pulsión se le deniega la admisión en lo consciente}. \footcite[][143]{@7103-FREUD1990} En este punto Freud introduce varias cuestiones relevantes para comprender cómo está pensando la estructuración psíquica, ya que lo que está en juego en dicha operación represiva son mociones pulsionales, es decir, mociones de deseo que conforman para Freud el \emph{núcleo} (irrepresentable) del Inconsciente, y que solo adquieren representación, manifestación, por medio de un \emph{representante}. En sus palabras: \enquote{Una pulsión nunca puede pasar a ser objeto de la conciencia; solo puede serlo la representación que es su representante. Ahora bien, tampoco en el interior de lo inconsciente puede estar representada si no es por la representación. Si la pulsión no se adhiriera a una representación ni saliera a la luz como un estado afectivo, nada podríamos saber de ella}.\footnote{\cite[][173]{@7031-FREUD2008} Freud en \emph{Pulsión y Destino de Pulsión} (1915) distingue el concepto de pulsión como esencialmente diferente a un \emph{estímulo fisiológico} en tanto el primero representa un estímulo para lo psíquico cuya fuente tiene origen en el interior del propio organismo y no en el mundo externo como ocurre en el caso del segundo. Se trata a la vez de una fuerza \emph{constante} de la cual no hay huída posible, es decir, se trata de una fuerza incoercible e indomeñable que pulsa por la satisfacción. Más adelante re-conceptualizará las mociones pulsionales no solo como orientadas por un \emph{principio de placer} sino que incluso insisten \emph{más allá}, se trata de otro orden incluso contrario al \emph{deseo}. Para profundizar sobre esto, ver \cite[]{@7104-FREUD1984}.}

Se destaca, en función de lo anterior, que de lo que se trata en la represión primordial, que inaugura el inconsciente, es de una \emph{moción afectiva}, pulsional, originaria que no ha sido desplazada o sofocada por la operación represiva sino que constituye aquello nunca realizado, aquello del orden de lo \emph{no nacido}. Son producciones anímicas que solo tienen acceso y resultan susceptibles de conocer a través de sus manifestaciones disfrazadas, es decir, solo a partir de la desfiguración que sobre las mismas ejerce el mecanismo represivo para su expresión. Los chistes, lapsus y actos fallidos, los sueños y los síntomas, en tanto producciones del inconsciente, son resultado de procesos de elaboración en los que se ponen en marcha dos mecanismos de funcionamiento: el \emph{desplazamiento} y la \emph{condensación}. En el primero caso, se desplaza la carga energética que inviste a una representación hacia otro contenido con el que guarda una asociación más o menos distante. En el segundo, una representación puede tomar sobre sí la \emph{investidura} o energía de otras representaciones. De esta manera, los contenidos reprimidos adquieren acceso a la conciencia por medio de la figuración enmascarada que obtienen a partir de su relación con otros contenidos anímicos.

A partir de estas conceptualizaciones, Lacan retoma los textos freudianos para destacar lo que nos aporta su descubrimiento, a saber, que de lo que se trata en el discurso psicoanalítico es de las incidencias del orden simbólico en la naturaleza humana, de los efectos del lenguaje en el individuo. El sujeto es definido así como aquello que va más allá de lo que el individuo puede experimentar subjetivamente; ese más allá marcado por una historia, ya escrita y desconocida, que se manifiesta en las escansiones, en los hiatos del discurso.\footcite[]{@7105-LACAN2000} Es mediante los juegos del lenguaje como el sujeto se nos presenta en la experiencia ya que el sujeto mismo es efecto de este encuentro con el lenguaje. Es en la retórica del lenguaje (desplazamientos sintácticos, metáforas, metonimia, sinécdoque, condensaciones semánticas) donde Freud, indica Lacan, nos enseña a leer las emergencias del sujeto del inconsciente. Las producciones del inconsciente (lapsus, síntomas, actos fallidos, chistes) adquieren la forma de una escritura, de una palabra a cifrar. Se evidencia en el trabajo del sueño que sigue las leyes del significante (metáfora/metonimia) para su formación, como también en el síntoma entendido como significante de un significado reprimido para la conciencia del sujeto. El inconsciente, nos dice Lacan, es \enquote{aquella parte del discurso concreto en cuanto transindividual que falta a la disposición del sujeto para restablecer la continuidad de su discurso consciente}. \footcite[][248]{@7105-LACAN2000} En la cita anterior vemos como se define al inconsciente en tanto estructurado como un lenguaje, homologando su lógica de funcionamiento con la lógica del significante; el inconsciente es un discurso y en tanto tal responde a los mecanismos de producción propios del lenguaje. El sujeto es efecto así de su encuentro con el significante, con aquellos significantes, marcas, huellas que le vienen del Otro y que determinan su emergencia en el campo simbólico. Y en tanto el discurso siempre nos viene del Otro el inconsciente mismo es definido como \enquote{el discurso del Otro}. \footcite[][254]{@7105-LACAN2000}

Lacan introduce así la función del lenguaje en tanto estructura que organiza el universo simbólico, como aquellos significantes que configuran la red de relaciones sociales y humanas a partir del cual el sujeto adviene y se significa. Al respecto, afirma: \enquote{Para nosotros lo importante es que en esto vemos el nivel donde \rdm{antes de la formación del sujeto, de un sujeto que piensa, que se sitúa en él} algo cuenta, es contado, y en ese contado ya está el contador. Solo después el sujeto ha de reconocerse en él}. \footcite[][28]{@7106-LACAN2006}

En línea con esto, la apuesta de Lacan consistió en resaltar que la experiencia subjetiva, no puede ser determinada ni definida a priori, sino que su emergencia se pone en juego en cada situación local en el encuentro con un Otro donde los procesos de subjetivación adquieren sentido y materialidad. Sin embargo, el ingreso al orden simbólico no se realiza sin más, sino que exige y conlleva una pérdida de la porción del ser, en tanto no hay posibilidad de que la estructura del lenguaje logre una representación plena de la realidad y del sujeto. En tanto función universal el significante conlleva la pérdida de lo más singular del sujeto en el efecto de su representación. El sujeto se concibe como efecto-producto de la inscripción significante que ejerce el Otro con su discurso, \enquote{reduciendo al sujeto en instancia a no ser más que un significante, petrificándolo con el mismo movimiento con que lo llama a funcionar, a hablar, como sujeto}. \footcite[][215]{@7106-LACAN2006} Esta relación dialéctica entre el sujeto y el Otro se engendra para Lacan en el encuentro con una \emph{falta} en el Otro, en los cortes propios de la lógica significante. Se produce en esta relación, en el encuentro con una ausencia fundamental, una brecha ontológica que Lacan denomina \emph{hiancia} y a partir de la cual la función del inconsciente, su emergencia, se evidencia a modo de \emph{tropiezo}, corte, con su aspecto de \emph{falla} y fisura, como aquello que no encaja en el orden del discurso. En la discontinuidad, en la vacilación del discurso es en donde el inconsciente se aparece como fenómeno, es allí que \enquote{una cosa distinta exige su \emph{realización} (\ldots). Lo que se produce en esta hiancia, en el sentido pleno del término \emph{producirse}, se presenta como \emph{el hallazgo.} Así es como la exploración freudiana encuentra primero lo que sucede en el inconsciente}. \footcite[][33]{@7106-LACAN2006}

Lacan destaca, en sus desarrollos, la función ontológica de la hiancia, como aquel espacio de apertura, de indeterminación en el que \emph{algo} de otro orden emerge, es la forma que adquiere la representación de una falta. Se trata del límite de la significación que no es, dice Lacan, \enquote{el no-concepto sino el concepto de la falta}. \footcite[]{@7106-LACAN2006} De esta forma, las producciones del inconsciente conjugan y vehiculizan aquello reprimido, no simbolizado e innombrable de otra manera, siendo emergencias subjetivas (de un sujeto acéfalo, pulsional) que ponen de manifiesto aquello no representado, no nacido, a la espera de su emergencia; como dice Lacan, aquello que \enquote{no es ni ser ni no-ser, es no-realizado}. \footcite[][38]{@7106-LACAN2006} Esta falla cuando se presenta, interpela al ser en su posición presentificando de este modo una imposibilidad radical, una \emph{falta constitutiva} que cuestiona la consistencia de su ser y pone de manifiesto la \emph{falta-en-ser}.

Se establece, de esta manera, una nueva ontología, que Lacan reconoce en Freud a partir de recuperar sus conceptualizaciones, y que implica un nuevo modo de pensar al sujeto y sus modos de constitución; en tanto su identidad ya no se concibe como una totalidad cerrada sino que emerge a partir de una \emph{spaltung (división)}, de un clivaje. Para Lacan \enquote{el inconsciente se manifiesta siempre como lo que vacila en un corte del sujeto --de donde vuelve a surgir un hallazgo, que Freud asimila al deseo}. \footcite[][35]{@7106-LACAN2006} Se trata en este sentido, del \emph{borramiento} que ejerce el significante sobre el sujeto en tanto tachado (\$), y que Freud atribuye a la operación de censura, a la represión, que inaugura a su vez la dimensión de la \emph{pérdida}, del \emph{deseo.} La \emph{represión primordial} sería, desde estos desarrollos, una operación originaria que marca el límite de la simbolización, de la representación y la manifestación consciente y que posibilita, al mismo tiempo, el ingreso del ser hablante al campo discursivo, al campo del Otro. El sujeto del inconsciente es producto del encuentro con el lenguaje, encuentro traumático que implica su \emph{afanisis (desaparición}). Lacan, en el Seminario XI, desarrolla en detalle los modos por medio de los cuales el ser humano es llamado a la \emph{subjetividad.} Si bien se parte de considerar el lugar del Otro como fundante del sujeto, esta relación, dice Lacan \enquote{se engendra toda en un proceso de hiancia}. \footcite[][214]{@7106-LACAN2006} de la cual el sujeto resulta en tanto dividido, o en tanto \emph{fading (desvaneciéndose)}, es decir que emerge en un movimiento que implica su misma desaparición. En palabras de Lacan: \enquote{el sujeto se manifiesta en ese movimiento de desaparición que califiqué de letal}. \footcite[][216]{@7106-LACAN2006} En tanto su emergencia se produce por el reconocimiento de una falta, de la falta en el Otro a la cual responde con su propia huida. En este punto Lacan introduce dos operaciones constitutivas de la subjetividad: la \emph{alienación} y la \emph{separación}.

En la primera operación se trata de un proceso de \emph{reunión}, de un \emph{vel}, dice Lacan, en la cual el sujeto se presenta, por un lado, como sentido producido, como efecto de un significante que lo nomina pero que, por otro lado, implica su división. Se trataría de una operación que implica la desaparición de una porción-del-ser a cambio del sentido, sentido por el cual el sujeto se instituye como tal. En sus propios términos, Lacan define a la \emph{alienación} como \enquote{ese \emph{vel} que condena (\ldots) al sujeto a solo aparecer en esa división que he articulado lo suficiente, según creo, al decir que \emph{si aparece de un lado como sentido producido por el significante, del otro aparece como afanisis [desaparecido]}}. \footcite[][218]{@7106-LACAN2006} Ahora bien, ese proceso de alienación, de identificación e inscripción subjetiva implica un segundo momento (lógico), una segunda operación que Lacan denomina \emph{separación}. Se trata del momento en el que el sujeto encuentra en el Otro una falta en los intervalos de su discurso, es decir, el momento en el cual se escabulle y desliza metonímicamente el \emph{deseo del Otro.} De esta manera, \enquote{el sujeto aprehende el deseo del Otro en lo que no encaja, en las fallas del discurso del Otro}. \footcite[][222]{@7106-LACAN2006} Es en este punto de \emph{intersección} de dos faltas, la falta-en-ser del sujeto y la falta que el sujeto encuentra en el Otro, que se produce la separación, la dimensión de la pérdida, en tanto el sujeto responde a esta falta (de goce, de completud) percibida en el Otro con su propia desaparición. Si el sujeto emerge producto de una operación de alienación en tanto proceso de identificación a un significante que lo nombra y lo instituye, al mismo tiempo este pone en evidencia que toda operación de representación siempre falla, siempre hay algo del sujeto que se escapa a su simbolización. Esta dimensión innombrable, que hace agujero en lo simbólico, es la que Lacan distingue como lo \emph{real.} En relación con esto, la falta, la hiancia ontológica, emerge del intento (siempre fallido) de simbolización de lo real, en la intersección entre lo simbólico y lo real. En palabras de Lacan \enquote{Si escogemos el ser, el sujeto desaparece, se nos escapa, cae en el sin-sentido; si escogemos el sentido, este solo subsiste cercenado de esa porción del sin-sentido que, hablando estrictamente, constituye, en la realización del sujeto, el inconsciente}. \footcite[][219]{@7106-LACAN2006}

De los procesos de alienación y separación destacados se produce un resto-producto, un residuo que se desprende de los límites de dicha operación simbólica, que marca el \emph{límite} y a la vez el \emph{exceso} de todo proceso de significación. Decanta así un objeto que pone de manifiesto la imposibilidad de significación total, los límites del saber establecido, a la vez que constituye aquel \emph{objeto-causa de deseo} que opera como motor de todo intento de significar lo que no puede ser significado, de todo nuevo intento de construcción socio-simbólica (que apunte a su significación). Se trata de aquel resto que se desprende a modo de heterogeneidad y que no puede ser reabsorbido adquiriendo el estatuto de un objeto-causa del deseo: el \emph{objeto a}.\footcite[]{@7106-LACAN2006} El objeto \emph{a} se manifiesta como la presencia de una ausencia fundante y constituye así la huella que positiviza la negatividad, la falta ontológica. Es el encuentro con la falta lo que inscribe la dimensión del deseo como la búsqueda incesante de lograr la completud imposible. El objeto \emph{a} al ser concebido como la positividad de un fundamento negativo cobra gran riqueza conceptual ya que permite reconocer y analizar qué modos encuentra el sujeto de relacionarse, siempre parcialmente y de manera singular, con ese resto. El objeto-causa del deseo posibilita la búsqueda incesante de producciones socio-simbólicas que colmen imaginariamente el vacío constitutivo. Es en el punto en el que el \emph{goce}, la \emph{jouissance}, se liga a lo imposible que surge de la dimensión del deseo. El objeto \emph{a} como objeto \emph{míticamente} perdido se presenta como aquello que soporta la promesa de una completud, aquel elemento que permitiría lograr la unidad en el sujeto ocultando la falta en el Otro. En este sentido el objeto \emph{a}, como resto del proceso de constitución subjetiva, funciona como una metáfora del sujeto en tanto completud, sujeto mítico previo a la perdida de la jouissance. De este modo se crea así la ilusión de una posible identificación con el objeto \emph{a} que permitiría al sujeto completo constituirse. Sin embargo este objeto solo puede presentarse como falta, como objeto-causa de un deseo siempre por venir siempre postergado, insatisfecho, en definitiva, imposible. Siguiendo a Stavrakakis se podría sostener que el objeto cumple una función simbólica al constituirse como \emph{soporte de la falta en lo simbólico mediante una promesa de un dominio imaginario de lo real}.\footcite[]{@7107-STAVRAKAKIS2007}

En función de lo anterior se abren múltiples interrogantes en torno a cómo se juegan los límites y la imposibilidad del orden simbólico y del sentido en toda estructuración psíquica, y, por lo tanto, en toda experiencia subjetiva; es decir, aquel \emph{más allá} de lo simbólico, del \emph{principio del placer}, que persiste y retorna en cada formación significante, manifestado en la \emph{compulsión a la repetición}.\footnote{Freud en \emph{Más allá del principio del placer} advierte sobre la relación existente entre la compulsión a la \emph{repetición} y el \emph{principio del placer} como no contradictorias. La repetición, como aquella exteriorización forzosa de lo reprimido, puede generar displacer en el yo, sin embargo, esto no contradice el principio de placer ya que puede ser \enquote{displacer para un sistema y, al mismo tiempo, satisfacción para el otro}. \cite[][20]{@7104-FREUD1984}.} El sujeto emerge como efecto de (des)estructura que conlleva, en el mismo movimiento, una pérdida.


En este punto es necesario destacar que, desde el psicoanálisis lacaniano, todo proceso de subjetivación se inscribe a partir del anudamiento de tres registros heterogéneos co-implicados: lo simbólico, lo imaginario y lo real. El orden \emph{simbólico} indica la presencia de una ley, el campo del lenguaje; lo \emph{imaginario} como el campo de lo especular, de las identificaciones y la significación. Pero como destacamos anteriormente a la vez implica la dimensión de imposibilidad, en donde el proceso de simbolización resiste, y que ha sido definido como \emph{lo real}. Estos tres registros son aspectos constitutivos y fundamentales para pensar al sujeto y su imbricación en los procesos socio-políticos, ya que la especificidad que asuma el anudamiento, siempre precario, de estos registros nos habla del modo singular por medio del cual cada sujeto construye una respuesta a la-falta-en-ser.

Si bien Lacan retoma los postulados del estructuralismo, para conceptualizar al sujeto y sus modos de constitución, su pensamiento va más allá al plantear la imposibilidad de constituir toda estructura como totalidad cerrada, clausurada en sí misma. La primacía de la falta en su pensamiento ha llevado a resaltar los límites de la significación, del orden simbólico, dando al concepto de hiancia una función ontológica. Es decir, desde este campo de pensamiento, la falta adquiere el estatuto de causa, en tanto fundamento ausente. El sujeto se concibe como descentrado, cuya identidad como totalidad centrada en sí misma y clausurada es, en última instancia, imposible; el corolario de esta perspectiva es que los procesos de constitución de identidades se consideran siempre \emph{contingentes, parciales} y \emph{precarias}, producto del juego infinito de relaciones diferenciales.


\section{Una ontología (política) de lo real}

Los desarrollos que aporta el discurso psicoanalítico en torno a conceptos como el de \emph{sujeto}, \emph{inconsciente}, \emph{goce}, \emph{fantasma}, entre otros, han aportado a avanzar y dar mayor inteligibilidad a la comprensión de procesos subjetivos, sociales y políticos por medio de la articulación con otros discursos afines. La composibilidad de campos de pensamiento heterogéneos a partir de determinadas categorías nodales posibilitó la producción conjunta de múltiples herramientas teórico-prácticas para realizar lecturas e intervenciones en procesos socio-políticos determinados teniendo en cuenta su especificidad. Al mismo tiempo, la consideración de la \emph{dimensión pulsional y afectiva} que se inscriben en estos procesos nos permite considerar la centralidad y especificidad que adquiere la subjetivación para el pensamiento en torno a lo político. Se destaca, de esta manera, la importancia de atender a los modos singulares y específicos que asumen los procesos de subjetivación \rdm{política} según las particularidades de su localización. Se ha destacado, para el análisis socio-político, la concepción lacaniana del sujeto como una de las mayores contribuciones del discurso psicoanalítico al pensamiento contemporáneo y al análisis político. Dicho campo de reflexiones produce no solo un cuestionamiento de la tradición filosófica humanista, sino que se posibilita un salto en el pensamiento político contemporáneo al introducir un nuevo modo de entender la subjetividad en el pos-estructuralismo y en la teoría marxista. A partir de esto ya no se piensa al sujeto como determinado en su esencia por la relación de clases ni como una unidad determinada por un juego infinito de posiciones interrelacionadas. Se introduce así un sujeto que, en tanto dividido y efecto del significante, se convierte en el \emph{locus} de diversos procesos políticos de identificación simbólico-imaginarias, siempre parciales. Sujeto cuya característica central es la falta, la ausencia de esencia que abre la lógica del deseo y de la permanente búsqueda en el campo socio-simbólico de objetos sociales y políticos de identificación. La falta constitutiva y el intento siempre fallido del sujeto de lograr la representación evidencian los puntos de encuentro entre el sujeto y lo social, entre la teoría lacaniana y el análisis político.\footcite[]{@7107-STAVRAKAKIS2007}

En este mismo sentido, otro de los conceptos elaborado por Lacan, y que resultó central para el pensamiento en torno a lo político, lo constituye la categoría de \emph{Lo Real} que se ha definido como aquel encuentro siempre fallido entre el sujeto y la realidad, la imposibilidad estructural de que la misma pueda ser asimilada, simbolizada plenamente. Lacan designa a este mal-encuentro como la \emph{tychè}, es decir, como aquella cita con lo real que siempre se escabulle. Para el psicoanálisis este encuentro fallido con lo real se inaugura en la experiencia subjetiva por medio del \emph{trauma}, de aquello que retorna bajo el aspecto de lo indescifrable e inasimilable. Al respecto Lacan afirma: \enquote{Nuestra experiencia nos plantea entonces un problema, y es que, en el seno mismo de los procesos primarios, se conserva la insistencia del trauma en no dejarse olvidar por nosotros. El trauma reaparece en ellos, en efecto, y muchas veces a cara descubierta}. \footcite[][63]{@7106-LACAN2006}

En este sentido, lo real debe ser identificado como aquella brecha inalcanzable que no puede ser simbolizada pero que sin embargo re-aparece en aquel más allá de la repetición. Es aquello que se presentifica a modo de \emph{ruido}, de \emph{extrañamiento} y accidente, como aquello \emph{de otro orden} que se evidencia en los estados oníricos, los síntomas, los tropiezos del discurso, entre otros. Se trata de ir a buscar \emph{más allá} de la representación, de aquello que la representación vela y recubre, haciendo las veces de lugarteniente de aquel vacío subyacente. Se trata de aquello que a modo de fuerza, gobierna nuestras actividades y hacia lo cual se orienta el trabajo analítico. en relación con esto Lacan afirma que: \enquote{Lo real está más allá del \emph{automaton}, del retorno, del regreso, de la insistencia de los signos, a que nos somete el principio de placer. Lo real es eso que yace siempre tras el \emph{automaton}, y toda la investigación de Freud evidencia que su preocupación es ésa}. \footcite[][62]{@7106-LACAN2006}

Lo real constituye aquel resto constitutivo de todo orden de discurso, aquello imposible de ser reabsorbido por los medios simbólico-imaginario de toda construcción socio-discursiva. Ahora bien: ¿qué relevancia incorpora este concepto psicoanalítico para el pensamiento político contemporáneo?

La ontología de la falta, de lo real, en la que se sostiene el pensamiento psicoanalítico conlleva implicancias en dos sentidos: por un lado, a nivel epistémico estos supuestos puntúan el hecho de que la teoría, como construcción de un saber, de un discurso, nunca puede absorber totalmente la experiencia, es decir, está en sí misma atravesada por una imposibilidad; por otro lado, esto conlleva implicancias ético-políticas en tanto requiere un trabajo que reconozca estos límites, y apunte a asumir un posicionamiento, una responsabilidad que tenga en cuenta la posibilidad del fracaso, del encuentro siempre fallido que implica toda construcción socio-simbólica, y poder dar lugar así a la emergencia de una verdad que forma parte de la experiencia pero que la excede, no posible de ser capturada en su totalidad por el saber establecido. Como ha sido destacado por Stavrakakis: \enquote{Aunque nunca podamos simbolizar plenamente \emph{lo real} de la experiencia en sí, es posible delinear (incluso de forma metafórica) los límites que impone a la significación y la representación, los límites que impone a nuestras teorías}. \footcite[][31]{@7003-STAVRAKAKIS2010}

A partir de tener en cuenta estos aspectos es que se apuesta a la \emph{composibilidad}\footnote{Este término es un neologismo propuesto por Badiou que resulta de la articulación de las palabras \emph{componer} y \emph{posibilidad}, para expresar la aptitud por medio de la cual la filosofía hace posible pensar la composición conjunta de discursos heterogéneos. Ver más en: \cite[]{@7072-BADIOU2002}.} y articulación entre múltiples discursos que permitan potenciar conceptos y categorías para lograr dar cuenta de la singularidad de los procesos subjetivos en cada situación. La comprensión y visibilización de determinados fenómenos, tanto subjetivos como socio-políticos, se posibilita, no por la construcción de un lenguaje totalizador, sino por la imbricación de discursos heterogéneos. Al tiempo, este anudamiento discursivo puede, mediante una tarea ético-política, posibilitar la construcción y hegemonización de nuevos proyectos políticos más emancipatorios. Se trata de pensar lo real como imposibilidad inherente a lo discursivo como tal pero que es necesario cifrar, escribir, bordear de un modo que es siempre singular y local.\cite[][]{@7077-FARRAN2009}

La incorporación de estos desarrollos por el campo de la teoría política contemporánea tuvo como correlato evidenciar el carácter radicalmente \emph{contingente} de todo orden discursivo y por tanto social. Al mismo tiempo, para el campo socio-político adquiere sentido en tanto es por medio de la lógica misma de la representación a través de la cual el sujeto solo llega a ser, en tanto acceda a ser representado por un significante. A su vez este real, si bien se concibe como manifestándose por fuera de la significación, su ausencia misma dentro del proceso funciona como \emph{causa}, es decir, la ausencia de significado es lo que causa el movimiento metonímico en la cadena significante en la promesa de alcanzar el significado perdido, imposible. The ilusión de llenar el vacío en el campo del lenguaje, de completar la falta, constituye la función activa del significante en la que, sin embargo, el significado siempre se desliza \emph{más allá}, porque la significación nunca es completa. Es mediante esta lógica que se ponen en marcha procesos de construcción socio-simbólica que posibilitan la producción permanente de proyectos políticos y objetos sociales de identificación.\footnote{\cite[]{@7107-STAVRAKAKIS2007}.} Al mismo tiempo, la imposibilidad de lograr la completud de la identidad pone en juego un proceso profundamente político: una serie de identificaciones, siempre fallidas pero constitutivamente necesarias. De esta manera, se evidencia la profunda relevancia que adquiere los procesos de identificación, tanto imaginaria como simbólica, para el pensamiento y el análisis social y político como así también para pensar y reflexionar acerca de los procesos de constitución de ambos órdenes.

Desde \emph{la izquierda lacaniana}, diversos han sido los trabajos que resaltan el aporte de la teoría psicoanalítica para la teoría y el análisis político. En esta línea de pensamiento encontramos los desarrollos de Stavrakakis, Alemán, Laclau, Badiou, Žižek, entre otros filósofos contemporáneos. Para estos autores, uno de los principales aportes del discurso psicoanalítico al campo de lo político tiene que ver con posibilitar una praxis que permita inscribir un nuevo modo de pensar lo político en tanto constitutivo y constituyente de la experiencia subjetiva. Experiencia dentro de la cual un sujeto se implica y asume un posicionamiento frente a lo que se presenta como el \emph{malestar en la cultura}, ligado a la imposibilidad de aprehender lo real, promoviendo la crítica a ciertos órdenes hegemónicos, la construcción de nuevos modos de \emph{ser con los otros}, entre otros posibles.


Retomando las lecturas de Alemán,\footnote{\cite[]{@7108-ALEMAN2010}.} se puede pensar que el concepto de \emph{real} conlleva considerar la realidad social como atravesada por una brecha, un hiato, imposible de capturar que se manifiesta y emerge en pequeñas fisuras, dislocaciones, síntomas, acontecimientos imprevistos, que hacen agujero en el saber. Lo real implica la imposibilidad de que la operación de representación sea total, ya que no existe un representante que logre significar de modo absoluto la experiencia. La realidad es concebida, a partir de estos desarrollos, como una configuración discursiva y por cuanto atravesada por una imposibilidad, lo que Laclau denominó la \emph{imposibilidad de la sociedad}, inaugurando un nuevo pensamiento sobre la ontología social a partir de la noción de Real lacaniano. Este pensamiento implica problematizar la dicotomía individuo/sociedad, para pensar la sociedad como una construcción socio-discursiva, como aquel gran Otro lacaniano en el que el sujeto se constituye como tal.\footnote{Lacan define al Otro como \enquote{el lugar donde se sitúa la cadena del significante que rige todo lo que, del sujeto, podrá hacerse presente, es el campo de ese ser viviente donde el sujeto tiene que aparecer} en \cite[][212]{@7106-LACAN2006}.} Es el espacio simbólico donde el sujeto como falta busca su completud (imposible).\footcite[]{@7003-STAVRAKAKIS2010} Parafraseando a Lacan esto consiste en el momento lógico en que el sujeto de la falta se encuentra con la falta en el Otro, donde el Otro ya no es centro de garantías, ya no puede dar al sujeto respuesta a la pregunta por el ser, abriendo así la lógica puramente política \rdm{inconsciente} de construir una respuesta parcial ante la falta. A partir de esto, la fantasía consiste en la construcción de un modo de llenar, taponar la falta en el Otro, constituye una construcción imaginaria en respuesta a la brecha entre lo simbólico y lo real, dando consistencia a nuestra realidad en un intento de suturar la distancia entre lo real y la realidad.

Lo anterior permite abrir diferentes interrogantes que movilizan a cuestionar e indagar en torno a los diversos modos (fantasmáticos) que una determinada sociedad o comunidad, según las particularidades de su contexto, construyen para tratar con este \emph{real} imposible. Es decir, no hay posibilidad de construir un discurso totalizador, que se clausure a sí mismo y que anule la manifestación de este resto heterogéneo que se desprende de toda operación de simbolización.

Desde una perspectiva donde la causa se considera un fundamento ausente, la \emph{falta} juega un papel estructural en tanto no es algo que pueda ser eliminado con el paso del tiempo o que, desde una dimensión epistémica, determinado conocimiento sobre la experiencia y la realidad puedan subsanar; sino que cumple una función ontológica, como posibilidad/imposibilidad de toda estructuración socio-simbólica, es decir, al tiempo que imposibilita su clausura constituye el fundamento de los modos parciales de institución del orden social. En este sentido la negatividad fundante del orden marca el límite de toda construcción simbólico-discursiva y de las relaciones imaginarias en las que se sostienen los procesos de subjetivación. Estos procesos simbólico-imaginarios conllevan, a su vez, un \emph{más allá} en el que se pone en juego una vertiente de goce pulsional inherente que resulta indispensable tener en cuenta para su lectura; es aquí donde el pensamiento del psicoanálisis resulta central para el pensamiento político contemporáneo. Stavrakakis, en relación con esto, afirma: \enquote{(\ldots) la teoría lacaniana, además de sus importantes contribuciones epistemológicas (\ldots) proporciona una serie de herramientas invaluables para el análisis de la realidad política y social (\ldots) también introduce un nuevo modo de teorizar el momento de \emph{lo político \emph{(\ldots)} como un encuentro con lo real}}. \footcite[][37]{@7003-STAVRAKAKIS2010}

Si toda experiencia social y subjetiva se encuentra atravesada por una brecha inconmensurable, por una negatividad fundante, la misma solo puede ser captada a medias, en un encuentro siempre fallido, a partir de los modos sintomáticos en los que adquiere materialidad. A su vez, el vacío, la falta, solo puede ser trabajada y rodeada en los límites del sentido, pero no fuera de él, es decir, solo en los límites de la significación algo de lo real puede circunscribirse, aunque solo parcialmente.

Sin embargo esta negatividad adquiere una dimensión de posibilidad, en tanto de lo que se trata en el orden social es de la construcción de nuevos sentidos que signifiquen la realidad y que instituyen un orden, siempre precario y contingente. En este punto ingresa la lógica política, en tanto comprende los distintos modos, siempre en conflicto, que los sujetos construyen para instituir un \emph{orden}, para hegemonizar sentidos y prácticas que instituyan a la sociedad como una totalidad homogénea (siempre imposible). Para Alemán,\footcite[][]{@7108-ALEMAN2010} la dignidad humana entonces consiste en poder encontrarse con esa \emph{nada} que nos interroga permanentemente sobre nuestro \emph{ser} y sobre la imposibilidad que se juega en todo lazo social. Es a partir de esto, que nuestras decisiones resultan apuestas sin garantías e implican operar una ruptura con lo dado y realizar la experiencia (política) de nuevos múltiples posibles. Toda construcción discursiva e ideológica, como así también los procesos de identificación que los mismos promueven, constituyen modos siempre parciales de abordar el vacío, modos a través de los cuales se desliza el objeto-causa del deseo, intentando articularse a una configuración significante; \enquote{se trata de un \emph{saber hacer}\footnote{Este aspecto será profundizado más adelante.} con lo parcial, saber construir con algo parcial, por ejemplo una hegemonía política}. \footcite[][49]{@7108-ALEMAN2010}

En función de estas conceptualizaciones vemos cómo se inscriben las reflexiones en torno a lo político en el pensamiento contemporáneo, es decir, pensar la dimensión política como aquella lógica que emerge de esta imposibilidad, de la lógica del no-todo del inconsciente. Se puede decir, siguiendo a Alemán, que \enquote{el inconsciente es una experiencia política. Es una experiencia política porque (\ldots) no hay ningún significante que agote la representación del sujeto}. \footcite[][171]{@7109-ALEMAN2006} En este sentido podemos decir, que la lógica política, en la medida en que implica y tiene en cuenta la dislocación, la dimensión de la falta, funciona en convergencia, y de manera homologable, a la lógica del inconsciente. En términos de Laclau, la dislocación es el punto en el que \enquote{la lógica del inconsciente, como lógica del significante, se muestra como una lógica esencialmente política (\ldots) y en que lo social, irreductible en última instancia al status de una presencia plena, se revela también como político. Lo político adquiere así el status de una ontología de lo social}. \footcite[][110]{@6999-LACLAU1990}



\section{El fantasma ideológico}

Lacan desarrolló el concepto de \emph{fantasma} haciendo referencia al modo que tiene el sujeto, mediante la función simbólica e imaginaria, de relacionarse con lo real, con la falta-en-ser. Funciona como un compuesto entre lo simbólico y lo real, por medio de la relación imaginaria que el sujeto mantiene con los objetos del mundo. El fantasma es la relación del sujeto del significante, el sujeto barrado \$, de la falta en ser, y el llamado a un complemento del ser. En términos de Miller, el fantasma \enquote{reúne un término propiamente simbólico, \$, y otro imaginario, \emph{a}, (\ldots) que compensa, repara, la pérdida indicada por la barra del \$}. \footcite[][286]{@7110-MILLER1998}

En un intento de colmar la falta en el Otro el sujeto recurre a una construcción simbólico-discursiva que vela la falta fundamental, surge en el encuentro con la falta en el Otro. Fantasía que causa el deseo en tanto sostiene una promesa imaginaria de eliminar la falta y posibilitar el encuentro con la jouissance imposible mediante la creación de un semblante de armonía y completud. Al tiempo que apunta a ocultar la imposibilidad misma de cubrir esta falta en el Otro, instituye esta incompletud no como imposible sino como prohibida y momentánea, es decir, como lo que en algún momento futuro permitiría reencontrar la jouissance perdida. Constituye, de esta manera, un modo de domesticar cierto goce, de bordear mediante lo simbólico algo de lo real.\footcite[]{@7107-STAVRAKAKIS2007}

En continuidad, Žižek reflexiona estos conceptos para pensar los procesos sociales e ideológicos, definiendo a la \emph{fantasía} como: \enquote{(\ldots) un escenario imaginario cuya función es proveer una suerte de apoyo positivo que llene el vacío constitutivo del sujeto. Y lo mismo es válido, (\ldots) para la \emph{fantasía social}: ella es la contrapartida necesaria del concepto de antagonismo, un escenario que llena los vacíos de la estructura social, ocultando su antagonismo constitutivo con la plenitud del goce}. \footcite[][262]{@7111-ZIZEK2003}

Žižek introduce el concepto de \emph{fantasía ideológica} que puede ser definida como una ilusión inconsciente que estructura la realidad, es decir, \enquote{el nivel fundamental de la ideología (\ldots) no es el de una ilusión que enmascare el estado real de las cosas, sino el de una fantasía (inconsciente) que estructura nuestra propia realidad social}. \footcite[][46-47]{@7111-ZIZEK2003} La ilusión no está del lado del saber, como se pensaba anteriormente, sino de la realidad misma, sostenida en esta ilusión. En este sentido, la ideología ya no se considera una representación ilusoria de la realidad que el sujeto debe erradicar y abandonar, sino que la realidad se concibe como implicando, en su consistencia ontológica, un cierto no-conocimiento, es decir, es la realidad misma la que se ha de concebir como ideológica; \enquote{\enquote{ideológica} es una realidad social cuya existencia implica el no conocimiento}. \footcite[][118]{@7111-ZIZEK2003}

Lo discursivo-ideológico, desde estos desarrollos, se concibe como aquellos sentidos que interpelan al ser a constituirse, dándole \emph{razones-de-sujeto} para asumir las funciones definidas por la estructura. El proceso de subjetivación, de constitución subjetiva, si bien, por un lado, implica una identificación significante en el cual el sujeto queda \enquote{prendido}, \enquote{cosido} al significante, por otro lado involucra la dimensión de la \emph{decisión}, en el cual el sujeto se encuentra forzado a una elección: la \emph{libertad} o la \emph{vida}. Se trata, como se desarrolló más arriba, de un proceso de alienación por el cual el hombre se encamina hacia una vida cercenada, en donde \enquote{si elige la libertad, ¡pum! pierde ambas inmediatamente, si elige la vida, tiene una vida amputada de libertad}. \footcite[][220]{@7106-LACAN2006}

El concepto de fantasma puede ser pensado y articulado al campo de lo socio-político cuando pensamos en la noción de \emph{ideología}. La \emph{fantasía ideológica} puede reconocerse como aquello que disimula el vacío, la causa ausente en torno a la cual se estructura la realidad social. Es decir, la sociedad está en sí misma, como el sujeto, atravesada por una escisión antagónica y la función del discurso ideológico es construir una imagen de la sociedad como no dividida y en la cual la relación entre sus partes sea percibida como orgánica y complementaria, es decir, un todo armónico y homogéneo. La fantasía ideológica resulta una categoría de gran riqueza para el análisis político en tanto posibilita que toda construcción discursivo-ideológica se considere como una forma parcial y contingente de construir la realidad, de instituir un orden. Al mismo tiempo, permite reconocer cómo en la fantasía ideológica se inscribe y anuda cierto \emph{goce}, es decir, que más allá del sentido pero imbricado en él, la ideología implica y manipula un cierto goce que se estructura en la fantasía. La función de la fantasía sería la de una pantalla que encubre la inconsistencia, la falta en el Otro, en el orden simbólico. Según Žižek \enquote{constituye el marco a través del cual tenemos experiencia del mundo como congruente y significativo --el espacio \emph{a priori} dentro del cual tiene lugar los efectos particulares de la significación}. \footcite[][169]{@7111-ZIZEK2003} Es a partir de este \emph{a priori} que se instituyen diversos objetos, parciales, objetos \emph{petit a} como una especie de rellenado del vacío donde lo real como fundamento negativo se positiviza. Es decir, el goce se anuda a determinadas producciones socio-políticas y de este modo la pulsión, en tanto energía psíquica destacada por Freud no solo determina la economía psíquica del sujeto sino que también estructura y organiza el orden social y político de una comunidad.

Estos conceptos aportan a pensar por qué ciertos proyectos políticos persisten por sobre otros, los procesos de transformación y cambio en el orden social, los modos de identificación política que se producen en determinados contextos, la institución y hegemonía de ciertos discursos-ideológicos en ciertas coyunturas, etc. Estos conceptos desarrollados permiten pensar y reconocer, mediante el análisis político, los modos en que se abrocha cierto goce a una producción de sentido, es decir, cierta \emph{jouissance} a una fantasía ideológica y las implicancias que este anudamiento tiene para la subjetivación política.

Siguiendo los desarrollos de Stavrakakis,\footcite[]{@7003-STAVRAKAKIS2010} se destaca una afinidad teórica entre las positivizaciones de lo real a través del \emph{objeto petit a} y lo que Laclau trabaja cuando define a los \emph{significantes vacíos} as modos de positivizar los límites de la significación. Es decir, \enquote{cualquier término que se convierta en el significante de la falta en un contexto político (\ldots). La política es posible porque la imposibilidad constitutiva de la sociedad solo puede representarse mediante la producción de significantes vacíos}. \footcite[][44]{@7112-LACLAU1996} Estas nociones son complementarias, ya que intentan conceptualizar cómo se positiviza el límite de la significación, la falta ontológica de todo orden simbólico.

Estas positivizaciones de lo real, como modos de \emph{trabajar} políticamente lo imposible, implica a su vez el establecimiento de lazos sociales, de procesos de identificación; este es otro de los aportes de la teoría psicoanalítica al campo del pensamiento político, ya que el psicoanálisis, en cuanto praxis, se interroga sobre los \emph{modos de ser con los otros}, los modos de institución del lazo social. En tanto el lazo social se constituye a partir de un acuerdo o pacto con otros, algo de lo más propio del sujeto, de su singularidad, queda por fuera.\footnote{Esto se vincula con lo que ha sido trabajado por Freud en \emph{El Malestar en la Cultura}.} Es en este punto que el vínculo con el otro siempre implica lo político en sí mismo. Esto a su vez se pone de manifiesto en la definición que nos brinda Alemán,\footcite[]{@7108-ALEMAN2010} retomando a Lacan, cuando se refiere al lazo, al discurso, como un modo de trabajar políticamente con lo imposible. El arte, el amor, la política, en tanto promueven el lazo social y la construcción de nuevos sentidos serían modos de negociar aquello irreductible, aquel resto heterogéneo que no puede ser reabsorbido en el juego de lo simbólico. En palabras de Alemán \enquote{Ningún período se salva del malestar en la cultura. Ningún período puede verdaderamente integrar en sus relatos el resto heterogéneo que lo ha constituido como período}. \footcite[][61]{@7108-ALEMAN2010}

A partir de esto es que podemos pensar la política ya no como lo que se reduce a un modo de administración de lo público y la búsqueda del consenso, sino como un modo de trabajar con la falta, con la \emph{pulsión de muerte}, es decir, con aquella compulsión a repetir lo nunca representado, lo reprimido, y que no entra en contradicción con el principio de placer y la satisfacción. \emph{Más allá} del orden discursivo existe un núcleo \emph{extimo}, es decir una intimidad-externa que nunca puede ser integrada por el lenguaje pero que retorna articulada a una porción de sentido expresada en el síntoma. Lacan se refiere a esto afirmando que entre la falta, el significante enigmático como aquello imposible de representar, y el término al que viene a sustituirse en una cadena significante se produce un abrochamiento que fija (en un síntoma) aquella significación inaccesible para el sujeto consciente.

\section{Hacia una lectura sintomática de lo político}

La noción de \emph{síntoma} ha sido central en los desarrollos del psicoanálisis, y ha adquirido diversas caracterizaciones en diferentes momentos. Desde sus inicios Freud pensaba al síntoma como una construcción significante que vehiculiza un deseo reprimido, es definido como una satisfacción sustituta de la pulsión. Lacan define, en un primer momento de su enseñanza, al síntoma como una formación significante que porta un mensaje cifrado, codificado dirigido a un Otro; tiene un valor de revelación y va en dirección al reconocimiento del deseo del sujeto.

El síntoma es el producto que emerge de la fisura, de la falta ontológica que subyace a toda estructura simbólica. Funciona como el punto de desequilibrio que desmiente todo intento de lograr una totalización, ya que todo proceso de clausura es siempre parcial en tanto implica un exceso, un resto que se desprende como producto. En este sentido, el síntoma social puede ser pensado como aquel elemento particular que subvierte el orden universal que es su fundamento mismo. Esta idea requiere posicionarse, como advierte Žižek, desde la \enquote{lógica de la excepción: cada Universal ideológico [es \enquote{falso} ] en la medida en que incluye necesariamente un caso específico que rompe su unidad, deja al descubierto su falsedad}. \footcite[][47]{@7111-ZIZEK2003} De este modo, un síntoma es el elemento disruptivo del orden establecido portando el sentido de un malestar, es un efecto de lo simbólico, un modo de tratamiento de lo real por lo simbólico.

En los últimos años de su enseñanza, Lacan opera un desplazamiento de la dimensión discursiva hacia lo real de la \emph{jouissance.} Propone así una nueva conceptualización del síntoma articulándolo a la noción de fantasma, ya que ambas nociones implican la presencia de cierto goce en el sujeto. Introduce la noción de \emph{Sinthome} para designar el aspecto innombrable que se anuda a toda formación simbólica. El sínthoma se define a partir de esto como una formación portadora de \emph{jouissance}, de goce en sentido. El síntoma es el modo en que el sujeto goza de su inconsciente, constituye así un \emph{sentido gozado} destacando, no solo la vertiente simbólica, sino que evidencia los límites del sentido y los modos en que esta imposibilidad, la falta ontológica, adquiere manifestación. Žižek se refiere a esto afirmando: \enquote{síntoma es el modo en que nosotros \rdm{los sujetos}\enquote{evitamos la locura}, el modo en que \enquote{escogemos algo (la formación de síntoma) en vez de nada (autismo simbólico radical, la destrucción del universo simbólico)} por medio de vincular nuestro goce a una determinada formación significante, simbólica, que asegura un mínimo de congruencia a nuestro ser-en-el-mundo}. \footcite[][110-111]{@7111-ZIZEK2003}

El sinthoma sería así un nudo entre el síntoma y la fantasía, en tanto esta última se define como una construcción inerte que implica un Otro tachado, un no-todo, incongruente, al tiempo que sutura a partir de la construcción de una ficción el vacío. El sinthoma sería aquel artificio que opera como elemento articulador de los tres registros que atraviesan la experiencia subjetiva y social (lo simbólico, lo imaginario y lo real), a modo de un cuarto elemento que permite anudar estas dimensiones. El éxito de una construcción significante, de esta manera, no puede ser atribuido completamente a sus posibilidades de efectuar una clausura discursiva, sino que depende de su eficacia para manipular cierto goce sintomático, es decir, de su capacidad para funcionar como sinthoma, en tanto invoca, organiza y regula cierta jouissance. Se destaca de esta manera, el estatus ontológico que subyace a toda formación sintomática, como la única sustancia y soporte positivo del ser, el punto que da congruencia al sujeto.

La satisfacción pulsional, la jouissance que se inscribe en el síntoma, permite entender por qué estos persisten a pesar de que expresan y manifiestan un malestar. En palabras de Stavrakakis: \enquote{La razón por la cual un síntoma (social) persiste, la razón por la cual nos resulta imposible librarnos de un síntoma que experimentamos conscientemente como algo doloroso y perturbador, estriba en que en otro nivel obtenemos de él cierto beneficio (primario o secundario), cierto goce (como algo opuesto al placer consciente)}. \footcite[][100]{@7003-STAVRAKAKIS2010}

Estas conceptualizaciones nos permiten pensar los síntomas, no ya como anomalías o desviaciones, sino como posibles modos singulares de lidiar con la falta constitutiva mediante la institución del lazo social permitiendo la regulación del goce pulsional. En función de esto, todo síntoma, a nivel subjetivo y social, instituye comunidades que se organizan según un modo particular de gozar definiendo formas y estilos de vida singulares. El sínthoma constituye un modo singular de hacer, que los sujetos producen, ante lo imposible, ante lo real, como aquel sin sentido que retorna a modo de agujero en lo simbólico, es el modo particular que tiene el ser hablante de \emph{saber hacer} con el goce. En este sentido, Lacan afirma que \enquote{uno solo es responsable en la medida de su saber hacer}, donde el saber hacer sería entonces el arte o artificio, \enquote{que le da al arte del que se es capaz un valor notable}. \footcite[][59]{@7113-LACAN2006} Es decir, desde esta perspectiva, el sínthoma constituye el modo por el cual, mediante lo simbólico, se inscribe y se bordea lo real posibilitando su tramitación a través de la invención de nuevos sentidos. Estos aportes permiten entender, no solo los procesos psíquicos y sus modos de estructuración, sino también los modos en que el goce configura y organiza el orden social y político.

\section{La subjetivación política}

Lo anterior tiene importantes implicancias en el análisis político, ya que la emergencia sintomática evidencia y presentifica la falta como fundamento ausente y constituye así un punto de ruptura, la \emph{tyché} que produce un corte con el \emph{automaton}, un corte en la insistencia de los signos en la que se sostiene toda fantasía ideológica. La dislocación producida inaugura el momento en el que el sujeto del inconsciente emerge y marca el momento de institución de \emph{lo político}, en tanto el mismo puede definirse como un encuentro, siempre fallido, con lo imposible, con lo real. En este espacio de indeterminación que se abre, el sujeto en un segundo momento lógico, realiza de modo retroactivo, un acto de invención que implica un forzamiento en el estado del saber, en el que una verdad singular emerge. Se producen significantes nuevos, supernumerarios, en los límites del sentido, del universal instituido. Es decir, se produce un forzamiento que permite la institución de un nombre para el exceso incontado, que no implica la clausura sino que se instituye como nominación del vacío, de lo real. Esta intervención en la que un significante nuevo surge es la marca del sujeto, de su decisión y elección, produciendo una reconfiguración del campo socio-discursivo en el que un nuevo modo de ser-con-los-otros se hace posible.

De este modo la experiencia política se inaugura a partir de un acontecimiento localizado en la singularidad de una situación, as aquella experiencia que surge en los bordes de lo real. Es aquello que ocurre en un sitio que es un múltiple singular (presentado pero no representado) al borde del vacío de una situación. Este acto subjetivo hace advenir elementos que no se encontraban contados en la situación más que como invisibles o in-existentes; aquello de lo múltiple indiscernible, que excede el sentido y el saber de una situación, es decir, lo \emph{genérico}.\footcite[]{@7072-BADIOU2002} Esto supone que no hay más fundamento de lo social que las decisiones que los agentes realizan, desde su localización específica, para la configuración de sus identidades y su realidad. Tal acto de intervención indica el carácter inmanentemente político de toda estructuración social, como así también de las \emph{subjetividades} que en la misma se instituyen. Es decir, asumir la función ontológica de la falta, posibilita el reconocimiento de la \emph{contingencia} en tanto toda representación, no solo en lo subjetivo sino en la institución de un orden social, es siempre parcial, lo que marca el carácter transitorio y temporal de toda construcción hegemónica, de todo estado de situación dada. El acontecimiento, como acto de quiebre del espacio social reglado, abre otro espacio-tiempo genérico a partir del cual se construyen nuevos modos de hacer y decir, nuevos modos de ser-con-los-otros que devienen en nuevas prácticas y acciones que luchan por hegemonizarse. Se trata de la invención singular que produce una nominación para lo incontado hasta el momento, que no implica la configuración de un saber absoluto, sino de una escritura para aquello que \emph{no cesa de no escribirse}.

A partir de estos movimientos de articulación y hegemonía, donde nuevos sentidos y prácticas advienen es que se instituyen nuevas identidades particulares. Estas construcciones que producen los sujetos, singulares y colectivos, operan como puntos de anclaje y referencia, tanto a nivel psíquico como social. La relativa estructuración que se lleva a cabo es producto de la institución de \emph{puntos de capitón}, puntos nodales, que abrochan cierto goce al sentido, adquiriendo así su eficacia en la construcción socio-política de la realidad. De esta manera, la inscripción de puntos nodales determina, retroactivamente, el ordenamiento de los significantes que circulan en el espacio social produciéndose su significación a partir de la articulación con un significante Uno, rígido, al que se abrocha cierta jouissance. Lo que se pone en juego, de este modo, es la lucha \rdm{política} por hegemonizar sentidos que logren definir y determinar la significación del resto de la cadena significante que estructura la \emph{fantasía ideológica.} Al respecto Žižek dice: \enquote{El espacio ideológico está hecho de elementos sin ligar, sin amarrar, \enquote{significantes flotantes}, cuya identidad está \enquote{abierta}, sobredeterminada por la articulación de los mismos en una cadena con otros elementos}. \footcite[][125]{@7111-ZIZEK2003} Este punto de inscripción, de cierto goce, es el punto de subjetivación de la cadena, el punto en el que, algo del sujeto se anuda al significante. En este sentido, estas identidades, parciales, se van re-configurando en el intento de hegemonizar un modo diferente de ser con los otros, de definir la realidad. Sentidos como \emph{Democracia}, \emph{Estado}, \emph{Pueblo}, \emph{Nación} adquieren significación retroactivamente a partir de la primacía que adquiere determinados significantes que articulan y organizan la cadena significante: \emph{Nacionalismo}, \emph{Comunismo}, \emph{Conservadurismo}, \emph{Liberalismo} pueden definirse como algunos sentidos reguladores de determinadas construcciones discursivo-ideológicas. La eficacia de cada uno de los discursos depende, como fue desarrollado anteriormente, de su capacidad para vehiculizar y regular cierto goce, de tramitar la satisfacción pulsional, inherente a la constitución subjetiva, que sin embargo puede estar orientado por el deseo o bien marcado por el predominio de la pulsión de muerte. Lo anterior nos permite pensar en torno a aquel exceso como aquel resto inasimilable que puede \enquote{estar jugando de manera mortífera o puede que cambie de estatuto y se lo trabaje políticamente}. \footcite[][50]{@7108-ALEMAN2010}

\section{Una ontología (ética) de la falta}

El aporte del psicoanálisis, en relación con esto, se encuentra ligado a la importancia de sostener una \emph{ética} de la falta, que consiste en asumir la incompletud de toda fantasía ideológica y de toda construcción socio-política. Es decir, el reconocimiento de lo \emph{real} en tanto límite y posibilidad de construcción de hegemonías políticas en tanto las mismas son siempre precarias y contingentes. Se trata, desde la ética que propone el psicoanálisis de ir más allá de la ética fantasmática de la armonía y la completud, y poder trabajar con lo real, sosteniendo la función ontológica de la falta. Constituye el gesto de atravesar la fantasía del orden social totalizado, sin fisura, y a partir de allí instituir la falta como causa ausente que a su vez presenta una potencialidad productiva. La posibilidad de invención que abre a \emph{lo político} ocurre al captar que la falta es una cuestión estructural productiva, es decir: \enquote{(\ldots) se trata de asumir la necesidad estructural de la falta en el Otro más allá de la contingencia de los significantes puestos en juego, para poder hacer algo, para movilizar una perspectiva que se sustente de manera ineludible en un goce propio y singular del cual solo cabe hacerse responsable}. \footcite[][8]{@7071-FARRAN2009}

Posibilitar de esta forma, una praxis política y un pensamiento sobre lo político a partir del cual es pensable una sociedad en la que \emph{lo común} no está dado a priori, sino que resulta de un momento contingente que se puede encontrar en el arte, el amor, y el orden específicamente político. Así, lo único verdaderamente Común sería la falta, el no fundamento (último) identitario, a partir del cual se pueda construir un proyecto político que asuma la distancia (imposible) entre el ser y la representación. Asumir esta distancia conlleva al mismo tiempo, en términos psicoanalíticos, una \emph{identificación al síntoma}, que significa reconocer un contenido particular detrás de toda universalidad abstracta. Es decir, el sujeto se identifica con aquel lugar desde el cual se denuncia como falsa la universalidad existente. Desde esta ética se propone identificar la universalidad con el punto mismo de exclusión constitutivo; \enquote{el punto de excepción/exclusión intrínseco, lo \enquote{abyecto} del orden positivo concreto, el único punto de verdadera universalidad}. \footcite[][244]{@7063-ZIZEK2005}

Así, el trabajo analítico en términos políticos consiste en poder indagar qué función cumplen en la estructura social determinadas formaciones discurso-ideológicas en tanto inscriben modos particulares de gozar, como así también modalidades específicas de anudamiento subjetivo, singulares y colectivos. A su vez, poder reconocer determinadas formaciones sintomáticas en tanto núcleos de goce, como factor explicativo de determinados fenómenos en ciertos contextos y construir a partir de allí prácticas específicas de intervención. El desafío de estos aportes consiste en poder reconocer en cada contexto socio-histórico los modos, siempre singulares, que cada comunidad instrumenta para trabajar \emph{lo real.} Al mismo tiempo se trata de poder identificar aquellos lugares y dispositivos que en determinados momentos posibilitan (o no) procesos de subjetivación política y en qué términos lo hacen. Es decir, si los espacios y dispositivos (existentes) realmente instituyen proyectos y posicionamientos políticos emancipatorios (que tiendan al reconocimiento de los límites y de la ética del deseo) o si por el contrario los mismos tienden a instaurar modalidades de reproducción de un orden de dominación, de un \emph{eterno retorno de lo mismo}\footcite[]{@7104-FREUD1984} que obstaculiza la invención de nuevos posibles.

El análisis deberá entonces poder aportar a dar visibilidad a aquellos espacios y lugares posibilitadores para la subjetivación (política) y a la apertura de interrogantes sobre los modos singulares de anudamiento que en los mismos se instituyen. Además, poder atender y reconocer las experiencias populares que resultan potenciales para la invención política, no perdiendo de vista que las mismas son proyectos atravesados por el antagonismo, por el no saber, lo que impide que existan garantías de éxitos. Proponemos entonces en función de este lugar ético y político interpelar al discurso psicoanalítico a identificarse con este lugar de causa-ausente, no solo al interior mismo de su práctica sino también, como lugar de permanente interrogación y cuestionamiento de los diversos discursos posibles en su localización específica. Esto significa insistir en el carácter contingente, como aquel elemento éxtimo y sintomático que denuncia lo imposible de toda relación ser/representación, evitando la institución de discursos totalitarios, clausurados y des-subjetivantes. Se trata de mostrar los \emph{impasses} de las distintas lógicas políticas y fantasmáticas en las que se anclan ciertas construcciones socio-políticas. Posibilitar de este modo la irrupción de lo igualitario y lo heterogéneo en la emergencia de nuevas praxis políticas emancipatorias. Esto requiere a su vez atender a los modos particulares que en cada contexto se construyen de \emph{ser-con-los-otros}, siempre diferentes y cambiantes.

\section*{Referencias}
\printbibliography[heading=none]   % Sin título automático



%%%%%%%%%%%%%%%%%%%%%%%
\ifPDF
\separata{capitulo6}
\fi

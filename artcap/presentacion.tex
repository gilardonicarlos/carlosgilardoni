\ifPDF
\chapter[\hspace{1.5pc}Presentación]{Presentación}
\chaptermark{Introducción}
\Author{Presentación}
\setcounter{PrimPag}{\theCurrentPage}
\fi

\ifBNPDF
\chapter[\hspace{1.5pc}Presentación]{Presentación}
\chaptermark{Presentación}
\Author{Presentación}
\fi

\ifHTMLEPUB
\chapter{Presentación}
\fi


Si muchas veces el trabajo intelectual parece darse en soledad, en el caso de este libro se trata de un pensar conjunto. A lo largo del tiempo hemos puesto en común nuestros trabajos individuales, los hemos sometido a discusión, hemos discrepado, compartido lecturas, y así, hemos tratado de escucharnos. Este texto surge de un ejercicio colectivo de pensamiento nucleado en torno al \gls{@glo234-petp} radicado en el \gls{@glo232-ciecs} de la \gls{@glo235-unc} y el \gls{@glo233-cne}.

Desde el pensar en común han surgido escritos singulares, diversos, plurales, que en su vacilación y fortaleza decidimos publicar para seguir dándole vueltas a preguntas, problemas, inquietudes, malestares y alegrías que nos convocan. La singularidad de cada texto no debe de dejar notar que los mismos están atravesados, de un lado, por una reflexión teórico-política y, de otro lado, por el trabajo desde autores que constituyen una constelación heredera de ciertas transformaciones surgidas del pensamiento francés de la década del `60. Autores como Althusser, Derrida, Foucault, Lacan, Rancière, Agamben, Nancy o Badiou han sido convocados. Cada texto, entonces, en la tensión entre singularidad y comunidad.

Publicamos para seguir la conversación infinita en la que estamos.

\ifPDF
\separata{presentacion}
\fi
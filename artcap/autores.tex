\ifPDF
\chapter[\hspace{1.5pc}Autores]{Autores}
\chaptermark{Autores}
\Author{Autores}
\setcounter{PrimPag}{\theCurrentPage}
\fi

\ifBNPDF
\chapter[\hspace{1.5pc}Autores]{Autores}
\chaptermark{Autores}
\Author{Autores}
\fi

\ifHTMLEPUB
\chapter{Autores}
\fi


\paragraph{Gala Aznarez Carini}


Licenciada en Psicología por la Universidad Nacional de Córdoba. Actualmente sus trabajos articulan el psicoanálisis lacaniano con el pensamiento político contemporáneo para pensar los diversos modos de subjetivación política según las especificidades de su localización. \href{mailto:gala\_az@hotmail.com}{\texttt{gala\_az@hotmail.com}}

\paragraph{Emmanuel Biset}


Licenciado en Filosofía y Licenciado en Ciencia Política por la Universidad Nacional de Río Cuarto. Doctor en Filosofía por la Universidad Nacional de Córdoba y por la Université Paris 8. Tesis doctoral: \enquote{Violencia, justicia y política. Una lectura de Jacques Derrida}. Investigador Asistente de CONICET sobre el problema de la justicia en el pensamiento político postfundacional. \href{mailto:bisetico@hotmail.com}{\texttt{bisetico@hotmail.com}}


\paragraph{Andrés Daín}


Licenciado en Ciencia Política. Profesor universitario. Doctorando en Ciencia Política. Becario de CONICET. Su investigación se titula \enquote{Análisis político-ideológico de las urbanizaciones privadas en Argentina}. \href{mailto:andresdain@gmail.com}{\texttt{andresdain@gmail.com}}


\paragraph{Roque Farrán}


Licenciado en Psicología por la Universidad Nacional de Córdoba. Doctorando en Filosofía por la Universidad Nacional de Córdoba. Becario Doctoral de CONICET. Su investigación se titula \enquote{El concepto de sujeto en Alain Badiou y Jacques Lacan. Dimensiones ontológicas y políticas}.  \href{mailto:roquefarran@gmail.com}{\texttt{roquefarran@gmail.com}}


\paragraph{Daniel Groisman}


Licenciado en Ciencia Política por la Universidad Católica de Córdoba. Maestrando en \enquote{Estudios Interdisciplinarios de la Subjetividad} por la Universidad de Buenos Aires. Doctorando en filosofía por la Universidad Nacional de Córdoba. Su investigación aborda la temática del sujeto y la verdad en y a través de la obra de Alain Badiou. Becario de la Secretaría de Ciencia y Tecnología de la UNC.  \href{mailto:danielgroisman@gmail.com}{\texttt{danielgroisman@gmail.com}}


\paragraph{Manuel Moyano}


Licenciado en Ciencia Política por la Universidad Católica de Córdoba. Investiga el problema de la experiencia en el pensamiento político de Giorgio Agamben.   \href{mailto:manumoyano@gmail.com}{\texttt{manumoyano@gmail.com}}


\paragraph{Juan Manuel Reynares}


Licenciado en Ciencia Política por la Universidad Nacional de Villa María. Actualmente es becario doctoral CONICET, con sede en el Centro de Estudios Avanzados de la Universidad Nacional de Córdoba. Está investigando la constitución de identidades políticas en la provincia de Córdoba desde el retorno a la democracia.
\href{mailto:juanmanuelreynares@hotmail.com}{\texttt{juanmanuelreynares@hotmail.com}}

\paragraph{María Aurora Romero}


Licenciada en Sociología por la Universidad Empresarial Siglo 21. Maestranda en Sociología en el Centro de Estudios Avanzados de la Universidad Nacional de Córdoba. Doctoranda en Ciencias Sociales en la Universidad de Buenos Aires. Becaria Doctoral de CONICET. Su investigación se titula \enquote{Entre el saber y el poder: el campo científico. Una perspectiva crítica al paradigma cientificista}. \href{mailto:maauroraromero@gmail.com}{\texttt{mmaauroraromero@gmail.com}}

\paragraph{Mercedes Vargas}


Licenciada en Psicología por la Universidad Nacional de Córdoba. Actualmente becaria doctoral FONCYT en el Centro de Investigaciones y Estudios en Cultura y Sociedad (CIECS-UE CONICET-UNC) en el marco de un proyecto PICT que se propone indagar la constitución de identidades políticas durante el primer peronismo desde una mirada \enquote{desde abajo}. Sus trabajos intentan incorporar el psicoanálisis lacaniano al análisis político para pensar los procesos de subjetivación política. \href{mailto:mer\_chan86@hotmail.com}{\texttt{mer\_chan86@hotmail.com}}

%%%%%%%%%%%%%%%%%%%%%%%
\ifPDF
\separata{Autores}
\fi

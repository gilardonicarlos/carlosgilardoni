\ifPDF
\chapter[Ontología del sujeto]{Ontología del sujeto}
\chaptermark{Ontología del sujeto}
\Author{Daniel Groisman}
\setcounter{PrimPag}{\theCurrentPage}
% encabezado para autor
\begin{center}
	\nombreautor{Daniel Groisman}\\
	\vspace{15mm}
\end{center}
\fi

\ifBNPDF
\chapter[Ontología del sujeto]{Ontología del sujeto}
\chaptermark{Ontología del sujeto}
\Author{Daniel Groisman}
% encabezado para autor
\begin{center}
	\nombreautor{Daniel Groisman}\\
	\vspace{15mm}
\end{center}
\fi

\ifHTMLEPUB
\chapter{Ontología del sujeto}
\fi

\section{Introducción}

Creo posible decir \emph{a priori} que cada una de las ontologías propuestas en este libro es la reducción de un campo de batalla, la consideración de un área geográfica parcial de un mapa más amplio: el posfundacionalismo\index[concepto]{Posfundacionalismo}. Este, a pesar de la diversidad que engloba, entraña una coincidencia: el deseo de reelaborar (sin comprometerse con un esencialismo\index[concepto]{Esencialismo!crítica al} y siendo conscientes de su parcialidad) las figuras filo-políticas\index[concepto]{Filosofía política!figuras} de la modernidad\index[concepto]{Modernidad!figuras filosóficas} que fueran abandonadas a partir de la crisis \emph{posmoderna}\index[concepto]{Posmodernidad!crisis} y sus antecedentes teóricos. Una de estas ha sido, sin lugar a dudas, la del sujeto\index[concepto]{Sujeto!teorías del}. Figura en la que encontramos, por ejemplo en Descartes\index[concepto]{Descartes, René!teoría del sujeto}, el intento de fundarla sobre la autoevidencia y la solidez de lo Mismo\index[concepto]{Mismo, lo (concepto)}.

La corriente del posfundacionalismo\index[concepto]{Posfundacionalismo} que aquí abordamos, a pesar de la deconstrucción\index[concepto]{Deconstrucción!del esencialismo} de los supuestos esencialistas, entiende que la reformulación de la categoría de sujeto\index[concepto]{Sujeto!reformulación posfundacionalista}, y no su abandono, es condición \emph{sine-qua-non} para un pensamiento político\index[concepto]{Pensamiento político!emancipatorio} de corte emancipatorio. Una de las razones es que el sujeto\index[concepto]{Sujeto!como ilegalidad necesaria}, digámoslo con Badiou\index[concepto]{Badiou, Alain!teoría del sujeto}, es una ilegalidad necesaria para que una verdad\index[concepto]{Verdad!en Badiou} advenga al mundo, y un mundo \emph{sin} verdades es un mundo muy parecido al que propone la legalidad contemporánea. Otra razón es que subjetividad\index[concepto]{Subjetividad!y política} y política\index[concepto]{Política!y subjetividad|see{subjetividad}} no van sino juntas, ya que todo enunciado sobre el sujeto\index[concepto]{Sujeto!enunciados políticos} articula representaciones explícitas o implícitas de los cuerpos en el espacio público\index[concepto]{Espacio público!representaciones del cuerpo} y la vida en común\index[concepto]{Vida en común!representaciones}, así como todo enunciado político\index[concepto]{Enunciado político!sujeto supuesto} conlleva un sujeto supuesto. Veamos un ejemplo: el sujeto paradigmático de las neurociencias\index[concepto]{Neurociencias!sujeto paradigmático}, que aparentemente pertenece al universo a-político de la medicina\index[concepto]{Medicina!sujeto neurocientífico}, es un sujeto que presupone que todo problema simbólico\index[concepto]{Simbólico, lo!problemas} tiene un correlato orgánico\index[concepto]{Correlato orgánico} que puede ser suturado si se produce y se encuentra un medicamento adecuado. ¿No es ese sujeto el caso más obvio de un aplanamiento de la dimensión simbólica\index[concepto]{Simbólico, lo!aplanamiento} y política\index[concepto]{Política!dimensión simbólica|seealso{Simbólico, lo}} de toda práctica, es decir de aquello que no encuentra correlación ni sutura? ¿Podría un sujeto así, un sujeto \emph{àla} farmacología\index[concepto]{Farmacología!sujeto}, poner en cuestión el problema eminentemente político de \emph{su} situación y de \emph{nuestra} situación como más-que-cuerpos\index[concepto]{Cuerpo!más-que-cuerpos}? Desde la lógica interna a las neurociencias\index[concepto]{Neurociencias!lógica interna} se deduce ciertamente que no porque se asume que la política no es lo que atraviesa a toda práctica con incidencia sobre lo común\index[concepto]{Común, lo!prácticas} sino simplemente una profesión que, análoga a esta concepción de la medicina\index[concepto]{Medicina!concepción política}, en lugar de administrar medicamentos administra los recursos del Estado\index[concepto]{Estado!administración de recursos}.

Las ontologías de sujeto\index[concepto]{Ontología!del sujeto} se diferencian entre sí, entre otras cosas, porque asumen de diversa manera los requerimientos de una época\index[concepto]{Epoca@Época!requerimientos}, o simplemente porque se ocupan de dispositivos y prácticas diferentes. Aunque lo que a su vez resulta importante es que estas también \emph{producen} performativa y retroactivamente los sujetos que postulan. Por ello decimos: no hay sujeto\index[concepto]{Sujeto!producción performativa} sino que hay producción y aparición de sujeto sobre el fondo de las (im)posibilidades que abre una época determinada\index[concepto]{Epoca@Época!(\textit{im})posibilidades}.

Nuestra tarea no pasa, allí, por descubrir la verdad del sujeto\index[concepto]{Sujeto!verdad como objeto} como objeto de estudio, la teoría más adecuada a la realidad última de éste, sino por \emph{decidir} argumentativamente qué \emph{ser} de sujeto está a la altura de nuestro tiempo\index[concepto]{Tiempo!adecuación del sujeto}. En esa decisión se manifiesta lo político de una ontología del sujeto\index[concepto]{Ontología!política del \textit{sujeto}}. En que su existencia en tanto tal es motivo de una disputa\index[concepto]{Disputa!ontológica} ya que toda verdadera disputa es un problema irresoluble sobre la existencia\index[concepto]{Existencia!problema irresoluble}. Como lo sostiene Badiou\index[concepto]{Badiou, Alain!teoría de la disputa},\footcite[][]{@7137-BADIOU2002} la polémica en general no se reduce a un conflicto de interpretaciones divergentes, como si se tratara de la manera de interpretar un objeto constituido, sino que implica el problema lógicamente previo: la existencia misma.\footnote{La frustración que cualquiera experimenta al intentar convencer a otro que políticamente está en la vereda opuesta responde precisamente a este motivo: no se trata de un problema óntico interpretativo (\emph{i.e}. más o menos gravámenes a la propiedad privada\index[concepto]{Propiedad privada!problema óntico}) sino de una cuestión ontológica (\emph{i.e.} la existencia o no de la propiedad privada\index[concepto]{Propiedad privada!cuestión ontológica} en cuanto tal).}

Si Lacan\index[concepto]{Lacan, Jacques!hontologie} decía \emph{hontologie}, jugando con la homofonía francesa entre ontología\index[concepto]{Ontología!en Lacan} y vergüenza\index[concepto]{Vergüenza!en ontología}, es para nuestra concepción de la ontología y ciertamente no para la suya \rdm{todavía bajo la necesidad epocal de anti-filosofía y pensando al ser como esencia (\emph{ousía}) y no como múltiple de múltiples (Badiou)}, porque la vergüenza adviene cuando quedamos al descubierto, evidenciados por enunciar algo que no está previamente determinado. Y es por eso precisamente que una ontología está transida por lo político\index[concepto]{Político, lo!en ontología}: por no estar reducida a un objeto constituido sino situada en la apertura de la pregunta misma por el \emph{Ser} (en este caso del sujeto).

\section{La versión cartesiana}


René Descartes\index[concepto]{Descartes, René} es considerado el filósofo que inaugura la modernidad\index[concepto]{Modernidad!inicio en Descartes}. Lo moderno de este pensador estaría condensado, entre otras cosas, en la forma de organización de su obra. \enquote{Se supone que Descartes da a ver, por el ordenamiento interno de su obra, aquello que el nacimiento de la ciencia moderna\index[concepto]{Ciencia moderna!nacimiento} requiere del pensamiento}.\footnote{Lacan citado en \cite[][41]{@7128-MILNER1999}} Su sistema filosófico, a pesar de que su lenguaje todavía pertenezca a la tradición que lo precede, se sitúa en un contexto de ruptura con la medievalidad\index[concepto]{Medievalidad!ruptura}. Más específicamente en la búsqueda de un método\index[concepto]{Método!cartesiano} que permita acceder a un conocimiento cierto\index[concepto]{Conocimiento!certeza} sin apelar a los recursos de la autoridad bíblica\index[concepto]{Autoridad!bíblica} o la lógica silogística aristotélica\index[concepto]{Lógica!silogística}. Es decir, un método forjado con leyes autónomas de la razón\index[concepto]{Razón!leyes autónomas}. Ya que hasta el momento, la filosofía \enquote{ha sido cultivada por los más excelentes ingenios que han vivido desde hace siglos, y, sin embargo, no hay nada en ella que no sea objeto de disputa y,~por consiguiente, dudoso}.\footnote{Descartes citado en \cite[][164]{@7114-CARPIO1980}.}

En ese estado de cosas, Descartes erige la duda\index[concepto]{Duda!método cartesiano}\footnote{La duda no es una duda como finalidad, una duda destructiva en el sentido escéptico\index[concepto]{Escepticismo!duda destructiva}, sino que la duda es el método (camino). Alexandre Koyré dice: \enquote{El escéptico y Montaigne padecen la duda. Descartes, en cambio, la ejerce; y ejerciéndola libremente la domina. A través de ella se libera}. \cite[][196]{@7115-KOYRE1984}. Traducción propia.} de todo lo sabido como método para llegar a un punto cero de apodicticidad\index[concepto]{Apodicticidad!punto cero}, al punto de Arquímedes\index[concepto]{Arquímedes!punto de apoyo} que le permite construir su universo de discurso. Como él mismo lo ilustra: \enquote{Arquímedes, para levantar la tierra y transportarla a otro lugar, pedía solamente un punto de apoyo firme e inmóvil; también tendré yo derecho a concebir grandes esperanzas, si tengo la fortuna de hallar solo una cosa que sea cierta e indudable}. \footcite[][119]{@7116-DESCARTES1941} Lo puesto en duda, para lo que bastará simplemente una sola razón, tendrá que ver con dos ámbitos: 1) el ámbito del conocimiento sensible o material\index[concepto]{Conocimiento!sensible} y 2) el ámbito del conocimiento racional o las ideas innatas\index[concepto]{Ideas!innatas}. La duda sobre el primer ámbito se pondrá en acto al \enquote{desligar el espíritu de los sentidos}\index[concepto]{Sentidos!Desligar del espíritu}, \footcite[][107]{@7116-DESCARTES1941} al desconfiar de las sensaciones hasta el punto de apelar a la imposibilidad de distinción entre el sueño y la vigilia\index[concepto]{Sueño y vigilia!indistinción}. En cuanto al ámbito del conocimiento racional\index[concepto]{Conocimiento!racional}, Descartes apunta que si bien este tiene que ver con la intuición de algo esencial (como el principio de identidad\index[concepto]{Principio de identidad}) y por ende inequívoco, podría suceder que hubiera un genio maligno\index[concepto]{Genio maligno}\rdm{cierto hasta que se pruebe lo contrario} que nos haya hecho de tal manera que siempre nos equivoquemos. Por lo tanto, de allí se deriva la necesidad de aplicar la llamada duda hiperbólica\index[concepto]{Duda!hiperbólica}, la duda de todo lo obtenido a fuer de razonar.

Llevado esto a cabo, desembarcados en el mar de la incertidumbre del conocimiento sensible y racional, en la más pura inseguridad ontológica\index[concepto]{Inseguridad ontológica}, Descartes logra echar anclas para refundar la racionalidad humana apelando a algo que según veremos sería pasible de resistir la duda: el \emph{cogito}\index[concepto]{Cogito} o la sustancia pensante\index[concepto]{Sustancia!pensante} en tanto tal. Para ello, la nueva definición del hombre\index[concepto]{Hombre!definición cartesiana}, su núcleo innegociable es, entonces, \enquote{ser una sustancia cuya esencia o cuya naturaleza toda es pensar}. \footcite[][173]{@7116-DESCARTES1941} Así es como da el paso:

\begin{quote}
	Pero advertí luego que, queriendo yo pensar, de esa suerte, que todo es falso, era necesario que yo, que lo pensaba, fuese alguna cosa; y observando que esta verdad: \enquote{yo pienso, luego soy}\index[concepto]{Cogito!pienso luego soy} era tan firme y segura que las más extravagantes suposiciones de los \index[concepto]{Escépticos} no son capaces de conmoverla, juzgué que podía recibirla sin escrúpulo, como el primer principio de la filosofía que andaba buscando.\footnote{Descartes, citado en \cite[][171]{@7114-CARPIO1980}.}
\end{quote}

En este pasaje por la duda del conocimiento racional y la duda sobre el conocimiento sensible, donde lo único que puede resistir los embates es el pensamiento \emph{qua} pensamiento\index[concepto]{Pensamiento!en sí mismo}, como vemos, el ser quedará homologado o identificado al pensar, ya que para pensar es preciso ser y se es porque al dudar de absolutamente todo no se puede dudar de que alguien duda: \enquote{yo}. Por lo cual, el uso del \enquote{luego} en la fórmula \enquote{pienso, luego existo} es simplemente expletivo. Es necesario recordar, no obstante, que al realizar esta operación Descartes queda entrampado en un solipsismo\index[concepto]{Solipsismo!cartesiano} que podemos leer así en las \emph{Meditaciones metafísicas}: \enquote{el conocimiento de mí mismo, tomado precisamente así, no depende de las cosas, la existencia de las cuales aún no me es conocida (\ldots)}. \footcite[][123]{@7116-DESCARTES1941} Solipsismo que podrá agujerear apelando a la figura de Dios\index[concepto]{Dios!en Descartes} y dando un salto lógico que va, sin mediación, del cogito (pura realidad psíquica que no tiene espacio) a la \emph{res extensa}\index[concepto]{Res extensa} (mundo externo sensible que está fundado en la existencia de Dios).

Lo que nos interesa recalcar, específicamente, es que esta salida del puro \textit{cogito} es posible a través de un círculo argumentativo en el que para afirmar la existencia de Dios y por ende de la \emph{res extensa} es necesario ya suponerlo de antemano, dado que la lógica \rdm{el \textit{cogito} es demostrado por Descartes por vía lógica y no a través de un acto de fe} pertenece a las ideas innatas que fueron implantadas en los hombres por Dios. Es decir que estamos frente a una petición de principio, una verdad que para ser demostrada ha de ser previamente supuesta como verdadera.

Descartes, podemos decirlo ahora expresamente, crea con este gesto el correlato filosófico del sujeto de la ciencia (Galileo, Kepler, Copérnico constituyen su horizonte de época). Un sujeto supuestamente consciente, reflexivo, autoevidente y \emph{auto-fundado}, que no obstante, en el momento previo a recibir cualidades de la consciencia (dudar, concebir, querer, imaginar, sentir), habría sido esbozado por Descartes como, según el axioma del sujeto de Milner elaborado con el tratamiento que da Lacan a la cuestión en el seminario 9, \enquote{distinto de toda forma de individualidad empírica}. \footcite[][35]{@7128-MILNER1999} Es decir, como un sujeto análogo a los objetos despojados de sus cualidades sensibles a través de su matematización.

Es precisamente ese sujeto vaciado de consciencia de sí, de su yo imaginario.\footnote{Lo imaginario siguiendo a Milner \parencite[][58-59]{@7128-MILNER1999}, es de estructura y tiene que ver con la \emph{buena forma}. El paso de la \textit{episteme} antigua (cosmos-objetos eternos) hacia la tecnociencia moderna (universo-objetos contingentes) significó que la buena forma cediera ante la \emph{mala forma}. Es decir, que el proceso de matematización y por ende literalización de la ciencia moderna implicaron la abolición de los rasgos imaginarios (eternidad, equilibrio, unidad, lo esférico) protectores de lo bello. Estos, sin embargo, siempre retornan bajo diversas formas en la cultura.}, del \emph{primer momento} cartesiano, en el que estaría interesado por sobre todo Lacan en su paralelo con el sujeto del significante o del psicoanálisis.\footnote{El recorrido de ontologización del sujeto en Descartes deja entrever para Lacan la relación del sujeto con el significante, es decir, su \emph{caída}, su desontologización, su quedar atrapado en las redes de un universo discursivo que, por su constitución incompleta, no puede otorgarle al sujeto un ser autónomo sino una identidad metonímica, en desplazamiento continuo, siempre en relación con otros significantes.} Ya que el inconsciente está estructurado como un lenguaje que no se define a partir de las cualidades imaginarias de la consciencia sino por su constante retirada y a la vez determinación de ese plano. Si como dice Milner siguiendo a Freud, el narcisismo es siempre una demanda de excepción para uno mismo, la introducción de la hipótesis del inconsciente no hace más que acabar definitivamente con esa excepcionalidad construida con las \emph{buenas formas} de lo imaginario. En este sentido, el inconsciente es un hijo legítimo del universo \emph{real} de la ciencia que acaba con los privilegios narcisistas del hombre (que los astros giren a su alrededor). Si \enquote{lo infinito es lo que dice \enquote{no} a la excepción de la finitud; el inconsciente es lo que dice \enquote{no} a la consciencia de sí como privilegio}. \footcite[][69]{@7128-MILNER1999}

Como quedó insinuado con Descartes, el sujeto de la ciencia se funda en el olvido del saber sobre su propia constitución. Así introduce la cuestión Lacan: \enquote{Este correlato, como momento, es el desfiladero de un rechazo de todo saber, pero por ello pretende fundar para el sujeto cierta atadura en el ser, que para nosotros constituye el sujeto de la ciencia, en su definición, término que debe tomarse en el sentido de puerta estrecha}. \footcite[][835]{@7117-LACAN1984} Esa \enquote{atadura al ser} es la que precisamente se obtiene del olvido del lugar vacío del deseo, de aquello que en lugar de ser es no-realizado. Pero para decirlo desde otra perspectiva, el acceso a la \enquote{cadena diacrónica de la ciencia}\footnote{La distinción de Bernard Baas y Amand Zaloszyc\footcite[][24-25]{@7118-BAAS1994} entre sincronía y diacronía tiene que ver con la voluntad de distinguir lo circular y \emph{siempre igual} (escepticismo) de lo lineal y progresivo (ciencia y conocimiento en sentido moderno).}, a la posibilidad de expandir el conocimiento en línea recta, forcluye la circularidad que le da lugar. Este es el \enquote{punto de indecidibilidad del sujeto de la ciencia}, \footcite[][24]{@7118-BAAS1994} y a eso se refiere Lacan cuando dice que \enquote{la lógica oficia de ombligo del sujeto},\footnote{Lacan citado en \cite[][]{@7118-BAAS1994}.} es decir que el sujeto se constituye en la relación con el Otro (que en Descartes toma el nombre de Dios). La indecidibilidad, en términos de Gödel, y a quien Lacan menciona en \enquote{La ciencia y la verdad}, tiene que ver con la imposibilidad intrínseca de cualquier sistema complejo de dar cuenta de su propia consistencia con los elementos del mismo sistema. Ya que para probar su consistencia, el sistema, tal como el sujeto cartesiano, debe apelar a un metasistema.

Ahora bien, es interesante notar que si bien dijimos que Descartes sería el filósofo moderno por excelencia en tanto su obra se organiza sistemáticamente como el pensamiento de la ciencia moderna requiere, el hecho de que su \emph{circulus in probando} no constituya una parte legítima y reconocida de su prueba de la existencia del sujeto como cogito, podría implicar que el de Descartes no es \emph{stricto sensu} un razonamiento secular-moderno en términos epistemológicos. Según Kordela,\footcite[][789-839]{@7119-KORDELA1999} el reconocimiento de la insuficiencia de la razón lógica para fundar su propio discurso es, específicamente, lo secular por excelencia, y un discurso teocrático se reconoce ya que esa brecha es siempre cubierta por la omnisciencia divina. De hecho, según la interpretación de esta autora, el \textit{dictum} lacaniano \emph{Dios es inconsciente} que tantas exégesis ha suscitado, implica que si para un paradigma discursivo teocrático Dios es un principio consciente de explicación que permite suplir las faltas de comprensión humanas, para un discurso secular moderno este se vuelve una necesidad inconsciente. Esto quiere decir que si entre la medievalidad y la modernidad hay un cambio real, lejos está de tratarse del cambio de un discurso fundado en Dios a un discurso fundado en la razón. El movimiento estaría más bien dado por el cambio de un discurso conscientemente fundado en Dios a uno que repudia su fundación en este pero que subrepticiamente lo introduce en el plano inconsciente. El psicoanálisis sería el discurso moderno por excelencia en tanto es aquel que más radicalmente pone en evidencia la insuficiencia de la lógica y del sistema simbólico para fundar un universo de discurso. En tanto todo discurso acarrea un inconsciente.

Incluso más, el psicoanálisis argumenta que el hecho de que el Otro no se pueda fundar lógicamente no implica ninguna liberación de su imperio por parte del sujeto sino que da cuenta más bien del carácter no-representable que adquiere ahora la contención del sujeto dentro del Otro. Que no haya meta-discurso (Dios) donde reparar la inconsistencia lógica de toda fundación requiere que el inconsciente haga algo con ese lugar vacío. En este sentido, para Kordela,\footcite[][793]{@7119-KORDELA1999} la recepción canónica del discurso ilustrado ha visto solamente la cara supuestamente liberadora de la muerte de Dios y no su metamorfosis. El psicoanálisis, ilustrado y romántico al mismo tiempo, tuvo en cuenta las dos caras del progreso.

\section{La sub-versión psicoanalítica}


El sujeto del psicoanálisis es un sujeto alienado no identificado. Alienado porque se sitúa entre medio de significantes y, por lo tanto, adquiere \enquote{valor de intervalo},\footnote{\cite[][]{@7120-LEGAUFEY2009} artículo disponible en línea: \url{http://elpsicoanalistalector.blogspot.com/2009/06/guy-le-gaufey-la-paradoja-del-sujeto.html}} habiendo perdido desde siempre la posibilidad de ser representado por un solo significante\rdm{en tanto el sujeto es siempre representado por un significante para otro significante} y no identificado porque justamente esa alienación en el significante, la imposibilidad de que un significante quede prendido en el cuerpo sin pasar por la instancia del Otro, es la que introduce la diferencia del sujeto consigo mismo y le vuelve imposible su auto-fundación. El poema de Alejandra Pizarnik con forma de ataúd, \enquote{Sólo un nombre}, podría ser un ejemplo gráfico de cómo el sujeto alienado en el campo del Otro queda sepultado en su cadena significante:

\begin{verse}
	alejandra alejandra \\debajo estoy yo\\alejandra
\end{verse}

Como se podría ver operando en el poema, el psicoanálisis desmitifica la calidad de puro agente del sujeto de la ciencia, introduciendo un quiebre al principio lógico de identidad y al principio de no contradicción (principios que para Descartes siguen siendo rectores). Esto se puede ver más nítidamente con la lingüística, donde Lacan realiza una operación de préstamo y modificación de algunos términos fundamentales. Nos referimos a los términos enunciado y enunciación que son tomados por él de Émile Benveniste y que tienen una historia que puede permitirnos ubicar mejor la operación de desmontaje del sujeto cartesiano por parte del psicoanálisis. Permítasenos este pequeño excurso.

Como es sabido, el lugar que otorga al sujeto el \emph{Curso de Lingüística General} de Ferdinand de Saussure es el de la portación pasiva del sistema o de la estructura de la lengua,\footnote{\enquote{la lengua no es una función del sujeto hablante; es el producto que el individuo registra pasivamente}. \cite[][63]{@6996-SAUSSURE2007}.} habilitando para toda una corriente de pensamiento las herramientas de una disciplina que abría la posibilidad de pensar la subjetividad humana fuera de la agencia y que, por otros medios, continuaba la línea de deconstrucción del sujeto cartesiano iniciada en filosofía a partir de~Nietzsche. La nueva lingüística, alejada de la corriente mentalista y de una idea esencialista del lenguaje centrada en la representación, ponía ahora el acento en el plano virtual de la lengua y no en su uso. La dimensión psicológica e individual del uso cotidiano de la lengua se torna superflua en tanto no produce efectos directos sobre el sistema y, los pocos que produce, son más bien fortuitos. Por lo demás, esta es su fortaleza, permitir abstraerse de las cualidades sensibles (históricas, sociales, etc.) de un sistema de signos determinado y establecerlo como un conjunto de relaciones lógicas de las que se desprenden invariantes. La significación de una palabra tomada en sí misma es pasajera, puede estar hoy y desaparecer mañana, las significaciones cambian con el uso, pero la lengua queda intacta. Este es el énfasis saussureano. El despojo de lo empírico genera la posibilidad de establecer leyes universales.

Dejando intacta la idea de que el individuo no es amo del sistema de la lengua \rdm{\enquote{lo que cambia en la lengua, lo que los hombres pueden cambiar, son las designaciones, que se multiplican, que se remplazan y que siempre son conscientes, pero jamás el sistema fundamental de la lengua}} \footcite[][98]{@7076-BENVENISTE2008} Émile Benveniste, intenta, sin embargo, poner el foco de atención en ese aspecto del lenguaje que Saussure estaría soslayando y que se ubica del lado del habla o la dimensión individual de la lengua: la enunciación. En \enquote{El aparato formal de la enunciación}, donde se presentan tres formas arquetípicas para su estudio, la enunciación es definida como el \enquote{poner a funcionar la lengua por un acto individual de utilización}, \footcite[][83]{@7076-BENVENISTE2008} esto es, como la apropiación única e irrepetible de la lengua que lleva a cabo un individuo al pronunciarse.

De lo que se tratará en este cambio de perspectiva que lleva a cabo Benveniste es de localizar los caracteres formales que operan en el acontecimiento que significa que quien habla se introduzca en su habla. Cuestión que marca, por un lado, la apertura de una relación al interior del discurso entre el locutor y su enunciación y, por otro lado, entre el locutor y un alocutario, \enquote{sea este real o imaginado, individual o colectivo}. \footcite[][88]{@7076-BENVENISTE2008} Esta relación implica, así, la presencia inexorable y fundante del diálogo, ya que cuando usamos el sistema de la lengua, lo hacemos determinados por un alguien que escucha. El acontecimiento único e irrepetible de una enunciación estará instanciado en marcas de persona (en su manifestación verbal o pronominal), en marcas de tiempo (el presente como configurador central del resto de los tiempos en el enunciado) y en marcas de espacio (el \emph{hinc et nunc} de una enunciación) y su estudio permite ver, entonces, las huellas de un sujeto en el discurso.

El sujeto para Benveniste, aun si continúa en cierta medida la vía estructuralista de registro pasivo de la estructura \rdm{la subjetividad no es más que \enquote{la emergencia en el ser de una propiedad fundamental del lenguaje}} también se patentiza en el momento activo de la enunciación, en tanto \enquote{es ego quien \emph{dice} \enquote{ego}}. \footcite[][181]{@7121-BENVENISTE2010} Para Benveniste, las dos instancias, pasiva y activa, virtual y actual, son fundamentales para una comprensión basta de la relación entre lenguaje y subjetividad. Y es por eso que decimos que en Benveniste hay un descompletamiento de la estructura que le da al sujeto parlante, o a la perspectiva desde la que el sujeto implica con su enunciación un acontecimiento discursivo, una relevancia mayor que la otorgada por su predecesor ginebrino. De hecho, a través del énfasis en la noción de discurso, de su consideración del uso de la lengua (pragmatismo discursivo) y de su llamado a ir más allá de Saussure y el signo lingüístico como único principio que determina la estructura y funcionamiento del lenguaje, Benveniste logra dar el puntapié inicial para el advenimiento de lo que luego será llamado postestructuralismo. Una nueva perspectiva que se abre para pensar la relación entre subjetividad y lenguaje como dialéctica continua entre lo mismo y lo otro, lo virtual y lo actual, la estructura y la agencia, el ser y el acontecimiento.

Lacan se introduce en este debate y no solo lleva a cabo la muy conocida reformulación del signo saussureano sino que, rescatando la producción de Benveniste, identifica sin embargo en él una relación especular entre enunciado y enunciación que habrá que romper en mil pedazos. Si para el lingüista es \enquote{yo} quien \emph{dice} \enquote{yo}, o sea hay un perfecto anudamiento entre lo enunciado y el sujeto de la enunciación, para Lacan, en cambio, \enquote{es a causa de que el sujeto dice yo \rdm{\emph{je}} que el sujeto en el decir desaparece, está en fading}. \footcite[][96]{@7129-GIUSSANI1991} Enunciado y enunciación son aquí dos dimensiones que no siempre se encuentran solapadas, entre otras cosas porque el deseo inconsciente (plano de la enunciación) es irreductible a la cadena significante (plano del enunciado), apareciendo más bien de manera enigmática en sus intervalos, en sus fallidos. De hecho, el intento de situar el deseo como causa, de saber algo sobre él no es para volverlo lingüísticamente formulable sino para reconocerlo como límite de lo que se puede decir (de ahí el \emph{medio decir}). Y ese quizá sea uno de los rodeos fundamentales de la clínica lacaniana, el rodeo del deseo. Separar al sujeto de lo inmediato de su demanda/su enunciado \rdm{no respondiendo a sus preguntas sin devolverle algo de su propio enunciado, ni dando consejos, ni fijando sentidos} para que se abra a la pregunta por su deseo inconsciente (enunciación).\footcite[][169]{@7016-MARQUESRODILLA2001}

Desde otra perspectiva, lo que está subvertido en el psicoanálisis es el esquema clásico de la comunicación propuesto por Jakobson: emisor-mensaje-receptor (más código y canal), ya que si el deseo atraviesa y corroe el decir del propio sujeto, entonces las instancias de la comunicación quedan languidecidas y de ninguna puede afirmarse que esté plenamente constituida. Si el mensaje bajo el esquema de Jakobson parece trasladarse por rieles precisos, ser un producto terminado que se agota en un contexto pasible de ser saturado, en Lacan el mensaje viene del Otro y va hacia el Otro sin nunca hallar una significación precisa. Esto es el resultado de una significación que aparece sobre todo retroactivamente y no se encuentra dentro del mensaje como si este llevara consigo una perla. Además, si el Otro (en todas sus formas) es tan fallido como el sujeto, cualquier significación que pasa por él vuelve en forma desestabilizada. Esta, de hecho, es la gran diferencia entre el Otro que postula Descartes (Dios) y el otro del sujeto del psicoanálisis. Si el Otro de Descartes es la perfección y la completud que le devuelve al sujeto una posición fija y amarrada al ser, el Otro del sujeto del psicoanálisis es una instancia que le devuelve el mismo signo de su incompletud y su carencia de ser.

Volviendo más estrictamente a Descartes, la deconstrucción psicoanalítica opera allí sobre la sutura del pensar a una realidad absolutamente presente en sí misma y cierta. Si aceptamos que el pensar es una actividad significante y que la lógica significante no puede sustraerse al deseo porque toda enunciación \emph{habla} del deseo sin hablar de él \rdm{y, por lo tanto, este le significa un escollo en su despliegue} entonces no queda más remedio que hacer del pensamiento un punto de desvanecimiento del sujeto, un punto en el que no se encuentra totalmente en casa. Al deshilar la sutura del ser del sujeto al pensamiento, el sujeto del psicoanálisis no solo queda dividido sino que queda en un orden simbólico incompleto, donde \enquote{el Otro no tiene los medios para contestar y darle al sujeto su signo de sujeto},\footnote{\cite[][4]{@7120-LEGAUFEY2009}; artículo extraído de: \url{http://elpsicoanalistalector.blogspot.com/2009/06/guy-le-gaufey-la-paradoja-del-sujeto.html}. Así como tampoco el sujeto puede responder por sí mismo ya que carece de un sí mismo. La reflexividad solo aparece \enquote{en la \enquote{confrontación con la imagen especular, se \emph{construye} con ella, cuando se produce el \thirdquote{moi} del lado del espejo}} \parencite[][9]{@7120-LEGAUFEY2009}.} aun si es de allí (del tesoro de significantes, la cultura, los discursos, etc.) de donde el sujeto toma inconscientemente la insignia que lo representa para otros sujetos significantes.

Por otro lado, si el orden simbólico es incompleto, esto alberga consecuencias para la verdad, las cuales se forcluyen en el campo de la propuesta cartesiana al garantizar, después de su duda hiperbólica, la existencia del Otro (Dios).\footnote{\enquote{El dios del que se trata aquí que hace entrar a Descartes en ese punto de su temática, es ese Dios que debe asegurar la verdad de todo lo que se articula como tal. Es lo verdadero de lo verdadero, el garante de que~la verdad existe}. \textcite[][23]{@7123-LACAN2000}, seminario no establecido, versión íntegra.} En el terreno del sujeto del psicoanálisis, al saberse en un orden simbólico incompleto (un Otro barrado), \enquote{no se puede saber la verdad de la verdad (\ldots), no hay garantías respecto a la verdad. {[}Y{]} como la verdad es una dimensión introducida en lo real por la palabra, es la palabra misma la que debe garantizar la verdad, a diferencia de la exactitud que se garantiza por su adecuación a lo real}. \footcite[][3-4]{@7124-WEISSE-SINFECHA}

En conclusión, si dijimos que el sujeto cartesiano homologa pensamiento a ser, el psicoanálisis rompe esta identidad. La lógica significante implica la escisión misma entre ser y pensar. Así lo articula finalmente Lacan: \enquote{No soy allí donde soy el juguete de mis pensamientos, pienso en lo que soy allí donde no me pienso pensar}. \footcite[][202]{@7125-LACAN1979} De esta manera es como surge el sujeto del inconsciente, en los lapsus del saber. Se trata de un sujeto que nunca sabe a ciencia cierta lo que dice.

Ahora bien, llevadas las cosas hasta el extremo de la disolución de la identidad del sujeto por parte del psicoanálisis, del énfasis en la alienación del sujeto en el significante, también es necesario decir que existe otra cara de esta operación. Lacan la llama \enquote{separación}, e implica una vuelta del sujeto sobre sí mismo, siendo \emph{se parare} un posible \emph{se parere} o parirse a sí mismo. Ciertamente no en un sentido de auto-fundación, de autonomía, sino de la producción de un saber ligado a la verdad del sujeto. Si la alienación al significante relacionaba al sujeto con el discurso del Otro (con sus significantes), la separación, en cambio, pone de relieve la pregunta por el deseo del Otro, que como no es un ser completo y autónomo es también \emph{deseante}. Si el Otro desea, si tiene falta, al asumirlo el sujeto consciente e inconscientemente, entonces, es posible acercarse mejor a la suya propia, que habrá sido velada por la producción de una respuesta fantasmática al \emph{primer encuentro} con ese deseo del Otro.

El proceso de separación tiene que ver con que no-todo en el sujeto posee carácter simbólico, no todo es interpretable y, por lo tanto, hay algo que muestra la \enquote{cara objetual del sujeto, su opacidad o coseidad, {[}y se trata del{]} límite de la significantización}, \footcite[][120]{@7016-MARQUESRODILLA2001} un resto-cosa. Ese objeto constitutivo del sujeto del psicoanálisis, llamado objeto a, es un resto que aparece en el proceso de un análisis, algo que tiene una fuerte dimensión real y no produce significaciones metonímicas, es más bien \enquote{el borde pulsional del sujeto}. \footcite[][118]{@7016-MARQUESRODILLA2001} Es un ser de goce llamado por Lacan plus-de-goce, significante de más excluido de la cadena significante que resulta de la operación de alienación del sujeto a la cadena significante. La separación, en tanto operación, implica una nueva relación con ese plus-de-goce, un pararse de otra manera frente a esa cara pulsional repetitiva. Donde, para efectuarlo, no queda más que atravesar los laberintos del fantasma que se ubica entre el sujeto y ese objeto también causa de deseo (\$\textless\textgreater a, sujeto barrado losange a). Como toda separación, la propuesta por el análisis conlleva una pérdida, que aquí se trata de un dejarse perder por parte del sujeto para asumirse como falta y poder hacerse cargo de su deseo.

Entonces, hecho el recorrido que diferencia el sujeto cartesiano y el del psicoanálisis, señalemos más claramente el punto fundamental en el que convergen: la piedra de toque que significa la duda respecto al pensamiento. En Descartes es la duda hiperbólica la que le permite encontrar la certeza del pensamiento y, por ende, de sí mismo como existencia. \enquote{Es en la enunciación del Yo dudo, donde se apoya la certeza. En el acto de pensar más allá de los contenidos pensados}. \footcite[][101]{@7122-GIUSSANI1991} Para Freud, es el campo discursivo dudoso en el que se relatan los sueños (la diferencia entre lo efectivamente soñado y lo narrado) lo que permite aseverar que hay allí un pensamiento, encontrar la certeza de un pensamiento inconsciente (\emph{unbewusste gedanke}) que también está más allá de los contenidos conscientemente pensados. La duda en el relato aparece en Freud, como dice Lacan, en tanto \enquote{signo de la resistencia} \footcite[][43]{@7106-LACAN2006} del sujeto a su inconsciente.

De la lectura lacaniana del cogito y de la lectura heideggeriana del cogito como inauguración de la violencia metafísica moderna (a pesar del rechazo mutuo de las ideas substancialistas del sujeto) se puede decir que Lacan, a diferencia de Heidegger \rdm{quien cree incluso necesario abandonar el concepto de sujeto ya que este arrastraría inexorablemente una estela metafísica}, \enquote{tematiza la experiencia cartesiana del cogito como acto inaugural del sujeto efecto del significante, el cual no es de ningún modo una substancia}.\footnote{ \cite[][86-87]{@7122-GIUSSANI1991}. Lacan \footcite[][30]{@7106-LACAN2006} le da al inconsciente el privilegio de lo no realizado, lo que no es ni ser ni no-ser, es decir algo que intenta siempre alejarse de la substancialización. Lo que a su vez, vale mencionarlo, otorga el estatuto ético a la práctica psicoanalítica. Ya que no se trata de ir a buscar algo inscripto en el sujeto de una vez y para siempre que sea determinable de antemano (un mensaje oculto donde cristalizar al sujeto) sino de comprobar que el sujeto solo aparece siempre en retrospectiva y como un efecto fallido del significante.} Y esto es posible por ese primer gesto de vaciamiento de lo imaginario que subrayamos en Descartes, ya que aun si sabemos del uso anacrónico del término, ninguna certeza puede sostenerse en ese registro.

En resumen, si ambos sujetos se tocan es porque ambos tocan en su experiencia algo de lo \emph{real}, llegando al límite de lo discursivo. Y si bien en el punto de la duda radical Descartes decide seguir \emph{hacia adelante}(abandonándola por la pura presencia/certeza del pensamiento) y Freud detenerse confiando en que allí hay un mundo que explorar, no hay que soslayar el punto previo de convergencia (el pensamiento desligado de sus cualidades, el pensamiento en su registro puramente significante) ya que es el suelo común que habilita las diferencias. El sujeto del psicoanálisis es imposible sin el sujeto cartesiano, porque es lo impensado de este.

\section{Y Badiou\ldots}


El sujeto en Badiou tampoco puede pensarse sino en relación de contigüidad con los dos sujetos anteriores. El mismo Badiou dice: \enquote{una filosofía es hoy posible, por tener que ser composible con Lacan} \footcite[][55]{@7126-BADIOU2007} y en referencia al sujeto cartesiano: \enquote{lo que localiza al sujeto es el punto en el que Freud solo se hace inteligible en la herencia del gesto cartesiano y en el que, a la vez, lo subvierte, des-localizándolo de la pura coincidencia consigo mismo, de la transparencia reflexiva}. \footcite[][473]{@7130-BADIOU2007}

Ahora bien, más allá de los enunciados explícitos de Badiou, algunos psicoanalistas lacanianos parecen preguntarse cómo queda posicionado realmente el sujeto en el recorrido de su obra y si hay algún privilegio de alguna de las dos partes, esto es: la herencia cartesiana (sujeto clásico de la filosofía) o la subversión freudo-lacaniana.

En principio, digamos que la centralidad del sujeto cartesiano para Badiou, su rescate de las críticas feroces a las que es sometido por casi todas las corrientes académicas actuales,\footcite[][9]{@7063-ZIZEK2005} proviene entre otras cosas de una voluntad renovada de ir contra el escepticismo que, cambiando de rostro en cada época, se caracteriza por una conformidad con la lógica de razonamiento circular que no asume el riesgo de \emph{salir}, esto es, de decidir por dónde romper el círculo. Si hay algo que pareciera sostener Badiou, digámoslo llanamente, es que no abrigar convicciones hoy equivale a ceder a los valores de un \emph{parlamentarismo} que enarbola las banderas de la neutralidad y la transparencia pero que esconde en ese mismo gesto su connivencia con el capital o lo peor de este.\footnote{No resulta difícil trazar un paralelo con la escena argentina, donde los políticos que se suponen más apegados a una ética parlamentaria son aquellos que en realidad se oponen sistemáticamente a que algo realmente cambie.} Materialismo democrático es, por cierto, el nombre badiouano de esa actitud natural que se impone como el sentido dominante de nuestra época. Es allí, entonces, donde Descartes se vuelve ejemplo de lo que significa sostener e indagar una convicción que nada a contracorriente de la actitud natural y que rompe el círculo vicioso de la duda escéptica.

La herencia cartesiana, por lo tanto, se asume en la lucha contra las formas de desaparición total del sujeto que no harían más que ceder a la naturalización del sistema. Porque donde no hay sujeto no hay ética, y si no hay ética no hay verdades que merezcan una indagación y una militancia, entendida esta última como el sostenimiento fiel de un indicio que se ha \emph{encontrado} en lo real y que, por la lógica retroactiva de su manifestación, \emph{habrá sido o no habrá sido un acontecimiento-verdad}. Ser sujeto para Badiou es ser agraciado por la in-corporación a una verdad acontecimental que se escapa a lo natural y automático del sistema, entendiendo por sistema/Estado/situación/normalidad la lógica de la repetición, de lo ya sabido, de lo ya establecido y legislado. Ser sujeto es ser el soporte finito de una verdad infinita y a la vez inmanente que adviene a una situación.

La recepción del sujeto cartesiano por parte de Badiou se diferencia de la recepción del psicoanálisis en que el primero parece no acudir plenamente al giro lingüístico para lo esencial de su subversión. De hecho, uno podría aseverar que ser y pensar para Badiou sí son lo mismo, pero con la diferencia de que ese pensar, ahora corroído por la multiplicidad sin Uno, no puede ser entendido bajo el resguardo de la plenitud y la presencia cartesianas. Badiou se aleja del giro lingüístico criticando la \enquote{concepción trascendental del lenguaje} \footcite[][103]{@7137-BADIOU2002} para sostener, en cambio, que el pensamiento y las verdades incluyen el lenguaje como un elemento más entre otros. \enquote{El nudo que une el pensamiento y el ser, nudo que se designa filosóficamente con el nombre de verdad, no posee una esencia gramatical (\ldots) se halla sometido a la condición del acontecimiento, del azar, de la decisión y de una fidelidad a-tópica, en vez de sometido a la condición que implican las reglas antropológicas y lógicas del lenguaje o de la cultura}. \footcite[][103-104]{@7137-BADIOU2002} Esta dislocación entre verdad y saber (o verdad y gramática) resulta en un abandono del examen de la verdad bajo la forma del juicio, gesto radical en el que según Badiou se reconoce la filosofía moderna.\footcite[][54]{@7137-BADIOU2002} Para Lacan, en cambio, si bien la verdad del síntoma del sujeto también es excluyente respecto a la lengua y en ese sentido tampoco habría para él un imperio absoluto de esta, no podría decirse que las condiciones biográficas de esa verdad sintomal no sean determinantes. Allí es donde la diferencia de dispositivos (psicoanalítico y filosófico)\footnote{La filosofía, para Badiou, no engendra verdades. Cuando decimos dispositivo filosófico, entonces, nos referimos a la instancia composibilitadora de las verdades que advienen en el campo artístico, científico, amoroso y político.} se hace evidente.

El sujeto de Badiou es un sujeto militante que se sustrae de las implicancias relativistas que parecieran suscitar el anclaje contextual-histórico, el despliegue de los condicionamientos subjetivos particulares, biográficos. La verdad, de esta manera, debe relocalizarse en el plano de lo común y desligarse de rasgos individuales, puramente lingüísticos, psicológicos y cognitivos así como de su carácter perspectivista; en tanto esta adquiere ahora un viso de \emph{eternidad} aun si surge de una situación particular y es infinita desde que sobrepasa el soporte finito de las subjetivaciones. La siguiente cita es aclaratoria: \enquote{si es cierto que toda verdad surge como singularidad, su singularidad es inmediatamente universalizable. La singularidad universalizable provoca una ruptura con la singularidad identitaria}. \footcite[][12]{@7018-BADIOU1999} En este sentido, como dijimos, es imposible no ver en Badiou una forma más \emph{filosófica} de eludir la subjetividad como realidad psíquica individual. Es por esto que psicoanalistas como Jorge Alemán,\footcite[][]{@7017-ALEMAN2000} preocupados por cómo es el sujeto \emph{real y efectivamente}, o cómo se manifiesta en la clínica, arguyen que Badiou, más allá de la introducción de lo indiscernible y de la multiplicidad sin Uno por el lado de la teoría de conjuntos, no se hace cargo verdaderamente de las íntimas conexiones entre el pensamiento filosófico y la lógica del todo.

En otras palabras, el reproche de los psicoanalistas a Badiou podría resumirse en la idea del sujeto sin objeto. Si el objeto es obviado por Badiou en tanto el sujeto se reduce a un simple trazo sin nombre propio, para el psicoanálisis, en cambio, el objeto es un rasgo constitutivo del sujeto. El objeto es justamente el nombre propio del sujeto. Esto puede ejemplificarse con la cuestión de la mirada, si la pensamos como metáfora de un sujeto, en tanto una mirada que mira sin objetos no logra constituirse como tal. Como dice Rodilla en relación con Descartes, cuestión análoga , sin embargo, a la crítica a Badiou: \enquote{El movimiento realizado por Descartes es semejante al de un sujeto que quisiera mirar su propia mirada. Trataría de acercar el ojo al espejo hasta casi dar con él, pero en ese momento, al desaparecer todo objeto, la visión misma se desvanecería. Lo mismo pasa con el pensamiento: cuando se prescinde de todo objeto de pensamiento para aprehenderse a sí mismo, se termina en el desvanecimiento del sujeto tal como Lacan lo plantea, o para poder evitarlo, se recurre a un \emph{Deus ex machina} que restablezca la consistencia amenazada}. \footcite[][141]{@7016-MARQUESRODILLA2001} Aquí residiría la cuestión, en que Badiou no estaría pasando al sujeto cartesiano por la operación de castración simbólica que implica la \emph{producción} de un objeto \emph{a} como resto y división del sujeto, como aquello que debe perderse para habilitar la entrada al orden simbólico y que una vez perdido queda como causa ausente. Para Jorge Alemán el sujeto de Badiou seguiría siendo un sujeto abstracto que ve sin engañarse con el señuelo del objeto, pudiendo mirar más allá, rectamente, sin anclaje, y revelándosele así el ser-en-tanto-ser. El sujeto del psicoanálisis, arguye, sin embargo, \enquote{no tiene un mirar trascendente, allende los objetos, para dar cuenta de su verdadero ser, sino que es un sujeto dividido, causado por la mirada del objeto. Por aquello que, desde todos los objetos, lo mira. Porque lo suyo no es el cálculo, sino el deseo y sus consecuencias: el síntoma y el fantasma}. \footcite[][220]{@7017-ALEMAN2000} Badiou estaría así, bajo esta lectura, a mitad de camino entre Descartes y Freud.

El mismo problema podría pensarse con la cuestión del vacío en su relación con el objeto \emph{a} y con el acontecimiento. Si en Badiou el vacío es el nombre propio del ser inconsistente en una situación, es decir, aquello que desde el punto de vista de lo establecido es imposible y que solo podrá advenir en la forma de un acontecimiento o ultra-uno, el vacío para Lacan es donde se introduce la estructura lingüística en el sujeto y donde adviene una creación de \emph{ser} del sujeto. La diferencia, advierte Rodilla,\footcite[][330]{@7015-MARQUESRODILLA2007} reside en los límites o bordes del vacío. Para Lacan el vacío queda delimitado en el sujeto, con bordes precisos, entre la pérdida de plenitud de goce \rdm{míticamente situada en el nacimiento o \emph{separación de Uno mismo}} y el objeto topológico creado sobre ese trasfondo. Objeto producido en un análisis, objeto plus-de-goce que no tendrá ser pero sí un semblante de ser por efecto del lenguaje.\footnote{Debe quedar claro que el objeto \emph{a} no es un objeto esencial, sino un objeto simbólico tejido con el entrelazamiento de la asociación libre y las intervenciones del analista.} Para Badiou, en cambio, el vacío es sin bordes, es ilimitado, no es ni global ni local,\footcite[][71]{@7130-BADIOU2007} es más bien lo inconsciente de una situación.

El acontecimiento, producto de ese vacío, requiere que haya una situación contada-por-uno y un subconjunto de elementos que no hayan sido \emph{tenidos en cuenta} (conjunto genérico). Ese subconjunto, si resulta verdaderamente en un múltiple acontecimental, implica un corte abrupto y una novedad para la situación, pero de ahí no se deriva una relación personal del acontecimiento con el sujeto, ya que el sujeto no encuentra en el acontecimiento un resto perdido por él que le indique algún camino sino que se inscribe en su apertura genérica, en su carencia de memoria. El objeto \emph{a}, en cambio, al estar en íntima relación con su propia biografía, no implica una novedad radical para el sujeto: \enquote{El acontecimiento propicio que es la producción y advenimiento del plus-de-goce no puede ser calificado \emph{sensu stricto} de acontecimiento porque se trata de la llegada de un lugarteniente o representante de la representación perdida y reconstruida verbalmente a lo largo de la cura. El plus-de-goce es una presencia vicaria, un \emph{hypokeimenón}, en el sentido que ya Aristóteles le da a este término}.\footnote{\cite[][331]{@7015-MARQUESRODILLA2007}. Otra diferencia importante entre Badiou y Lacan es que si para el primero la subjetivación no se produce con la ley, con el orden simbólico, ya que \enquote{la ley no prescribe que haya sujeto} \footcite[][331]{@7015-MARQUESRODILLA2007}, para el psicoanálisis, en cambio, no hay constitución de sujeto sin metaforización de la Ley, y es solo en referencia a esta que se puede pensar en los distintos modos de subjetivación: psicosis, neurosis, perversión y sus variantes (aun si lo anterior puede ser matizado y variado según los tiempos de la enseñanza de Lacan).}

Sucede que Badiou, en realidad, no se sumerge en las categorías de objeto \emph{a} y de goce por estar demasiado prendidas a la pulsión de muerte y al cuerpo --significantes predominantes en una época que legitima su relativismo con la idea de que solo hay cuerpos, lenguajes y finitud.\footnote{El materialismo democrático, nombre que como dijimos decide darle Badiou a esta era global, se construye a sí mismo como el paradigma de la liberación de los individuos al goce pleno del cuerpo y a su regulación siempre cada vez más permisiva a través de lenguajes jurídicos y culturales particulares.} En ese contexto Badiou propone dar un lugar central a la verdad del amor, la ciencia, la política y el arte como sitios donde son posibles las torsiones de lo que hay, a lo que desde el punto de vista de la creencia natural solo es una conjunción finita entre cuerpos y lenguajes. Pero el problema \emph{empírico} que se presenta es que los individuos surgidos de la matriz materialista democrática padecen sus exigencias y sus desventuras y, por lo tanto, la pregunta que se hacen los analistas también es válida: ¿de qué amor estaríamos hablando, por ejemplo, sin hablar de los objetos \emph{a}, sin hablar de los tropiezos del inconsciente de cada uno de los sujetos en singular?, ¿se puede hablar del amor como un procedimiento genérico de verdad, sin ataduras, sin tener en cuenta \enquote{los nuevos síntomas y las nuevas angustias}?

El imperativo de Badiou, difícil de aceptar si uno se atrinchera en una ortodoxia clínica, es el de rehabilitar la filosofía sin desfallecer la anti-filosofía de Lacan, privilegiando una lógica eminentemente más colectiva, con pretensiones universalistas. Y (\ldots)  ¿no es esta en definitiva una gran apuesta que hay que leer en la totalidad de su gesto, es decir en el contexto de una época de dispersión y de retraimiento de las narrativas colectivas? Porque (\ldots)  ¿acaso Badiou no sabe de los mecanismos de repetición de los sujetos, acaso no sabe que hay \emph{goce} en el amor y también en la pasividad, el sometimiento, la servidumbre voluntaria? ¿Se trata de hacer caso omiso de ese \emph{dato} de la subjetividad, siempre reactualizándose en los consultorios, para esconderlo debajo de la alfombra? ¿No se tratará, en cambio, de un intento de heroizar la existencia, una existencia demasiado apegada al consumo masivo de aparatos electrónicos que prometen suplir lo que no pueden y al sufrimiento diario de los cuerpos martirizados, hambreados, masacrados por los \emph{imperativos} de la economía? ¿La filosofía de Badiou, acaso no nos da una perspectiva dignificante del sujeto, una bocanada de aire fresco en medio de un \emph{apocalipsis} para abrir el pensamiento a otros horizontes?

Como dijimos en un principio, no hay sujeto \emph{a priori} sino producción performativa y aparición retroactiva de múltiples sujetos sobre el trasfondo de lo (im) posible de una situación. Partiendo de ese supuesto anti-esencialista, no se puede ver en el terreno del sujeto más que una disputa onto-política. Por ello, creer que Badiou ignora el aspecto de la humanidad enmudecida en los goces mortificantes y autistas y en la animalidad deshonrosa es, más bien, ignorar nosotros por dónde elige romper el círculo de la que llama con un coraje más bien extemporáneo: doxa contemporánea.


\section*{Referencias}
\printbibliography[heading=none]   % Sin título automático


%%%%%%%%%%%%%%%%%%%%%%%
\ifPDF
\separata{capitulo7}
\fi

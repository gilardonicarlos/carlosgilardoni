\ifPDF
\chapter[\hspace{1.5pc}Ontología de la sobredeterminación]{Ontología de la sobredeterminación}
\chaptermark{Ontología de la sobredeterminación}
\Author{Andrés Daín}
\setcounter{PrimPag}{\theCurrentPage}
% encabezado para autor
\begin{center}
	\nombreautor{Andrés Daín}\\
	\vspace{15mm}
\end{center}
\fi

\ifBNPDF
\chapter[\hspace{1.5pc}Ontología de la sobredeterminación]{Ontología de la sobredeterminación}
\chaptermark{Ontología de la sobredeterminación}
\Author{Andrés Daín}
% encabezado para autor
\begin{center}
	\nombreautor{Andrés Daín}\\
	\vspace{15mm}
\end{center}
\fi

\ifHTMLEPUB
\chapter{Ontología de la sobredeterminación}
\fi




\section{Introducción}

En el presente capítulo nos proponemos avanzar en la elaboración de una ontología política que se haga eco de las principales implicancias del pensamiento político posfundacionalista y que esté orientada a poner en tensión el encorsetamiento impuesto al análisis político por parte de la ciencia política canónica. Una ontología que dé cuenta de una aproximación antiesencialista a los procesos de creación y fijación de sentido, al tiempo que asuma el carácter constitutivamente abierto de toda identidad. A la vez que también evidencie la relativa estructuralidad que actúa como superficie de inscripción de toda significación, su carácter intrínsecamente fallido y su condición de estar hegemónicamente suturada. Una ontología de la significación que cuestione toda parcelización teórica que actúe como legitimadora de la delimitación de un campo de objetos y prácticas denominadas \emph{a priori} más como \emph{políticos} y así nos permita pensar  al análisis político  \linebreak como un \emph{modo de ve}r que como un \emph{qué ver}. En definitiva, una ontología que nos posibilite avanzar en la conformación de un nuevo campo de intervención politológica a partir de una concepción del modo en que se crean las estructuras socialmente significativas; o sea, una ontología que legitime al análisis político para dar cuenta de las operaciones ideológicas que definen los objetos y las prácticas sociales, sean o no definidas por las gramáticas tradicionales como \emph{políticas}. En conclusión, una ontología de la sobredeterminación.

Una \emph{ontología}, en cuanto se cuestiona por el modo en que los objetos y las prácticas sociales adquieren su \emph{ser}; ya que pretende dar cuenta de los procesos de creación/fijación de sentido. De la \emph{sobredeterminación}, en cuanto esta es la categoría central que nos permite evidenciar la lógica que orienta cómo se relacionan los sentidos \rdm{constitutivamente dislocados} entre sí y cómo se relacionan estos con la relativa estructuralidad que actúa como superficie de inscripción de todo proceso significativo. \emph{Política}, porque se inscribe en el marco de una ontología política de lo social, a partir de la cual asume que todo acto de significación está basado en un acto de exclusión y que la operación hegemónica es la condición de posibilidad de cualquier sentido social en la medida en que \emph{sutura} el contexto donde estos se fijan.

En esta dirección, el presente trabajo se estructura básicamente en tres partes. En primer término, plantearemos brevemente dos hitos fundamentales en el modo de pensar la significación desde una gramática posfundacionalista; por un lado, la lingüística estructuralista de origen saussureano, y por el otro, la crítica derridiana a la misma idea de estructura, intervención ineludible para cualquier aproximación antiesencialista. Este primer apartado concluirá con la presentación de la categoría de discurso que habilita nuestra ontología de la sobredeterminación La segunda parte girará en torno a la propia noción de sobredeterminación, y aquí nuevamente nos encontraremos con dos momentos centrales; el inaugural, a partir de los desarrollos del psicoanálisis freudiano, particularmente en referencia a la interpretación de los sueños, y el de la traslación althusseriana de la sobredeterminación al campo general del pensamiento social estructuralista. Finalmente, en un tercer y último momento, plantearemos nuestra propuesta de una ontología de la sobredeterminación.

\section{La irrupción del lenguaje en las ciencias sociales}

Los desarrollos teóricos en el campo de la lingüística en las primeras décadas del siglo pasado comenzaron a habilitar una nueva forma de pensar no solo la propia estructuración del lenguaje, sino, sobre todo, el conjunto de las relaciones sociales en la medida en que permitieron pensar cualquier sistema de significación. Particularmente, en el marco de la teoría política \emph{postestructuralista} la noción central de \emph{discurso} está directamente inspirada en la idea de \emph{estructura} de la lingüística saussureana y postsaussureana.

Ferdinand de Saussure va a emprender un análisis estrictamente formal del lenguaje desplazando \rdm{hasta su anulación} el rol del \emph{referente} en el proceso de significación. De esta manera, Saussure comienza a pensar el lenguaje como un sistema de diferencias, donde no hay términos positivos ni sustancia, pasando el lenguaje a ser pura \emph{forma}. La identidad de un término, de un \emph{signo} (unidad lingüística) deja de depender del \emph{referente} y se constituye a partir de un relacionamiento diferencial con el resto de los signos que conforman la lengua, de modo que la identidad de los \emph{eventos} está dada por su pertenencia a la estructura y no por la existencia de un vínculo natural o esencial entre el \emph{significante} (imagen acústica) y el \emph{significado} (concepto); diferenciándose así de la teoría \emph{referencial} del significado, para la cual las palabras denotan objetos y, por tanto, el lenguaje es reductible a una nomenclatura. Precisamente, este sería el aporte principal del \emph{corte} saussureano: \enquote{la ruptura se situaría entonces esencialmente en el nivel de la definición de una teoría del valor, en los principios generales de descripción, en la abstracción del procedimiento. Su noción de sistema es la expresión de la construcción de una vía abstracta, conceptual, puesto que un sistema no se observa, y sin embargo cada elemento lingüístico depende de él}. \footcite[][64]{@6995-DOSSE2004} Como podemos observar, en Saussure existe una presencia de la estructura a partir de sus efectos en cada elemento del sistema y, a su vez, cada evento no puede ser significado con independencia de su partencia estructural. Estamos ya \rdm{como procuraremos demostrar más adelante} en la lógica de la \emph{sobredeterminación} estructural.

Para Saussure el vínculo de asociación de los términos implicados en el signo lingüístico es de carácter psíquico, la unión entre el significante y el significado es arbitraria y la entidad lingüística solo existe gracias a la asociación entre el concepto y la imagen acústica. En otros términos, el signo no une una cosa con su nombre, sino un concepto con su imagen acústica. Ahora bien, el carácter arbitrario del signo no debe llevarnos a la burda conclusión de que cualquier asociación entre el significante y el significado es posible: \enquote{el símbolo de la justicia, la balanza, no podría reemplazarse por otro objeto cualquiera, un carro, por ejemplo}. \footcite[][145]{@6996-SAUSSURE2007} Esta limitación a la arbitrariedad del signo va a ser pensado por la teoría política postestructuralista en relación con una relativa estructuralidad que limita/habilita cierta fijación (parcial y contingente) entre significado y significante. Asimismo, tampoco debemos ubicar al sujeto en el lugar de portador de dicha arbitrariedad. No se trata de una cuestión de libre elección; por el contrario, las posibilidades del individuo son prácticamente nulas. La noción de arbitrariedad pretende dar cuenta del carácter \enquote{\emph{inmotivado}, es decir, arbitrario con relación al significado, con el cual no guarda en la realidad ningún lazo natural}, \footcite[][146]{@6996-SAUSSURE2007} pero es impuesto con relación a la comunidad lingüística que lo emplea. Por lo tanto, el relato del \emph{contrato} \rdm{por lo menos en los términos que la tradición liberal lo ha pensado} para dar cuenta de esta arbitrariedad del signo no encuentra lugar en la propuesta saussureana: \enquote{precisamente porque el signo es arbitrario no conoce otra ley que la de la tradición, y precisamente por fundarse en la tradición puede ser arbitrario}. \footcite[][153]{@6996-SAUSSURE2007} Estamos frente a la paradójica relación\emph{mutabilidad}/ \emph{inmutabilidad} del signo lingüístico. Lo que gobierna todo cambio es la persistencia de la materia vieja, de modo que siempre hay una infidelidad parcial del pasado El signo puede alterarse porque se continúa. Se produce, en definitiva, un continuo \emph{desplazamiento} de la relación entre el significado y el significante.

Con la finalidad de condensar aún más el argumento saussureano podemos presentar su conclusión fundamental: el lenguaje \enquote{gira todo él sobre identidades y diferencias, siendo estas la contraparte de aquéllas}. \footcite[][230]{@6996-SAUSSURE2007} Así, podemos sintetizar el orden de la lingüística estructural en torno a dos principios fundamentales: el primero indica que \enquote{\emph{en la lengua no hay más que diferencias} (\ldots) pero en la lengua \emph{sólo hay diferencias sin términos positivos}}, \footcite[][247]{@6996-SAUSSURE2007} lo que revela el carácter puramente relacional y diferencial de las identidades lingüísticas, mostrando por un lado que \enquote{la lengua no puede ser otra cosa que un sistema de valores puros} \footcite[][235]{@6996-SAUSSURE2007} a la vez que constituye un sistema en el cual ningún elemento puede ser definido independientemente de los otros. Y el segundo principio indica que la lengua es \enquote{una forma, no una sustancia}, \footcite[][237]{@6996-SAUSSURE2007} donde cada elemento se define exclusivamente mediante reglas de combinaciones y sustituciones con los otros elementos, poniéndose en evidencia el carácter formal del lenguaje.

\section{El retiro del fundamento}

Los \emph{Cursos} de Saussure no tardaron en generar un fuerte impacto en el mundo intelectual, alcanzando un importante nivel de influencia en las ciencias sociales, ya que rápidamente se procuró exportar el modelo lingüístico al campo de las ciencias humanas en general, a partir de la idea de que cualquier sistema de significación puede ser abordado en los mismos términos en que Saussure estudió el lenguaje. En este sentido, entendemos que es importante insistir sobre algunas cuestiones fundamentales de la lingüística estructural. Por una parte, la asunción saussureana de un enfoque formal para abordar la significación, la consecuente desaparición del referente en el proceso de significación y el continuo desplazamiento de la relación entre el significante y el significado. Por otro lado, sus conclusiones en torno a la identidad como reverso de la diferencia y sobre la arbitrariedad del vínculo entre significante y significado; sobre los límites de dicha arbitrariedad y sobre las posibilidades del sujeto en dicha situación. Y finalmente, la centralidad de la pertenencia estructural para comprender dicho proceso, al tiempo que se habilita a pensar la estructura como algo no observable ni aprehensible directamente sino a través de sus implicancias.

Sin embargo, el marco saussureano pronto comenzó a mostrar insalvables fisuras. Será 1967 el año que verá aparecer dos críticas centrales desde genio de una misma persona: Jacques Derrida. La primera se encuentra en su ya clásica obra \emph{De la gramatología} donde sitúa claramente al estructuralismo saussureano dentro de su (devastadora) crítica a la \emph{metafísica de la presencia} y al \emph{logocentrismo}. Pero, a los fines que estamos abocados, el texto central será un breve trabajo publicado ese mismo año en su libro \emph{La escritura y la diferencia}, el cual terminará por configurarse como el \emph{locus} del pensamiento posfundacionalista. Nos referimos a \emph{La estructura, el signo y el juego en el discurso de las ciencias humanas}. Dicho artículo se erigirá como una bisagra fundamental para el desarrollo del estructuralismo y será bautizado por algunos autores como el texto fundacional del postestructuralismo, su \emph{locus classicus}.

En este trabajo Derrida no se ocupa directamente de Saussure; por el contrario, realiza una lectura de Claude Lévi-Strauss y a partir de la misma \emph{deconstruye} una noción central dentro de las corrientes estructuralistas: la propia noción de estructura, o más bien, la \emph{estructuralidad} de la estructura; y es en este sentido que podemos hacer extensible sus conclusiones al conjunto del pensamiento estructuralista en general y a la lingüística saussureana y postsaussureana en particular.

Derrida comienza el texto indicando que en la historia del concepto de estructura se ha producido un \emph{acontecimiento}, que tendría la forma exterior de una \emph{ruptura} y de un \emph{redoblamiento}. \emph{Acontecimiento} que ha consistido en pensar la misma estructuralidad de la estructura, cuestión que las corrientes estructuralistas parecieron siempre haber dado por supuesto o bien creyeron haber neutralizado y reducido \enquote{mediante un gesto consistente en darle un centro, en referirla a un punto de presencia, a un origen fijo}. \footcite[][383]{@6997-DERRIDA1989} Por una parte, dicho \emph{acontecimiento} se presenta como una \emph{ruptura}, como una \emph{dislocación} que es parte ya del conjunto de una época y que encuentra sus momentos culmines en los nombres de Nietzsche, Heidegger, Freud, entre otros. No solo bajo la forma de un discurso filosófico sino que también se trata fundamentalmente de \enquote{un momento político, económico, técnico, etc.}. \footcite[][388]{@6997-DERRIDA1989} Y por la otra, se muestra también como un \emph{momento} que ha estado \emph{siempre ya} allí, y no de una simple deriva de los desarrollos intelectuales contemporáneos. De esta forma se produce una apertura a un nuevo modo de pensar la estructuralidad de la estructura, una \emph{irrupción} del fundacionalismo desde dentro, un cambio paradigmático sin la necesidad de \enquote{restringir de una manera histórica los momentos de aparición de tales acontecimientos a nuestro propio tiempo}. \footcite[][31]{@6998-MARCHART2009}

A la estructura siempre se le ha asignado un centro cuya función principal no solamente era organizarla (el orden es condición \emph{sine qua non} de la estructura) sino fundamentalmente hacer del principio de organización de la estructura un límite al \emph{juego} de la  \linebreak estructura, un límite al juego de los desplazamientos y de las sustituciones hacia el interior de la misma.El centro de una estructura posibilita su propia organización al permitir el juego de los elementos en el interior de dicha totalidad, a la vez que plantea los propios límites de dicho juego: en tanto que el centro habilita el juego de los elementos de la estructura, a la vez \enquote{es el punto donde ya no es posible la sustitución de los contenidos, de los elementos, de los términos}. \footcite[][374]{@6997-DERRIDA1989} Veamos, por ejemplo, el caso del lenguaje. Si la lógica relacional y diferencial es la que guía el proceso de constitución de la identidad de todos y cada uno de sus signos, entonces la condición de posibilidad del lenguaje es que este sea un sistema cerrado, ya que sin dicha \emph{sistematicidad} sería imposible la significación porque el sistema no podría determinar si una diferencia está dentro o fuera del mismo.

En definitiva, el centro es condición de posibilidad de la estructura porque le brinda un punto a partir del cual se organizan sus elementos constitutivos, pero al mismo tiempo escapa a la propia estructuralidad de la estructura. Como afirma nuestro autor: \enquote{del centro puede decirse, paradójicamente, que está \emph{dentro de} la estructura y \emph{fuera de} la estructura. Está en el centro de la totalidad y sin embargo, como el centro no forma parte de ella, la totalidad tiene \emph{su centro en otro lugar}. El centro no es el centro}. \footcite[][384]{@6997-DERRIDA1989} Por lo tanto, el deseo del centro en la constitución de la estructura apela a una presencia central imposible porque desde siempre dicha presencia \enquote{ha estado deportada fuera de sí en su sustituto {[}pero éste{]} (\ldots) no sustituye a nada que de alguna manera le haya pre-existido}. \footcite[][385]{@6997-DERRIDA1989} Así, como está tanto dentro como fuera, el centro de la estructura ha recibido tanto los nombres de origen o de fin, pero todos sus movimientos y reemplazos quedaron siempre inmersos en una (H)istoria del sentido \enquote{cuyo origen siempre puede despertarse, o anticipar su fin, en la forma de la presencia}. \footcite[][384]{@6997-DERRIDA1989}

Toda la historia del concepto de estructura,\footnote{Según Derrida, la historia del concepto de estructura posee la edad de la \emph{episteme} de la ciencia y la filosofía occidentales, pero tiene \enquote{sus raíces en el suelo del lenguaje ordinario, al fondo del cual va la episteme a recogerlas para traerlas hacia sí en un desplazamiento metafórico}. \cite[][388]{@6997-DERRIDA1989}.} hasta el \emph{acontecimiento} que refiere Derrida, \enquote{debe pensarse como una serie de sustituciones de centro a centro, un encadenamiento de determinaciones del centro} \footcite[][385]{@6997-DERRIDA1989} que actúa como fundamento, como principio, como lo invariante de una presencia y que ha ido recibiendo diferentes nombres: \emph{eidos}, \emph{arché}, \emph{telos}, \emph{energeia}, \emph{ousía}, conciencia, Dios, hombre, etc.

La inexistencia del centro y su requerimiento, hace pensar que el centro no puede tener la forma de un ente-presente ni de un lugar fijo \enquote{sino una función, una especie de no-lugar en el que se representaban sustituciones de signos hasta el infinito}. \footcite[][385]{@6997-DERRIDA1989} El centro, en definitiva, tiene la forma de una \emph{falta}.

Por lo tanto, Derrida está descartando lo que él mismo denomina la \emph{hipótesis clásica} sobre cómo pensar el límite de la totalización. No se trata entonces de que la totalización sea imposible debido a las limitaciones del sujeto cognoscente situado frente a una infinitud empírica; no es entonces que \enquote{la pluralidad no puede fundarse porque siempre será \enquote{demasiado plural} para que alguien la funde}. \footcite[][32]{@6998-MARCHART2009} Por el contrario, Derrida propone una nueva hipótesis, que algunos.\footcite[Véase][32]{@6998-MARCHART2009} han bautizado como \emph{posclásica} La imposibilidad de la totalidad no se debe a la infinitud de un campo, sino a que la propia naturaleza del campo excluye la totalización. En palabras de nuestro autor:


\begin{quote}
	(\ldots) este campo es, en efecto, el de un juego, es decir, de sustituciones infinitas en la clausura de un conjunto finito. Este campo tan solo permite tales sustituciones infinitas porque es finito, es decir, porque en lugar de ser un campo inagotable, como en la hipótesis clásica, en lugar de ser demasiado grande, le falta algo, a saber, un centro que detenga y funde el juego de las sustituciones (\ldots) . Ese movimiento del juego, permitido por la falta, por la ausencia de centro o de origen, es el movimiento de la \emph{suplementariedad}. No se puede determinar el centro y agotar la totalización, puesto que el signo que reemplaza al centro, que lo \emph{suple}, que ocupa su lugar en su ausencia, ese signo se añade, viene por añadidura, como \emph{suplemento}. El movimiento de la significación añade algo, es lo que hace que haya siempre \enquote{más}, pero esa adición es flotante porque viene a ejercer una función vicaria, a suplir una falta por el lado del significado. (\ldots) La \emph{sobreabundancia} del significante, su carácter \emph{suplementario}, depende, pues, de una finitud, es decir, de una falta que debe ser \emph{suplida}.\footcite[][397-398]{@6997-DERRIDA1989}
\end{quote}

Este \emph{descentramiento} de la estructura lleva a Derrida a una conclusión fundamental. Los límites de toda totalidad (la \emph{estructuralidad} de la estructura, la \emph{sistematicidad} del sistema) están penetrados por una \emph{indecidibilidad} constitutiva, están constitutivamente \emph{dislocados} pero no por razones \emph{empíricas}, no porque sea demasiado amplio y complejo, sino porque hay algo que le falta \rdm{un centro, un fundamento \emph{último}} que es justamente lo \enquote{que hace posible la pluralización misma al hacer \emph{imposible} la realización final de una totalidad}. \footcite[][33]{@6997-DERRIDA1989} Por lo tanto, si toda estructura está penetrada por una indecidibilidad constitutiva, ningún sistema, producto de dicha indecidibilidad de sus fronteras, puede hallarse totalmente protegido. De modo que toda identidad \emph{intra} sistema estará constitutivamente dislocada y esta dislocación pondrá en evidencia su radical contingencia, demostrándose \enquote{la imposibilidad de fijar con precisión \rdm{es decir, en relación con una totalidad necesaria} tanto las relaciones \emph{como las identidades}}. \footcite[][37]{@6999-LACLAU1990}

Dicha ausencia tiene un status cuasi-trascendental, ya que niega la posibilidad de un fundamento último (trascendental) y dicha negación es una verdad necesaria para todas las fundaciones empíricas. Este es el momento en que frente a la ausencia central el lenguaje invade al conjunto de las ciencias sociales y todo se convierte en discurso, es decir, todo comienza a ser pensado como un \enquote{sistema en el que el significado central, originario o trascendental no está nunca absolutamente presente fuera de un sistema~de diferencias}. \footcite[][385]{@6997-DERRIDA1989} Ahora bien, este cuestionamiento a la existencia de un fundamento último no solamente tiene implicancias empíricas al favorecer a una mejor comprensión de una determinada coyuntura al \emph{reactivar} los procesos mediante los cuales ciertos fundamentos \rdm{entre una pluralidad} devienen en dominantes \rdm{en \emph{hegemónicos}, como procuraremos demostrar más adelante}, sino que fundamentalmente nos arroja luz sobre la diferencia entre las \emph{condiciones de posibilidad} de cualquier fundamento precario y los determinantes empíricos del mismo. En otras palabras, este tipo de cuestionamiento del estatus ontológico de los fundamentos nos muestra que la imposibilidad de un fundamento último es al mismo tiempo \enquote{la condición de posibilidad de los fundamentos en cuanto presentes, vale decir, en su objetividad o \enquote{existencia} empírica como seres ónticos}, de forma que la pluralización de los fundamentos es consecuencia \enquote{de una imposibilidad radical, de una brecha radical entre lo óntico y lo ontológico que es preciso postular a fin de dar cuenta de la pluralidad en la esfera óntica}. \footcite[][30]{@6998-MARCHART2009} La imposibilidad de un fundar definitivo o trascendental es una verdad necesaria \rdm{un supuesto ontológico} para todas las fundaciones empíricas, pero como al mismo tiempo se sigue sosteniendo la necesidad del fundamento, este sigue presente en su ausencia y sigue operando en tanto fundamento, pero siempre sobre la base de su propia ausencia. Quizás sea esta la diferencia cardinal entre el \emph{pos}fundacionalismo y cualquier forma de \emph{anti}fundacionalismo.


\section{Discurso, hegemonía y el retorno de lo político}

A partir de esta irrupción del lenguaje en las ciencias sociales, fue generalizándose la idea de que \emph{discurso} hace referencia \enquote{a un punto de vista desde el cual era posible redescribir la totalidad de la vida social}. \footcite[][3]{@7000-LACLAU2004} Si el lenguaje es un sistema de diferencias donde la identidad de cada elemento es puramente relacional y, consecuentemente, todo acto de significación implica a la totalidad del lenguaje, entonces se puede pensar que dicho carácter relacional no es exclusivo del lenguaje y, por tanto, caracteriza a todas las estructuras significativas, es decir, a todas las estructuras sociales. Obviamente, esto \enquote{no significa que todo sea lenguaje en el sentido restrictivo de habla o escritura sino, más bien, que la estructura relacional o diferencial del lenguaje caracteriza a todas las estructuras significativas}. \footcite[][124]{@6999-LACLAU1990}

Desde este punto de vista, el discurso consistirá en una construcción social y política,\footnote{Desde este punto de vista, el pensamiento político postestructuralista se enmarcaría dentro del \emph{constructivismo social}, a la vez que expresa su oposición a cualquier forma de \emph{idealismo}.  \cite[Véase][]{@7002-GROPPO2009}. El propio Lacan, a quien muchos han acusado despectivamente de \emph{sofista}, asume dicho supuesto ontológico. Para un desarrollo de este argumento y su llamativa coincidencia con otra postura teórica aparentemente distante como la de Cornelius Castoriadis. \cite[Véase][]{@7003-STAVRAKAKIS2010}.} en una totalidad parcial y precariamente estructurada, donde la identidad de cada elemento es relacional, pero no por ello es arbitraria, sino que dicho proceso transcurre sobre un determinado orden de relaciones sistemáticas que definen los valores diferenciales y no a partir de una necesidad exterior al sistema que las estructura.

Un rasgo fundamental de esta noción de discurso emerge tras descartar la imposibilidad lógica de toda totalidad: el carácter constitutivamente abierto de toda formación discursiva. Estamos frente a un límite de la lógica diferencial y relacional saussureana que permite la constitución de toda identidad. Como dicha totalidad no puede presentarse nunca bajo la forma de una positividad simplemente dada y delimitada, la lógica relacional entre los elementos que la conforman va a ser constitutivamente incompleta y estará atravesada por la contingencia. Si asumimos la imposibilidad de cierre de toda formación discursiva, existe por lo tanto, un \emph{exterior} a ella que subvierte toda identidad e imposibilita toda \emph{sutura}\footnote{Para un desarrollo de este concepto lacaniano, \cite[][]{@7005-MILLER2008}.} definitiva. La existencia de ese exterior \rdm{que no significa la introducción de un elemento extradiscursivo} es la condición de posibilidad de todo discurso porque le permite fijar parcialmente las identidades; pero a la vez, es su condición de imposibilidad, al \emph{subvertir} toda identidad e imposibilitar cualquier cierre definitivo. Así, la \enquote{tensión irresoluble interioridad/exterioridad es la condición de toda práctica social}. \footcite[][151]{@7025-LACLAU2006}

De este modo, desde la perspectiva \emph{posfundacionalista} que sostenemos, la significación de un objeto o práctica social, la constitución de su identidad, es un proceso de fijación parcial atravesado por la lógica relacional y diferencial (y no por \emph{el referente} ni por substancia alguna). La fijación \emph{es} parcial, afirmamos. \emph{Fijación}, en cuanto la significación consiste en una estabilización entre significante y significado; y \emph{parcial}, ya que dicha fijación es siempre precaria y contingente producto de la indecidibilidad constitutiva de sus fronteras: es así, pero \emph{siempre} pudo haber sido de otro modo; y es así, pero \emph{siempre} podrá ser de otro modo en el futuro.\footnote{Nótese que estamos afirmando que la \emph{historicidad} de toda fijación es intrínseca al concepto mismo. Se trata de una historicidad \emph{interna}. Véase \cite[][]{@7028-BISET2010}. Este es el sentido estricto de nuestra noción de contingencia. No negamos una fijación definitiva de un concepto solamente porque este siempre se encuentre situado históricamente, porque cambie su sentido en función del contexto. De lo que se trata no es de un \emph{exceso de sentido} sino por el contrario, como indicamos con nuestra lectura de Derrida, se trata de la existencia de una \emph{falta}, de una \emph{ausencia} que impide toda fijación definitiva. Esto mismo, en el campo de la Historia de las Ideas, se conoce como la \emph{tesis de la esencial refutabilidad} de los conceptos (políticos), la que reza la imposibilidad de un fijar definitivo (intrínseco carácter aporético) a la vez que afirma que la posibilidad de dicha fijación siempre se da en relación con una determinada comunidad. \cite[Véase][245-253]{@7026-PALTI2007}} Asimismo, la propia superficie de inscripción sobre la que se produce dicho proceso de significación de identidades particulares también se configura discursivamente y por lo tanto, ella misma está atravesada por una indecidibilidad de sus fronteras.

De la aplicación literal del formalismo se deriva lógicamente otra característica de nuestra noción de \emph{discurso}: el \emph{discurso}, en tanto totalidad significativa, trasciende la separación entre lo lingüístico y lo no lingüístico. De hecho, a partir de la crítica postestructuralista, se debe descartar toda diferenciación sustancial entre estos campos. De modo que el \emph{discurso} no se reduce a una mera combinación de habla y escritura, sino que, por el contrario, habla y escritura son tan solo algunos de sus componentes. En definitiva, el término \emph{discurso} permite resaltar el hecho de que \enquote{toda configuración social es una configuración significativa}. \footcite[][114]{@7026-PALTI2007} Consecuentemente, se rechaza también cualquier distinción entre \emph{prácticas discursivas} y \emph{no discursivas}. Si todo objeto se conforma como objeto discursivo, en la medida que no hay objeto que se dé por fuera de toda superficie discursiva de emergencia, toda diferenciación entre lo que habitualmente se denominan aspectos lingüísticos y prácticos de una práctica social o bien es considerada incorrecta o bien se da bajo la forma de diferenciaciones internas a la producción social de sentido, estructurada en forma de totalidad discursiva. Asimismo, distinguir dos actos, por ejemplo una enunciación de una acción, según la oposición lingüístico/extralingüístico no agota su realidad, porque a pesar de que efectivamente sean diferentes en estos términos, ambos comparten algo \rdm{su pertenencia a un discurso} que es justamente lo que permite su comparación. Por lo tanto, si pertenecen a una misma totalidad, esta es anterior a la distinción lingüístico/extralingüístico.

Finalmente, es pertinente marcar otra cuestión fundamental que apunta hacia la misma dirección. Asumir una aproximación discursiva, que supone que todo objeto se constituye como objeto de discurso, no tiene relación alguna con la problemática en torno a la existencia de un mundo exterior al pensamiento, ni con la diferenciación entre \emph{realismo} e \emph{idealismo}. Afirmar el carácter discursivo de todo objeto \enquote{no implica en absoluto poner su \emph{existencia} en cuestión}. \footcite[][115]{@6999-LACLAU1990} De hecho, la existencia de los objetos es tan independiente de su significación que, por ejemplo, esta es el punto de~partida de los enfoques predominantes de la ciencia política \linebreak  (y de los de las ciencias sociales en general). Pero si todos los objetos adquieren su \emph{ser} gracias al discurso, entonces ¿no hay objetividad o realidad anterior al discurso? ¿Aquellos objetos sobre los que no se habla, no se escribe o no se piensa, simplemente no existen?\footnote{Al respecto es interesante el siguiente pasaje de la crítica de Boron a Laclau: \enquote{Encerrado en sus propias premisas epistemológicas, la única escapatoria que le queda a Laclau para dar cuenta del carácter contradictorio de lo real \rdm{que estalla ante sus propios ojos} es postular que las contradicciones de la sociedad son meramente discursivas y que no están ancladas en la naturaleza objetiva (algo que no debe confundirse con el \enquote{objetivismo}) de las cosas. Conclusión interesante, si bien un tanto conservadora: las contradicciones del capitalismo se convierten, mediante la prestidigitación \enquote{posmarxista}, en simples problemas semánticos. Los fundamentos estructurales del conflicto social se volatilizan en la envolvente melodía del discurso, y de paso, en estos desdichados tiempos neoliberales, el \linebreak capitalismo se legitima ante sus víctimas, pues sus contradicciones solo serían tales en la medida en que existan discursos que lacanianamente las hablen. La lucha de clases se convierte en un deplorable malentendido. No hay razones valederas que la justifiquen: ¡todo se reduce a un simple problema~de comunicación! (\ldots) El mundo exterior y objetivo se constituye a partir de su transformación en objeto de un discurso lógico que le infunde su soplo vital y que, de paso, devora y disuelve la conflictividad de lo real. La explotación capitalista ya no es resultado de la ley del valor y de la extracción de la plusvalía, sino que solo se configura si el obrero la puede representar discursivamente} \cite[][]{@7027-BORON1996}. La imposibilidad (o negación) de comprender el carácter material del discurso por parte de Boron es tan evidente que no admite ningún comentario más.} De lo que se trata es de no incurrir en una confusión básica \enquote{entre el ser (\emph{esse}) de un objeto, que es histórico y cambiante, y la entidad (\emph{ens}) de tal objeto, que no lo es}. \footcite[][117-11]{@6999-LACLAU1990} El \emph{ser} de una cosa solo puede encontrarse en relación con una determinada configuración discursiva (o \enquote{juego de lenguaje}), fuera de ella, los objetos solo tienen \emph{existencia}, no un \emph{ser}.\footnote{Un ejemplo interesante es el que ofrece el propio Geras. Afirmar que un terremoto es producto de la ira de Dios es una superstición característica de las sociedades \enquote{primitivas}, mientras que expresar que es un fenómeno natural, simplemente es decir lo que \enquote{realmente es}. Como destaca el propio Laclau, resulta evidente la tendencia autoritaria en dicha ejemplificación. Por supuesto que en ciertos contextos discursivos \rdm{como el que caracteriza a nuestra sociedad} resulta correcto calificar de \enquote{supersticioso} aquella creencia. Sin embargo, otra es la cuestión al contraponerlo a lo que \enquote{realmente es}, porque estamos negando la posibilidad de que nuestra visión pueda cambiar, que nuestra idea actual de lo natural en un futuro resulte insuficiente o errónea; y al mismo tiempo, estamos afirmando que ahora tenemos un acceso transparente, verdadero y no mediado por ninguna teoría, al fenómeno \enquote{terremoto}.}

Ahora bien, esto nos sitúa frente a un clásico problema acerca de la necesidad de distinguir entre significado y acción. A partir de la filosofía analítica wittgensteiniana, la separación entre \emph{semántica} \rdm{que se ocupa del significado de las palabras} y \emph{pragmática} \rdm{que se refiere a cómo se usa una palabra en diferentes contextos} tiende a ser cada vez menos nítida, hasta el punto de que se asume que el significado de un término depende por completo del contexto en el cual se lo emplea, ya que es el uso el que ayuda a determinar su sentido. Esto lleva a sostener que la semántica de un término depende directamente de su pragmática, y dicha separación no es más que una mera distinción analítica. En definitiva, toda identidad se configura en el contexto de una acción. El carácter performativo del lenguaje y la difuminación de la separación entre semántica y pragmática, entre significado y uso, nos habilita a rechazar lo que algunos han denominado como el carácter \emph{mental} del discurso. Y a la vez, sostener que todo discurso tiene un carácter \emph{material}, porque de lo contrario, volveríamos a una clásica dicotomía entre la existencia de un espacio objetivo conformado al margen de todo discurso y un discurso consistente en la mera expresión del pensamiento. De esta manera, afirmar que el mundo objetivo se estructura en secuencias relacionales no supone una visión teleológica e incluso tampoco supone un sentido precisable. Consecuentemente, \enquote{la materialidad del discurso no puede encontrar el momento de su unidad en la experiencia o la conciencia de un sujeto fundante} \footcite[][148]{@7025-LACLAU2006} como predicaba el marxismo clásico, que depositaba en la clase obrera la posibilidad de superación revolucionaria del orden capitalista, porque el discurso tiene una existencia objetiva y no subjetiva. En este mismo sentido, la articulación de los \emph{elementos} de un discurso \rdm{operación que los transforma en \emph{momentos} de una totalidad (imposible)} tampoco puede consistir en puros fenómenos lingüísticos, \enquote{sino que debe atravesar el espesor material de instituciones, rituales, prácticas}. \footcite[][148]{@7025-LACLAU2006}

Consecuentemente, la noción laclauniana de discurso que tomamos se hace eco de, por un lado, los desarrollos de la lingüística estructural desde Saussure en adelante al asumir la identidad como reverso de la diferencia, al \emph{desesencializar} todo proceso de significación y al reconocer un continuo desplazamiento del significante. Y por el otro, incorpora las principales críticas a la metafísica de la presencia al poner en cuestión cualquier idea de totalidad debido a la indecidibilidad constitutiva de sus fronteras, al negar la existencia de un fundamento último y al asumir el carácter material del discurso tras sostener que todo \emph{ser} se configura discursivamente. Semejante desplazamiento ontológico no debe hacernos caer en el error del relativismo, ya que no arroja como resultante un borramiento de los fundamentos, lo que supondría una nueva forma de \emph{anti}fundacionalismo, sino que, por el contrario, motiva el debilitamiento del estatus ontológico de dichas categorías. Como aseveráramos más arriba, a la vez que se afirma la imposibilidad de un fundar definitivo, se sostiene la \emph{necesariedad} del fundamento (precario, fallido y contingente).

Consecuentemente, una implicancia central del posfundacionalismo es la desestabilización de la frontera entre lo social\footnote{Frente al rechazo de la idea de sociedad como totalidad transparente e idéntica consigo misma, se emplea la noción de \emph{lo social} para dar cuenta de la sustantiva diferencia ontológica, de su radical contingencia y de la indecidibilidad de sus fronteras. En palabras del propio Laclau: \enquote{Frente a esta visión esencialista, hoy en día tendemos a aceptar la \emph{infinitud de lo social}, es decir, el hecho de que todo sistema estructural es limitado, que está siempre rodeado por un \enquote{exceso de sentido} que él es incapaz de dominar y que, en consecuencia, la \enquote{sociedad} como objeto unitario e inteligible que funda sus procesos parciales, es una imposibilidad}. \cite[][104]{@6999-LACLAU1990}.} y lo político que el pensamiento occidental decimonónico había trazado rígidamente. Para llevar adelante dicha empresa, la teoría política posfundacionalista acusa, por un lado, a las ciencias sociales de haber disuelto lo político en lo social\footnote{La Ciencia Política, en su proceso de autonomización e institucionalización disciplinar, asumió acríticamente esta estricta escisión entre lo político y lo social. Un ejemplo aún vigente de esto es la delimitación del análisis político (en tanto área central de la Ciencia Política orientada a la investigación empírica) a un conjunto de objetos y prácticas que conforman un subsistema de lo social.} y, por el otro, propone un desplazamiento en una dirección diametralmente opuesta, desarrollando una ontología \emph{política} de lo social. La incorporación de la distinción entre \emph{la} política y \emph{lo} político \rdm{la política como subsistema social y lo político en tanto momento instituyente y fundante de lo social} habilita afirmar el carácter eminentemente político de toda identidad social.

Estamos, en definitiva, en condiciones de pensar la política desde un lugar diferente. La política, en su sentido más \emph{radical}, encuentra aquí su condición de posibilidad y de imposibilidad al mismo tiempo. Se produce, en definitiva, el \emph{retorno de lo político}. Lo político puede ser pensado, entonces, a partir de la misma imposibilidad de toda totalidad y de la consecuente imposibilidad de toda fijación de sentido definitiva. Si el estructuralismo habilitó a pensar relacional y diferencialmente toda identidad y la crítica postestructuralista dio cuenta de la \emph{falta} de la estructuralidad de la estructura al mismo tiempo que de su \emph{necesariedad}, lo político puede comprenderse como el momento de \emph{sutura}, de cierre (precario) imprescindible para detener el infinito juego de las diferencias que de otro modo sucedería (ya que el sistema no podría determinar si un objeto cualquiera le pertenece o no en la medida en que siempre sería otra diferencia más). Lo político, entonces, reside en la delimitación de aquello que habitualmente denominamos \emph{sociedad}, consiste en la fijación de los límites de la totalidad, con el fin de restringir el juego de las diferencias hacia su interior y así hacer posible la significación. Pero, afirmamos, condición de posibilidad y de imposibilidad a la vez. Lo político, como movimiento instituyente, también es una operación cuya plenitud deviene en imposible en la medida en que siempre se tratará de un fundar precario y, por tanto, fallido; poniendo en evidencia la radical contingencia de toda fijación de sentido, incluido el propio concepto de política (en su acepción agonal tradicional).

En esta dirección, la lectura laclauniana de la noción de \emph{hegemonía} nos permite dar cuenta de esta forma de entender lo político. Los discursos se constituyen en hegemónicos cuando logran posicionar una fuerza social particular como la representante de una totalidad que, debido a su radical inconmensurabilidad, es imposible; de modo que la práctica hegemónica brinda ese punto de \emph{sutura} que le permite al sistema constituirse como tal. Por lo tanto, la operación política por excelencia es aquella que \emph{tiende} a \emph{borrar} su propia intervención invisibilizando y \emph{sedimentando} los actos de su institución originaria, pero dejando siempre una \emph{huella}, un rastro. En otras palabras, existe hegemonía cuando lo particular se presenta como universal, pero no entendiendo el universalismo y el particularismo como dos nociones opuestas sino como \enquote{dos posiciones diferentes (\enquote{universalizante} y \enquote{particularizante}) que~dan forma a una totalidad articulante hegemónica}. \footcite[][301]{@7029-LACLAU2000} De lo~que se trata para entender la política, es cómo un elemento dentro del sistema asume ese rol de \enquote{significante trascendente} \rdm{el \emph{points de~capiton} lacaniano} a partir del cual toma sentido toda la cadena~de significación.


\section{Significación y hegemonía}

En el modo en que lo hemos expuesto, las diferentes aproximaciones a los procesos de significación conforman una trayectoria signada por la progresiva desubstanciación de todo proceso social de creación de sentido. En este devenir, hemos situado sus inicios en las reflexiones de Saussure en torno al lenguaje y destacamos un punto de inflexión central en la crítica de Derrida a la idea de totalidad.

Saussure inició el proceso de anulación del referente y la consecuente concepción de la identidad como reverso de la diferencia; poniendo en el centro de la escena a las \emph{relaciones} por sobre los \emph{elementos} en el proceso significativo. Si lo relevante para comprender una identidad particular son las relaciones que esta establece con el resto de las identidades, entonces la pertenencia a una totalidad (a una estructura, a un sistema) es condición de posibilidad de toda identidad. Y el corolario de esto es que dicha totalidad solo es aprehensible a partir de sus efectos sobre los elementos particulares que la conforman. La intervención derridiana en este nuevo modo de pensar la significación supuso, sin dudas, un giro fundamental. Su crítica a la estructuralidad de la estructura nos sitúa frente a una cuestión importante a la hora de pensar cualquier proceso de creación de sentido. Producto de la indecidibilidad de las fronteras de cualquier sistema, todo significado se encuentra constitutivamente dislocado, cuestionado por algo exterior a él que lo niega, pero en ese mismo movimiento le permite \emph{ser}. Es sumamente importante insistir con el argumento de que dicha indecidibilidad es constitutiva y que, por tanto, no se debe a razones ónticas, en el sentido de que toda totalidad siempre estará rodeada por un \emph{exceso de sentido} que la pondrá en tensión. En otras palabras, no se trata~de un problema empírico sino que, por el contrario, es la existencia de una \emph{falta}, la ausencia de un centro que detenga y fundamente el \emph{juego} lo que hace posible la pluralización de fundamentos.

Solamente en el contexto de este recorrido es que podemos comprender, en primer lugar, el tan trillado \emph{dictum} acerca de que \enquote{todo es discurso}; o más bien, que todo \emph{ser} se configura discursivamente. Por lo tanto, si toda identidad se configura relacionalmente no habiendo sustancia ni esencia alguna que ponga límite al juego diferencial, entonces se presenta como fundamental pensar cómo se relacionan los sentidos (constitutivamente abiertos y dislocados). Y en la medida en que el \emph{ser} de todo objeto o práctica social se construye en su interacción con otros, el modo en que concibamos dicho relacionamiento adquiere lógicamente un estatus ontológico. La cuestión sería ciertamente distinta desde el punto de vista de una epistemología, ya que toda epistemología se sostiene en una escisión entre sujeto (cognoscente) y objeto (cognoscible) suponiendo que el \emph{sentido} del objeto de algún modo guarda cierta independencia con el sujeto; aunque dicha distancia respecto del objeto devenga en inaprensible producto de las limitaciones cognitivas del sujeto, vinculadas las más de las veces a la \emph{mediación} del lenguaje. Por el contrario, desde una mirada ontológica la frontera entre el sujeto y el objeto tiende a disolverse, y el lenguaje nunca puede ser pensado como algo que se interpone en el camino, sino como la condición de posibilidad de todo \emph{ser} a partir del reconocimiento de su carácter material y performativo.

Y en segundo término, desde este lugar es que adquiere relevancia la \emph{diferencia} política y se presenta como pertinente pensar lo político como el momento hegemónico propiamente dicho. La operación hegemónica, en los términos que la hemos planteado siguiendo la propuesta teórica de Laclau, se presenta como el movimiento más relevante a la hora de comprender las configuraciones significativas particulares, en la medida en que se erige como la condición de posibilidad de la propia formación política que actúa como superficie de inscripción de dicha singularidad. En este sentido, aproximarse a la significación supone dar cuenta de la imposibilidad de lo social (o, más bien, de toda totalidad). Como afirmamos anteriormente, la estructuralidad de la estructura tiene la forma de una \emph{falta}, pero al mismo tiempo requiere de un centro para detener el juego de las diferencias que guía toda significación. Asimismo, también dijimos que dicho punto de \emph{sutura} lógicamente solo puede ser llevado a cabo por una diferencia interior al sistema, lo que muestra su rol vicario al adjudicarse una universalidad imposible. Y este es el momento de \emph{lo político}, el momento de la presencia del fundamento (tan necesario como imposible) precario y contingente; pero al mismo tiempo, el momento de su retiro,~de su \emph{sedimentación}, de su propio borramiento, del ocultamiento de~los actos de su institución originaria. El momento, en definitiva, de \emph{lo político} como \emph{hegemonía}. Bajo este razonamiento encuentra su estricto sentido otro lugar común del pensamiento político posfundacional acerca de que \enquote{todo es político}.

Ahora bien, ¿a través de qué concepto podemos pensar la relación entre la operación hegemónica y las formaciones particulares? Más aún, si afirmamos desde Saussure en adelante que la estructura no puede observar directamente ¿a través de qué concepto podemos dar cuenta de su existencia? ¿Cómo podemos mostrar sus efectos sobre las formaciones particulares? En definitiva, si la sutura hegemónica es la condición de posibilidad de todas las identidades hacia el interior de una formación social determinada ¿de qué modo aquella se relaciona con estas? ¿Cómo pensar la relación entre objetos y prácticas sociales, entre sentidos, si sostenemos su apertura constitutiva, su continuo desplazamiento y la imposibilidad de una estabilización definitiva? ¿Cómo pensar estas relaciones cuando siempre son estas las que están mostrando la ausencia de esencia o substancia en los elementos? O al revés ¿cómo pensar los elementos cuando la identidad de estos depende \enquote{absolutamente} de las relaciones que estos puedan establecer?

\section{Sobredeterminación}

Estos interrogantes nos sitúan sin escalas en el objetivo principal de este artículo. Estos interrogantes nos exigen avanzar hacia una nueva ontología política. Una ontología que nos permita comprender el modo en que relacionan sentidos constitutivamente dislocados con una relativa estructuralidad hegemónicamente configurada que define el contexto donde dicho proceso significativo se lleva a cabo. Se trata, por tanto, de pensar en una ontología de la sobredeterminación que nos posibilite dar cuenta de esta relativa estructuralidad, del momento político por excelencia que atraviesa todo acto de creación/fijación de sentido.

Desde este lugar, una ontología de la sobredeterminación tiene implicancias de suma importancia para el análisis político por lo menos en dos direcciones principales. En primer término, una ontología de la sobredeterminación pondrá en tensión la frontera que la ciencia política canónica tradicionalmente ha impuesto al análisis político al circunscribirlo a un conjunto más o menos delimitable y delimitado de objetos y prácticas denominadas como \emph{políticas}. Si la lógica que orienta los procesos de significación \rdm{en los términos en que la hemos planteado} atraviesa todas las identidades, por tanto no hay \emph{a priori} un campo predefinido de objetos y prácticas \emph{políticas}. Justamente, se niega la existencia de una especificidad o una substancia de algunos sentidos que permita identificarlos como \emph{políticos} y, por el contrario, se busca sostener, desde una ontología política de lo social, que todo proceso de significación es \emph{político}. En este sentido, una ontología de la sobredeterminación nos permite llevar hasta las últimas consecuencias una ontología política de lo social en la medida en que pone en tensión toda taxonomía de los objetos y prácticas sociales. En segundo término, y como deriva de lo anterior, una ontología de la sobredeterminación da un paso más en la afirmación de un nuevo campo de intervención para el análisis político: el del análisis ideológico. Desde este punto de vista, la intervención politológica pondrá en evidencia dos cuestiones centrales. Por un lado, el análisis político va a consistir en escrutar las operaciones ideológicas que contaminan una identidad particular y que están invisibilizadas producto de la sedimentación; y por el otro, al mostrar el modo en que la relativa estructuralidad está presente a partir de sus efectos sobre las configuraciones particulares, el análisis político dará cuenta de la operación política-ideológica por excelencia: la hegemonía.

Como es sabido, la noción de la sobredeterminación fue introducida por Sigmund Freud en sus investigaciones sobre la formación de los sueños. Posteriormente, Louis Althusser la trasladó al campo~de la teoría marxista para (re)pensar la especificad de la dialéctica marxista en relación con la hegeliana y así plantear una novedosa respuesta a los problemas que atravesaron las discusiones en el seno del marxismo desde las últimas décadas del siglo \textsc{xix}. Desde allí, el pensamiento político postestructuralista ha incorporado la categoría de la sobredeterminación tanto en sus desarrollos teóricos como en muchas de sus intervenciones empíricas. Sin embargo, notablemente se ausenta una discusión centrada en la propia categoría,\footnote{En otro lugar, hemos intentado mostrar cómo en una serie de investigaciones empíricas realizadas en nuestro entorno académico se ausenta tanto una discusión en torno a dicha categoría, así como también se hace un uso de la lógica de la sobredeterminación muchas veces de manera implícita. \cite[Véase][]{@7030-DAIN2007}.} cuestión obviamente ineludible si pretendemos avanzar en una ontología orientada al análisis político que dé cuenta del modo en que se relacionan y fijan los sentidos. En esta dirección, en el presente apartado presentaremos brevemente el uso de dicha noción en el contexto de su emergencia en el campo de la teoría psicoanalítica y en su extensión posterior al conjunto de las ciencias sociales de la mano de Althusser. Así procuraremos mostrar cómo la sobredeterminación, en tanto lógica subyacente que orienta todo proceso de significación, puede erigirse como una ontología que nos habilite expandir y redefinir el campo de intervención politológica y dar cuenta empíricamente del momento político fundamental (la operación hegemónica) a partir del estudio de identidades particulares, sean o no reconocidas como \emph{políticas}~por las parcelizaciones teóricas llevadas a cabo por las gramáticas tradicionales.


\section{Emergencia: Sigmund Freud}

La sobredeterminación es un concepto surgido en el contexto del psicoanálisis que permite evidenciar la compleja relación existente entre el inconsciente y ciertos mecanismos psíquicos tales como las elaboraciones oníricas, los chistes y los actos fallidos. Particularmente, Freud incorporada la sobredeterminación en sus estudios sobre la interpretación de los sueños. En estos trabajos, Freud concluye en que el proceso de elaboración de los sueños radica en la transformación del sueño \emph{latente} en sueño \emph{manifiesto} y la tarea de \emph{interpretación} del psicoanálisis consiste justamente en la operación contraria; esto es, en llegar desde el contenido manifiesto del sueño a las ideas latentes, desanudando la compleja trama forjada por la elaboración. Básicamente, la elaboración del sueño es un proceso de \emph{deformación onírica} que supone el tránsito de lo \emph{latente} a lo \emph{manifiesto}, y se produce a través de dos mecanismos fundamentales: la condensación y el desplazamiento.

La \emph{condensación} da cuenta del hecho de que el contenido manifiesto del sueño es más breve que el latente, constituyendo \enquote{una especie de traducción abreviada del mismo}. \footcite[][184]{@7032-FREUD1969} Esta operación se desarrolla a través de tres procedimientos fundamentales. En primer término, la tarea de condensación supone la \emph{exclusión} de determinados elementos latentes que sencillamente desaparecen como producto de la elaboración onírica. Asimismo la condensación también implica la \emph{fragmentación}, procedimiento mediante el cual se toman pedazos de elementos complejos del latente. Y, finalmente, tenemos el trabajo de condensación propiamente dicho que se realiza a través de la \emph{fusión} de elementos latentes a partir de ciertos rasgos comunes.

El resultado de la condensación son los \emph{puntos nodales}, a través de los cuales se reúnen toda una serie de pensamientos oníricos. Los puntos nodales tienen la característica de ser \emph{multívocos} con referencia a la interpretación del sueño, ya que \enquote{cada uno de los elementos del contenido del sueño aparece como \emph{sobredeterminado}, como siendo el subrogado de múltiples pensamientos oníricos}. \footcite[][291]{@7031-FREUD2008} Pero no solamente los elementos del sueño están sobredeterminados por los pensamientos oníricos, sino que los pensamientos oníricos particulares también están sobredeterminados en el sueño por varios elementos. De este modo, a partir del trabajo de interpretación, se puede llegar a vincular un elemento del sueño con varios pensamientos oníricos, y también puede suceder que un pensamiento onírico concreto esté relacionado con varios elementos del sueño. Por lo tanto, la elaboración del sueño es un proceso complejo que nos impide relacionar cada pensamiento onírico singular o cada grupo de ellos como si se tratara de una abreviación presente en el contenido del sueño. Es interesante aquí la comparación que establece el propio Freud para mostrar esto último: el proceso de elaboración onírica no puede ser pensado \enquote{a semejanza de un electorado que designa un diputado por distrito, sino que toda la masa de pensamientos oníricos es sometida a una cierta elaboración después de la cual los elementos que tienen más y mejores apoyos son seleccionados para ingresar en el contenido onírico, valga como analogía la elección por listas}. \footcite[][292]{@7031-FREUD2008} De esta forma, los elementos del sueño se configuran a partir de un conjunto de pensamientos oníricos, y cada uno de ellos aparece sobredeterminado por referencia a los pensamientos oníricos. En otras palabras, podríamos afirmar que el contenido \emph{manifiesto} del sueño aparece sobredeterminado por el contenido \emph{latente}.

El otro mecanismo destacado por Freud es el \emph{desplazamiento}, el cual arroja como resultante que aquello central en los pensamientos oníricos pierda su protagonismo o ni siquiera esté presente en el sueño. Esta operación se manifiesta fundamentalmente de dos maneras. Por un lado, haciendo que un elemento latente sea reemplazado por \enquote{algo más lejano a él; esto es, por una alusión}. \footcite[][187]{@7032-FREUD1969} Desde este punto de vista, el chiste también supone una operación de desplazamiento, pero a diferencia del sueño, debe ser fácilmente reconocible el objeto al que alude, lo cual se presenta como condición de inteligibilidad del chiste ya que de lo contrario no causaría gracia. La alusión del desplazamiento onírico no tiene esta limitación que presentan los chistes ya que ofrece \enquote{relaciones por completo exteriores y muy lejanas con el elemento que reemplaza, y resulta de este modo ininteligible, mostrándosenos, en su interpretación, como un chiste fracasado y traído por los cabellos}. \footcite[][187]{@7032-FREUD1969} El otro modo de expresión del desplazamiento es a través del descentramiento, \enquote{motivando que el acento psíquico quede transferido de un elemento importante a otro que lo es menos, de manera que el sueño recibe un diferente centro y adquiere un aspecto que nos desorienta}. \footcite[][187]{@7032-FREUD1969} Se trata, por tanto, de una traslación que implica un descentramiento. De esta forma, el sueño está de algún modo \enquote{\emph{diversamente centrado}, y su contenido se ordena en torno de un centro constituido por otros elementos que los pensamientos oníricos}. \footcite[][311]{@7031-FREUD2008} Por lo tanto, en el proceso de elaboración del sueño se exterioriza un poder psíquico que le sustrae intensidad a ciertos elementos latentes, a la vez que brinda nuevas valencias \enquote{\emph{por la vía de la sobredeterminación}} \footcite[][303]{@7031-FREUD2008} a otros, haciendo de esta manera, que estos alcancen el contenido onírico. En la elaboración de los sueños, entonces, acontece \enquote{\emph{una transferencia y un desplazamiento de las intensidades psíquicas} de los elementos singulares, de lo cual deriva la diferencia de texto entre contenido y pensamientos oníricos}. \footcite[][292]{@7031-FREUD2008} La operación de desplazamiento arroja como resultante que el contenido del sueño exprese un aspecto diferente al núcleo de los pensamientos oníricos al tiempo que \enquote{el sueño solo devuelve \{refleja\} una desfiguración \{dislocación\} del deseo onírico del inconsciente}. \footcite[][314]{@7031-FREUD2008}

De este modo, estamos ya en condiciones de afirmar que la sobredeterminación no es una mera forma de evidenciar la multicausalidad ni la pluralidad. No se trata de una noción que dé cuenta de algún tipo de cadena causal o de una determinación a partir de la superposición de diferentes influencias. Por el contrario, se trata de un modo de pensar las relaciones que trasciende ampliamente la lógica causalística que orienta ontológica y epistemológicamente a la enorme mayoría de los paradigmas hegemónicos de las ciencias sociales contemporáneas. Y como está claro, en el contexto de una ontología antiesencialista de la significación en los términos que el pensamiento político posfundacional ha pensado la cuestión, la manera en que concebimos las relaciones entre sentidos es fundamental. Si rechazamos, desde Saussure en adelante, cualquier vínculo esencial entre significante y significado y proclamamos la arbitrariedad de dicho vínculo a la vez que sostenemos, consecuentemente, el carácter relacional de toda identidad, entonces se presenta de manera imperiosa la necesidad de avanzar en una ontología de las relaciones entre objetos y prácticas configuradas discursivamente. En este contexto debe nuestra propuesta de una ontología de la sobredeterminación.


\section{Protagonismo: Louis Althusser}

Louis Althusser, a partir de sus lecturas del psicoanálisis freudiano y lacaniano,\footnote{cf. \cite[][171 nota 46]{@7051-ALTHUSSER1965} y del mismo autor,\cite[][17-48]{@7033-ALTHUSSER1993}.} introduce la noción de sobredeterminación al campo intelectual marxista. Althusser presenta la sobredeterminación no solo como un término que permite repensar la propuesta marxista en su relación con Hegel, ni tampoco simplemente para dar cuenta de la \emph{ruptura epistemológica} de Marx. La sobredeterminación es el \emph{suplemento}\footcite[][]{@7034-DAIN2011}
que permite dar cuenta de la especificidad de la causalidad estructural, habilitando un nuevo modo de pensar todo proceso de significación social. Plantea una nueva forma de concebir lo social, una manera de trascender la visión de lo social como un espacio plano donde reina una causalidad mecánica transitiva y, por tanto, donde un efecto determinado puede siempre ser atribuido a una causa objeto.\footcite[][197]{@7051-ALTHUSSER1965}

La intromisión althusseriana de la sobredeterminación se produce en el contexto de los debates marxistas que, desde la II Internacional en adelante, estuvieron atravesados por la preocupación en torno a la delimitación tanto de la intensidad como acerca del modo en que la estructura económica define el conjunto del orden social. Althusser comienza sus indagaciones sobre dicha cuestión sosteniendo que se trata de un problema ya resuelto por la práctica política marxista, siendo la obra y la práctica de Lenin su clara demostración. Lenin, en tanto hombre de acción, opera en un \emph{momento actual}, en una situación política concreta donde la \emph{necesidad histórica} se realiza, y a través de su práctica y su reflexión muestra las características estructurales de la Rusia zarista (estado semifeudal y semicolonial y, sin embargo, imperialista: \emph{el eslabón más débil de la cadena imperialista}), evidenciando las articulaciones esenciales y los eslabones de los que depende la posibilidad y el resultado de toda práctica revolucionaria. En definitiva, el leninismo es un ejemplo de un análisis de una realidad estructural concreta, \enquote{en el \emph{desplazamiento} y las \emph{condensaciones} de sus contradicciones, en su unidad paradójica, que constituyen la existencia misma de ese \enquote{ \emph{momento actual}} que la acción política va a transformar, en el sentido fuerte del término, de un febrero en un octubre 17}. \footcite[][147]{@7051-ALTHUSSER1965} La intervención leninista muestra que la contradicción principal marxista no se presenta empíricamente de manera literal. En realidad, la contradicción real se confunde con el (infinito) conjunto de circunstancias particulares en las que se expresa \enquote{que no es discernible, identificable ni manuable \emph{sino a través de ellas y en ellas}}. \footcite[][79]{@7051-ALTHUSSER1965} Dicha contradicción muestra que la revolución está a la orden del día, pero no puede directamente ni exclusivamente producir efectivamente la revolución. Solamente desde un burdo mecanicismo economicista podría esperarse semejante efecto. Para que pueda activarse la contradicción principal que Marx pudo especificar, debe producirse una acumulación de circunstancias \rdm{cuyos orígenes y sentidos no tienen necesariamente una dirección revolucionaria, pudiendo incluso a ser opuestos a tal fin} que puedan fusionarse en una unidad de ruptura y siendo, a su vez, cada circunstancia tomada por separado la fusión de una acumulación de contradicciones.

De otro modo sino \enquote{¿{[}c{]}ómo es posible (\ldots) que las masas populares, divididas en clases (proletarios, campesinos, pequeños burgueses) puedan, consciente o confusamente, lanzarse al asalto general del régimen existente?}. \footcite[][80]{@7051-ALTHUSSER1965} La respuesta no está en el efecto simple y directo de la contradicción en Capital y Trabajo. Por cierto que esta contradicción general se encuentra activa en cada una de esas contradicciones y hasta en su propia fusión, pero no se puede afirmar que todas esas contradicciones y su fusión sean un puro reflejo, un simple epifenómeno del orden estructural. Tampoco debemos caer en el empirismo o, peor aún, en la irracionalidad del \emph{así es} o del azar. Las condiciones existentes a través de las cuales se expresa la contradicción principal son sus condiciones de existencia, pero lo existente no es un concepto empírico, es, por el contrario, un concepto teórico, sustentado en la esencia misma del \emph{todo complejo siempre-ya-dado}. Este conjunto de condiciones es la existencia del todo en un momento determinado, en un \emph{momento actual}, como dijera Lenin; es decir, la compleja relación de condiciones de existencia recíprocas entre las articulaciones de la~estructura de un todo. Por lo tanto, el problema de la determinación estructural ha sido resuelto por la práctica marxista-leninista y lo que Althusser pretende es enunciar a nivel teórico dicha solución. Si se puede hablar a nivel teórico de las condiciones, escapando al empirismo y a la irracionalidad, es porque \enquote{el marxismo concibe las \enquote{condiciones} como la existencia (real, concreta, actual) de las contradicciones que constituyen el todo de un proceso histórico}. \footcite[][172]{@7051-ALTHUSSER1965} Justamente por esto Lenin no cayó en el empirismo al invocar las \emph{condiciones existentes} de Rusia; por el contrario, analizó~la presencia del todo complejo del Imperialismo en la Rusia zarista, la Rusia del \emph{momento actual}.

El nudo de la cuestión, según Althusser, se encuentra en el problema de la \emph{inversión} marxista de la dialéctica hegeliana. El filósofo argelino propone una nueva lectura acerca de la relación entre Marx y Hegel respecto a la dialéctica, a partir de la convicción de que la propia caracterización de Marx de la \emph{especificidad} de su método dialéctico en términos de \emph{inversión} es visiblemente insuficiente. El problema de la \emph{inversión} de la dialéctica hegeliana refiere a la misma naturaleza de la dialéctica considerada en sí misma y no \rdm{como muchos han malinterpretado} a la naturaleza de los objetos (la Idea, en el caso de Hegel, y el mundo de lo real, en el caso de Marx) a los cuales se trata de aplicar dicho método. En este sentido, la problematización althusseriana de la expresión metafórica de la \emph{inversión} y, consecuentemente, su intento por determinar la especificidad y la naturaleza de la dialéctica marxista, en contraposición a la hegeliana, lleva directamente a repensar el núcleo central de la misma: esto es, el mismo concepto de \emph{contradicción}. Según Althusser, se trata de evidenciar y describir las diferencias en las estructuras fundamentales de la dialéctica (la negación, la negación de la negación, la identidad, etc.) marxista respecto a la hegeliana. En este contexto debe entenderse la intromisión del concepto de sobredeterminación. Será a través de esta operación teórica que Althusser pretende dar cuenta de las determinaciones y de la propia estructura de la dialéctica marxista; en definitiva, de la especificidad de la estructura de la contradicción marxista. La noción de sobredeterminación le permite teorizar a Althusser aquella complejidad que la práctica marxista ya había reconocido.

La clásica enunciación marxista acerca de la determinación estructural expresada en el famoso pasaje de la \emph{Introducción a la crítica de la economía política} demuestra que aquello que en una primera lectura se muestra como una relación simple,  se trata en realidad de una relación compleja, ya que todo modo de producción se refiere siempre a un modo de producción en un determinado estado del desarrollo social, o sea que todo modo de producción se engendra en un todo social estructurado. De modo que Marx está evidenciando que toda categoría simple siempre implica la existencia de un todo estructurado de la sociedad. Y más aún, lejos de ser originaria, la simplicidad es solo consecuencia de un proceso complejo, es producto de un largo proceso histórico y nunca puede situarse como el punto de partida: \enquote{la \enquote{Introducción} no es más que una larga demostración de la siguiente tesis: lo simple no existe jamás sino en una estructura compleja}. \footcite[][162]{@7051-ALTHUSSER1965}

Ciertamente, estamos en las antípodas de la dialéctica hegeliana, en tanto proceso simple de dos opuestos, cuya unidad originaria simple se divide en dos contrarios. Para Hegel, la dialéctica se sostiene en el \enquote{supuesto radical de una unidad originaria simple, desarrollándose en el seno de ella misma por la virtud de la negatividad y no restaurando nunca, en todo su desarrollo, más que esta unidad y esta simplicidad originarias, en una totalidad cada vez más \enquote{concreta}}. \footcite[][163]{@7051-ALTHUSSER1965} No se trata de una \emph{inversión} de este supuesto de una unidad originaria; por el contrario, estamos frente a su simple supresión y frente a su reemplazo por un supuesto teórico totalmente diferente: \enquote{en lugar del mito ideológico de una filosofía del origen y de sus conceptos orgánicos, el marxismo establece en principio el reconocimiento de la existencia de la estructura compleja de todo \enquote{objeto} concreto, estructura que dirige tanto el desarrollo del objeto como el desarrollo de la práctica teórica que produce su conocimiento. No existe una esencia originaria, sino algo siempre-ya-dado}. \footcite[][164]{@7051-ALTHUSSER1965} Pero al mismo tiempo también estamos en las antípodas de la economía política clásica que \enquote{pensaba los fenómenos económicos como dependientes de un espacio plano, donde reinaba una causalidad mecánica transitiva, de tal modo que un efecto determinado podía ser referido a una causa objeto, a otro fenómeno} \footcite[][197]{@7051-ALTHUSSER1965} ya que, frente a esta concepción empirista, Marx contrapone su concepción de una \emph{región} compleja y profunda definida por una estructura. Estamos frente a la destrucción de todas las teorías clásicas de la causalidad; en definitiva, estamos frente a la \emph{ruptura epistemológica} de Marx.

Ahora, entonces, se puede plantear el problema con claridad.  \linebreak Se trata de desarrollar el herramental teórico que permita comprender las formas específicas en que se manifiesta la contradicción principal, cómo esta se encuentra determinada por las formas y~las circunstancias históricas concretas. Si la contradicción entre las~relaciones de producción y las fuerzas productivas está empíricamente especificada por la superestructura y por la situación histórica interna y externa (que a su vez remite al propio pasado nacional e internacional), entonces hay que dar cuenta teóricamente de dicha relación. En palabras de Althusser:


\begin{quote}
	\emph{¿Por medio de qué concepto puede pensarse el tipo de determinación nueva, que acaba de ser identificada como la determinación de los fenómenos de una región dada por la estructura de esta región?}De manera más general: \emph{¿por medio de qué concepto o de qué conjunto de conceptos puede pensarse la determinación de los elementos de una estructura y las relaciones estructurales existentes entre estos elementos y todos los efectos de estas relaciones, por la eficacia de esta estructura?} Y \emph{a fortori}, \emph{¿por medio de qué concepto o de qué conjunto de conceptos puede pensarse la determinación de una estructura subordinada por una estructura dominante? Dicho de otra manera ¿cómo definir el concepto de una causalidad estructural?}.\footcite[][201]{@7051-ALTHUSSER1965}
\end{quote}

La respuesta a tamaños cuestionamientos es elaborada por Althusser a partir de sus lecturas del psicoanálisis freudiano y de los aportes de la lingüística estructuralista desde Saussure en adelante. Pero como él mismo se encarga de aclarar, Marx \emph{practicó} dichas preguntas a partir de su teoría de la historia y de la economía política, pero sin generar el concepto en una filosofía de igual rigor. Marx fue el teórico que tuvo la audacia y la capacidad para plantearse el problema de la determinación de los elementos del todo por la estructura del todo, y lo hizo aún sin contar con ningún concepto filosófico elaborado para responderlo. Antes de plantearlo como problema, más bien Marx \emph{produjo} este problema, ya que se ocupó de resolverlo prácticamente aunque sin disponer del arsenal conceptual apropiado, lo cual lo llevó a recaer en esquemas anteriores, necesariamente inadecuados al planteamiento y a la solución de este problema. Y la noción de la que ahora se dispone gracias al psicoanálisis de Freud y la que permite resolver a nivel teórico el problema de la determinación estructural es la de la sobredeterminación.

Así, pareciera que la excepción se convierte en regla. El (in)finito conjunto de contradicciones y su fusión en principio de ruptura o de inhibición histórica, hacen que la contradicción principal se exprese en la práctica como una contradicción sobredeterminada, lo que a su vez constituye la \emph{especificidad} de la con\-tradicción marxista. Esto es, que no estamos frente a una simple contradicción ni a una simple sobredeterminación, sino frente a una \emph{contradicción sobredeterminada} porque es una sobredeterminación cuyo fundamento es una contradicción; se trata, entonces, de una \enquote{\emph{acumulación de~determinaciones eficaces} (surgidas de las superestructuras y de circunstancias particulares nacionales e internacionales) \emph{sobre la determinación en última instancia por la economía}}. \footcite[][92]{@7051-ALTHUSSER1965}

Pero no debemos confundirnos con la idea de que la sobredeterminación esté basada en situaciones aparentemente singulares y aberrantes de la historia. Por el contrario, es \emph{universal}, en el sentido de que nunca la contradicción principal actúa en \enquote{estado puro} creando superestructuras que posteriormente se separan cuando han realizado su obra: \enquote{ni en el primer instante ni en el último, suena jamás la hora solitaria de la \enquote{última instancia}}. \footcite[][93]{@7051-ALTHUSSER1965} Esto significa ni más ni menos que la superestructura no puede ser pensada como un mero reflejo, como un simple epifenómeno estructural: \enquote{las contradicciones secundarias son necesarias a la existencia misma de la contradicción principal, que constituye realmente su condición de existencia, tanto como la contradicción principal constituye a su vez la condición de existencia de las primeras}. \footcite[][170]{@7051-ALTHUSSER1965} Por tanto, como el propio Marx lo expresara, no existe producción sin sociedad.

Claramente, Althusser orienta su planteamiento teórico hacia la elaboración de una \enquote{\emph{teoría de la eficacia de las superestructuras y otras \enquote{circunstancias}}}. \footcite[][93]{@7051-ALTHUSSER1965} En este sentido, afirmar que la contradicción principal siempre está sobredeterminada significa que su eficacia depende del conjunto de circunstancias sociales en las cuales esa contradicción opera. Evidentemente, esto se aleja de cualquier forma de economicismo, ya que no se puede sostener que~esas contradicciones y su fusión sean su puro fenómeno porque emergen de las relaciones de producción, las cuales son uno de los términos de la contradicción pero, al mismo tiempo, también son su misma condición de existencia. Esto significa que las diferencias que configuran cada una de las instancias en juego, al fusionarse en una unidad real, no se disipan como un puro fenómeno. Al conformar esta unidad\emph, constituyen la unidad fundamental que las alienta, pero al hacerlo también evidencian la naturaleza de dicha unidad. Esto es, que la contradicción principal es, por una parte, inseparable de la estructura del cuerpo social en el que ella actúa, inescindible de las condiciones formales de su existencia; y que, por la otra, la misma contradicción es \emph{afectada} por dichas circunstancias, \enquote{determinante pero también determinada en un solo y mismo movimiento, y determinada por los diversos \emph{niveles} y las diversas \emph{instancias} de la formación social que ella arma; podríamos decir: \emph{sobredeterminada en su principio}}. \footcite[][81]{@7051-ALTHUSSER1965} En definitiva, una contradicción no existe más que en y a través de las circunstancias en las cuales se realiza, por lo que no puede pensarse jamás al margen de sus condiciones de existencia. Específicamente, la contradicción entre Capital y Trabajo nunca es simple ya que siempre está delimitada por las circunstancias históricas concretas en las que se ejerce; definida tanto por la superestructura, como por la situación histórica interna y externa. \enquote{¿Qué queda por decir sino que la contradicción aparentemente simple está \emph{siempre sobredeterminada}?}. \footcite[][86]{@7051-ALTHUSSER1965}

Entonces, la lógica de la sobredeterminación es la que permite dar cuenta del hecho evidente de que cada coyuntura histórica es única a pesar de que las contradicciones presentes sean las mismas. Pero no debemos caer en la confusión de que este reconocimiento del carácter complejo de las contradicciones en toda formación social lleve a Althusser a reemplazar una concepción monista de la historia por una concepción \enquote{pluralista}. Por el contrario, la sobredeterminación permite pensar la articulación compleja entre tales contradicciones. La noción de sobredeterminación es central porque permite \enquote{designar, al mismo tiempo, la ausencia y la presencia, es decir, \emph{la existencia de la estructura en sus efectos}}, \footcite[][203]{@7035-ALTHUSSER1967} permite pensar la contradicción principal en sus propias condiciones de existencia, es decir en su propia inserción en la estructura dominante del todo complejo. La estructura no es algo \emph{exterior} que vendría a modificar el aspecto, la forma y la relación de los objetos: \enquote{\emph{La ausencia de la causa en la \enquote{causalidad metonímica} de la estructura sobre sus efectos no es el resultado de la exterioridad  \linebreak de la estructura en relación con los fenómenos} (\ldots) ; \emph{es al contrario, la forma misma de la interioridad de la estructura como estructura, en sus efectos}}. \footcite[][204]{@7035-ALTHUSSER1967}

Como vemos, no hay lugar para las lecturas mecanicistas puesto que, tanto sea respecto de su sentido como de sus efectos, la contradicción deja de ser unívoca, en la medida en que \enquote{refleja en sí, en su misma esencia, su relación con la estructura desigual del todo complejo}. \footcite[][173]{@7051-ALTHUSSER1965} Pero no significa que ahora sea equívoca, sino que se muestra determinada por la complejidad estructural que le establece su papel. Como señala el propio Althusser, la contradicción marxista es \enquote{compleja-estructural-desigualmente-determinada} \footcite[][174]{@7051-ALTHUSSER1965} o, expresado de manera algo más elegante, diríamos sencillamente: sobredeterminada. Althusser despega a Marx de la lectura economicista porque fue esta la que estableció \enquote{una vez para siempre la jerarquía de las instancias, fijó a cada una su esencia y su papel y definió el sentido unívoco de sus relaciones}. \footcite[][177]{@7051-ALTHUSSER1965}

De este modo, la sobredeterminación da cuenta de la \emph{ruptura epistemológica} de Marx ofreciendo la posibilidad para emprender una crítica a las concepciones clásicas del relacionamiento entre objetos y prácticas sociales en términos de mera determinación, tanto en su sentido mecánico transitivo de origen cartesiano y vinculado actualmente al pensamiento de raíz positivista, como así también en su forma leibniziana de \emph{causalidad expresiva} que domina todo el pensamiento hegeliano. Althusser, al \emph{suplementar} la teoría marxista con la noción de sobredeterminación, va a terminar abriendo la puerta a un nuevo modo pensar cualquier proceso de significación que, por un lado, renuncie a toda escisión ontológica entre un espacio de conformación identitaria y otro de relacionamiento e interacción y, por el otro, permita aproximarse a los procesos de creación de sentido más en términos de contaminación, hibridización e implicancia mutua.


\section{Hacia una Ontología de la Sobredeterminación}

Como hemos procurado poner de relieve, una ontología de la sobredeterminación está orientada a pensar las relaciones entre objetos y prácticas socialmente significativas. Un modo diferente de pensar las relaciones a, por ejemplo, los sistemas mecanicistas. Estos modelos, de origen cartesiano, piensan a lo social como una suerte de espacio plano y tienden a subsumir la causalidad a una eficacia \emph{transitiva}, al modo de bolas de billar que chocan unas con otras. De esta forma, las relaciones entre los elementos se producen en términos de una clara exterioridad, existiendo siempre algo que escapa a la influencia de las relaciones, en la medida en que la (pre)existencia del objeto es condición de posibilidad de su relacionamiento con otros elementos. Desde esta lógica causa-efecto, la causalidad siempre tiene una temporalidad inherente ya que se presupone una necesaria \rdm{y \emph{lógica}} preexistencia de la causa respecto de su efecto. Asimismo, nuestra propuesta ontológica también se diferencia con otro gran sistema de la filosofía moderna específicamente concebido para pensar la influencia del todo sobre sus partes. Nos referimos al modelo cuyas raíces se encuentran en el concepto leibniziano de \emph{expresión} y que ha sido de suma influencia en el pensamiento hegeliano. En términos generales, este presupone que siempre el todo es \enquote{reductible a un principio de interioridad único, es decir, a una esencia interior, de la que los elementos del todo no son entonces más que formas de expresión fenomenales},\footnote{\cite[][197-209]{@7035-ALTHUSSER1967}, específicamente pág.~202.} de tal modo que si cada parte constitutiva de la totalidad es \emph{expresión} de la totalidad entera, entonces la condición absoluta es que el todo no sea una estructura.

Como vimos, el concepto de sobredeterminación es introducido por Freud para pensar las relaciones simbólicas entre las estructuras psíquicas del sujeto y sus expresiones a través de los sueños. Saussure también problematiza el vínculo entre la totalidad y sus partes constitutivas al intentar comprender la relación de la lengua con el signo lingüístico desde una aproximación estrictamente formal llevándolo a la conclusión de que la lengua como sistema de diferencias solo se expresa en cada uno de sus elementos constitutivos. Como hemos procurados mostrar, la cuestión de cómo pensar la relación entre el todo y sus partes también es clave en el marxismo estructuralista de corte althusseriano, donde este problema adquiere la forma de cómo comprender la relación entre el todo-complejo-estructurado y las identidades particulares. Y, al trasladar la categoría de la sobredeterminación al campo del marxismo para ofrecer una nueva respuesta al problema de la determinación estructural, Althusser terminará por habilitar una nueva forma de concebir las relaciones entre totalidad y particularidades socialmente significativas.

Acerca del modo en que cada uno de estos autores pensó la relación del todo con sus partes ya se ha dicho mucho. Por ejemplo, las críticas lacanianas a Saussure que mostraron la preeminencia del significante sobre el significado. Los duros embates desde el campo posfundacionalista contra el marxismo en general, al cuestionarle la propia idea de \emph{contradicción}, y contra Althusser en particular, criticado por la inconsistencia lógica de su propuesta, específicamente sobre la coherencia de su tesis de la determinación en última instancia por parte de la economía con su lectura sobre la condición sobredeterminada de la contradicción principal. \footcite[Por ejemplo,][133-136]{@7025-LACLAU2006}

Sin embargo, seguimos acechados por los mismos espectros. La diferencia es que ahora, después del \emph{acontecimiento} señalado por Derrida, no podemos suponer la estructuralidad como algo inherente a la estructura sino que, por el contrario, es justamente su ausencia la que habilita el \emph{juego} de fundamentos precarios. Si bien seguimos sosteniendo la necesariedad de algún orden del todo, este no puede darse por supuesto. Por esto, desde una ontología de la sobredeterminación, pensamos al todo a partir de su propia imposibilidad y, por tanto, como \emph{parcialmente} estructurado y \emph{hegemónicamente} suturado; y, producto de la indecidibilidad de los límites, asumimos que toda particularidad siempre está constitutivamente \emph{dislocada}. Además ahora debemos preguntarnos sobre la relación del todo con las partes sin poder trazar una frontera nítida entre ambos, haciendo que la relación pierda la temporalidad propia de la lógica causalística tanto en su sentido mecánico-transitivo como \emph{expresivo}. Así, la cuestión del todo y sus partes aparece \emph{reactivada} actualmente de un modo \emph{radicalmente} distinto. Por lo tanto, está claro ya que nuestra idea de estructura poco tiene que ver con la perspectiva marxista, quien la define a partir de la contradicción entre Capital y Trabajo, mientras que desde nuestra mirada antiesencialista y postestructuralista toda estructura está constitutivamente abierta, por lo que no hay ningún contenido ontológico \emph{a priori}.

Desde una ontología de la sobredeterminación, en los términos en que la estamos planteando, el modo de relacionamiento entre la estructura fallida y los elementos dislocados está definido por los mecanismos señalados por Freud en sus estudios sobre la interpretación de los sueños. Es a través de la condensación y el desplazamiento que el todo parcialmente estructurado contamina las identidades, pero en la medida en que dicha estructuralidad solo se expresa a través de sus efectos no podemos dar cuenta de ella sino a partir de su presencia en cada particularidad. Y, tras asumir el abismo entre lo \emph{óntico} y lo \emph{ontológico}, entre el \emph{ser} y el \emph{ente}, y su corolario de que no hay \emph{ser} más allá de su manifestación como \emph{ente}, entonces podemos afirmar que se disuelve la frontera entre contexto y significación de modo que la relativa estructuralidad no es más que su misma capacidad de sobredeterminar identidades particulares.

Esto nos sitúa frente al argumento central de nuestra propuesta. En primer término, lo que hemos procurado plasmar es que nuestra ontología encuentra su primera condición de posibilidad en el marco de una aproximación discursiva de lo social que asuma una visión antiesencialista del lenguaje y, por tanto, de toda estructura socialmente significativa. En otras palabras, la condición de posibilidad de una ontología de la sobredeterminación es la asunción de que toda estructura e identidad social lo es justamente por ser significativa, y dicho significado se configura en el contexto de una superficie de emergencia discursivamente edificada. Desde Saussure y su aproximación formal al estudio de la lengua, el referente ha sido progresivamente anulado para pensar la significación y a partir de allí entendemos la arbitrariedad del vínculo entre significante y significado y su continuo desplazamiento. Por lo tanto, el sentido no depende de ningún tipo de vínculo esencial o natural con la cosa significada, no se puede desprender de su mera \emph{existencia} \rdm{la cual es claramente independiente y anterior a cualquier significación} sino por el contrario, el \emph{ser} de toda práctica u objeto social se conforma discursivamente. No hay, por tanto, ningún tipo de literalidad. Y es justamente en este sentido que afirmamos el carácter \emph{material} de nuestra noción de discurso y, consecuentemente, de nuestra ontología de la sobredeterminación.

En definitiva, el estatus ontológico de la sobredeterminación muestra que todo objeto o práctica social se configura discursivamente. Si los hechos \emph{existiesen} con independencia de su significación social o si su sentido estuviese de algún modo inscripto en su propia materialidad \rdm{le fuera, del algún modo, inmanente} o bien lo elementos no podrían ser más que la \emph{expresión} de una totalidad trascendental, no podrían ser más que la manifestación de una totalidad que les imprime su propia esencia o fundamento. O bien, no podría haber más que relaciones de determinación, puesto que su interrelación no podría afectar \emph{sustantivamente} su propia identidad, por lo menos no sino bajo alguna de forma de \emph{desviación}, \emph{ocultamiento} o \emph{desnaturalización}. Exactamente lo contrario, desde el punto de vista epistemológico y ontológico, sucede en el conductismo y en todos aquellos enfoques de la ciencia política que asumen, más o menos explícitamente, una concepción referencial y substancial del lenguaje.

Este modo de concebir el lenguaje nos sitúa frente a otra cuestión central que una ontología de la sobredeterminación procura poner de relieve. A saber, la primacía ontológica de las relaciones por sobre los elementos. Este protagonismo de las relaciones debe entender en un doble sentido. Por un lado, implica una desubstanciación radical de toda identidad. Una vez más, Saussure fue quien puso en el centro de la escena el momento relacional para comprender el proceso de significación al punto de afirmar que la identidad de cada signo lingüístico depende enteramente de sus vínculos \rdm{reglados por el sistema lingüístico} con los otros eventos de la lengua y nada tiene que ver con el referente. Desde nuestro punto de vista, la sobredeterminación, en tanto lógica que permite comprender el modo en que se relacionan las identidades particulares con un entorno parcialmente estructurado está mostrando la centralidad del momento relacional en el proceso de fijación de sentido, está evidenciando hasta qué punto un significado particular está contaminado, a través de mecanismos de condensación y de múltiples operaciones de desplazamiento, por el conjunto de la formación social hegemónicamente configurada. La otra consecuencia de esta preeminencia de las relaciones es que no debe hacernos pensar en una linealidad ordenada temporalmente al modo de un primer momento de sutura hegemónica y uno posterior de sobredeterminación de los elementos particulares. Pero, sin embargo, la sobredeterminación nos permite pensar la realidad en cuanto proceso, al igual que lo pretendido por la dialéctica hegeliana (y marxista) pero sin los resabios esencialistas que esta acarreaba. Y al contrario de lo que sucede con la noción de determinación que es algo que ocurre en un momento determinado (y determinable). Si la significación de una práctica social es un proceso, todo acto de significación pasa a ser metafórico, en el sentido de que todo punto de origen y de fijación siempre es arbitrario y solo definible analíticamente.

Consecuentemente, una ontología de la sobredeterminación muestra la tensión entre la interioridad y la exterioridad de toda identidad. En sentido estricto, ontológicamente no hay una frontera que distinga nítidamente el interior del exterior de una singularidad. Y esto es lo que una ontología de la sobredeterminación permite mostrar, ya que desde este posicionamiento se procura evidenciar el carácter híbrido de toda identidad, al mostrar de qué modo esta siempre está contaminada por la relativa estructuralidad que define su contexto. Así, la sobredeterminación nos permite dar cuenta de la existencia de aspectos sumamente relevantes que escapan a la posibilidad de observación directa, ya que permite mostrar cómo el todo hegemónicamente estructurado existe solo mediante sus efectos sobre identidades particulares. La sobredeterminación permite mostrar en un mismo movimiento la presencia de la estructura mediante sus efectos, a la vez que su ausencia~ya que esta no existe más allá de sus implicancias sobre singularidades, permitiendo a su vez pensar la significación en sus propias condiciones de existencia, es decir en su propia inserción en el todo complejo. De esta forma, la estructura no es algo exterior que vendría a modificar el aspecto, la forma y la relación de los objetos sino que la relativa estructuralidad que actúa como superficie de inscripción de toda identidad solo se encuentra en su misma interioridad.

Como esperamos quede claro, una ontología de la sobredeterminación tiene una serie de implicancias fundamentales para el análisis político. Una ontología de la sobredeterminación cuestiona toda clasificación de los objetos y las prácticas sociales y, por tanto, el análisis político no puede ya sustentarse en la existencia de un objeto que reclame como propio. Desde este punto de vista, no hay \emph{a priori} un mundo de objetos \emph{políticos} que requieran de un tipo específico de intervención para poder ser comprendidos; en todo caso, el propio proceso que orienta la configuración significativa es susceptible de ser leído \emph{políticamente}. Por lo tanto, estamos frente a una clara \emph{radicalización} del análisis político en la medida en que se disuelven las fronteras impuestas por las gramáticas canónicas. Una ontología de la sobredeterminación define más al análisis político como un \emph{modo de intervención}, como un \emph{modo de ver y de pensar} los procesos de fijación de sentido, y, por ende, no legitima el análisis político a partir de un objeto de estudio propio y exclusivo como pretenden hacerlo la ciencia política y las ciencias sociales. Desde este nuevo lugar, el análisis político situado en una ontología de la sobredeterminación encuentra un nuevo espacio legítimo para su propia intervención, el del análisis ideológico. Desde este nuevo lugar, el análisis político consistirá en un nuevo modo de aproximación a los objetos y prácticas sociales (sean o no comprendidas por las gramáticas tradicionales como \emph{políticas}) que procurará poner en evidencia el modo en que cada particularidad se encuentra contaminada (es decir, sobredeterminada) por aquellos discursos devenidos en hegemónicos. Entonces, el análisis político consistirá en escrutar las operaciones ideológicas, expresadas a través de los mecanismos de condensación y desplazamiento, que muestran la presencia de la relativa estructuralidad a través de sus efectos; poniendo en evidencia el carácter material del discurso, al dar cuenta de cómo las operaciones ideológico-discursivas hacen que una práctica u objeto social \emph{sea} lo que efectivamente \emph{es}.

En conclusión, una ontología de la sobredeterminación pone en evidencia el atravesamiento del poder en todo proceso de configuración de sentido, por lo menos desde dos puntos de vista. En primer término, si la sobredeterminación supone operaciones como la condensación y el desplazamiento (es decir, mecanismos tales como la exclusión, la fragmentación y la fusión así como reemplazos, traslaciones de relevancia y descentramientos) generando que algunos elementos tengan una preeminencia sobre otros y pasando las relaciones a ocupar un lugar definitivamente central en un contexto atravesado por la contingencia, entonces podemos afirmar que la sobredeterminación da cuenta de la presencia del poder en todo proceso de significación. Desde este lugar, el análisis político posfundacionalista pondrá en evidencia la radical contingencia inerradicable de cualquier fijación precaria de sentido, mostrando en este mismo movimiento la especificidad de~cualquier práctica u objeto social al dar cuenta de los procesos que la legitimaron y la hicieron válida al demostrar cómo otros contenidos o sentidos fueron excluidos. Por lo tanto, desde una ontología de la sobredeterminación podemos mostrar el sentido radical en que el pensamiento político ha venido pensando la noción de contingencia desde Derrida en adelante.

Y en segundo término, en la medida en que nuestra ontología de la sobredeterminación nos permite dar cuenta de la presencia de la relativa estructuralidad que actúa como superficie de inscripción de todo proceso de configuración identitaria, pone en evidencia la operación política por excelente representada por la sutura hegemónica, que es la condición de posibilidad de dicha totalidad parcialmente estructurada. En definitiva, una ontología de la sobredeterminación nos permite dar cuenta del \emph{juego} de la hegemonía, al permitirnos a través del análisis político-ideológico de una identidad particular mostrar su presencia y su ausencia. Permite pensar cómo se ocupa de manera vicaria la presencia central, como una identidad tan particular como el resto de los elementos de una formación social determinada asume una función universal y limita el \emph{juego} hacia el interior de la misma. Y esto muestra su situación paradojal, de ser parte del sistema y, por lo tanto, formar parte de él y, sin embargo, ocupar un lugar que no le es preexistente. Así, el lugar vacío del centro es ocupado lógicamente por una particularidad intra-sistema, pero para asumir el rol central debe mostrarse como algo más que una mera singularidad, debe mostrarse como universal, es en este sentido que la hegemonía está dentro (por ser identidad tan particular como el resto) y fuera (al presentarse como universal y mostrarse como diferente del resto de los elementos del sistema). Desde este lugar, el análisis político dará cuenta de la \emph{diferencia} política al \emph{reactivar} lo \emph{sedimentado}, al mostrar que la existencia del momento político solo puede reconocerse a través de sus efectos sobre configuraciones particulares.

\section*{Referencias}
\printbibliography[heading=none]   % Sin título automático




%%%%%%%%%%%%%%%%%%%%%%%
\ifPDF
\separata{capitulo2}
\fi

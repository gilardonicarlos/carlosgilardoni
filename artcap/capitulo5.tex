\ifPDF
\chapter[Ontología de la inoperancia...]{Ontología de la inoperancia. La política en el pensamiento de Giorgio Agamben}
\Author{Manuel Moyano} % poner nombre del autor
\setcounter{PrimPag}{\theCurrentPage}
\else
\ifHTMLEPUB
\chapter[Ontología de la inoperancia...]{Ontología de la inoperancia. La política en el pensamiento de Giorgio Agamben}
\fi
\fi

%%%%%%%%%%%%%%%%%%%%%%%
% Empezar con \section{Introducción al Lorem Ipsum}
\nombreautor{Manuel Moyano}

\epigraph{(\ldots) \emph{el problema del ser está universalmente referido al constituyente y a lo constituido}.}{Carta de Heidegger a Husserl del 22 de octubre de 1927}


\section{Introducción}

Una nueva ontología recorre el campo de la teoría política: el antiesencialismo. Heredera de la crítica heideggeriana a la metafísica y de los desarrollos postestructuralistas del pensamiento francés de los '60, en esta nueva ontología se juega el estatuto de la \emph{diferencia}, en tanto se permite su despliegue contra la fijación de cualquier esencialismo. Y esto también sucede al interior de la constelación teórica que (se) la intenta pensar. Si la falta de una esencia última y determinante de la existencia de lo que hay es la matriz que da inteligibilidad a sus desarrollos, los de la nueva ontología, la pregunta que de allí se desprende es ¿qué habilita esta falta?

En el presente trabajo intentaremos delinear dos \emph{posibles} lógicas de respuesta a esta pregunta: la decisión y la inoperancia. En este sentido, será nuestro objetivo implicarnos en los desarrollos de Giorgio Agamben para dar cuenta de cómo aquella lógica de la decisión se corresponde con el paradigma de la soberanía en cuanto organización del poder en Occidente, mientras que la inoperancia se ancla en su correspondencia a la comunidad en tanto posibilidad de desactivación del paradigma soberano. Por lo tanto, la apuesta que en este trabajo intentaremos sostener recorriendo la obra de Agamben en términos generales, es que la política pensada como comunidad solo se vuelve posible a partir de un despliegue de la inoperancia en tanto apertura a la politicidad del ser mismo.

Como se desprende de dicha apuesta, hay aquí una problemática asunción que asimila el ser y la política. En este sentido, es necesario remarcar que la condición ontológica de la política en Agamben proviene no solo de los alcances que este pretende darles a sus tesis sobre la política,\footnote{\cite[96-98]{@7088-GALINDOHERVAS2003}. El autor muestra en estas páginas por qué el pensamiento agambeniano puede ser considerado una \enquote{ontología política}, no sólo por la utilización de términos ontológicos para referir a los problemas políticos sino también por el alcance que pretende darle a sus tesis sobre la política.} sino también por la singular aprehensión de la política como \emph{lo que hay}. La búsqueda de una ontología política propiamente dicha será, en el autor, la posibilidad de que la experiencia de lo que hay (y allí, lo que \emph{es} sería al mismo tiempo la \emph{experiencia} de ello) sea la marca de la existencia política.

El dispositivo teórico que aquí privilegiamos para dar cuenta de dicha asimilación es justamente el término inoperancia. La originalidad del planteamiento agambeniano residirá, para nosotros, en que dicho término se copertenece a una nueva comprensión de la vida. \footnote{Como lo menciona Rodrigo Karmy Boltón, esta nueva vida adquiere diversas. denominaciones en relación con la inoperancia. Así, \enquote{“forma-de-vida”, “vida feliz” o “vida eterna”, serán las denominaciones específicas de dicha inoperosidad}. \cite{@7089-KARMYBOLTON2010}.} De este modo, el autor realiza un anclaje entre tres términos esenciales a su pensamiento los cuales son vida, ser y política que aquí pretendemos rescatar para dar cuenta de una ontología de la inoperancia.\footnote{Este anclaje es uno más entre otros, los cuales también se corresponden a éste mismo. La mejor forma de comprender esta estrategia teórica es, tal como Roque Farrán lo ha definido, como un nudo. Así, como dice el autor, la articulación entre diferentes términos o discursos se realiza de la siguiente forma: \enquote{cada discurso [o término] (como un hilo) pasará en algún punto por encima de otro de manera alternada (\ldots) pero luego ese otro pasará por encima del primero, y así también sucederá en relación con un tercero y a un cuarto (como en un trenzado infinito). Por lo tanto, no hay metalenguaje, como no hay discurso \textit{determinante en última instancia} de la inteligibilidad de los demás (sutura); cada discurso cumple esa función en algún punto de cruce y hace a la consistencia nodal del conjunto \rdm{virtualmente infinito}}. Véase el capítulo~\ref{cap:nodal}, pág.~\pageref{cap:nodal} de Roque Farrán \enquote{Ontología nodal} en el presente libro. En este sentido, nuestra intervención teórica pretende reconstruir ese anudamiento en la obra de Agamben (produciéndolo de este modo también). El término escogido, inoperancia, es, por lo tanto, otra de las formas en que el nudo (se) expone.} El estatuto de la diferencia, en consecuencia, es así circunscripto al nivel de la inoperancia entendida esta como potencia, es decir, como pura posibilidad. Pero esta no mantiene una relación de exterioridad respecto de lo que hay, sino que más bien ella misma es la marca en que se da la existencia: existir políticamente será entonces concebir la vida como pura posibilidad. Esta \enquote{ontología de la potencia}, como la denomina Galindo Hervás,\footcite[205]{@7088-GALINDOHERVAS2003} restituye a la vida su vocación específica: la inoperosidad, esto es, la falta de obra. El punto central será, entonces, pensar la política y la vida humana como una existencia genérica en la potencia sin realizar un paso al acto. El ser que allí se traza es un ser que declina su potencia de ser (pasar al acto) en beneficio de su potencia de no ser (no pasar al acto, retardarse en la potencia).\footcite[205]{@7088-GALINDOHERVAS2003} Por lo tanto, el lugar de la imposibilidad del ser \rdm{el no ser} es justamente la condición de posibilidad de este mismo ser: de lo que se trata es de ser allí donde este declina su pretensión soberana de ser, donde se declara sin obra y sin fin. El estatuto ontológico de una vida que allí se resguarda, le devuelve a esta su condición de perteneciente no ya a tal cualidad, a tal forma de vida, sino a la forma de vida misma. La vida será ahora inseparable de sus formas-de-vida.\footnote{\cite[18 y ss.]{@7091-AGAMBEN2001}. Véase también,\footcite{@7092-KARMYBOLTON2010} donde el autor muestra cómo la lectura de la diferencia entre acto y potencia, en tanto diferencia ontológica, que realiza Agamben de Aristóteles se corresponde a la especial atención del italiano a los comentarios del filósofo cordobés Averroes, pensador cuyo olvido la tradición nos ha legado.}

Ahora bien, tal como habíamos anunciado al principio, la falta de una fijación esencial que se presenta en esta ontología también puede habilitar el movimiento inverso de la inoperancia, esto es, la decisión. En este sentido, es preciso que delineemos el recorrido que nos proponemos en este texto. Podemos designarlo esquemáticamente como un tridente: en primer lugar, buscamos realizar un acercamiento a la diferencia desde su estatuto ontológico antiesencialista. Para ello buscamos reconsiderar algunas tesis inscriptas por Oliver Marchart como propias del pensamiento político posfundacional (en tanto dicho autor realizar una formulación explícita de la política como la diferencia ontológica),\footcite{@7093-MARCHART2009} para así complejizar sus análisis mostrando cómo en su postulación se mantiene una concepción soberana de la política dado que se privilegia la decisión como tránsito en que se produce la diferencia ontológica.

En segundo lugar, desarrollamos los dos modelos de poder que incluye el paradigma soberano (la biopolítica y la gloria) para dar cuenta de cómo este a pesar de su antiesencialismo circunscribe en los hombres algo así como una vida desnuda que intenta politizar continuamente, y por esto mismo la produce. En esta instancia mostramos cómo las tesis de Agamben sobre el poder en Occidente se corresponden a la habilitación de la decisión como punto de sutura parcial entre la falta de esencia del existir políticamente y los circuitos del poder soberano. Desarrollamos \emph{in extenso} los problemas de la saga \emph{Homo Sacer}, principalmente\footcite{@7101-AGAMBEN2003} y\footcite{@7102-AGAMBEN2008} publicados en 1995 y en 2008 respectivamente.

Finalmente, en un tercer momento, postulamos que la inoperancia es la posibilidad que otorga el pensamiento de Agamben para re-pensar la política como una actividad humana que se sustrae a las redes de la soberanía. En este punto, nos explayamos sobre las posibilidades de la inoperancia como una existencia política que desactivando el poder soberano, nos permite una vida propiamente política sin necesidad de dispositivos soberanos que regulen la comunidad. En este punto, con una intención deconstructiva, remitimos las tesis de Agamben sobre la inoperancia a los postulados de\footcite{@7094-AGAMBEN2003} cuya publicación data de 1990.

\section{El posfundacionalismo y la ontología} %2.1.

En 1927, Martin Heidegger se preguntaba en \footcite{@7095-HEIDEGGER2009} por el concepto de ser y su relación con el ente. La cuestión era planteada del siguiente modo: \enquote{{[}a{]}quello de que se pregunta en la pregunta que se trata de desarrollar es el ser, aquello que determina a los entes en cuanto entes, aquello \enquote{sobre lo cual} los entes, como quiera que se los dilucide, son en cada caso ya comprendidos. El ser de los entes no \enquote{es} él mismo un ente}. \footcite[15--16]{@7095-HEIDEGGER2009} De este modo, si el ser se diferencia de los entes, pero no como otro ente, implica que en ellos ya acaece aquél como determinante, pero no como una sustancia primigenia en el sentido de anterior y desligada de los entes, al modo de otro ente, sino al modo de una diferencia en sus propias mismidades. Esta misma diferencia del ser respecto del ente, es la que se da entre lo ontológico y lo óntico: mientras que lo ontológico refiere a la diferencia interna que determina lo óntico, siendo así su condición de posibilidad, lo óntico no puede estabilizarse a sí mismo sino a partir de referir a lo ontológico. Así, hay regiones ónticas, las cuales son determinadas ontológicamente. Es esta diferencia lo que en una primera lectura de la obra temprana de Heidegger implica la diferencia ontológica.

Esta diferencia es asumida al interior del pensamiento político posfundacional, según Marchart, como la diferencia entre \emph{lo} político y \emph{la} política.\footcite{@7093-MARCHART2009} En este sentido, la política entendida como un conjunto específico de reglas, significados y procedimientos, se vería imposibilitada a ser idéntica a sí misma por cuanto se encuentra abierta desde su interior a un exceso que funciona como el suplemento de su mismidad: lo político. Un elemento exterior a la política funciona como el momento de institución (determinación) de ella (como también de los otros subsistemas: lo social, lo cultural, etc.). Pero ese elemento, lo político, es también en sí mismo inconmensurable, esto es, indeterminado. Por ello, la diferencia entre lo político y la política es ella misma una imposibilidad de estabilización, o determinación definitiva, tanto de uno como del otro. La política en sí es insuficiente, incompleta, ya que no puede basarse en un fundamento último, una esencia, que la determine, tal como en Heidegger un ente no puede ser determinado por otro ente. Surge en la política una carencia de fundamento último, una imposibilidad de decir \enquote{tal cosa es la política}. Por esto mismo, la política no puede adquirir una determinación esencial que la estabilice, lo que sugiere que para mantenerse como tal, habrá de relacionarse con algo que la excede: su propia institución como lo político.\footcite[19--20]{@7093-MARCHART2009} Esto no refiere a la inexistencia de fundamentos, sino más bien a la \emph{imposibilidad} de un fundamento último; pero a su vez a la \emph{posibilidad} de múltiples fundamentos que, contingentemente y en conflicto permanente, intentan fundarla. Así, la sociedad es~políticamente signada por el doble movimiento de la diferencia política, ya que, como afirma Marchart,

\begin{quote}
	(\ldots) por un lado, lo político, en tanto momento instituyente de la sociedad, opera como fundamento suplementario para la dimensión infundable de la sociedad; pero, por el otro, este fundamento suplementario se retira en el \enquote{momento} mismo en que instituye lo social. Como resultado de ello, la sociedad estará siempre en busca de un fundamento último, aunque lo máximo que puede lograr es un fundar efímero y contingente por medio de la política (una pluralidad de fundamentos parciales)\footcite[23]{@7093-MARCHART2009}.
\end{quote}

Para el posfundacionalismo \emph{a là} Marchart, la política sería entonces la diferencia consigo misma, lo que abre el espacio a lo político entendido como la falta de la propia política, y por ello como un movimiento o pliegue interno y externo a la política, where como en todo pliegue dos elementos heterogéneos entre sí quedan solapados en una misma instancia.

Hay en esta lectura dos asunciones importantes. En primer lugar, una asunción política: la política como tal, como transparencia y estabilidad significante, es imposible dado que en su interior un elemento negativo la excede señalándole su propia insuficiencia, su falta. Pero ese elemento produce un pliegue, ya que tuerce la política sobre sí misma, haciéndola re-inventarse sobre su apertura misma, volviéndola sobre sí, trayendo sobre la base de su discontinuidad la posibilidad de su contingente y efímera estabilidad siempre renovada. El juego entre la política y lo político, defendido por este autor, refiere a un movimiento circular de presencia/ausencia, de fundación/des-fundación, de positividad/negatividad, donde la continuidad entre los polos de la relación se tensa en un corte entre ambos donde se indistinguen, dado que se produce una instancia de imbricación mutua, de recíproca contaminación, digamos una \enquote{zona de indistinción} \footcite[19]{@7101-AGAMBEN2003} entre los dos elementos de la relación.

A su vez, hay todavía una asunción más fuerte que la anterior, ya que la incluye: la ontología \emph{es} política. Esto se debe para Marchart a~que si lo ontológico refiere al abismo que supone el ser en general, a la falta de fundamento último, (\ldots) únicamente \emph{lo político} puede intervenir como suplemento del fundamento ausente. Y ello implica que cualquier ontología (posfundacional) \rdm{cualquier \emph{hauntologie}} será necesariamente una ontología \emph{política}, la cual ya no puede ser subordinada al estatus de una región de la indagación filosófica.\footcite[216]{@7093-MARCHART2009}

Por lo tanto, vemos que \enquote{la ontología \emph{política} no equivale sino a una ontología de \emph{lo político}}. \footcite[219]{@7093-MARCHART2009} En otras palabras, la naturaleza del ser-\emph{qua}-ser sería de índole política por cuanto el ser como diferenciante, y como tal infundable, se corresponde con el movimiento de lo político ya que este supone una instancia de ausencia \rdm{o, más bien, \emph{ausentificación}} y al mismo tiempo de presencia \rdm{o, más bien, \emph{presentificación}} siempre contingente y parcial. Esta asunción supone que \enquote{{[}n{]}o \enquote{todo es político}, pero el fundamento/abismo de todo es \emph{lo político}}. \footcite[223]{@7093-MARCHART2009} Pero también queda solapada en esta asunción del \enquote{ser-\emph{qua}- \emph{lo político}} \footcite[220]{@7093-MARCHART2009} la necesidad de \emph{la} política, en tanto región particular/óntica del ser general/ontológico. Ello se debe a que no hay un acceso puro e inmediato al ser en general, sino por vía de los entes particulares. Así, el lugar óntico que detenta el lugar imposible de lo ontológico, será la política en tanto espacio óntico-regional que ocupa la imposibilidad del todo, asumiéndose a sí como una instancia contingente. De este modo, \enquote{el ascenso de una ontología regional al siempre precario estatus de una ontología general solo puede basarse, en definitiva, en una decisión contingente}. \footcite[226]{@7093-MARCHART2009} Por lo tanto, la diferencia entre \emph{lo} político (en tanto diferencia del todo con respecto a sí mismo) y \emph{la} política (en tanto estabilización contingente de la diferencia del todo) es ella misma una diferencia \emph{política}. Y si esta diferencia es lo ontológico, el ser- \emph{qua}-ser, ello acarreará una \enquote{doble inscripción de lo político}: \footcite[227]{@7093-MARCHART2009} político es lo que nombra la ausencia del fundamento último, pero es también el nombre de la \emph{distancia} entre dicha ausencia y la presencia contingente de la política. El problema de esta definición será, como veremos, que en dicha distancia, ausencia y presencia se confunden e intercambian sus papeles eternamente, dando lugar a la constante necesidad de una decisión contingente propia del paradigma soberano.

\section{La decisión y la zona de indistinción entre ser-ente} %2.2.

En un artículo titulado \enquote{Contra la diferencia política}, \footcite{@7074-BISET2010} Emmanuel Biset realiza una crítica al pensamiento de la diferencia política propia del posfundacionalismo presentado por Marchart que aquí quisiéramos rescatar con respecto a dos frentes: en la reducción de la estructura ternaria de la diferencia en Heidegger a una estructura dual y en la asimilación paradójica de la decisión como fundamento político.

Respecto de la reducción dualista de la estructura ternaria de la diferencia ontológica, esta queda clara a partir del doble movimiento en que Marchart presenta la diferencia política. Así, \enquote{de un lado, tenemos la imposibilidad de un fundamento último, y por ende un fundamento negativo; de otro lado, tenemos los fundamentos contingentes de lo social}. \footcite[179]{@7074-BISET2010} En este sentido, \enquote{lo político (\ldots) solo aparece como negatividad de las fundaciones parciales}.\footcite{@7074-BISET2010} Por lo tanto, lo que la diferencia política en el pensamiento de Marchart trae a colación no es solo la imposibilidad de un fundamento último, sino también la imposibilidad de una ausencia radical de fundamentos. En otras palabras, la negatividad de lo político como tal no es accesible sino por medio de las regiones ónticas del ser-político. No hay, entonces, una ontología pura sino más bien una ontología contaminada por la multiplicidad de regiones ónticas. Lo que el pensamiento posfundacional según Marchart señala es el ser de los entes, pero se detiene en la dualidad de dicha diferencia: el ser será lo que señala el acabamiento del ente, su exterioridad constitutiva (y como constitutiva, también interior a los entes). ¿Qué sucede en dicho planteamiento? El ser es reducido a su diferencia respecto del ente, y por lo tanto, en cuanto que ser- \emph{qua}-ser es convertido en un \enquote{todo indiferenciado}. \footcite[184]{@7074-BISET2010} Esto significa que el ser en cuanto tal solo adquiere su matiz \rdm{su propiedad de diferencia} con base en la presencia de los entes, siendo el ser sin el ente un todo indiferenciado. La diferencia, por lo tanto, será generada a partir de la presencia contingente de los entes. Hay aquí una preeminencia de lo óntico por sobre lo ontológico, ya que la diferencia será posible por la presencia del ente, donde el ser será inteligible solo por su inscripción al seno de aquel, que devendrá en ausencia: en lo que Marchant llama lo político. Lo que esta dualidad olvida es la diferencia irreductible del ser en cuanto tal: los modos de ser del \emph{es}.\footcite{@7074-BISET2010}

Con base en una genealogía de la diferencia ontológica en los textos de Heidegger, Biset muestra cómo dicha diferencia tiene una estructura ternaria: en primer lugar la diferencia entre los entes (diferencia óntica), en segundo lugar la diferencia entre el ser de los entes y los entes (diferencia óntica-ontológica), y en tercer lugar la diferencia entre el ser de los entes y los modos de ser del mismo ser (diferencia ontológica). De acuerdo al autor, a esta última diferencia \enquote{Heidegger la nombra como \emph{sentido del ser}}. \footcite[185]{@7074-BISET2010} El sentido del ser referirá no ya a la diferencia de este respecto del ente, sino a la pluralidad irreductible que supone la unicidad del ser.\footnote{De este modo se comprende la preeminencia del Dasein (ser-ahí), el ente cuyo modo de ser es interrogarse por el sentido del ser, que en los textos tempranos de Heidegger se le otorga. La figura del Dasein será central por cuanto es el ente en el cual se posibilita el paso de lo óntico a lo~ontológico, del ente al ser. Así, “[e]s el Dasein quien abre la diferencia ontológica: si la vía de acceso al ser es el ente, se trata de indagar en un ente que escape a su carácter óntico. El Dasein abre la diferencia ontológica no porque la nombre, sino porque la realiza. Pero sólo puede realizar la diferencia porque es él mismo esta diferencia. La diferencia ontológica está en el Dasein porque es el único ente que se caracteriza por comprender el~ser. De modo que el Dasein comprendiendo su ser comprende el ser en general, es la transición en sí mismo de lo óntico a lo ontológico, porque es ónticamente lo ontológicamente diferente. (\ldots) la cuestión del ser es el mismo Dasein” \parencite[186]{@7074-BISET2010}. En otras palabras, la posibilidad del ser --de su sentido, es decir, de sus modos de ser- viene dada por la existencia de un ente que puede realizar la \enquote{trascendencia} de hacer asequible el ser-\textit{qua}-ser. No es la intención de este trabajo señalar el problemático lugar que implica la preeminencia óntica-ontológica del Dasein en la obra temprana de Heidegger, pero su referencia es ineludible, como señala Biset, por cuanto retoma la diferencia ontológica en cuanto tal.}

De este modo, señala Biset, olvidando la diferencia del ser en cuanto tal, el posfundacionalismo de Marchart termina por entificar al ser dado que si el ente es fundado por su referencia al ser,  \linebreak como tal, es decir, como fundante, no puede ser pensado sino como externo al ente y así como otro ente.\footcite[195]{@7074-BISET2010} Pero también, extremando esta interpretación y según lo dicho anteriormente, el ser será fundado en cuanto tal por su referencia al ente y así nuevamente entificado. En otras palabras, si la diferencia ontológica es reducida a una dualidad entre ser y ente, lo que resulta de ella es una doble entificación del ser: por un lado, se lo presenta como lo fundante del ente por cuanto es la referencia necesaria de todo ente (el límite de su entidad en una instancia negativa), y así como algo externo al mismo, lo que lleva a entificarlo en una \emph{desligadura respecto del ente}; y, por otro lado, se lo entifica dado que solo el ser es tal en su referencia al ente, es decir, en la diferencia de aquel respecto de este, y así se vuelve un paradójico \emph{fundamento-fundante-fundado}. Lo que en última instancia se produce con esta entificación del ser es una \emph{zona de indistinción} entre ambos ya que no puede discernirse lo fundante de lo fundado por cuanto ambos se implican en su mutua necesariedad. En otras palabras, si el ser funda a los entes en cuanto ellos refieren a él para ser tales (\enquote{la política es\ldots}, \enquote{la economía es\ldots}, \enquote{la ética es\ldots}), el ser no puede sino estar por fuera de ellos (para ser de todos al mismo tiempo, es decir, para ser el \emph{es} de todos ellos, necesita estar más allá de ellos), pero ello también implica que el ser necesita de los entes por cuanto ellos serán la única posibilidad real de hacer efectiva su presencia que no será ni más ni menos que la ausencia de aquellos. El problema de esta interpretación de la diferencia ontológica es that reduce la diferencia en sí a una diferencia entre el ser y el ente, pero como tal diferencia se presenta a partir de la relación entre ambos, en el punto de su entrecruzamiento se vuelven indiferenciados y así entificados.

Por esto, en esta lectura de la diferencia ontológica, \emph{el ser es la ausentificación de la presencia de los entes}. Esto supondrá que \emph{la relación} entre ser y ente será \emph{lo entitativo}, y en el límite de dicha relación, se produce una \enquote{zona de irreductible indeferenciación}.\footnote{Sintagma acuñado por \cite[19]{@7101-AGAMBEN2003}} entre ambos dado que se co-pertenecen: el ser no es sino en referencia al ente, el ente no es sino en referencia al ser. Allí, fundante y fundado intercambian sus papeles constantemente, y preparan el campo para un paradójico fundamento: la decisión.

Para entender el privilegio que el posfundacionalismo le otorga a la decisión, debemos volver a su asunción de la política como ontología. Si, como habíamos dicho, asumir que la imposibilidad de un fundamento último y la contingencia de los fundamentos parciales es la ontología, y decidir que ese movimiento es el juego entre lo político y la política,

\begin{quote}
	(\ldots) esto conlleva un argumento circular. Desde una decisión se nombra a la diferencia política, pero política es justamente esa decisión primera. Esto conlleva a un desplazamiento metonímico de la primera decisión al llamarla política. En otros términos, ¿por qué llamar a la diferencia ontológica política? O mejor, ¿por qué llamar política al juego entre ausencia de fundamentos últimos y decisión contingente? Sólo se puede responder a esto, desde la perspectiva analizada, sosteniendo que una decisión es la que establece esta nominación, es decir, que en última instancia es arbitraria. Pero esta arbitrariedad se repliega sobre sí porque termina estabilizando un sentido de lo político como decisión infundada.\footcite[190]{@7074-BISET2010}
\end{quote}

Por lo tanto, la asunción de Marchart sobre la política como ontología solo adquiere su sentido en una paradoja: la ontología es política porque por medio de una decisión infundada, así se establece. Pero la paradoja surge en tanto se supone lo mismo que~la decisión genera: supone que la \emph{decisión}, que es la que suplementa la ausencia de fundamento último con fundamentos parciales y contingentes, es lo político \emph{en el mismo momento} en que ha decidido que lo político es el juego de ausencia y presencia contingente. Así la misma suposición es la decisión, y ello hace caer al argumento en una circularidad por cuanto se entiende como político a aquello mismo que se hace en el acto de la nominación: se dice la ontología es lo político porque en el mismo momento en que supone qué es lo político está decidiendo el sentido de lo político.

Y esta paradoja no se produce solo en el desplazamiento conceptual de definir a la ontología como política, o viceversa, sino en la misma interpretación de la diferencia ontológica: si la ausencia de fundamento último hace necesaria una decisión infundada de fundamentos contingentes, se realiza \enquote{un salto entre el plano ontológico y el plano óntico}. \footcite{@7074-BISET2010} Es decir, de lo ontológico se pasa necesariamente a lo óntico contingente por medio de una decisión, y así resulta contaminada la ontología general por una ontología regional. Ello nos conduce a un callejón sin salida, que recientemente señalábamos en la entificación del ser, por cuanto lo ontológico se vuelve lo óntico y viceversa, es decir, \emph{se indistinguen}. Esta circularidad se puede resumir del siguiente modo: suponer que la ausencia propia del ser habilita la necesidad de entes siempre contingentes, hace que la presencia de los entes genere la ausencia propia del ser.

El problema central de esta circularidad se debe a que el ser es pensado \emph{en relación con} el ente. Si lo que se buscaba era desustancializar al ente, es decir, mostrar la imposibilidad de un ente primero y esencial (la imposibilidad de un fundamento último), al introducir la necesidad de entes o fundamentos parciales y contingentes, \emph{se realza la figura de la decisión} en tanto tránsito del ser al ente, y del ente al ser. Y ello se debe a que en ella confluyen tanto el ente como el ser, la ausencia y la presencia, la imposibilidad y la posibilidad. Por lo tanto, la decisión coincide punto por punto con la relación. ¿Qué se produce en la decisión sino la relación misma entre la imposibilidad de un fundamento último y los fundamentos contingentes? ¿Qué asegura la relación entre el ser y los entes sino~la misma decisión de relacionarlos? En otras palabras, se presupone la relación entre el ser y los entes en el mismo momento en que dicha relación se está realizando. Así la decisión se vuelve omnipresente por cuanto se inscribe en el entrecruzamiento del ser y del ente. Es en ella donde se entifica al ser, dado que la constitución de los entes dependen de aquel, pero a su vez aquél depende de la necesaria presencia constituida y contingente de estos. Lejos de producirse una diferencia irreductible entre ambos polos, a lo que se llega es a una desustancialización entre ambos donde se vuelven irreductiblemente indiferenciados. Es en el \emph{entre} del ser y del ente donde la decisión infundada vuelve indiferenciados al constituyente como a los constituidos, es en ella donde quedan solapados en un umbral de indiferenciación, y así entificados (en el sense de determinados en su solapamiento). Su omnipresencia, no problematizada por Marchart, es consecuencia de su ilocalizable espacialidad. Si se la quiere tomar por su presencia, se diluye por el lado de su ausencia; si se la busca por el lado de la imposibilidad, aparece en forma de contingencia. \emph{La política pensada como institución contingente y parcial de los instituidos, lleva a la aporía de sobresaltar la decisión instituyente-destituyente.}

Así, toda decisión parte de un presupuesto paradójico: se presupone aquello mismo que con la decisión se crea. Y esto se debe a que la lógica de la decisión presenta un imposible posible, esto es, una instancia donde lo instituyente (lo negativo) se mixtura con lo instituido (lo positivo) y dan lugar a una instancia de destitución (el punto ciego entre lo negativo y lo positivo). ¿Qué crea una decisión sino lo mismo que presupone para darse como forma de decisión? Desde el individuo racional moderno, que se presupone a sí para crearse como sí mismo en el acto decisorio, hasta el pueblo que se presupone en el discurso soberano para ser creado como tal.

\section{La aporía soberana: un supuesto paradójico} %3.1.

La importancia de resaltar la omnipresencia de la decisión en el pensamiento político posfundacional \rdm{al menos el predominante en autores como Marchart, Laclau y Mouffe}, se encuentra en el estrecho lazo que la une con el paradigma vigente de la organización histórica del poder en Occidente: la soberanía. Ha sido el italiano Giorgio Agamben, a partir de la publicación de su trabajo\footcite{@7101-AGAMBEN2003} en 1995, quien ha sacado a luz esta problemática tan compleja de la soberanía. A pesar de que el autor no refiera en modo alguno al posfundacionalismo, ni siquiera a los autores principales del que este pensamiento se hace eco, pueden extraerse claras conclusiones a partir de su crítica a la soberanía que atañen al primado que dicho pensamiento le otorga directa o indirectamente a la decisión. Establecer este contrapunto nos ayudará a develar lo que por inoperancia entendemos.

Será nuestra intención, por lo tanto, mostrar la correspondencia de estas complejas relaciones topológicas entre el ser y el ente con la estructura de la soberanía implicadas en la relación entre excepción y norma jurídica. Esta analogía viene justificada no solo por el carácter plenamente jurídico de la decisión,\footcite[32]{@7101-AGAMBEN2003} sino también por la necesidad de mostrar los puntos de anclaje entre la ontología y la configuración del poder político en Occidente que Agamben desarrolla a lo largo de sus trabajos. De este modo, asumimos que hay una expresa analogía entre la relación ser-ente y la relación excepción-norma jurídica. Para ello debemos, en primera instancia, dar cuenta de la estructura de la excepción que define la soberanía y así, en una segunda instancia, anudar su problemática a la comprensión de la diferencia ontológica que el posfundacionalismo de Marchart esgrime.

En base a una exégesis de las formulaciones schmittianas sobre la soberanía, Agamben analiza la paradoja que encierra la creación de un orden jurídico por parte del soberano. Dicha paradoja se resume en la estructura de la excepción que hace vigente un orden jurídico. Así, la primera connotación de la decisión que debemos nombrar es su carácter excepcional, es decir, su condición de infundada. Por lo tanto, la pregunta a realizar es: ¿cómo se produce la excepción? ¿Cuál es su estructura lógica?

Con respecto a ello, Agamben nos dice

\begin{quote}
	(\ldots) la excepción es una especie de la exclusión. Es un caso individual que es excluido de la norma general. Pero lo que caracteriza propiamente a la excepción es que lo excluido no queda por ello absolutamente privado de conexión con la norma; por el contrario, se mantiene en relación con ella en la forma de la suspensión. \emph{La norma se aplica a la excepción desáplicandose, retirándose de ella}. El estado de excepción no es, pues, el caos que precede al orden, sino la situación que resulta de la suspensión de este.\footcite[30]{@7101-AGAMBEN2003}
\end{quote}

Desde el pensamiento agambeniano, por lo tanto, la relación por la cual una norma se establece, y por la cual además se mantiene, es una relación de excepción. Ella \enquote{incluye algo a través de su exclusión}. \footcite[31]{@7101-AGAMBEN2003} La relación lógica de la excepción con la norma no es el antecederla, sino más bien el hecho de crearla y también el hecho de conservarla. La regla es fundada por medio de una excepción, ya que su propia institución como orden es posible solo en una excepción, es decir, en una instancia de institución radical que no depende más que del acto de institución \emph{per se}. Pero una vez creada la norma, ella no corta su vínculo con la excepción que le dio origen, sino que se conecta a ella excluyéndola. Es decir, la incluye por medio de su exclusión. Por lo tanto, la única posibilidad de aplicación de la regla sobre la excepción es la suspensión de la regla, la retirada de ella. En ese entrecruzamiento de regla y excepción, donde la regla se mantiene en la forma de su suspensión, habita el soberano. Y en lugar del orden jurídico positivo, lo que tenemos es la suspensión del orden en la figura del soberano. El orden se mantiene, así, en su suspensión excepcional. En ese umbral que supone la relación entre la norma y la excepción, coinciden ambos polos de la dicotomía y así se indistinguen. La excepción es, por lo tanto, la ausencia de la norma, pero la norma solo se presenta asumiendo su ausencia como excepción. De este modo, la validez del orden jurídico se ubica en este espacio soberano donde lo que está fuera y lo que está dentro de la norma se vuelven indiferenciados, es decir, coinciden en este estado de excepción.\footcite[32]{@7101-AGAMBEN2003}

Esta relación de excepción que se teje entre la norma y la excepción que la crea se ancla en el soberano por cuanto es él, según la tesis schmittiana sobre la soberanía,\footcite{@7096-SCHMITT1987} quién decide la situación normal o la excepción. La decisión, por lo tanto, compete al soberano desde el momento en que él mismo es quien puede crear un orden, mantenerlo e incluso derogarlo. El soberano en cuanto tal, definido a partir de la estructura de la excepción, es inasible. Es puramente su decisión, la cual es indivisible y absoluta \rdm{de ahí el carácter de soberano} por cuanto es infundada, es decir, es \emph{sui generis}, no depende de nada ni nadie más que de su propia decisión. En última instancia, el soberano es inasible ya que es puramente la \emph{forma} de la decisión, una instancia desustancializada, inesencial, que por encontrarse en el límite de todo orden \rdm{donde se confunde lo instituyente y lo instituido}, escapa a cualquier determinación esencial.

Y esta formulación de la soberanía, como un umbral de indistinción entre la excepción y la norma que determina la suspensión de esta última, problematiza el juego entre lo político y la política expuesto por Marchart, ya que lo que de allí resulta es la mutua contaminación de dichos elementos heterogéneos por cuanto ambos se necesitan. Esa instancia se corresponde con la decisión que no pertenece ni a lo político ni a la política exclusivamente, sino al punto de su entrecruzamiento (el cruce de la imposibilidad de fundamento último y la multiplicidad de fundamentos parciales y contingentes). El soberano es absoluto por cuánto está aquí y allá también, está en el borde de cada uno de los polos de la dicotomía. No es solo la excepción que funda el orden, es también el propio orden ya que para que este se pueda aplicar debe suspenderse como tal. Esto significa que para que una norma tenga una validez general, dicha validez debe ser sustraída al caso particular en que se aplica ya que debe realizar un \emph{salto} entre lo general y lo particular. Así, una ley solo puede aplicarse en el paso de su validez general a su aplicación concreta, y en este paso anula su condición de derecho y lo vuelve un hecho. Si la norma estaba formulada a nivel general, su aplicación concreta refuerza la excepción que le dio origen: en cada aplicación concreta de la ley lo que se repite es la creación ilegal de la misma, su soberana decisión de afirmar que tal caso está regido por la ley. Por lo tanto, si reafirma su carácter excepcional, la norma es válida en su invalidez intrínseca.
\footnote{Respecto de ello, afirma Galindo Hervás: \enquote{Cuando no hay normas previstas a las que acudir o, mejor aún, cuando se es plenamente consciente de que incluso la aplicación de normas implica una implícita decisión discriminadora del caso excepcional y del caso normal, ya que de la norma misma no se deriva la imputación, es cuando la decisión brilla en su carácter de \textit{arché}}. \footcite[56]{@7088-GALINDOHERVAS2003}.} Se crea así una situación que no es de hecho ni de derecho, y \enquote{como tal, el orden jurídico en sí mismo es esencialmente ilocalizable {[}tal como el soberano{]}}. \footcite[32]{@7101-AGAMBEN2003} Del mismo modo que, como habíamos mostrado, el ente no es determinado en sí sino por referencia al ser inscripto en su seno, lo que aquí se produce es una constitución de la norma por su relación con la excepción. Pero esta excepción la excede, mostrando la pura ausencia que está apresada al interior de la norma, en palabras de Scholem, la \enquote{vigencia sin significado} \footcite[70]{@7101-AGAMBEN2003} del derecho. Esto implica que la ley ya no vale en cuanto a su contenido sustancial, sino en cuanto a su pura forma de ley o, lo que es lo mismo, en cuanto a su pura decisión infundada.

Reconduciendo estos análisis a la paradoja mostrada anteriormente al pensar el ser en relación con the ente, podría afirmarse desde la perspectiva agambeniana que los entes ya no se justifican por su substancialidad inherente, sino que justamente estos en cuanto infinitamente insustanciales dada su implicación en el ser que los trasciende y abre, encuentran su justificación en la decisión que en cada caso los (re)une al ser mismo. Pero si el ser no era tal sino por la presencia contingente de aquellos, se realiza también así la necesidad de un algo más que lo relacione con los entes: la decisión. Hay entonces una doble estructura de la excepción/decisión, como lo político en el pensamiento de Marchart: la excepción es la ausencia de norma última que da origen a la(s) norma(s), pero~es a su vez la propia suspensión de esta(s) es también la excepción. Por lo tanto, la decisión no solo instituye, sino que viven lo instituido destituyéndolo ya que siempre es más que lo instituido, más radical aún. Así, \emph{lo instituido solo sobrevive repitiendo esa institución originaria y destituyéndose de este modo}. Nuevamente, la decisión es omnipresente por su inasible condición.

Ahora bien, el punto fundamental que debe remarcarse es que esta imposibilidad de asir a la decisión, lo que le otorga su primado teórico y político, no es una condición natural, por así decirlo, de la decisión. Antes bien, ella se produce a partir de la paradoja de~un supuesto que la supone en la misma condición de su darse. Esto significa que la decisión no es un elemento claro y trasparente derivado de un sujeto decisorio sustancial, sino más bien que tanto la decisión como su sujeto (el soberano) sobreviven en la pura forma de su darse insustancial, donde el soberano se constituye como tal suponiendo su identidad en el acto de la decisión que le otorga dicha identidad. De este modo, como advierte Galindo Hervás, \enquote{(\ldots) es detectable una dialéctica entre voluntad y sujeto que arruina la pretendida estabilidad de ambos: la voluntad de decisión sobre la forma de la propia existencia (que determina la existencia de la unidad política) presupone ya dicha identidad (bien de forma real, inmediata, bien por representación)}. \footcite[38]{@7088-GALINDOHERVAS2003}

En consecuencia, la voluntad de decisión y el sujeto de la soberanía quedan entramados en la compleja relación topológica por la cual ambos se constituyen (siempre de forma inestable) presuponiéndose: la voluntad de decisión presupone un sujeto decisorio en su darse para así crear en su acto al sujeto, y el sujeto decisorio presupone una voluntad de decisión en el acto de su misma decisión. Esto se corresponde, en última instancia, al carácter existencial con que es asumida la decisión en los planteamientos schmittianos. Y esta asunción es legitimada a partir de \enquote{la conciencia de la vaciedad de fundamento sustancial y pérdida de soberanía que define a la Modernidad}. \footcite[50]{@7088-GALINDOHERVAS2003} Vemos de este modo que la urgencia y necesidad de la decisión en Schmitt surge claramente de la ausencia de fundamento último, al igual que en el planteamiento de Marchart.

Ahora bien, la presentación de la decisión como aporía determinante de la soberanía bajo una rúbrica lógico-formal como hasta aquí hemos hecho, podría hacernos perder de vista el objetivo central de nuestra crítica agambeniana a tal planteamiento. Al fin de evitar semejante paso en falso es necesario reconstruir las implicancias estatales y gubernamentales que la aporía soberana trae consigo.

\section{Biopolítica y gloria, nuda vida y \emph{oikonomía}} % 3.2.

En una de sus primeras clases del curso\footcite{@7097-FOUCAULT2007} de 1978, Michel Foucault situaba lo que para él era indispensable analizar allí. Así, decía contra el marxismo de la época \rdm{exaltante del Estado}

\begin{quote}
	(\ldots) lo importante para nuestra modernidad, es decir, para nuestra actualidad, no es entonces la estatización de la sociedad sino más bien lo que yo llamaría \enquote{gubernamentalización} del Estado. (\ldots) Gubernamentalización del Estado que es un fenómeno particularmente retorcido porque, si bien los problemas de la gubernamentalidad y las técnicas de gobierno se convirtieron efectivamente en la única apuesta política y el único espacio real de lucha y las justas políticas, aquella gubernamentalización fue, no obstante, el fenómeno que permitió la supervivencia del Estado.\footcite[137]{@7097-FOUCAULT2007}
\end{quote}

Así, contra la sobrevaloración del Estado por parte de lo que había denominado modelo \enquote{discursivo-jurídico} \footcite[100]{@7098-FOUCAULT2003} que exaltaría el análisis de la soberanía estatal como el centro de su funcionamiento, Foucault opondría en este curso el análisis de las tácticas generales de la gubernamentalidad para dar cuenta de la especificidad de aquél, reinscribiendo de este modo sus estudios en lo que había denominado el modelo \enquote{estratégico} \footcite[113]{@7098-FOUCAULT2003} que analizaría al mismo en un campo múltiple y móvil de relaciones de poder-saber.

A primera instancia, pareciera que la obra de Agamben encontraría su lugar adecuado en lo que Foucault denominó modelo discursivo-jurídico para analizar el Estado y su configuración moderna, implicando de este modo un análisis centrífugo. Pero si se trae a colación la persistencia de la noción de biopolítica \rdm{de cuño foucaltiano}, se puede ver que, cuanto menos, Agamben implica en sus estudios tanto el modelo jurídico tradicional como el estratégico foucaultiano.

Si Foucault había comprendido por biopolítica \enquote{la administración de los cuerpos y la gestión calculadora de la vida}; \footcite[169]{@7098-FOUCAULT2003} donde la política por medio de diversas y múltiples prácticas, discursos, cálculos, instituciones anudados en grandes dispositivos de poder, toma por \emph{objeto} a la vida para desarrollarla, multiplicarla, asegurarla en sexualidades, educaciones, es decir, en \emph{buenas formas de vivir}; se extrae la conclusión general que la vida es el objeto del poder. Ahora bien, si se enlazan estos análisis descentrados con las analíticas centradas propias de la soberanía como hace Agamben, se puede afirmar, en palabras de Galindo Hervás, que \enquote{(\ldots) él {[}Agamben{]} pretende investigar conjuntamente los procedimientos totalizantes de los Estados modernos (\emph{técnicas políticas} de tutela de la vida) y las técnicas de individualización objetivas (\emph{tecnologías del yo} foucaultianas), ya que considera que las implicaciones de la mera o \enquote{nuda vida} en la política constituyen el núcleo originario y oculto en la comprensión de la soberanía estatal} \footcite[94]{@7088-GALINDOHERVAS2003}.

Por esta razón, Agamben legitima su proyecto teórico \emph{más allá} de Foucault, en el estudio del derecho como zona privilegiada \enquote{en que se tocan las técnicas de individualización y los procedimientos totalizantes}. \footcite[15]{@7101-AGAMBEN2003} Así, ubica el centro de su investigación en la soberanía por cuánto en ella coincide el fenómeno estatal como procedimiento individualizante y totalizante a la vez. De este modo, la biopolítica en el autor viene a designar la aportación específica del poder soberano, dado que el elemento jurídico (el juego de excepción y norma ya analizado) se asienta en la vida de los hombres. Ahora bien, ¿qué entendemos por vida? Y más lejos aún, ¿es esa vida, objeto de la soberanía, un dato natural o su más íntima producción?

Partiendo de la distinción griega entre \emph{zoè} y \emph{bìos}, simple vida natural y vida cualificada, Agamben muestra cómo el sentido de dicha distinción es debilitado a partir de la instauración de la biopolítica. Ello se debe a que en el proceso biopolítico lo que se suscita es la gestación de lo \emph{bío}(en relación con lo) \emph{político}. Por lo tanto, el poder político asume allí un papel positivo, es decir, productivo, ya que encuentra su quehacer en la conversión de la vida natural en una forma de vida buena. De este modo, lo que esta biopolítica realiza es la \emph{relación} entre vida y política, donde esta afirma como fin suyo el producir aquella por medio de regulaciones, cuidados, aseguramientos. En otras palabras, los dispositivos políticos-económicos toman por objeto a la vida natural para reconvertirla en alguna forma de vida (sexuada, educada, etcétera); de algún modo lo que buscan es incluirla en el centro de sus cálculos para darle una consistencia positiva. Hasta aquí se podría decir que en este proceso se basa la biopolítica descripta por Foucault.

En este sentido, el aporte específico de Agamben será entonces haber señalado que esa producción de una vida politizada por medio de dispositivos técnicos y administrativos (y, en este sentido, económicos) lo que genera es \emph{un umbral de indistinción entre la vida como simple vida y la vida como forma de vida ya politizada}. En otras palabras, el encuentro entre la vida y los dispositivos políticos se realiza en un espacio intermedio donde lo biológico se confunde con lo político. Y si lo político, en la interpretación agambeniana, reconducía a la soberanía como el lugar donde se implicaban las técnicas individualizantes y los procedimientos totalizantes del poder, la consecuencia de ello es que la soberanía tiene como objeto inmediato la vida indistinguible de los hombres. Pero ella no es un dato natural sobre la que el poder actúa, antes bien es su más íntima producción. El carácter específico de esta producción no será tanto la gestación de formas de vida determinadas, sino el aislamiento de un algo en los hombres que debe ser politizado: la nuda vida. Respecto de ello, dice Galindo Hervás,

\begin{quote}
	(\ldots) para el italiano, que en esto sigue a Benjamin, lo característico de la política moderna y de la soberanía propia del Estado no es tanto que incluyan la \emph{zoé} en la \emph{polis}, cuanto que en ellos (y por ellos) la nuda vida, originariamente al margen de lo jurídico, coincide con el espacio político. Es decir: la vida se reduce a la referencia que sostiene el derecho, careciendo de significado \emph{per se} o, mejor, reduciéndose todo su significado al impuesto desde y por el derecho.\footcite[95]{@7088-GALINDOHERVAS2003}

\end{quote}

De este modo, si el poder tiene una faz específicamente positiva, es decir, una connotación de productividad y no ya una negativa de prohibición, censura, etcétera, como planteaba Foucault en su crítica a la \enquote{hipótesis represiva}, la aportación específica de Agamben sería señalar que \emph{el poder produce una negatividad radical: la nuda vida}. Y ello repite la doble inscripción \rdm{positiva y negativa a la vez} en que habíamos definido a la decisión. Por lo tanto, hay una analogía expresa entre el estado de excepción en que habita el soberano y la nuda vida: tanto la excepción como la nuda vida señalan el lugar de un exceso infinito que como tal, es decir, como imposible de asir completamente, convierten a todo lo que se asienta sobre ella en una maquinaria compulsiva y circular de decisión. En otras palabras, el poder soberano incluye a la nuda vida en sus cálculos para excluirla en la forma del referente al cuál debe aplicarse. Al igual que la norma se aplica sobre la excepción desaplicándose, la política se aplica sobre la vida retirándose de ella. Esta vida reducida a la pura nada, necesitada de una \emph{decisión} que la politice, es el objeto presupuesto del soberano sobre el cuál actúa. Pero si esta actuación no puede ser sino en la forma de una excepción, es decir, en la forma de una decisión, lo que resulta de ello es que la nuda vida producida por el mismo soberano queda \emph{abandonada} a la absolutez de éste, siendo ahora sobre ella todo posible, incluso su contracara negativa: la muerte.

En este sentido, el soberano comporta un gesto análogo al de la biopolítica: si en ella \enquote{los hombres quedan reducidos a vida corporal, de suyo sometible a normas científicas cuya competencia poseen los expertos}, \footcite[91]{@7088-GALINDOHERVAS2003} se presupone lo mismo que se gesta: nuda vida, vida biológica inclasificable en cuanto tal pero por ello mismo objeto de la clasificación política. Así, la nuda vida no es un dato sobre el que trabaja el poder, sino su más íntimo producto que al ser el presupuesto por el cual puede referirse a ella, le es externo también: excluida e incluida a la vez, tal es el espacio de la nuda vida. Por lo tanto, la correspondencia estructural entre nuda vida y soberano otorga la matriz comprensiva de la biopolítica estatal: \enquote{(\ldots) el Estado {[}el soberano{]} no existe fuera de los cuerpos de los individuos que lo componen {[}nuda vida{]}, de ahí que la política asuma como objeto la protección y reproducción de la vida de esos cuerpos (\ldots) los mecanismos de dominio se tornan inmanentes al campo social, difusos en el cuerpo y en los cerebros de cada individuo, intensificándose por ello la eficacia de los aparatos normalizantes. \emph{El Estado alcanza con ello una simultánea posibilidad de proteger y eliminar la vida}}.\footcite[94-95]{@7088-GALINDOHERVAS2003}[(Las cursivas son nuestras.)]

Llegamos así a la más profunda conclusión agambeniana: vida y muerte en la biopolítica occidental se trenzan en una estructura soberana que haciéndoles perder cualquier significación concreta gestan vida muerta, y ello \enquote{señala el punto en que la biopolítica se transforma necesariamente en tanatopolítica}. \footcite[180]{@7101-AGAMBEN2003}

Ahora bien, en esta analítica de la soberanía que reconduce a la paradoja en que se funda el Estado moderno, Agamben parecería haber agotado sus instancias críticas solo sobre la figura de aquel perdiendo así de vista la gubernamentalidad que acompaña y consolida lo estatal. En este sentido, su obra \textit{El reino y la gloria. Una genealogía teológica de la economía y del gobierno complementa el proyecto crítico iniciado en Homo sacer: el poder soberano y la nuda vida}, ya que en ella desarrolla el problema del gobierno en relación con el Estado por medio de una genealogía de la teología cristiana. El desarrollo de dicha genealogía para comprender el funcionamiento del poder político viene justificado por la gran influencia que el paradigma teológico tuvo en la conformación de la sociedad occidental.\footcite[13]{@7102-AGAMBEN2008} Por razones de espacio, no podremos aquí reconstruir la enorme arqueología con que Agamben ha estructurado sus análisis, pero será necesario dar cuenta de las conclusiones generales que de ella se extraen para anudarlas a las implicancias biopolíticas ya tematizadas.

El problema que la teología cristiana deja en herencia al pensamiento político será, en los análisis agambenianos, fundamentalmente la relación entre la unidad y la multiplicidad. Esta problemática, que hoy en día nos compete integralmente por cuanto implica el juego visceral de la diferencia, fue asumida en los primeros siglos de la teología cristiana a partir de la dificultad de establecer la relación entre Dios y el mundo. Contrariamente al paradigma teológico-político enunciado por Schmitt según el cual la moderna doctrina del Estado está basada en conceptos teológicos secularizados, lo que Agamben pretende demostrar es que la teología funda un paradigma no político, sino económico con el cual es articulada la relación entre Dios y el mundo, lo divino y lo profano, la eternidad y la historia; que será el basamento sobre el que se asienta el Estado moderno.\footcite[16]{@7102-AGAMBEN2008} Si la historia humana es~el ínterin previo a la segunda venida de Cristo, la relación que Dios establecerá con el mundo en este preludio temporal estará basada, de acuerdo a Agamben, en una \emph{oikonomía} en el sentido gestional y administrativo con que los griegos habían acuñado el término. Mientras que la hipótesis más común ve en esta \emph{oikonomía} teológica el \enquote{plan divino de la salvación}, \footcite[46]{@7102-AGAMBEN2008} el autor pretende ver en ella la articulación de un ensamble de reflexiones sobre actividades que poco o nada tienen que ver con la voluntad divina de la salvación. Antes bien, estos pensamientos son articulados estratégicamente en los primeros siglos de la teología cristiana a los fines de evitar una recaída en el politeísmo, ya que por medio de ellos se busca remediar las implicancias del dogma trinitario (el hecho de que Dios es uno y tres al mismo tiempo) para así mantener el monoteísmo.\footcite[71-72]{@7102-AGAMBEN2008} De este modo, dice Agamben: \enquote{{[}c{]}on un desarrollo ulterior de su significado también retórico de \enquote{disposición ordenada}, la economía es ahora actividad \rdm{en esto realmente misteriosa} que articula en una trinidad y, a su vez, mantiene y \enquote{armoniza} en unidad al ser divino}. \footcite[77]{@7102-AGAMBEN2008}

Por lo tanto, para evitar la fractura en el ser divino, los teólogos emplearon el término \emph{oikonomía} para designar la triplicidad de Dios, y distinguieron así un segundo plano donde Dios es uno en cuanto a su ser. Pero la división que se evitó en el plano ontológico, se efectuó entre el obrar de Dios (su \emph{oikonomía}) y su ser (su ontología). Así, en Dios queda separado \enquote{el ser y el obrar, la sustancia y la praxis}. \footcite[99]{@7102-AGAMBEN2008} En consecuencia, dirá el autor, la problemática de la reflexión teológica a lo largo de los siglos será re-unir esta fractura entre la ontología y la praxis, también denominada trinidad inmanente y trinidad económica.\footcite[268]{@7102-AGAMBEN2008} Y el dispositivo central con que se ha podido pensar esa re-unión ha sido la \emph{voluntad}, lo que le otorga a esta su primado central en la metafísica occidental.\footcite[104]{@7102-AGAMBEN2008} Lo que ella resuelve, siempre de un modo precario por cuanto implica una contaminación de los dos polos de la oposición, es la relación \enquote{entre un ser incapaz de acción y una acción sin ser \rdm{y entre los dos, como apuesta, la idea de libertad}}. \footcite[108]{@7102-AGAMBEN2008} La justificación de un gobierno del mundo viene dada, por lo tanto, en la articulación de una \emph{oikonomía} sin ser, esto es, en una praxis librada a su pura contingencia: solo porque no hay un principio estable que la guíe, el mundo y su historia necesitan un gobierno. Pero la aporía en que se halla contenido este, está dada por el hecho de que se encuentra librado a su propia \enquote{an-arquía}, es decir, a su falta radical de fundamento. Dios-uno se presenta como el centro ausente de la \emph{oikonomía}-múltiple, y así hace necesaria la perpetua proliferación de las medidas administrativas y gestionales de esta. Es allí donde el gobierno moderno coincide con las postulaciones teológicas cristianas.

De este modo, partiendo del dogma trinitario se introduce la \emph{oikonomía} como el obrar divino, pero entendido como una forma de relación y no ya como sustancia. Así, la máquina gubernamental que ha consolidado el paradigma teológico-económico incluye al ser-fundamento por medio de su exclusión: lo presenta en su centro en la forma de la ausencia. Esta articulación entre Dios como sustancia ausente y el gobierno divino del mundo, se consolida en el paradigma gubernamental moderno como la relación entre Reino y Gobierno, Estado y economía. La \emph{gestión} estatal de la vida que define a la biopolítica adquira así un nuevo matiz: el Estado gobierna la vida de los hombres por medio de su propia ausencia, lo que repite el gesto soberano de presentarse por medio de su disolución. Pero el centro ausente de esta maquinaria gubernamental bipolar queda oculto por medio de un dispositivo que la enceguece en su ignota y resplandeciente luz: la gloria.\footcite[323]{@7102-AGAMBEN2008} Dice Agamben:

\begin{quote}
	(\ldots) la gloria es el lugar en que la teología trata de pensar la inaccesible conciliación entre trinidad inmanente y trinidad económica, \emph{theología} y \emph{oikonomía}, ser y praxis, Dios en sí y Dios para nosotros. (\ldots) \emph{La economía glorifica al~ser, como el ser glorifica la economía}. Y solo en el espejo de la gloria ambas trinidades parecen reflejarse la una en la otra; solo en su esplendor parecen coincidir por un instante el ser y la economía, el Reino y el Gobierno.\footcite[364-365]{@7102-AGAMBEN2008}
\end{quote}

La gloria, que el autor identifica en las doxologías y aclamaciones de los cantos y alabanzas inauguradas por el cristianismo,\footnote{La finalidad última de realizar una arqueología de la gloria, en Agamben, viene dada por la necesidad de dar respuesta a la siguiente pregunta que ni politólogos ni filósofos han formulado: \enquote{[s]i el poder es esencialmente fuerza y acción eficaz, ¿por qué necesita recibir aclamaciones rituales y cantos de alabanza, vestir coronas y tiaras molestas, someterse a un inaccesible ceremonial y a un protocolo inmutable; en una palabra, inmovilizarse hieráticamente en la gloria: él, que es esencialmente operatividad y \textit{oikonomía}?}. \cite[343]{@7102-AGAMBEN2008}. Así, el autor mostrará cómo la gloria es el centro último del poder por cuanto revela y esconde al mismo tiempo la inoperosidad central \rdm{la ausencia} que se halla contenida en él.} se pliega en el carácter bifrontal con que el gobierno toma a su~cargo la vida de los hombres. La economía con que Dios se hacía cargo del mundo, queda resguardada al peligro que le supone su falta de fundamento por medio de la glorificación que se produce en su seno. La glorificación resguarda el centro vacío del poder. Y solo porque discurre en ambos sentidos, el gobierno glorifica al Estado y este glorifica a aquél, puede mantenerse unido lo que desde siempre está separado. La gloria justifica la existencia del ser, por cuánto si ella no se diera en cada una de las praxis de la vida cotidiana, el ser de Dios no solo estaría ausente y más allá de la mundanidad, sino que no existiría como tal. Sólo porque hay una glorificación del ser de Dios, la praxis de la \emph{oikonomía} no solo gira en torno a un vacío, sino que más radicalmente aún, lo produce. Esta producción infinita del vacío, que en Agamben es hallada a partir de una arqueología de la gloria/glorificación, muestra la paradoja de una operosidad que solo puede hacer un retardo de su ser inoperoso central. La gloria \rdm{doxológica, aclamativa y ceremonial} es el manto con que se cubre el ser, y solo por ello, porque reconcilia la gubernamentalidad con el vacío de todo poder \rdm{porque solo en una alabanza el ser y el hacer pueden coincidir}, se reinscribe en la soberanía de la decisión que funda Occidente.

\section{Inoperancia: una ontología política} %4.

\begin{quote}
	Empezamos a comprender por qué la doxología y el ceremonial son tan esenciales para el poder. Lo que en ellos está en cuestión es la captura y la inscripción de la inoperosidad central de la vida humana en una esfera separada. La \emph{oikonomía} del poder pone firmemente en su centro, en forma de fiesta y de gloria, lo que aparece a sus ojos como la inoperosidad del hombre y de Dios, inoperosidad que no se puede mirar. La vida humana es inoperosa y sin objetivo, pero precisamente esta \emph{argía} y esta ausencia de objetivo hacen posible la operosidad incomparable de la especie humana. El hombre se ha consagrado a la producción y al trabajo porque en su esencia está totalmente privado de obra, porque él es por excelencia un animal sabático. Y así como la máquina de la \emph{oikonomía} teológica solo puede funcionar si inscribe en su centro un umbral doxológico en el que Trinidad económica y Trinidad inmanente se relacionan de manera incesante y litúrgica (es decir política), así el dispositivo gubernamental funciona porque ha capturado en su centro vacío la inoperosidad de la esencia humana. Esta inoperosidad es la sustancia política de Occidente, el nutriente glorioso de todo poder.\footcite[428--429]{@7102-AGAMBEN2008}
\end{quote}

A lo largo de todo este trabajo, hemos ido reconduciendo todas las instancias problemáticas a un dualismo donde mostrábamos cómo progresivamente los dos términos de la dualidad quedaban contaminados mutuamente, y progresivamente se confundían entre ambos dando lugar a una situación paradójica. Así lo hemos hecho con el ser y el ente, la excepción y la norma, la vida natural (\emph{zoé}) y la vida política (\emph{bíos}), Dios-uno y \emph{oikonomía}-múltiple, inoperosidad y operosidad. En el \emph{entre} de todos ellos, donde se producía su relación, lo que se encuentra es una ausencia presente que, según la extensa cita que abre este apartado, no se puede mirar: la inoperosidad de la vida. Los términos utilizados por Agamben en dicho extracto, remiten claramente a un cierto esencialismo y a un humanismo en su postura que tendrá importantes consecuencias ontológicas y políticas para su pensamiento. En primer lugar, la inoperosidad central de la vida la remite a la estructura definida como nuda vida. Con ella, como mostramos, el poder establece una relación de excepción por cuánto la incluye en sus cálculos en la forma de su exclusión. Esto significa ni más ni menos que el referente del poder es la vida aislada de las múltiples formas de vida concreta. Y este aislamiento será según Agamben la cuota metafísica distintiva de los dispositivos de poder en occidente: así como en ellos lo que se hace es aislar la vida \rdm{su inoperosidad} de sus múltiples formas concretas \rdm{operosas}, en el ámbito metafísico lo que se hace es aislar el ser puro entre los múltiples significados del ser. Este ser puro, denominado \emph{haplós}, se corresponde al adjetivo \enquote{nuda} con que el autor define la vida que se produce en Occidente.\footcite[Véase][96-98]{@7088-GALINDOHERVAS2003} Por lo tanto, cuando se establece la analogía radical entre vida y ser, se entiende por qué para el autor la metafísica no puede ser sino la política, y viceversa. Ello se debe a que la política, en su forma extrema de la soberanía, aísla la vida para así politizarla. Lo que se realiza en el dispositivo gubernamental es, entonces, el abandono de la nuda vida.

Contrario al aislamiento metafísico de la vida, Agamben propone un replanteamiento de la ontología que deje sin sustento a la soberanía.\footcite[Véase][205]{@7088-GALINDOHERVAS2003} De lo que se tratará es de pensar cómo la inoperosidad central de la vida es ella misma una \emph{forma de vida}, y no ya el centro oculto por medio del cual es necesario la politización de la misma. Esto se corresponde a la diferencia ontológica no al modo de una diferencia del ser con el ente, sino a una diferenciación intrínseca del ser. Así:

\begin{quote}
	Agamben pretende abandonar un pensamiento en el que las posibilidades inagotables del ser se reduzcan y remitan a la efectividad de los entes o, lo que es lo mismo, un pensamiento en el que la nuda vida en tanto que \emph{posibilidad} permanezca oculta en \emph{formas de vida} que niegan su carácter de ilimitada apertura. Un pensamiento tal impide la experiencia del individuo como permanente posibilidad, como vacío disponible. Si toda determinación es ya una negación, se comprende que sugiera un genérico modo de existencia en la potencia.\footcite[Véase][207]{@7088-GALINDOHERVAS2003}
\end{quote}

Por lo tanto, el correcto modo de entender una ontología de la inoperancia es remitirlo a la potencia radical que la inoperosidad supone. La connotación ontológica que define a la potencia es el poder \emph{no} no-ser. Esto significa que el existente es aquél que pudiendo realizar el acto, también puede no realizarlo. Mientras que la soberanía puede realizar el acto en el punto que se desprende de su potencia de no realizarlo, la inoperancia esencial de los hombres, es decir, el \emph{poder ser} que allí queda apresado es justamente el motor de la soberanía, su condición inasible: esto es, la decisión absoluta. No se trata tanto ahora de mostrar la imposibilidad, sino la radical \enquote{posibilidad de} que ello implica. Por lo tanto, lo que la inoperancia contenida en el centro del poder soberano viene a posibilitar es la apropiación de ella, la radical posibilidad de restituir la vida a sus formas de vida, el ser a sus modos de ser. Y sobre el plano existencial de la potencia, del \emph{poder ser}, que prima siempre sobre lo efectivo que es, se restituye al hombre a su \emph{facticidad}. El objeto no es ya un acto (politizar la vida), sino la potencia misma. La vida, por cuanto es posibilidad de, facticidad, ya es en sí misma política. Presenta el vacío que es en tanto posibilitante, en tanto despliegue de su pura exposición.\footnote{En este punto de pura autoexposición del ser y de la vida, donde la obra se descompone en su inoperosidad y coinciden plenamente, Agamben reconoce e inscribe su proyecto teórico en el campo de la inmanencia absoluta abierto por Spinoza y continuado por Deleuze. \cite[][85]{@7099-DELEUZE2007}.}

El ser que se desprende de esta vida política es un ser que corroe la soberanía, impide su decisión por cuanto él es desde siempre una decisión: ser- \emph{así}. En este sentido, la inoperancia vendría a retomar en el proyecto agambeniano lo que se había anunciado en \emph{La comunidad que viene}: \enquote{el ser-así no es una sustancia, de la \emph{así} expresaría una determinación o una cualificación. El ser no es un presupuesto que esté antes o después de su cualidad. El ser, que es así, irreparablemente, \emph{es} su \emph{así} y solo su modo de ser} \footcite[][81]{@7094-AGAMBEN2003}
. En la inoperancia, el ser es la decisión de existir. De este modo, sostendremos por lo ya dicho, que el pensamiento agambeniano presenta una ontología política por cuánto considera a la vida como \emph{poder ser} (o, lo que es lo mismo, poder no no-ser): solo porque la vida humana es inoperosa, no tiene destino ni obra que le pertenezca de modo alguno, está radicalmente abierta al poder ser, donde lo político se resuelve por el lado del poder, y lo ontológico por el lado del ser. Así, el ser es siempre poder y el poder es siempre ser. Este \emph{poder-ser}, que por medio de los dispositivos políticos-económicos, queda apresado como el combustible más preciado del soberano sobre el que separa un ser (una vida) del resto de los modos de ser (de las formas de vida), una vez restituido al hombre le abre su posibilidad de ser (rompe con la entificación del ser). A este respecto, afirma Agamben: \enquote{{[}a{]} la potencia y a la posibilidad, en cuanto diferente de la realidad efectiva, parece serle inherente la forma del \emph{cualsea}, un irreductible carácter de cualquieridad}. \footcite[][37]{@7094-AGAMBEN2003} Esto significa, ni más ni menos, que lo diferente a la realidad efectiva de los entes no es solo un ser que en cada caso se presenta como la contracara negativa de ellos. Antes bien, es la cualquieridad, la multiplicidad de lo común. La ontología de la inoperancia abre en la actualidad de los entes, la cualquieridad de su potencia. Esta ontología no refiere a la vida como un ser \emph{en} potencia, sino más bien, como afirma Rodrigo Karmy Bolton, como \enquote{un ser \emph{de} potencia}. \footcite[][9]{@7089-KARMYBOLTON2010} La vida, como el ser de potencia, es política por cuanto no se agota nunca en ninguna forma de vida, en ninguna identidad, en ninguna propiedad. El esencialismo que de aquí se desprende es uno de tipo particular: el hombre es inmediatamente inesencial, su propiedad más íntima es su impropiedad, lo que lo abre desde siempre a lo común, la cualquieridad comunitaria. Esta reformulación de la ontología desactiva la separación efectuada por el soberano, ya que de lo que se trata es de aferrarse a la impropiedad que nos constituye. En otras palabras, la política implicará restituir al hombre, a sus formas de vida, su nuda vida. Imposibilitar la separación de la vida, del ser, de la política, de lo que ya desde siempre es ser-vida-política.

Por lo tanto, en contra de la escisión teológica cristiana entre ser y praxis, la inoperancia designa una praxis específicamente ontológica: poder ser inoperoso. Y esto supone restituir al ser su pura diferencia, no ya respecto de los entes efectivos, sino en sí. La inoperancia implicada así, lleva a desarticular los dispositivos biopolíticos y gloriosos con que la vida es apresada, separada y ocultada en la forma soberana. Conlleva, a su vez, impedir una decisión inaudita por cuanto lo que es, ya es en sí mismo una decisión. La propia singularidad del \enquote{ser cualsea} con que Agamben había definido la ontología de una comunidad,\footcite[Véase][7 a 9]{@7089-KARMYBOLTON2010} viene dada por esta condición inoperosa de la vida.\footnote{La \emph{désoeuvrement} (el término francés que traducido al español significa inoperancia o desocupación), dice Agamben, \enquote{{[}n{]}o puede ser ni la simple ausencia de actividad ni (como en Bataille) una forma soberana y sin empleo de la negatividad. La única forma coherente de entender la desocupación {[}inoperancia{]} sería pensarla como un modo de existencia genérica de la potencia, que no se agota (como la acción individual o la colectiva, entendida como la suma de las acciones individuales) en un \emph{transitus de potentia ad actum}}. \cite[][82]{@7101-AGAMBEN2003}. En este sentido, la inoperancia en Agamben diferirá de aquella presentada por Nancy dado que este la seguirá pensando en el sentido batailleano de lo sagrado, esto es, como puro gasto. Véase \cite[][43-46]{@7100-NANCY2000}. El problema de ello sería, para Agamben, que lo sagrado como puro gasto repite el gesto soberano de un algo aislado en los hombres que no tiene finalidad alguna sino que funciona como el combustible que permite el eterno circuito de la politización de la vida. Véase \cite[][144-147]{@7101-AGAMBEN2003}. Pensar en cambio la inoperancia como potencia impide este proceder de la soberanía por cuanto lo que ha de politizarse, ya es en sí mismo político. Y ello desactiva la maquinaria soberana.} La vida pertenece a la vida misma, y en ello consiste la política.

Por todo ello, solo en la facticidad de ser tal cual se es, es pensable algo así como una política sustraída a la metafísica soberana que aísla un núcleo sobre el cuál operar. Pero ello no reconduce al mundo de los entes, sino que se busca el ser y la pluralidad de sus modos de ser pero sin relación a los entes (se mantienen, pero sin relación).\footcite[][82]{@7094-AGAMBEN2003} Esta forma inactiva de la existencia \rdm{existir en la decisión de ser} no busca ya glorificar el centro ausente de la maquinaria gubernamental biopolítica, sino antes bien volver inoperosas las cosas, esto es, acabar con el circuito interminable de la tecnificación mostrando su radical apertura a la pluralidad, a la diferencia \emph{per se}. El \emph{factum} político no es ya una relación, una vinculación, sino pura exposición del ser: \emph{poder ser}.

\section{Conclusión}


En este trabajo hemos intentado mantener una intermitencia entre tres términos \rdm{ser, vida y política} para mostrar las dos direcciones posibles que una ontología antiesencial efectúa sobre el campo de la teoría política. En este sentido, hemos opuesto la lógica de la decisión a la de la inoperancia como respuestas posibles ante la conciencia de una falta de esencia que determine el ser político. La apuesta de este trabajo comenzó recuperando los desarrollos de Oliver Marchart para establecer el contrapunto con la obra agambeniana, con el objetivo de mostrar que la problemática de las categorías políticas modernas no se agotaba una vez que estas eran deconstruidas. Los análisis agambenianos sobre la soberanía y la gubernamentalidad mostraban, en cambio, que la insustancialidad de las prácticas políticas modernas era justamente lo que motorizaba continuamente dichos dispositivos de poder. Para ser consecuentes con nuestro objetivo, en todo momento buscamos mostrar cómo en la lógica decisionalimperante en dichos dispositivos, que responde a una ontología del acto, se producía un determinado ser (indiferenciado, es decir, entificado), una determinada política (la biopolítica) y una determinada vida (la nuda vida). Finalmente, sin ánimos de exhaustividad, intentamos delinear los postulados en que habría de pensarse una ontología de la inoperancia que posibilitara una concepción de la vida como inmediatamente política y ontológica: esto es, como la instancia en que el ser es necesariamente político. De este modo, buscamos respondernos la pregunta central sobre la politicidad de la ontología, en otras palabras, la politicidad de la existencia misma. El concepto de inoperancia trabajado por Agamben nos otorgó el marco para circunscribir dicho intento. Los interrogantes y los caminos que aquí se abren son infinitos, por lo que este trabajo solo presenta una posibilidad entre muchas más. Nunca fue otra la apuesta aquí esbozada.

\section*{Referencias}
\printbibliography[heading=none]   % Sin título automático



%%%%%%%%%%%%%%%%%%%%%%%
\ifPDF
\separata{capitulo5}
\fi

\ifPDF
\chapter[\hspace{1.5pc}Ontología genealógica]{Ontología genealógica}
\chaptermark{Ontología genealógica}
\Author{María Aurora Romero}
\setcounter{PrimPag}{\theCurrentPage}
% encabezado para autor
\begin{center}
	\nombreautor{María Aurora Romero}\\
	\vspace{15mm}
\end{center}
\fi

\ifBNPDF
\chapter[\hspace{1.5pc}Ontología genealógica]{Ontología genealógica}
\chaptermark{Ontología genealógica}
\Author{María Aurora Romero}
% encabezado para autor
\begin{center}
	\nombreautor{María Aurora Romero}\\
	\vspace{15mm}
\end{center}
\fi

\ifHTMLEPUB
\chapter{Ontología genealógica}
\fi


\section{Introducción}

El presente texto buscará ensayar una forma de comprender la empresa genealógica de Michel Foucault en el intersticio de modalidades de veridicción, formas de gubernamentalidad y procesos de subjetivación, que se anudan en una ontología crítica del presente. La propuesta de una lectura de \emph{lo ontológico} del pensamiento político foucaultiano arribará en la problematización de lo que somos en tanto \emph{diferencia} de nuestro presente. Tal lectura buscará desplazarse de aquellas interpretaciones que buscan distinguir, esquematizar y compartimentar la (no)obra de Foucault en momentos teóricos o etapas cronológicas, que responderían a distintas metodologías para abordar diferentes objetos de estudio. El movimiento del presente trabajo buscará ensayar una forma de grilla de análisis donde se superponen y se constituyen mutuamente las dimensiones del saber, del poder y del sujeto. Consideramos tal forma de abordaje deudora del pensamiento foucaultiano, sin por ello inscribirnos ni los debates que versan sobre la correcta interpretación de la obra foucaultiana, como tampoco en las disputas que buscan imputarle o negarle ciertos principios de coherencia o consistencia teórica y/o metodológica. El siguiente trabajo busca apropiarse de unas herramientas teóricas, para reutilizarlas, para forzarlas hasta desfigurar la forma de un pensamiento como el de Foucault. Así, la propuesta será problematizar \emph{a partir}, \emph{alrededor}, o hasta incluso \emph{por fuera} de Foucault como una experiencia de pensamiento, que sabiéndose perspectiva de las maneras de ser lo que somos, ficciona las posibilidades de transformarnos. En este sentido, intentamos acercarnos a Foucault, tal como él consideraba la mejor forma de \emph{reconocer} el pensamiento de un autor, es decir, utilizarlo e incluso deformarlo para llevarlo hasta sus límites.\footnote{\enquote{La única marca de reconocimiento que se puede testimoniar a un pensamiento como el de Nietzsche es precisamente utilizarlo, deformarlo, hacerlo chirriar, llevarlo al límite. Mientras tanto, los comentaristas se dedican a decir si se es fiel al texto, algo que carece del menor interés}. \cite[Véase][610]{@7038-FOUCAULT2010}.}

La indagación sobre la dimensión ontológica de la genealogía foucaultiana comenzará recogiendo la pregunta por la Ilustración como un diagnóstico del presente a partir de las formas contingentes en las que nos hemos constituido de determinadas maneras. En un movimiento que subvierte y se reinscribe en la tradición kantiana, la empresa de Foucault buscará reabrir la pregunta por la Ilustración como el legado de trabajo crítico a asumir. La lectura foucaultiana de Kant sobre la \emph{Aufklärung} exalta la reflexión del presente como \emph{diferencia}, donde la \emph{crítica} deviene como la actitud específica de Occidente, que toma la forma de sospecha sobre aquellos excesos de poder que la razón posibilita y reproduce históricamente. Una forma de pensamiento crítico deviene a partir de una problematización ontológica de nuestras maneras históricas, y por tanto, contingentes de ser, pensar y actuar.

Como propuesta de una ontología \emph{genealógica}, a continuación, se revisará la recepción foucaultiana de la noción de genealogía de Nietzsche en cuanto análisis de las emergencias y procedencias. La ligazón que une Nietzsche a Foucault, se hallará en perspectiva genealógica de la historia, como una interrogación filosófica que anuda el presente y el pasado, la actualidad y el origen, para mostrar la contingencia de aquellos universales antropológicos que se han instaurado en el tiempo. En un movimiento desde un presente problemático, la genealogía se lanza al pasado para mostrar, por un lado, la emergencia del acontecer que sigue atravesándonos, y por otro lado, la procedencia de las marcas actuales de aquel conjunto disperso de fallas y fisuras sobre la que nos hemos constituido. Sin embargo, la genealogía como opción no solo metodológica, sino también táctica y estratégica, aborda nuestras maneras de ser, pensar y actuar, buscando en el pasado que estamos dejando de ser, la posibilidad futura de devenir en otras formas.

En el marco de la reflexión, se explorará la relación entre poder y libertad, ya no desde su oposición, sino en la co-pertenencia en la que se constituyen mutuamente. Una ontología política de lo que somos, se relacionará para Foucault con un \emph{principio de~libertad} como \emph{poder}, y por tanto, como una \emph{capacidad de hacer}, que abre la posibilidad de resistencia. En el pensamiento foucaultiano, libertad, poder y resistencia lejos de oponerse, se imbrican en la posibilidad de crear prácticas de sí y con otros, que nos hagan ser más libres. La noción de \emph{gubernamentalidad} enmarca la forma de pensar cierto conjunto de prácticas a partir de las cuales se puedan definir estrategias que los individuos en su libertad puedan establecer unos con otros. En este sentido, la ontología histórica foucaultiana como \emph{crítica} consiste en pensar un nuevo régimen de gobierno de libertades, como un gesto que invita a ficcionar nuevas formas de vida como modalidades de existencia.

En la forma genealógica de un pensamiento crítico, consideramos que es posible encontrar entre lo óntico y lo ontológico, un Foucault que ficciona, que juega sobre los límites, los bordes y las fronteras de lo que hemos sido y seguimos siendo hoy. Como esbozo y aporte del presente texto, abordaremos la relación que se entreteje entre pensamiento e historia, donde se abre la posibilidad de problematizar la forma de una \emph{ontología genealógica}. La problematización de las condiciones de posibilidad históricas sobre las que se apoya una ontología de nosotros mismos, puede ser entendida como un pliegue ontológico de la historia, donde ciertas relaciones estratégicas entre libertades como poder y ciertos juegos de verdad como prácticas, se encuentran siempre constituyendo nuestras maneras de ser. Una ontología genealógica, de como un permanente devenir histórico, no solo comprenderá las diferencias y transformaciones en las variantes condiciones empíricas, sino que abordará las necesarias diferencias de la pluralidad de \emph{formas} de ser. En incluso, sobre una ontología genealógica de los modos transcendentalmente contingentes (o históricos) de ser, se puede vislumbrar la posibilidad de apertura a nuevas formas de subjetivación.

\section{\emph{Aufklärung}: una ontología crítica del presente}


\epigraph{\emph{No sé si hace falta decir hoy que el trabajo crítico todavía implica la fe en la Ilustración; pienso que sigue necesitando el trabajo sobre nuestros límites, es decir, una labor paciente que le dé forma a la impaciencia por la libertad}}{Michel Foucault}


Una fe en la Ilustración, un \emph{ethos} filosófico, deviene en una actitud límite para el pensamiento crítico de un \emph{nosotros}. La lectura foucaultiana de Kant sobre la \emph{Aufklärung} busca mostrar la emergencia de un pensamiento que introduce la reflexión del presente como \emph{diferencia}. En un movimiento de subvierte y se reinscribe e la tradición kantiana, la empresa de Foucault buscará reabrir la pregunta por la Ilustración como un legado de trabajo crítico a asumir, en el punto en que la actualidad que somos, nos diferencia. \emph{Aufklärung} es para Foucault una pregunta crítica sobre ¿Qué es nuestra actualidad?, una pregunta por lo que nos pasa en el límite de lo que hoy somos. A partir de esta pregunta, desarrolla su trabajo dentro de un pensamiento crítico que tomará la forma de una ontología de nosotros mismos, de una ontología de la actualidad. La cuestión gira en torno a interrogar el presente como un signo de la actualidad de las Luces, en términos de nuestra pertenencia a ellas. El sello de la Ilustración para Foucault se halla en la resistencia a la autoridad, por lo que inscribe su continuidad con los fines de las Luces, donde la actitud crítica aparece toda vez que con ella se busca \emph{lo impensado} dentro de los propios términos de la Ilustración.\footcite[][]{@7036-BUTLER2001} La actitud propia de la modernidad será entendida como una crítica permanente de nosotros mismos en tanto sujetos de la Ilustración. El \emph{ethos} filosófico legado por Foucault nos impone transitar en la reflexión de nuestras fronteras, donde la actitud límite sortea los afueras y los adentros, esquivándolos. De tal \emph{ethos} filosófico emerge \enquote{una prueba histórico-práctica de los límites que podemos franquear} sobre nosotros mismos en nuestra impaciencia por la libertad.\footcite[][104]{@7037-FOUCAULT2002}

¿Quiénes somos? Constituye la pregunta que nos remite hacia el pasado a recorrer la genealogía de nuestras identidades, para poder pensarlas en su propia precariedad histórica. Pero además, ¿quiénes somos? nos lleva, hacia adelante a una conversión ética de nosotros mismos, a la autoinvención política de nuevas subjetividades. \enquote{La historicidad de nuestro ser (\ldots) no conduce a un relativismo de valores y a un nihilismo de la acción, sino a la provocación de nuestras libertades, desafiadas por la invención de nuevas modalidades de ser}. \footcite[][129-130]{@7039-GROS2007} La pregunta por la \emph{Aufklärung} muestra una doble orientación del pensamiento a partir del valor otorgado al presente, y a través del anudamiento de teoría y práctica que de él se deriva.\footcite{@7040-LEBLANC2008} La vuelta sobre la idea kantiana de una ontología crítica del presente busca, no solo comprender, sino dibujar los límites de lo que funda el espacio de nuestro discurso. El trabajo genealógico fuerza una historización de la razón y la verdad para mostrar las contingencias y las discontinuidades de las condiciones de posibilidad sobre las que se fundan los eventos. La crítica como práctica, expone los contornos de un horizonte epistemológico para mostrar cómo aparecen por primera vez puestos en relación con su propio límite.\footcite[][5]{@7036-BUTLER2001} El ejercicio de tal pensamiento crítico como experiencia, no solo prescinde de las tradicionales ambiciones normativas, sino que busca indagar, presionar y contraatacar a aquello que nos es dado como universal, necesario y obligatorio, con la singularidad contingente de la arbitrariedad coactiva que realmente lo posibilita.

De la misma manera que Kant buscaba diferenciar el hoy respecto del ayer, para Foucault de lo que se trata hoy es de deslindar posibilidades de ruptura y cambio de la contingencia histórica que nos hace ser lo que somos. En términos de Revel \enquote{plantear la cuestión de la actualidad, pues, equivale a definir el proyecto de una crítica práctica en la forma de franqueamiento de lo posible}. \footcite[][13]{@7041-REVEL2008} Esta actitud filosófica debe traducirse en un trabajo de investigación histórica que permita desanudar los acontecimientos partir de los cuales nos hemos constituido en sujetos que se reconocen a través de lo que hacen, dicen y piensan, es decir, las formas históricamente singulares en que han sido problematizadas las generalidades de nuestra relación con nosotros mismos y con los demás. El \emph{ethos} filosófico como crítica permanente de nuestro ser histórico, tiene la finalidad genealógica de extraer de \enquote{la contingencia que nos hizo ser lo que somos la posibilidad de ya no ser, hacer o pensar lo que somos, hacemos o pensamos}. \footcite[][102]{@7037-FOUCAULT2002} La contingencia que expone en su diagnóstico del presente se radicaliza mostrándose necesaria en la brecha abierta entre un pasado diferente al presente disciplinario, donde se apuesta hacia la posibilidad de nuevas formas de sujeción en un futuro.

\emph{Aufklärung} para Kant, \emph{crítica} para Foucault deviene como la actitud específica de Occidente a partir de la gubernamentalización de la sociedad, donde la interrogación sobre ¿qué es nuestra actualidad?, sobre ¿qué (nos) pasa? toma la forma de desconfianza, de sospecha sobre aquellos excesos de poder que la razón posibilita y reproduce históricamente. El nexo entre la gubernamentalización y la crítica, se puede hallar en la pregunta de \enquote{cómo no ser gobernado de \emph{esa forma}}, \footcite[][7]{@7042-FOUCAULT1995} donde se anudan las dimensiones de poder, verdad y sujeto. Si la gubernamentalización de la sociedad es el movimiento donde se trata de \emph{sujetar} a los individuos a través de unos mecanismos de poder/saber, la crítica se hallaría en el reverso de este movimiento, como el derecho a interrogar a la verdad acerca de sus efectos de poder como al poder acerca de sus discursos de verdad. \enquote{La crítica tendrá esencialmente por función la desujeción en el juego de lo que se podría denominar, con una palabra, la política de la verdad}. \footcite[][8]{@7042-FOUCAULT1995} De este modo, la posibilidad de desujeción se halla dentro de la racionalización, sin que por esto haya que asumir que la \enquote{fuente para la resistencia (\ldots) esté alojada en el sujeto o conservada de una manera fundacional}. \footcite[][10]{@7036-BUTLER2001}

A partir del diagnóstico de nuestra actualidad, la pregunta por la Ilustración recoge una dimensión ontológica de lo político en su pliegue con el saber y la subjetividad. La ontología histórica del presente se pregunta por la Ilustración como un trabajo crítico sobre las interrelaciones constitutivas de dispositivos de poder y regímenes de saber que nos constituyen como sujetos a través de las relaciones que entablamos nosotros mismos y con los otros. De este modo, la dirección que abre una ontología genealógica, apunta no a descubrir lo que somos, sino rechazar lo que somos con el objetivo de creación de libertad, porque debemos librarnos de la doble coerción política de individualización y totalización de las estructuras del poder moderno. El problema ético, político y filosófico significativo ya no es liberar al individuo del Estado, sino de liberarnos a nosotros mismos de estas formas de poder que nos han impuesto un tipo de individualidad a través de nuevas formas de subjetivación que nos permitan ser más libres.\footcite[][]{@7043-FOUCAULT1986}

\section{Sentido histórico y genealogía: \emph{Herkunft} y \emph{Entstehung}}


\epigraph{\emph{Si fuese pretencioso, pondría como título general de lo que hago: genealogía de la moral}}{Michel Foucault}


En el prólogo de su \emph{Genealogía de la Moral}, Nietzsche se pregunta: \enquote{¿quiénes \emph{somos} nosotros en realidad?}. \footcite[][22]{@7044-NIETZSCHE2009} Nos invade siempre un extrañamiento que nos impide conocernos, nos dice \enquote{en lo que a nosotros se refiere no somos \enquote{los que conocemos}}. \footcite[][22]{@7044-NIETZSCHE2009} Foucault recoge la pregunta nietzscheana para preguntarse ¿qué nos pasa? como un diagnóstico de nuestra actualidad, como una genealogía de las procedencias y emergencias que nos hacen ser lo que somos. La ontología histórica foucaultiana, como un abordaje genealógico, comenzará su movimiento desde el presente para recorrer la historia de las emergencias y las procedencias de las prácticas. Foucault toma, deforma y utiliza el pensamiento nietzscheano con la necesidad de negarse a otorgar un origen metafísico a \emph{lo dado}, en una rehabilitación de la genealogía nietzschiana que condena todo fundamento metafísico de la historia, y todo uso de la historia como fundamento del presente. Por tanto, el objeto de la genealogía no se encontrará jamás del lado del origen (\emph{Ursprung}) sino del de la invención (\emph{Enfindung}).\footnote{Foucault retoma el tema nietzscheano de la diferencia entre el origen e invención. El autor considera que se debe optar \emph{Herkunft} y \emph{Entstehung}, de entre los términos traducidos todos como origen, por considerar que \enquote{indican mejor que \emph{Ursprung} el objeto de la genealogía}. \cite[][24]{@7045-FOUCAULT2008}.} El comienzo como invención, como producción humana en un determinado momento de la historia, se orienta a disolver las invariantes propias de los racionalismos. La genealogía trabaja con positividades, más allá y más acá de las palabras y los hechos. Más allá de estos para encontrar sus condiciones de posibilidad, al mismo tiempo, que más acá para desarmar la urdimbre en la que se constituyeron.\footcite[][87]{@7046-DIAZ2005} La genealogía busca restituir los acontecimientos en su singularidad, en el trabajo continuo sobre las diversidades, las dispersiones y las rupturas de la historia.

Foucault toma la noción nietzscheana de \emph{Herkunft} (procedencia), para abordar el análisis de las \emph{marcas} actuales de los acontecimientos pasados. La indagación sobre la procedencia, desde el presente de aquellas marcas singulares que se entretejieron y entrecruzaron en el pasado, busca desanudar ese lugar donde \enquote{el alma busca unificarse, allí donde el Yo se inventa una identidad}. \footcite[][26]{@7047-FOUCAULT2008} Lejos de ser una categoría del orden de la semejanza, el análisis genealógico como procedencia busca \enquote{conservar lo que ha sucedido en su propia dispersión, (\ldots) descubrir que en la raíz de los que conocemos y de lo que somos no hay ni ser ni la verdad, sino la exterioridad del accidente}. \footcite[][27-28]{@7047-FOUCAULT2008} La peligrosa herencia trasmitida a través de la procedencia, no es una adquisición que se acumule, sino un conjunto de fallas y fisuras, que perturban al heredero dejándolo en una situación de fragilidad. La búsqueda de la procedencia no fundamenta, sino que desestabiliza lo inmóvil, fragmenta lo unido para mostrar la heterogeneidad de lo que pensábamos conforme a sí mismo. La \emph{Herkunft} se inscribe en el cuerpo, el cuerpo como superficie de inscripción de los acontecimientos se vuelve el lugar donde se disocia el yo, como un \enquote{volumen en perpetuo derrumbamiento}.\footcite[][30]{@7047-FOUCAULT2008} La genealogía, como análisis de la procedencia, se halla en el intersticio donde cuerpo e historia se pliegan, donde el cuerpo se impregna de historia y la historia se presenta como destruyendo el cuerpo.

Sin embargo, la recepción foucaultiana de la genealogía de Nietzsche no solo apunta a la procedencia como marca actual (\emph{Herkunft)}, sino que también aborda la \emph{Entstehung} como \emph{la emergencia del acontecer}. La procedencia como marca actual no puede dar cuenta del punto de surgimiento, ya que esta no es más que \enquote{el episodio actual de una serie de sometimientos}. \footcite[][33]{@7047-FOUCAULT2008} Al igual que sería equivocado buscar la procedencia en una continuidad sin interrupción, también lo sería el intento de explicar la emergencia por el último episodio. Frente a la metafísica que sitúa el presente en el~origen para imputar un destino que se manifestaría desde el primer momento, la genealogía restablece el juego azaroso inscripto en unos sistemas de sumisión. La emergencia aflora en y desde un determinado estado de fuerzas. La emergencia no es más que la entrada en escena de las fuerzas, su irrupción, el empuje por el que saltan a un primer plano. La emergencia designa un lugar de enfrentamiento, en realidad, define un no-lugar como la pura distancia de la no pertenencia de los adversarios al mismo espacio. La emergencia abre el intersticio entre quienes se enfrentan como un vacío donde intercambian sus amenazas.

La historia será para Foucault el instrumento de la genealogía. La ontología foucaultiana del presente se funda en una historia \emph{efectiva}, en \enquote{el sentido histórico {[}que{]} da al saber la posibilidad de hacer, en el movimiento mismo de su conocimiento, su genealogía}. \footcite[][54]{@7047-FOUCAULT2008} Hay que librar una batalla contra \enquote{el desplegamiento metahistórico de las significaciones ideales}. \footcite[][13]{@7047-FOUCAULT2008} Hay que adueñarse de la historia para hacer de ella un uso genealógico. Para Foucault tal empresa implica ejercer un uso de la historia rigurosamente anti-platónico. La apropiación foucaultiana de Nietzsche recuperará, contra la historia como reminiscencia, un uso paródico, frente la historia continuista como tradición, un uso disociativo, y, por último, frente a la historia como conocimiento, un uso sacrificatorio de la verdad. Para liberar la historia del modelo metafísico y antropológico de la memoria, para mostrar cómo ciertas realida\-des,~identidades y verdades que aparecen como condiciones universales, no se han sino establecido a través de una serie de sometimientos. Recuperar en el abordaje genealógico \emph{el sentido histórico}, implica reconocer que \enquote{vivimos, sin jalones ni coordenadas originarias, en miríadas de acontecimientos perdidos}. \footcite[][50-51]{@7047-FOUCAULT2008} La historia \enquote{nos enseña a reírnos de las solemnidades del origen}, \footcite[][19]{@7047-FOUCAULT2008} y por tanto, nos revela la pequeñez meticulosa e inconfesable de la serie de fabricaciones e invenciones que nos constituyen de manera contingente. Para Foucault \enquote{la historia será \enquote{efectiva} en la medida en que introduzca lo discontinuo en nuestro mismo ser}. \footcite[][47]{@7047-FOUCAULT2008}

El análisis genealógico tomará por objeto el saber histórico en conjunción con unas relaciones de poder, como el estado de fuerzas donde se inscribe. Por lo tanto, es la diferencia que separa al pasado del presente, lo que tomará la historia genealógica para interrogar la voluntad de poder que ha determinado una cierta constitución de lo verdadero. El principio genealógico analizará los discursos en tanto acontecimientos, y por tanto, aborda los puntos de discontinuidad y exterioridad para mostrar sus condiciones~de aparición y transformación. El análisis de estas formaciones dispersas, discontinuas y regulares de los discursos, busca captar el poder en su dimensión de afirmación, es decir, analizando los efectos de verdad que el poder produce y que le permiten reproducirse. El poder produce verdad, al mismo tiempo que, solo a partir de cierto régimen de verdad es posible el ejercicio del poder. Esto plantea que nos encontramos disciplinados a través de la producción de verdad a la que el poder nos somete, al mismo tiempo, que solo podemos ejercer el poder a través de ella. \enquote{No hay ejercicio del poder sin una cierta economía de los discursos de verdad que funcionan en, a partir y a través de ese poder}. \footcite[][34]{@7048-FOUCAULT2000} El análisis genealógico se orienta en la dirección de los comportamientos, de las luchas, de los conflictos y de las tácticas, que hacen aparecer un saber político regularmente formado por una práctica discursiva, que en su articulación con otras prácticas encuentra su especificación, sus funciones y la red de sus dependencias. Así, la genealogía funciona como un método de crítica social inmanente, que analiza las relaciones entre sistemas de verdad y modalidades de poder que configuran un \emph{régimen político de verdad.} Una policía discursiva que introduce la disciplina como un control de la producción del discurso, fija y reactiva las reglas que establecen cuáles enunciados podrían caracterizarse como verdaderos o falsos para un discurso dado. Sin embargo, el método genealógico es también una forma de intervención política, como una tentativa de liberar los saberes históricos de la sujeción de un orden de discurso que se presenta como unitario, formal y científico. Como una insurrección de los saberes, apunta ya no solo contra los contenidos, los métodos y los conceptos de una ciencia, sino contra los efectos de poder centralizadores dados al funcionamiento institucional de una sociedad.

En este marco, la genealogía, a la vez como método y finalidad, será el redescubrimiento meticuloso de las luchas y la memoria de estos enfrentamientos, que permiten configurar las tácticas actuales para enfrentarlo. La genealogía posibilitará hacer visible el saber histórico de las luchas en el acoplamiento de los saberes sepultos de la erudi­ción y los saberes descalificados por la jerarquía del cono­cimiento. La genealogía en su finalidad de restituir saberes sometidos, se conforma como una anti-ciencia que abre la posibilidad de hacer entrar en juego los saberes locales, discontinuos y descalificados, contra una policía discursiva que los pretende filtrar, jerarquizar y ordenar en nombre de un conocimiento verdadero. La historia se convierte en un saber que funciona en un campo de luchas, donde el combate político y el saber histórico se encuentran entrelazados mutuamente: \enquote{la historia nos aportó la idea de que estamos en guerra, y nos hacemos la guerra a través de la historia}. \footcite[][161]{@7048-FOUCAULT2000} Por lo tanto, una genealogía como táctica, no solo busca la huella de los acontecimientos singulares en el pasado, sino que instaura la cuestión de la posibilidad de los acontecimientos en la actualidad.

Entre los análisis históricos que Foucault realiza siguiendo el modelo nietzscheano, la constitución del sujeto adquiere una centralidad plena en relación con el saber, el poder y la moral.\footnote{\enquote{Creo que en Nietzsche se encuentra un tipo de discurso en el que se hace un análisis histórico de la formación misma del sujeto, el análisis histórico del nacimiento de un tipo de saber, sin admitir jamás la preexistencia de un sujeto de conocimiento}. \cite[][18]{@7049-FOUCAULT2008}.} El estudio de las relaciones de poder busca describir una anatomía, una microfísica del poder, que expondrá cómo el individuo es constituido a través de una red circulante, configurada por unos sistemas de saber y unos mecanismos de poder disciplinarios. La genealogía le ha posibilitado a Foucault entender el sujeto como efecto de dispositivos de poder que diagraman tanto a la sociedad como a nosotros mismos. \enquote{El individuo es, sin duda, el átomo ficticio de una representación ideológica de la sociedad; pero es también una realidad fabricada por esa tecnología de poder que se llama disciplina}. \footcite[][198]{@7054-FOUCAULT2004} Los sujetos son el \emph{efecto} de unas tecnologías de poder, o como diría Deleuze, \enquote{{[}e{]}l sujeto es siempre algo derivado. Nace y se desvanece en la espesura de lo que dice y de lo que ve}. \footcite[][173]{@7050-DELEUZE1996} La historia de la microfísica del poder punitivo, puede ser entendida como una ontología histórica del alma moderna. El alma es el espacio que aprisiona el cuerpo. En nombre del alma se pasa por el cuerpo, en nombre de una humanización, se atraviesa el cuerpo a través de un reticulado disciplinario. La genealogía del alma moderna busca hacer visible el desarrollo de una sucesión de sujeciones y resistencias. A través del trabajo genealógico se buscará exponer \enquote{cómo funcionan las cosas en el nivel de la sujeción presente, (\ldots) que someten nuestros cuerpos, gobiernan nuestros gestos y dictan nuestras conductas, esos procesos que nos constituyen como sujetos}. \footcite[][104]{@7046-DIAZ2005} Un análisis de la \emph{Herkunft} que nos hizo ser lo que somos, buscará siempre, \enquote{disociar el YO y hacer pulular, en los lugares y posiciones de sus síntesis vacías, mil acontecimientos ahora perdidos}. \footcite[][26]{@7047-FOUCAULT2008}

La genealogía foucaultiana tendrá como catalizador de su análisis el poder y sus resistencias como fuerzas que entran en relación. Tal análisis insistirá en abordar el poder en una nueva vertiente positivista, como productora de saberes y subjetividades a través de continuas reacomodaciones de fuerzas. Nos dice Deleuze al respecto: \enquote{el poder más que reprimir \enquote{produce realidad}, y más que \emph{ideologizar}, más que abstraer u ocultar, produce verdad}. \footcite[][104]{@7047-FOUCAULT2008} De este modo, antes de analizar el poder desde sus efectos represivos, se lo debe entender como una función social compleja, que emerge desde ciertas relaciones, en las cuales los individuos se constituyen. Frente a las concepciones jurídico-discursivas del poder que se configuran como \emph{teorías} que se preguntan por lo que el poder~es, la genealogía funcionará como una analítica de \emph{cómo} el poder opera. En esta perspectiva, la naturaleza del poder se entiende como la estructura total de las acciones, como disposiciones para producir posibles acciones. El poder es una forma de actuar sobre la acción de otro, como una forma de conducirlo. El poder se ejerce y solo existe en acto, entonces su ejercicio consiste en guiar~las posibilidades de conducta y disponerlas con el propósito de obtener posibles resultados, donde no solo se inhiben acciones, sino que se incrementan las posibilidades de otras. Cuando se comprende el poder como una forma de gobierno, como el efecto de una acción sobre otra, es posible pensar el ejercicio de la libertad como práctica. El poder como ejercicio y la libertad como poder se anudan y se pliegan posibilitando una conversión del poder, una reversión de ciertas las formas de opresión que nos constituyen sujetándonos.

\section{Poder y libertad: formas de gobierno y modalidades de existencia}

\epigraph{\emph{Cuando se define el ejercicio del poder como un modo de acción sobre las acciones de los otros (\ldots) se incluye un elemento muy importante: la~libertad. El poder solo se ejerce sobre sujetos libres, y solo en tanto ellos sean libres}}{Michel Foucault}

El abordaje genealógico foucaultiano se enfrentará a la concepción liberal que opone la libertad al poder, mostrando como el poder y la libertad no son más que las dos caras de una misma moneda. Las prácticas de libertad como relaciones de poder entre sujetos libres se inscriben en una cuestión propiamente política. Foucault comprende las relaciones de poder como \emph{relaciones estratégicas entre libertades.} La libertad en su forma plena y positiva posibilita aquel poder que ejercemos sobre nosotros mismos en el poder que ejercemos sobre los demás. Deleuze enfatiza, que \enquote{no basta con que la fuerza se ejerza sobre otras fuerzas o sufra sus efectos, se precisa que la fuerza se ejerza sobre sí misma} \footcite[][181]{@7047-FOUCAULT2008} a través de ciertas reglas facultativas (de relación consigo mismo) como una fuerza que se auto-afecta y habilita así un tipo de subjetivación, donde devienen formas específicas de gobierno entre hombres libres. En este sentido, \enquote{la condición material de existencia del poder, será la realización, el propio ejercicio de la libertad}. \footcite[][111-112]{@7055-ARANCIBIA2005} La libertad como ejercicio de un poder, y el poder como ejercicio de una libertad, manifiesta la forma política en la que nos constituimos como sujetos a través del gobierno ejercido sobre nosotros mismos y sobre otros. La libertad se manifiesta como en sí misma política, tanto en las prácticas que ejercemos sobre nosotros mismos, donde se establece una relación de dominio sobre uno, como en las prácticas que entablamos con otros, como la capacidad de afectar las acciones de los otros. De este modo, las relaciones de poder como disposiciones para producir posibles acciones, como campos de posibilidades de afectar y ser afectados por otros, se (des)funda en libertad: \enquote{En este juego la libertad bien puede aparecer como la condición para ejercer el poder (al mismo tiempo que es su precondición, ya que la libertad debe existir para que el poder pueda ser ejercido, y a la vez ser su apoyo permanente, ya que sin la posibilidad de resistencia, el poder podría ser equivalente a la imposición física)}. \footcite[][254]{@7056-FOUCAULT2001}

Foucault entrelaza el ejercicio mismo del poder con el de la libertad, la condición de posibilidad del poder es esta. La libertad como \emph{precondición} marca la imposibilidad de un fundamento último de la sociedad en el pensamiento foucaultiano, es decir, solo porque hay libertad el fundamento del fundamento se vuelve imposible. Si existen relaciones de poder, emergen de ellas las de resistencia, como un entramado de relaciones estratégicas entre libertades, de allí que toda relación de poder es siempre reversible. El complejo juego de las relaciones de poder y prácticas de libertad configuran un orden de lo social que se constituye siempre de una manera provisoria y precaria. Las relaciones de poder están profundamente enraizadas en el nexo social, sin que esto signifique que exista \enquote{un principio de poder primario y fundamental que domina a la sociedad hasta en su último detalle}. \footcite[][256]{@7056-FOUCAULT2001} El trabajo genealógico de la ontología del presente busca \enquote{restituir las condiciones de aparición de una singularidad a partir de múltiples elementos determinantes, de los que no aparece como el producto sino el efecto} \footcite[][16]{@7042-FOUCAULT1995}
como un trabajo de inteligibilización pero sin un principio de clausura.

En la medida en que el poder no puede ya ser pensado como potencialmente absoluto y arbitrario, la sociedad no puede ser pensada como aquello ajeno al poder que al autoregularse funcionaría como límite exterior del poder. Foucault lo plantea muy claramente \enquote{el poder no es una institución, no es una estructura y no es cierta fuerza de la que algunos estarían dotados; es el nombre que uno atribuye a una situación estratégica compleja en una sociedad dada}. \footcite[][89]{@7059-FOUCAULT2010} No hay sociedad sin relaciones de poder, no hay~sociedad sin gobierno, nos constituimos como sujetos solo a través de las relaciones de poder que entablamos con otros, que no son sino relaciones entre libertades. Vivir en sociedad para Foucault \enquote{es vivir de manera tal que sea posible la acción de uno sobre la acción de otros} \footcite[][256]{@7056-FOUCAULT2001} de allí que para él pensar en \enquote{una sociedad sin relaciones de poder solo puede ser una abstracción}.\footcite[][256]{@7056-FOUCAULT2001} La separación de \emph{Sociedad} y \emph{Estado} responde a una tecnología liberal de gobierno, donde la noción de sociedad cumple una función paradójica de principio de autolimitación del gobierno al mismo tiempo que campo de intervención permanente.

Frente al individuo liberal postulado como un sujeto continuo e idéntico a sí mismo, preconstruido en una relación de exterioridad con el poder, Foucault entiende que el individuo es un efecto del poder a partir formas de subjetivación donde se constituye, transforma, resiste y tiene la posibilidad de autoinventarse. Los modos de subjetivación no se refieren solo a la manera en que el sujeto se forma, sino también a cómo deviene en formador de sí. El yo se forma a sí mismo en formas que de alguna manera ya están operando. En el pensamiento foucaultiano, los modos de subjetivación son una invitación al ejercicio de una libertad práctica, pero no de los actos o de las intenciones, sino una libertad de escoger una manera de ser.\footcite[][213]{@7057-RAJCHMAN1990} Las prácticas de sí como prácticas de libertad sirven como una conversión del poder, como una manera de controlar y delimitar el abuso de poder. Paul Veyne muestra que el trabajo de una ontología del presente \enquote{es hacer el diagnóstico de los actuales posibles, y al hacerlo, erigir la carta estratégica}. \footcite[][7]{@7060-VEYNE1987}

La ontología histórica foucaultiana como \emph{crítica} consiste en pensar un nuevo régimen de gobierno de libertades, como un gesto que invita a ficcionar nuevas formas de vida como modalidades de existencia. En el pensamiento foucaultiano libertad, poder y resistencia lejos de oponerse, se imbrican en la posibilidad de crear prácticas de sí y con otros, que nos hagan ser más libres. Sólo en las relaciones de poder-libertad podemos entablar prácticas con otros y con nosotros mismos que nos amplíen nuestro espacio de acción y lucha. Lo político deviene a través del gobierno de sí y~de los otros como el lugar donde es posible pensar otro régimen de politicidad como otro de régimen de libertad. La sociedad se constituye a través de relaciones de poder-saber contingentes, por tanto, la resistencia ha de ser pensada en la manera de \enquote{cómo no ser gobernado de \emph{esa forma}}. La crítica foucaultiana es sobre una forma específica de gobierno. No se plantea aquí \enquote{la posibilidad de una radical anarquía} donde la cuestión sería \enquote{cómo volverse radicalmente ingobernable}. \footcite[][6]{@7036-BUTLER2001} Sólo reconociendo que las relaciones de poder/saber/subjetivación fundan lo social de manera histórica, transitoria y precaria es posible la crítica, la resistencia ya no al Poder sino a una forma de gobierno que nos sujeta de determinada manera. Una ontología política de los somos se relacionará para Foucault con un \enquote{principio de libertad, donde esta se define no como un derecho a ser, sino como una capacidad de hacer}. \footcite[][316]{@7068-FOUCAULT2010} En este revés del concepto de libertad como poder, es decir, como \emph{capacidad de hacer} se hallaría la condición de resistencia, cambio y transformación en nosotros mismos. La noción de \emph{gubernamentalidad}, como una confluencia de las técnicas de sí y las técnicas de gobierno ejercida sobre los otros, muestra la dirección de pensar cierto conjunto de prácticas a partir de las cuales se puedan definir estrategias que los individuos en su libertad puedan establecer unos con otros.

\section{Pensamiento e historia: hacia una ontología genealógica}


\epigraph{\emph{Tres dominios de la genealogía son posibles. Primero, una ontología histórica de~nosotros mismos en relación con la verdad, a través de la cual nos constituimos como sujetos de conocimiento; segundo, una ontología histórica de nosotros mismos en relación con el campo de poder, a través de la cual nos constituimos como sujetos que actúan sobre otros; y tercero, una ontología histórica en relación con la ética, por medio de la cual nos constituimos como agentes morales}}{Michel Foucault}


La propuesta foucaultiana se dirige hacia una ontología histórica de nosotros mismos que buscará responder cómo hemos sido constituidos como sujetos de saber, como sujetos imbricados en relaciones de poder, y por último, como sujetos morales de nuestro accionar. Tres genealogías en constante implicación; saber, poder y sí mismo conjugan una dimensión ontológica junto a una histórica. En el camino de ensayar una lectura que dé cuenta de la dimensión ontológica del pensamiento político foucaultiano, se vuelve necesario mostrar el lugar que ocupa la historia como genealogía, al mismo tiempo que el abordaje genealógico posibilita una forma pensamiento donde emerge una ontología de los modos de ser que nos constituyen como sujetos. Una ontología crítica se esforzará en mostrar cómo nos hemos constituidos en sujetos en relación con la verdad, el poder y la ética, a partir de un abordaje genealógico. Tres ámbitos de la genealogía a partir de una ontología histórica, tres ejes de una ontología genealógica buscarán indagar en nuestras maneras de ser, hacer y pensar. La problematización de las condiciones de posibilidad históricas sobre las que se apoya una ontología de nosotros mismos, nos permiten entrever un anudamiento ontológico de la historia, como un pliegue histórico del ser, donde ciertos juegos de verdad y ciertas relaciones de poder nos constituyen (y nos siguen constituyendo) de determinada manera. Revel alumbra esta relación mostrando que si \enquote{la historia no es memoria sino genealogía, entonces el análisis histórico en realidad no es sino la condición de posibilidad de una ontología crítica del presente}. \footcite[][58]{@7041-REVEL2008} En este marco, se vuelve necesario interrogar cuáles son las implicaciones de la aparición, en sus últimos escritos, de esta dimensión ontológica anudada con la historia a partir de un abordaje genealógico.

Deleuze de algún modo alumbra esta interrogación sobre la ontología de nosotros mismos, a partir de cómo Foucault toma la pregunta por el pensamiento como un modo de experimentar y problematizar; donde saber, poder, y sí mismo funcionan como una triple problematización del pensamiento. El ejercicio de pensar el saber o los estratos, como los llama Deleuze, es inventar el entrelazamiento de la disyunción del ver y del hablar, que implica poder ver y hablar en el intersticio abierto por los dos. Problematizar el Ser-saber implica poder abordar las visibilidades que hacen perceptible una realidad en su límite con los enunciados que hacen inteligible lo que acontece, donde la frontera que los separa los pone en relación. Problematizar el Ser-poder es \enquote{emitir singularidades, lanzar los dados}, esta metáfora del azar de la tirada de dados muestra cómo pensar siempre procede de un afuera, como el lazo que, de un modo permanente, reencadena las tiradas de lo aleatorio.\footcite[][152]{@7053-DELEUZE2008} En este sentido, problematizar el poder es tejer relaciones de fuerza, que se constituyen como acciones sobre otras acciones, donde emerge el pensamiento como estrategia. Por último, pensar la dimensión del Ser-sí mismo, es para Deleuze un proceso de transformación de lo lejano y de lo próximo que constituye el \emph{espacio del adentro}, que de manera simultánea se hallará co-presente con el espacio del afuera en la línea de pliegue.

El sujeto como forma, se encuentra sujetado a un control relativamente externo, y sujetado a una identidad relativamente interior. El sujeto como un pliegue de la exterioridad, constituye su interioridad exterior. Las fuerzas en el hombre se pliegan en una dimensión de \emph{finitud en profundidad}, donde el pliegue constituye al mismo tiempo el \emph{espesor} y el \emph{vacio}. \enquote{La subjetivación se hace por plegamiento}, la relación que entablamos con nosotros mismos es fuerza plegada, pensamiento que se dobla en y a través de su exterioridad.\footcite[][137]{@7053-DELEUZE2008} En este marco, Deleuze considera que se abre la posibilidad de situar el tiempo afuera, para poder pensar el afuera como tiempo bajo la condición de pliegue. Nos dice: \enquote{el pliegue del afuera constituye un sí mismo, y el propio afuera un adentro coextensivo. Había que pasar por el entrecruzamiento estrático-estratégico para llegar al pliegue ontológico}. \footcite[][148]{@7053-DELEUZE2008} La lectura deleuziana de Foucault alumbra la compleja dimensión ontológica del pensamiento foucaultiano en cuanto \emph{pliegue del pensamiento}:

\begin{quote}
	Pensar es plegar, es doblar el afuera en un adentro coextensivo a él. (\ldots) Si el adentro se constituye por el plegamiento del afuera, existe una relación topológica entre los dos: la relación con sí mismo es homóloga de la relación~con el afuera, y las dos están en contacto por medio de los estratos, que son los medios relativamente exteriores (y por lo tanto, relativamente interiores). Todo adentro se encuentra activamente presente en el afuera, en el límite de los estratos.\footcite[][154-155]{@7053-DELEUZE2008}
\end{quote}

\emph{Pensar es plegar}, el pensamiento se afecta a sí mismo al descubrir el afuera como su propio impensado. Pensar es alojarse en el presente que funciona como límite de lo que puedo ver y decir hoy, a la vez pensar el pasado en tanto se condensa un adentro, en la relación que tenemos con nosotros mismos. Interrogar la pertenencia a nuestra propia actualidad implica tratarla como un acontecimiento del que debemos mostrar la irrupción de su singularidad arbitraria. Foucault al preguntarse por las condiciones de posibilidad de la constitución de los discursos, de los objetos y de~los sujetos, problematiza las relaciones de fuerza y las estrategias sociales, que operan finalmente como productoras de subjetividad. La subjetividad es un pliegue de la exterioridad de los poderes y de los saberes epocales a través de los cuales nos constituimos. Saber, poder, y sí mismo como formas irreductibles y permanente implicación configuran una ontología histórica de lo que somos. Lo histórico aparece en la perspectiva foucaultiana para mostrar cómo estas formas de ser no establecen condiciones universales, sino que se piensa la realidad \enquote{desde un fondo móvil de fuerzas, que adquieren formas determinadas, pero siempre provisorias en el marco de su emergencia}. \footcite[][19]{@7052-COLOMBANI2008} El Ser-saber y el Ser-sí mismo (como proceso de subjetivación) están determinados por relaciones de fuerzas, atravesados por singularidades variables en cada estrato y época. Por lo tanto, en esta ontología de Foucault, la historia aparece como \enquote{prueba de que las cosas carecen de esencia o, mejor dicho, que la esencia de las cosas no se encuentra en algún lugar más allá del tiempo sino en la propia historia}. \footcite[][85]{@7058-MOROABADIA2006}

En la tarea genealógica foucaultiana no hay lugar para la búsqueda de un fundamento originario ni una esencia anterior a la exterioridad, ni una verdad que anteceda al conocimiento positivo de algo. El origen solo puede entenderse como un comienzo azaroso, irrisorio e irónico, como una invención que surge de la confrontación histórica.\footcite[][85-86]{@7046-DIAZ2005} En este marco, por fuera de toda visión metafísica esencialista, o incluso aún más, en un movimiento que la subvierte, una forma de ontología contiene en sí misma la historia, en el sentido en que las formas de ser no emergen históricamente sino que dichas formas son en sí mismas históricas. La centralidad del análisis foucaultiano se halla en las condiciones históricas de posibilidad que configuran lo existente de una determinada manera. Deleuze afirma enfáticamente aclarando que \enquote{si bien es verdad que las condiciones no son más generales o constantes que lo condicionado, sin embargo, lo que le interesa a Foucault son las condiciones}. \footcite[][151]{@7053-DELEUZE2008}

La filosofía histórica de Foucault no esconde cómo nuestra actualidad esté determinada por el pasado del que deriva, sino que condena toda concepción que entienda que dicho pasado como \enquote{el correlato de estructuras universales y esenciales que se repiten, necesariamente en el presente}. \footcite[][85]{@7058-MOROABADIA2006} En relación con esto, podríamos comenzar a esbozar una ontología genealógica que aborda el pasado desde la actualidad, donde el pasado ya no funciona como fundamento del presente, sino que la historia pone de manifiesto el carácter abierto del momento actual, su falta de fundamento, su permanente estado de indeterminación. Incluso la ausencia de un fundamento último opera como un proceso irreductiblemente histórico, como un acaecer del acontecimiento, en lo que Foucault denomina como \emph{acontecimientalización}, es decir, el acontecimiento como \emph{la irrupción de una singularidad histórica} que todavía sigue atravesándonos en nuestra actualidad.\footcite[Véase][9-11]{@7041-REVEL2008} En esta línea podemos proseguir pensando una ontología genealógica a través de las formas de ser dinámicas, como modos transcendentalmente contingentes (y en este sentido, también necesariamente históricos) de ser de una determinada manera. Una ontología genealógica de naturaleza histórica como un permanente devenir, no solo comprenderá las diferencias y transformaciones en las variantes condiciones empíricas, sino que abordará la pluralidad de \emph{formas} de ser como contingentes de manera necesaria.

La genealogía propone una nueva relación con el presente que afronta lo acaecido desde una problemática determinada por la actualidad, para franquear y transformar la experiencia de ese presente como una realidad incuestionable y natural. La genealogía se lanza al pasado para mostrar la ausencia de un fundamento último, y en este movimiento busca disolver la base y el sustento~de un problema constitutivo del presente que merece ser destruido. Los efectos de verdad que la genealogía busca instaurar residen en la posibilidad de mostrar que todo tiene una historia, y por tanto, lo que hoy puede parecer esencial, es en realidad contingente, y por tanto, susceptible de ser transformado. Así, una ontología genealógica como historia eficaz actúa como un instrumento al servicio del presente capaz de promover nuevas realidades. Deleuze explicita claramente que el gran principio histórico de Foucault consiste en entender que \enquote{toda formación histórica dice todo lo que puede decir y ve todo lo que puede ver}, para preguntarse \enquote{en cuanto a nuestro presente, ¿qué es lo que somos hoy capaces de decir, qué somos capaces de ver?}. \footcite[][157]{@7047-FOUCAULT2008} Una ontología genelógica arriba en la problematización de lo que somos en tanto \emph{diferencia}~de nuestro presente. Abadía plantea cómo \enquote{frente a una historia de la similitud que rastrea en el pasado los fundamentos de la moderna racionalidad, la genealogía tiene su origen en una serie de preguntas que interpretan el pasado como diferencia}. \footcite[][86]{@7058-MOROABADIA2006} La historia opera en esta ontología genealógica de nosotros mismos ya no para afirmar lo que somos sino para mostrar aquello en lo que diferimos, es decir, \enquote{no establece nuestra identidad, sino que la disipa en provecho de eso otro que somos}. \footcite[][154]{@7047-FOUCAULT2008}

\begin{quote}
	Al historiar la cuestión crítica, Foucault ha descubierto una especie de imposibilidad que es, no lógica, sino histórica, la imposibilidad, no de un círculo cuadrado o de un dios inexistente, sino la imposibilidad de lo que ya no es o de lo que todavía no es, aunque es posible pensarlo. No lo que no tiene sentido, sino lo que todavía no lo tiene o ya no lo tiene más. Es esta coacción o esta exclusión histórica es lo que el trabajo del pensamiento debe hacer ver.\footcite[][216]{@7057-RAJCHMAN1990}
\end{quote}

Una ontología genealógica, en cuanto problematización de nuestras formas de ser históricas, posibilitaría plegar la imposibilidad histórica de lo que todavía no somos para inventarnos de nuevo. Foucault allá en \emph{Las palabras y las cosas} ya nos advertía que \enquote{una historicidad profunda penetra en el corazón de las cosas, las aísla y las define en su coherencia propia}, \footcite[][8]{@7079-FOUCAULT2005} por tanto, se vuelve necesario no solo entender, sino también subvertir la forma en que \enquote{hemos sido atrapados en nuestra propia historia}. \footcite[][244]{@7056-FOUCAULT2001} Una ontología genealógica de nuestras maneras de ser, hacer y pensar, nos posibilita problematizar \enquote{las condiciones históricas que motivan nuestra conceptualización} como una forma de \enquote{conciencia histórica de nuestras circunstancias actuales}. \footcite[][242]{@7056-FOUCAULT2001} Deleuze de vuelta nos alumbra mostrándonos que el principio general que guía el pensamiento foucaultiano consiste en advertir, dadas unas fuerzas en los hombres, con qué otras fuerzas se entran en relación en una formación histórica específica, y qué forma deriva de ese compuesto de fuerzas.\footnote{Nos dice Deleuze a continuación: \enquote{Según Foucault, se trata de una relación de fuerzas, en la que las fuerzas regionales afrontan, unas veces, fuerzas~de elevación al infinito (despliegue), a fin de constituir una forma-Dios, y otras, fuerzas de finitud (pliegue), a fin de constituir una forma-Hombre. Una historia nietzscheana más que heideggeriana, una historia restituida a Nietzsche, o restituida a la \emph{vida}}. \cite[][166]{@7053-DELEUZE2008}.} El pensamiento del afuera abre la posibilidad a la resistencia de la vida: \enquote{no se sabe lo que puede el hombre \enquote{en tanto que está vivo}, como conjunto de \enquote{fuerzas que resisten}}. \footcite[][122]{@7053-DELEUZE2008} El \emph{sí mismo} como la \emph{relación} que entablamos con nosotros mismos, como una relación de la fuerza consigo misma, se constituye como un pliegue. En la lectura deleuziana, solo a partir del pliegue de~la relación que entablamos con nosotros mismos, es que se abre la posibilidad de reorientar la vida o la muerte frente al poder. En los diferentes plegamientos, se constituyen modos de existencia, y se habilita la posibilidad de inventar modos de vivir capaces de resistir los poderes y saberes instaurados:

\begin{quote}
	Pensar el pasado contra el presente, resistir al presente, no para un retorno, sino \enquote{a favor, eso espero, de un tiempo futuro} (Nietzsche), es decir, convirtiendo el pasado en algo activo y presente afuera, para que por fin surja algo nuevo, para pensar, siempre, se produzca pensamiento. El pensamiento piensa su propia historia (pasado), pero para librarse de lo que piensa (presente), y poder finalmente pensar \enquote{de otra forma} (futuro). \footcite[][154-155]{@7053-DELEUZE2008}
\end{quote}

En la tarea de la filosofía de pensar la diferencia que marca nuestro presente, Foucault nos invita a distinguir lo que somos actualmente, como una experiencia que nos posibilitaría emerger~de ella transformados.\footcite[][13]{@7061-FOUCAULT2009} Pensamiento e historia anudan una forma de~abordaje ontológico que entenderá que \enquote{la historia es lo que nos separa de nosotros mismos, y lo que debemos franquear y atravesar para \emph{pensarnos} a nosotros mismos}.\footcite[][154-155]{@7047-FOUCAULT2008} En la finalidad genealógica de franquear las fronteras de lo que aún somos se halla la posibilidad de irrupción de \emph{ser de otra manera}. La empresa de realizar una ontología genealógica radica en una apuesta política, en tanto \emph{pensamiento resistencial}, capaz de producir un orden nuevo~en lo visto y en lo pensado. Una ontología genealógica se constituye como una forma de pensamiento en los límites de nuestras formas de ser, que abre la pregunta por las condiciones de posibilidad indefinidas de transformarnos a nosotros mismos.

Butler considera que la cuestión de la libertad en Foucault se articula con la \emph{virtud}, a partir de un movimiento que se \enquote{pone en juego mediante el pensamiento y, en efecto, mediante el lenguaje, y que hace que el orden contemporáneo de ser sea empujado hasta su límite}. \footcite[][8]{@7036-BUTLER2001} Pensamiento y resistencia se convierten en un \emph{ethos}, que invita a transitar un nuevo modo de subjetivación en los límites de la ficción tecnológica del biopoder. Butler entiende que \enquote{la cuestión de sí inaugura una actitud tanto moral como política} como una virtud que \enquote{tendrá que ver con objetar esa imposición del poder, su precio, el modo en que se administra, a quienes la administran}. \footcite[][6]{@7036-BUTLER2001} Butler busca mostrar que si el \enquote{poder establece los límites de lo que un sujeto puede \enquote{ser}, más allá de los cuales ya no \enquote{es} o habita en un ámbito de ontología suspendida}, la resistencia a esta coerción no puede sino consistir en una \enquote{estilización de sí en los límites del ser establecido}. \footcite[][9]{@7036-BUTLER2001} Es decir, la autora expone la posibilidad de desobedecer los principios a partir de los cuales nos formamos, arriesgarnos al ejercicio de una virtud como una práctica \emph{deformarnos} en cuanto sujetos para formarnos en desujeción, como una resistencia que busca \enquote{actuar con artisticidad en la coacción}. \footcite[][12-13]{@7036-BUTLER2001} En una estética del existir aparece el legado a asumir en el trabajo crítico sobre los límites. La clave de la resistencia se alojará en un nuevo pliegue que posibilite nuestra propia reinvención como sujetos en el marco de una \emph{poiesis}. En términos de Arancibia la forma de pensamiento desplegada por Foucault arriba hacia \enquote{una concepción genealógica, estética y trágica de lo político}, que \enquote{traería consigo otro campo de posibilidades, otro régimen de politicidad}. \footcite[][114-115]{@7055-ARANCIBIA2005}

Nos constituimos como sujetos atravesados por las múltiples tensiones entre libertad y poder, contingencia y determinación, historia y condición. Nos constituimos como sujetos a través, alrededor y en ambos lados de estas fronteras, en una serie de desplazamientos donde nos desdoblamos y anudamos continuamente. Sólo la crítica como \enquote{el arte de la inservidumbre voluntaria, el de la indocilidad reflexiva} \footcite[][7]{@7042-FOUCAULT1995}
posibilita el ejercicio de intentar \emph{pensar de otro modo} como la condición misma para la creación de la libertad. En el momento en que una actitud, un ethos crítico, deviene, \enquote{la libertad surge en los límites de lo que uno puede ser, en el preciso momento en que la desujeción del sujeto tiene lugar dentro de las políticas de la verdad}. \footcite[][8]{@7036-BUTLER2001} El ejercicio de nuestra libertad está dado por pliegue de prácticas de sí, técnicas de gubernamentalidad y modalidades de veridicción que com-posibilitan la creación de uno mismo. Una ontología genealógica podría abrir el intersticio para la experiencia de un pensamiento, que como \emph{forma de vida} plegada, nos posibilite devenir en otras formas de subjetivación que nos hagan ser más libres.

\section*{Referencias}
\printbibliography[heading=none]   % Sin título automático


%%%%%%%%%%%%%%%%%%%%%%%
\ifPDF
\separata{capitulo3}
\fi

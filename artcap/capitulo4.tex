\ifPDF
\chapter[\hspace{1.5pc}Ontología de la distorsión. Algunas notas sobre la política en la obra de Jacques Rancière]{Ontología de la distorsión. Algunas notas sobre la política en la obra de Jacques Rancière}
\chaptermark{Ontología de la distorsión...}
\Author{Juan Manuel Reynares}
\setcounter{PrimPag}{\theCurrentPage}

% encabezado para autor
\begin{center}
	\nombreautor{Juan Manuel Reynares}\\
	\vspace{20mm}
\end{center}
\else
\ifHTMLEPUB
\chapter{Ontología de la distorsión. Algunas notas sobre la política en la obra de Jacques Rancière}
\fi
\fi

\section{Introducción}

Si bien todo establecimiento de fronteras es arbitrario, podemos incluir la obra de Rancière\index[concepto]{Filosofía política!emancipación!en Rancière}\index[concepto]{Rancière} al interior de un conjunto de pensadores contemporáneos cuya reflexión surge de la crítica a la existencia de un fundamento trascendental que dé orden y justificación a la totalidad\index[concepto]{Ontología crítica!ausencia de fundamento}. Haciéndonos eco de la propuesta de D. Scavino,\footcite[][]{@7062-SCAVINO2010}\index[concepto]{Scavino} nos es posible considerar que la filosofía de nuestros días se encuentra articulada en torno a tres cuestiones centrales: el estatuto de la verdad\index[concepto]{Verdad}, la crítica al totalitarismo\index[concepto]{Filosofía política!crítica al totalitarismo}\index[concepto]{Totalitarismo|see{Crítica al totalitarismo}} y la propuesta de una ética que se defina en la relación aporética entre universalidad y particularidad.

El acontecimiento filosófico fundamental que nos sirve de base para presentar estas críticas a la Verdad\index[concepto]{Verdad!crítica a la correspondencia}, lo Común y el Bien, es el giro lingüístico\index[concepto]{Giro lingüístico}, es decir, \enquote{que el lenguaje deja de ser un medio, algo que estaría entre el yo y la realidad, y se convertiría en un léxico capaz de crear tanto el yo como la realidad}\index[concepto]{Giro lingüístico!relación significante-referente}. \footcite[][12]{@7062-SCAVINO2010} De esta manera, no existe una relación privilegiada entre el referente y el significante, entre la cosa y la palabra, que habilitaría así a ejercicios de descubrimiento de la realidad tal cual es. Si asumimos entonces en toda su complejidad al giro lingüístico, ¿cómo comprender a la verdad\index[concepto]{Verdad!significado|see{Crítica al totalitarismo}}? Esta deja de ser considerada como una relación de correspondencia entre la idea, la palabra, la teoría y la realidad pensada como externa, y pasa a ser así siempre una verdad \emph{constituida}\index[concepto]{Verdad!crítica a la correspondencia!en Rancière}\index[concepto]{Giro lingüístico!verdad constituida}.

De allí podemos inferir que en la constitución de esa verdad está presente siempre una dimensión de poder\index[concepto]{Poder!relación con verdad}. O lo que es lo mismo, toda verdad es política, está cruzada por tensiones entre distintas concepciones de la realidad, la que en última instancia no posee mayor ser que el dado a través de la simbolización. Scavino delinea allí un elemento común a este conjunto de pensadores dentro de los que ubicamos a Rancière (aun cuando seguramente este se desmarcaría de cualquier ejercicio de etiquetamiento intelectual), que es la denuncia de la relación implícita entre una filosofía que se dice dueña de la verdad y el totalitarismo político. La seguridad de poseer la única interpretación correcta de lo que nos rodea habilita a su poseedor a imponer sus consecuencias a todos, sin detenerse en particularidades.

Si es imposible presentar a la verdad como adecuación de lo dicho a la cosa, debido a la inerradicable mediación constitutiva del lenguaje, tampoco es posible postular, con base en una verdad fuera de todo debate, un orden político último y absoluto, encarnación de alguna razón universal que a través de distintos dispositivos justifique la irrupción de la historia sobre los cuerpos\index[concepto]{Cuerpos|see{Cuerpo}}\index[concepto]{Ontología crítica!ausencia de fundamento}. En ese mismo sentido, tampoco habrá un principio último de todas las cosas que fundamente una postura ética universal. Este es el contexto general de la reflexión filosófica de las últimas décadas donde se inscribe Jacques Rancière como un pensador de las posibilidades políticas de la emancipación, sin caer en los abismos filosóficos que reivindican un proyecto en última instancia totalitario, producto de una hipotética relación privilegiada entre una interpretación y la positividad del ser.

Las contribuciones de Rancière a la reflexión política contemporánea han sido puestas en discusión en múltiples ámbitos teóricos y políticos, y hay así numerosas miradas sobre las categorías que introdujo a lo largo de su obra. Entre ellas, ha primado un conjunto de interpretaciones que consideran, en términos generales, que al interior de la argumentación de Rancière la noción de política es relegada\index[concepto]{Política!}. Aquí desarrollaremos sucintamente algunas de esas perspectivas que derivan en puntos muertos de la lectura, y que notamos particularmente en la obra de Žižek y algunas premisas de Marchart.\footnote{\cite[][]{@6998-MARCHART2009}; y \cite[][]{@7063-ZIZEK2005}.}\index[concepto]{Žižek}\index[concepto]{Marchart} Allí, la política se erige como una expresión esporádica, accidental, en contraste con un orden social positivo, cuya incompletud es señalada \emph{a posteriori} por la irrupción momentánea, efímera del sujeto político\index[concepto]{Sujeto político}\index[concepto]{Subjetividad|see{Sujeto político}}. Esta caracterización habilita entonces a Žižek a plantear que la postura política que deviene de esta argumentación solo alcanza la crítica a lo dado, pero escapa a poder incidir de algún otro modo sobre el orden social.

Nuestro planteo intenta desmarcarse de estas lecturas, y sostiene así una clave de lectura de la obra rancièrana sobre la política (que se entrelaza con la pedagogía y con la estética, otras dos ramas fundamentales de la literatura), a partir de la noción de una ontología de la \emph{distorsión}\index[concepto]{Ontología crítica!distorsión de!en Rancière}\index[concepto]{Política!como distorsión}. Específicamente: la condición de posibilidad de la política tal como la piensa Rancière es la de que todo orden social se erige en torno a, través de, una torsión. Esta es la de la relación inconmensurable, pero necesaria, entre la igualdad de los que hablan y se entienden, y la desigualdad propia de las relaciones sociales\index[concepto]{Igualdad!relación con el lenguaje}\index[concepto]{Desigualdad!social}. Esto, veremos en lo que sigue, implica la siempre necesaria explicación de la organización jerárquica de las diferencias sociales, de donde se deriva la ausencia de un fundamento natural para el establecimiento de una ordenación inmutable de funciones y lugares sociales.

Por lo tanto, antes que un orden social positivo donde se verificaría una irrupción accidental de la política, como se deja ver en los planteos de Žižek y Marchart, encontramos en la obra rancièrana una tensión ineliminable entre la igualdad propia del lenguaje y la desigualdad de las relaciones sociales. Es esta distorsión, que subyace a toda distribución de diferencias sociales, la que es pasada por alto en los análisis críticos, o bien en ciertas interpretaciones comunes, de la noción rancièrana de política. Precisamente, esa distorsión ontológica nos permite comprender, agregamos que agragamos, la co-constitución de la política y la policía, como lógicas del ser-juntos humano.\index[concepto]{Política!en Rancière}\index[concepto]{Policía|seealso{Política}} A partir de allí, completaremos nuestra lectura sobre el trabajo de Rancière, con las dimensiones performativa y estética del sujeto político, y con la relación dinámica entre la subjetivación y la identificación.\index[concepto]{Sujeto político!subjetivación} Hacia el fin, consideramos útil la caracterización de la emancipación como apuesta rancièrana en el presente.\index[concepto]{Filosofía política!emancipación!en Rancière}

\section{Igualdad y distorsión}

Para comenzar, veamos la manera en la manera en que se estructura el argumento sobre la inconmensurable relación entre la igualdad de las inteligencias y la desigualdad de las relaciones sociales, y cómo ello supone una distorsión constitutiva del orden social.\index[concepto]{Igualdad!de las inteligencias!en Rancière}\index[concepto]{Desigualdad!tensión con igualdad}\index[concepto]{Ontología crítica!distorsión!} Nuestro punto de partida es que en la caracterización de la igualdad, Rancière sostiene todo el andamiaje de su argumento. En el final de este apartado, trayendo a colación algunas premisas de la reflexión de Žižek sobre la propuesta teórica-política de Rancière, enfatizamos la importancia de la importancia de tener presente el carácter constitutivo de la distorsión, para, adelantamos, no caer en miradas dicotómicas sobre un orden policial positivo y una irrupción política esporádica.\index[concepto]{Žižek!en Rancière}\index[concepto]{Policía!}

Existe una relación lógica en la interacción que sostiene todo mandato. El hecho de dar y recibir órdenes implica cierta jerarquía, distribución y disposición de cuerpos, funciones y prerrogativas.\index[concepto]{Cuerpo}\index[concepto]{Desigualdad!} Social]. Por lo tanto, entonces, la posibilidad de todo orden social está sustentada en esa interacción específica. Ahora bien, la posibilidad de que se sostenga tal interacción del mandato reside en la relación lógica que presupone la mutua comprensión de los interlocutores. Es decir, cuando se explica a un subordinado qué debe hacer, se da por sentado que comprende que \emph{debe} hacerlo. La situación de habla explicativa lleva implícito el hecho de que tanto quien habla como quien obedece se entienden uno al otro. Mismo al mismo tiempo, cuando se explica por qué trata aquel subordinado debe obedecer, cuando se pretende dotar de legitimación la relación ratio de obediencia, nuevamente se que reconoce que existe una relación de obediencia interna la población es igual de inteligencias compartida por el.\index[concepto]{Igualdad!de las inteligencias}\index[concepto]{Ontología política!relación social}

Toda vez que hemos asentado la necesidad de explicar la relación de obediencia que se ubica en la base del mandato, es posible un signo de tesis de gran importancia para el planteo de Rancière, aquel plan de la ausencia de un signo trascendente, de un principio de gobierno asentado en causas naturales, de fuerza, o de saber.\index[concepto]{Ontología crítica!ausencia de fundamento}\index[concepto]{Rancière} Si así fuera, el orden determinado por un fundamento \emph{a priori} no precisaría de ninguna explicación. De esta forma, las relaciones de desigualdad social mi entran en tensión continua con el presupuesto relacional igualitario sobre la dimensión del lenguaje, como no dice, Rancière, \enquote{(\ldots) esta igualdad es simplemente la igualdad de cualquiera con cualquiera, (\ldots) la pura pura contingencia de todo el orden social (\ldots) (\ldots) para obedecer una orden (\ldots)) hay que comprender y hay que comprender que hay que obedecerla. Y para hacer eso, ya es preciso ser igual a quien nos manda. Es esta igualdad la que carcome todo el orden natural}.\index[concepto]{Igualdad!relación con el lenguaje!en Rancière}\index[concepto]{Desigualdad!Tensión con igualdad}\footcite[][30-31]{@7064-RANCIERE2010}

Aun cuando uno de los extremos de la interacción del mandato asume la capacidad de decir al otro lo que debe hacer, o por analogía, lo que se espera de él, los valores a respetar, las cuentas a mantener, \emph{esa es una asunción arrogada}, sin un sustento trascendental que lo fije en ese punto.\index[concepto]{Ontología crítica!ausencia de fundamento} En esa instancia, ese interlocutor se atribuye la capacidad de representar la racionalidad específica que dota de sentido a la relación. Los mismos términos, las palabras, los sentidos otorgados a las prácticas que se utilizan en la relación son aquellos que devienen de esa capacidad arrogada. El que da las órdenes participa en el \emph{logos} comunitario \emph{como un semejante}, de distinta manera que lo hace el que las recibe, que solo es parte de ese \emph{logos} en términos de \emph{lo útil}\index[concepto]{Comunidad!Logos}\index[concepto]{Igualdad!relación con el lenguaje}. En esa situación paradojal reside el núcleo del conflicto en toda relación política. No existe fundamento para la relación de subordinación no. El lo permite al lo por lo subordinado a exigir en lo que cualquier momento lo que participa en el \emph{logos} comunitario y que se dé cuenta de ello, es decir que todos somos potencialmente lo que son capaces de demandar que se nos trate como un tratado. La presupuestalidad de las inteligencias es igual es el fundamento constitutivo fundamentalmente abierto de toda práctica política.\index[concepto]{Política!}\index[concepto]{Sujeto político} Y por ello se observa, agregamos, en la necesidad de exterioridad de la ley, su necesidad de escritura escrita.%%

Si debemos explicar algo, si la obediencia requiere legitimación, la igualdad presupuesta tanto en el enunciado como en la recepción está atada a la explicación de la desigualdad. Y si hay que dar razones, ella, por lo tanto, no tiene fundamento de por sí. Como es arbitraria la lengua, es arbitraria la desigualdad.\footnote{Podemos completar esta noción de arbitrariedad de la lengua con algunas de las contribuciones al respecto presentes en Joseph Jacotot, educador francés de principios del siglo XIX. Existe una tensión esencial entre dos lógicas contradictorias, \enquote{la lógica igualitaria implicada en el acto de la palabra y la lógica desigualitaria inherente a la relación social} Al rechazar la noción de una razón inmanente a la lengua, lo que se deja ver es la imposibilidad de reducir la relación entre los significantes y los significados a una univocidad condensada en una ley intrínseca al lenguaje. Existe una constitutiva polisemia en la relación entre significante y significado, lo~que implica que todo enunciado y toda recepción estén marcados por un querer decir \rdm{un sentido dado} y un querer escuchar --un sentido \emph{recibido}. Siempre que hablamos o escribimos estamos presuponiendo un sujeto capaz de codificar lo dicho, en ausencia de una ley universal sobre significados unívocos en la lengua, de un \enquote{diccionario primero {[}que{]} asegura la~verdad}. De allí que siempre que hablemos estaremos presuponiendo la igualdad del repertorio, lo que inhabilita a pensar en un fundamento que determine de una vez y para siempre los límites asumidos por la comunidad. Es posible observar aquí entonces cómo una perspectiva surgida del giro lingüístico se entronca con la reivindicación de la igualdad como un fundamento abisal, que revela la radical ausencia de un fundar único y trascendente y habilita, en cambio, el juego de fundamentos contingentes.} La ausencia de fundamento de la disposición de cuerpos y funciones obliga a explicar esa división, lo que solo puede realizarse si se tiene como condición la igual capacidad de las inteligencias. Completamos con este pasaje central en la argumentación rancièrana: \enquote{(\ldots)  lo que se trata de explicar, lo que pone en marcha la máquina explicativa, es la desigualdad, la ausencia de razón que tiene necesidad de ser racionalizada, la facticidad que requiere ser ordenada (\ldots) ese arbitrario de la lengua que un sujeto razonable traspasa a otro sujeto razonable supone este otro arbitrario que es el arbitrario social}. \footcite[Versión digital][65]{@7065-RANCIERE2007} De esta forma, a la desigualdad social, siempre presente, se le contrapone, paradójicamente complementaria, la comunidad de los iguales como \enquote{una sociedad \emph{inconsistente} de hombres trabajando a la creación \emph{continua} de igualdad (las cursivas son nuestras)}. \footcite[Versión digital][66]{@7065-RANCIERE2007}

Pensar que existe la posibilidad de una ausencia de la distancia entre la dominación y su legitimación, volviendo así innecesaria la explicación, sostendría el carácter no mediado de la obediencia, desconocería la arbitrariedad de la jerarquización social. Ello solo sería posible si las partes al interior de esa comunidad actuasen guiadas por el espíritu de la ley, si no fuese necesaria dejarla \emph{escrita} y refrenar el comportamiento de los hombres y las mujeres a través de ella. En este caso, la tensión entre las relaciones sociales desiguales y la comunidad de semejantes sería desplazada por la estabilidad propia de una jerarquización \emph{natural}, y la política no tendría allí ningún sentido. Precisamente ello es lo que pretende una de las figuras de la filosofía política que obtura la política \rdm{la arquipolítica}. La pretensión platónica, en \emph{La República}, de anular el espacio abierto por el pueblo democrático y presentar así un espacio clausurado, saturado, donde no hubiese posibilidad de distorsión, se topa con la exterioridad de la ley. De allí que Platón abogue por una eticidad a través de la que toda parte cumpla su función en la república sin ser \emph{refrenado por la ley}. \enquote{La ciudad buena es aquella donde el orden del cosmos, el orden geométrico que rige el movimiento de los astros divinos, se manifiesta como \emph{temperamento de un organismo}, donde el ciudadano no actúa según la ley sino según \emph{el espíritu de la ley}, el soplo vital que la anima}.\footnote{\cite[][91]{@7064-RANCIERE2010} (las cursivas son nuestras).} La arquipolítica pretende entonces imbricar integralmente la \emph{physis} en el \emph{nomos}, la naturaleza en la ley. Allí la libertad del pueblo, que vendría a desbaratar la distribución plena, se remplaza por la prudencia {[}\emph{sophrosyne}{]} de los artesanos, el hacer cada uno su parte, contribuyendo así a la saturación comunitaria de la ciudad ideal.

La única manera de lograr una República que anulara la ausencia de fundamento, para Platón, era sostener la obediencia a la ley sin la necesidad de su escritura. La escritura expresa la diferencia primera, la imposibilidad de una armonía natural de órdenes, disposiciones de cuerpos y funciones, porque implica la necesidad de explicar y justificar la obediencia. Esa diferencia primera es la que impone el juego continuo de la política, que surge así a partir de la distorsión básica en la configuración de las relaciones sociales. En este sentido se refiere Rancière cuando nos dice que

\begin{quote}
	(\ldots) no hay fuerza que se imponga sin tener que legitimarse, es decir, sin tener que reconocer una igualdad irreductible para que la desigualdad pueda funcionar. Desde el momento en que la obediencia debe pasar por un principio de legitimidad; desde el momento en que tiene que haber \emph{leyes que se impongan como leyes}, e instituciones que encarnen lo común de la comunidad, el mandato debe suponer una igualdad entre el que manda y el que es mandado (\ldots) \emph{la sociedad desigualitaria no puede funcionar sino gracias a una multitud de relaciones igualitarias}\ldots.\footnote{\cite[][72-73]{@7066-RANCIERE2007} (las cursivas son nuestras)}
\end{quote}

Así la reflexión rancièrana sitúa en la base de todo orden social un desfasaje constitutivo, una distorsión fundante que podemos encontrar en las múltiples relaciones paradojales que configuran la vida en comunidad.

Lo que hemos planteado hasta aquí señala la imposibilidad de erradicar la dimensión conflictiva que informa todo ordenamiento de diferencias en lo social, enfatizando así la necesaria importancia ontológica de la distorsión. Ahora retomaremos una cuestión recién mencionada, aquella de la semejanza y la utilidad, por lo que se vuelve importante enfatizar la relación entre comunidad, \emph{logos}, política e igualdad. Esta última, expuesta en estos términos, adquiere una relación muy específica con la noción de comunidad. Antes que suponer una noción estable y suturada de \emph{comunidad}, que supondría una distribución fija y fundada de lugares, cuerpos y funciones, la igualdad de cualquier ser parlante con cualquier otro viene a irrumpir en ella. De allí que para nuestro autor, a la comunidad hay que pensarla en un continuo \emph{arreglo de cuentas} entre la pretensión igualitaria y la diferencia jerárquica, por lo que \enquote{la comunidad de iguales no puede nunca darse cuerpo, sino es con cierto \emph{reestucado}, con la obligación de \emph{recontar} los miembros y las filas, \emph{tapar las fisuras} de la imagen\ldots}.\footnote{\cite[][50]{@7065-RANCIERE2007} (las cursivas son nuestras). Aquí radica el núcleo de la imposibilidad de alcanzar un fundamento de lo Común. La igualdad a la base de las relaciones sociales introduce una distorsión que no es posible de \enquote{resolver} \emph{a priori}.}

La distorsión que impide la plenitud de la distribución de las partes está dada por su necesaria relación con la comprensión mutua del mandato que venimos rastreando en otros pasajes de la obra rancièrana. Toda comunidad está basada en las partes que la componen, en una determinada distribución de esas partes que poseen funciones, y donde algunas son superiores ante otras inferiores. Pero ese reparto se enfrenta constitutivamente con la necesaria comprensión de la situación de dominio por parte de aquellos que no son semejantes, es decir, que no son considerados capaces para hablar con sentido en torno a los asuntos comunes. Tal presuposición viene a distorsionar la desigualdad implícita en la distribución de cuerpos y funciones. Pero la jerarquía que reconocen los que participan en cierta distribución de la cuenta de la comunidad depende siempre de considerarlos a los otros semejantes, \enquote{no porque se es útil a los iguales se entra a su comunidad, sino porque se es semejante}, \footcite[][54]{@7065-RANCIERE2007} la cuestión se concentra entonces en la \emph{imagen} de lo igual, es decir, en el reconocimiento de la palabra válida, de quiénes son aquellos que pueden contar como partes de la comunidad. Sólo de esa forma lo que digan será palabra al interior del \emph{logos}, y no el ruido propio de animales, serán argumentaciones reconocidas como tales. Es en este punto, en la imposibilidad de determinar \emph{a priori} quiénes son considerados interlocutores válidos en torno a cierta racionalidad de lo común, que emerge la política.

La distorsión que se encuentra a la base de cualquier orden social no permite así la consecución de un \emph{logos} absoluto, porque para dar forma a una distribución específica de cuerpos y lugares que interactúan en relaciones sociales de por sí desigualitarias, se debe presuponer la misma capacidad de comprender de todos. Hay un paso en falso entre lo \emph{semejante} y lo \emph{útil} que vuelve retrazable la distancia entre uno y otro de estos aspectos de toda comunidad. Ahora bien, Rancière no detiene su reflexión en la denuncia de la incapacidad de un orden definitivo. Haciendo uso de esta indeterminación radical, desarrolla una caracterización del sujeto político que enfatiza el rol ontológico de la distorsión, y el carácter fundacional y siempre conflictivo de la intervención política, poniendo en tensión la distribución de los cuerpos y la reivindicación igualitaria, como veremos enseguida, bajo los nombres de la policía y la política.

Antes de entrar en ello, como mencionamos en las notas introductorias, consideramos que enfatizar la distorsión constitutiva permite echar luz sobre algunas interpretaciones de la obra rancièrana que la plantean como la defensa de una noción de la política solo como una práctica que irrumpe sobre una positividad plena, clausurando \emph{a posteriori} de esa manera la posibilidad de plenitud de ese orden. Paradigmática en este sentido es la lectura de Žižek, quien considera que en la perspectiva rancièrana existe una \enquote{brecha entre el orden global positivo (\ldots)  y las intervenciones políticas que perturban ese orden\ldots}. \footcite[][185]{@7063-ZIZEK2005} Lejos de ello, consideramos que Rancière no sostiene que el orden social es positivo, sino que, como acabamos de ver, está constitutivamente distorsionado por las presuposiciones paradojales de la igualdad y la desigualdad que no reconocen medida justa posible entre ellas. Debido entonces a esta tensión que se encuentra a la base de todo orden social, adelantamos, el sujeto político se erige en la verificación de la igualdad que alcanza a aquella parte del todo que no es contada como tal.

\section{La co-constitutividad de la policía y la política}


Como vemos, es posible identificar hasta este punto de la argumentación dos lógicas diferentes que rigen la articulación entre las partes presentes en la sociedad, como modos de ser-juntos humano, \enquote{dos tipos de partición de lo sensible}: una lógica policial y otra política.\footnote{Es esta contribución de la obra rancièrana la de mayores efectos en el debate teórico-político contemporáneo, y al mismo tiempo, parece, la que ocasiona mayores problemas.} La primera de ellas retoma su nombre de la noción foucaultiana como técnica de gobierno. Es aquella que pretende distribuir de manera jerarquizada y organizada funciones, cuerpos, prerrogativas, modos de ser, hacer y decir. En algunos pasajes de su obra, Rancière considera que esta lógica hace a la \enquote{constitución simbólica de lo social},\footnote{\cite[][5]{@7080-RANCIERE1997} versión digital, traducida por M. C. Galfione. El original, \enquote{Onze thèses sur la politique}, fue publicado en \emph{Filozofski vestnik}, N° 2, año 1997.} donde debemos destacar dos puntos. En primer lugar, siempre se refiere cierta parte de lo sensible, que define así las formas del tener-parte. La policía se sitúa sobre una división al interior de lo sensible, y al mismo tiempo establece el modo en que se distribuyen, se reparten, las partes exclusivas. En segundo lugar, la policía se caracteriza por la ausencia de vacío y de suplemento. Es decir por la clausura y por la pretensión de plenitud. La lógica del uno a uno indica la relación unívoca de cada identidad con su referente. La policía refiere entonces al orden, al conjunto de procesos por medio del que se logra un cierto ordenamiento de cuerpos y prerrogativas. Hasta aquí las críticas de Žižek tienen sentido, parecería que la lógica policial instituye un orden positivo del ser, pero debemos completar la definición rancièrana, ya que a esta organización de lo sensible se agrega el grupo de prácticas que construyen el consentimiento de las partes, y los distintos sistemas de legitimación de tal distribución. La obediencia que existe al interior de este modo de partición de lo sensible siempre se acompaña de cierta pretensión de legitimidad. Por ello, \emph{la igualdad de las inteligencias estará siempre paradójicamente a la base de esta distribución policial, socavándola}, y es esta distorsión la que impide hablar de un orden social positivo donde irrumpe la política de manera esporádica, accidental, marginalista. Más bien es la distorsión a la base de todo reparto de lo dado el que da lugar al modo de ser-juntos-humano de la política, la segunda lógica que mencionamos al principio de este apartado, precisamente aquel que da a la igualdad de cualquier ser parlante con cualquier otro la actualidad en forma de casos.

Teniendo en cuenta entonces qué implica la política y también qué incluye la policía, nos interesa profundizar las relaciones que Rancière establece entre ambas lógicas a lo largo de su obra.\footnote{Rancière nos advierte del uso que da a este par de categorías. Hay un uso excesivo de ellas, en que él mismo incurre en función de su crítica al llamado \enquote{retorno de la política}. Ese uso excesivo, que hace notar críticamente Žižek, y retoma al pasar Marchart, apunta a una división tajante entre ambas lógicas, al modo, nos dice Rancière, \enquote{de oposiciones conocidas: espontaneidad y organización o acto instituyente contra orden instituido. Se trataría, en pocas palabras, de oponer una esfera de actos puros de la igualdad al orden del mundo}. Es clara aquí la tendencia a pensar la dicotomía, que se acentúa al considerar el autor que la política siempre viene después. Pero esa posterioridad de la política, que es así porque actualiza una igualdad que ha sido dañada por el orden social, supone no solo la primacía ontológica de la igualdad de las inteligencias, sino también la distorsión a la base de cualquier organización de diferencias sociales. Entonces, lo que hay es distorsión, que no acepta la división, presente en las figuras de la filosofía denunciadas por Rancière, entre apariencia y realidad. \enquote{La oposición política/policía vuelve a poner en cuestión todo principio de una repartición positiva de las esferas y de las maneras de ser (\ldots)  Es una configuración efectiva de lo dado, de lo que es visible, y entonces de lo que puede ser dicho de lo dado y hecho en relación con lo dado. Se sigue igualmente que no hay de un lado la esfera de las instituciones policiales, y del otro las formas de manifestación puras de la subjetividad igualitaria auténtica (\ldots)  La distinción de la política y de la policía opera en una realidad que conserva siempre una parte de indistinción. Es una manera de pensar la mezcla. No hay un mundo político puro y un mundo de la mezcla. Hay una distribución y una redistribución}. \cite[][]{@7081-RANCIERE2004}.} Precisamente, al partir de una premisa que hace hincapié en la distorsión implícita a la base de todo orden social, no es posible pensar estas lógicas de la policía y la política como exteriores, sino más bien como constitutivas una de otra, es decir co-constitutivas. Retomando la crítica a la crítica de Žižek, asumir estas lógicas por separado va de la mano con la concepción de la política como irrupción en el terreno positivo del ser. También encontramos este forzamiento en la interpretación de la política en Rancière en un uso común de su obra, cuando es puesta en consonancia con otros autores de la \enquote{constelación posfundacional}. Marchart, quien rastrea algunas coordenadas del proyecto rancièrano, considera la oposición entre la policía y la política como una irrupción de esta última de manera accidental en el marco de un orden social establecido bajo la lógica policial. Si bien este autor no trabaja en profundidad la obra rancièrana, analiza tangencialmente varias de sus cuestiones centrales. Una de ellas es la diferencia entre la política y la policía: mientras esta última es \enquote{el conjunto de procedimientos mediante los cuales se organizan los poderes, se establece el consenso\ldots, la primera es, precisamente, la que se \emph{opone} al control (\ldots) la política surge en el punto de reunión conflictivo entre la condición (no política) de la igualdad y la lógica del control}.\footnote{\cite[][161]{@6998-MARCHART2009}. Las cursivas son nuestras.} Entre ambas oraciones encontramos un cambio en el énfasis que puede provocar ciertos puntos muertos en la lectura rancièrana, y ello se resume en que, para el tratamiento de Marchart, ese \enquote{punto de reunión conflictivo} no es constitutivo del orden, o bien del \enquote{monopolio} de la lógica del control, sino que adviene a posteriori, oponiéndose al control, efectuando allí una ruptura. Al no detenerse en la inconmensurabilidad que recién postulamos en la obra del filósofo argelino, la ruptura política se comprende contra la fachada de una distribución específica de funciones y diferencias sociales, sin mayores especificaciones sobre su dinámica. Proponer un orden social positivo donde la brecha de imposibilidad que señala la subjetivación política implica su radical incapacidad de alcanzar una plenitud absoluta, pero que siempre llega \emph{a posteriori}, deja así abierta la puerta para el no cuestionamiento radical de ese orden, lo que supone entonces una posición política, al decir de Žižek, marginalista e histérica, situada en los confines de la práctica política. De esta manera, las dos miradas que estamos rastreando tienen como punto en común el considerar que para Rancière el conflicto, la brecha del acto político, no está allí desde el principio, \emph{ab initio}, es decir, no es ontológico sino accidental.

A diferencia de estas conclusiones, nos proponemos enfatizar la caracterización del orden social en la obra rancièrana como la imagen distorsionada de un cálculo que siempre deja resto. Todo orden descansa en la distorsión de la inconmensurabilidad de dos lógicas necesarias pero imposibles de \enquote{poner en orden}. En este mismo sentido, la co-constitutividad de las lógicas de la policía y la política señala que la política, antes que una ruptura a posteriori, es necesaria comprenderla como la \emph{verificación} de la igualdad trayendo al acto un orden social constitutivamente distorsionado. Ello mismo, como veremos ahora, habilita a caracterizar al sujeto político en sus dimensiones performativa y estética. Precisamente, la política tiene un rol central en la configuración de las formas de ser, pensar y decir, por lo que se encuentra \emph{in nuce} en toda delimitación de la experiencia política.

\section{Distorsión y sujeto político: comunidad de litigio}

Como ya hemos planteado, nuestra lectura apunta a enfatizar el carácter co constitutivo de las lógicas de la policía y la política, dejando de lado un tratamiento dicotómico de estos términos como plantea Žižek, argumentando en primer término que debemos comprender la distorsión de lo social. Para comenzar, traemos a colación una noción rancièrana de lo político. La alusión a este término es más bien extraña en la obra de Rancière, se desarrolla con detenimiento solo en algunos pasajes. Si la policía se refiere al proceso de jerarquización, de establecimiento unitario y saturado de diferencias sociales junto con su legitimación, y la política alude a la práctica de verificación de la igualdad, lo político, como tercer elemento de esta presentación, se refiere aquí al encuentro entre estos dos procesos heterogéneos. La forma de que sea posible el encuentro, teniendo en cuenta esta heterogeneidad de las lógicas mencionadas, es que no haya una negación absoluta de una sobre la otra,\footnote{Precisamente, esta negación absoluta estaría a la base de las propuestas criticadas por Žižek.} sino que esa relación asuma la caracterización de un daño. La policía no niega radicalmente la igualdad, sino que al establecer la jerarquía propia de toda relación social le hace daño, de allí que lo político, como encuentro entre estas dos lógicas, \enquote{es el escenario sobre el cual la verificación de la igualdad debe tomar la forma del tratamiento de un daño}.\footnote{\cite[][1]{@7082-RANCIERE2004}, versión digital.}

De esta manera, la emergencia política de subjetividades adquiere la forma de una comunidad de litigio. El sujeto político conforma un espacio común en torno a una distorsión radical. Para completar este punto, utilizaremos algunos ejemplos de las investigaciones del autor. Rancière recupera en su análisis las luchas obreras en Francia durante la década de 1830,\footnote{Hay trabajos donde Rancière muestra el resultado de estas investigaciones, tales como \cite[][]{@7083-RANCIERE1991}; y \cite[][]{@7084-RANCIERE2010}.} se sitúa en la práctica de los trabajadores y observa el modo en que, ante la declaración en la Carta promulgada en 1830 de que todos los ciudadanos son iguales, en vez de denunciar la desigualdad real en que se encontraban y vociferar contra la mentira de la letra de la ley, los obreros buscaban verificar esa igualdad, cambiando las causas que motivaban su situación. No había allí una oposición de la frase al hecho, sino de la frase de la Carta a las frases de los patrones que proclamaban la desigualdad, creando en la distancia entre ellas un lugar polémico en que nuevas razones tenían lugar. Los obreros pretendieron verificar el poder performativo de la frase que instaura su cualidad de ciudadanos franceses, generando, a través de la práctica de la huelga, un efecto de demostración de razones. La contradicción entre lo real y lo formal se erige en un lugar fructífero desde el cual el sujeto político pretende verificar su igualdad. Estos casos de lucha de los obreros franceses en la década de 1830 manifiestan una contradicción lógica pero los sujetos no se detienen en ella, sino que performan en ese lugar imposible la demostración polémica que permite crear una comunidad de litigio. Es aquí entonces que hay subjetivación, es decir, emergencia de sujetos políticos. Como vemos, este mundo común que se constituye a partir de la práctica política no es un mundo informado por el consenso, sino que consiste en una comunidad de reparto {[}\emph{partage}{]} en el doble sentido de distribución y división, en que esa pertenencia \emph{común} se realiza por la polémica, por el \emph{conflicto}.

En el no-lugar que deviene de la aparente contradicción de una norma igualitaria y una situación de desigualdad, los obreros franceses construyeron un espacio en que sus razones podían ser consideradas válidas. Como vemos, este espacio común, donde la lógica policial de la jerarquía se ve interrumpida, se constituye alrededor de un conflicto. Y el sujeto que emerge allí lo hace en nombre de un daño.\footnote{El actualizar la lógica igualitaria implica \enquote{una fractura primera por la que se introducen en la comunidad de seres parlantes aquellos que no estaban incluidos}. \cite[][67]{@7065-RANCIERE2007}. De allí que todo acto político devenga de un daño por el que eran excluidos aquellos cuyas palabras eran mero ruido para el \emph{logos} comunitario. Pero es importante destacar que esa fractura supone proyectar hacia atrás la presuposición igualitaria. Esta verificación política crea un lugar con un principio de argumentación y un espacio polémico. La posibilidad del daño se sustenta sobre la distorsión primigenia del orden social sin fundamento, de allí el juego con la palabra francesa \enquote{\emph{tort}}, que alude tanto al daño como a la distorsión. Siempre se niega esta igualdad en torno a un nombre, pero no es propio de ningún nombre el llevar \emph{a priori} tal daño. No hay así sujeto político por antonomasia.} que el orden policial realiza sobre el principio de la igualdad. Esta situación adquiere una forma constitutiva de lo social: toda relación social, jerárquica, ejerce un daño sobre esta presuposición universal, que, sin embargo, solo puede verificarse en el caso \rdm{particular} de la emergencia excepcional de la política.\footnote{Nos interesa remarcar que esa universalidad no preexiste a las prácticas políticas, sino que son estos casos los que permiten comprobarla. La igualdad entonces no es una universalidad que se aplique en casos particulares, sino que debe ser verificada en la práctica. Ese proceso de verificación da lugar a demostraciones del daño y, por lo tanto, a la constitución de comunidades de litigio. Podríamos así decir que no hay más igualdad, universal, que la que se verifica siempre a través de la práctica política, particular.} Una vez más, la distorsión ontológica es la condición de posibilidad para pensar al sujeto político en este conflictivo espacio abierto por la tensión entre las lógicas policial y política.

Ahora bien, la política no solo se refiere a aquello que se dice sino también y sobre todo, a quién está investido de la legitimidad para hacerlo. Hay una lógica de la política, radicada en la dualidad del \emph{logos}, que es al mismo tiempo palabra legítima y cuenta de ella. De allí que la política siempre apunte a la institución de un escenario común donde tratar el litigio. \enquote{En el corazón de toda argumentación y de todo litigio argumentativo políticos hay una disputa primordial que se refiere a lo que implica la inteligencia del lenguaje}. \footcite[][68]{@7064-RANCIERE2010}

En este sentido, la práctica política apunta a una reconfiguración de la partición de lo sensible.\footnote{La co-constitutividad de la policía y la política se juega allí. La división de lo sensible es denominada por Rancière como \enquote{ese sistema de evidencias sensibles que pone al descubierto al mismo tiempo la existencia de un común y las delimitaciones que definen sus lugares y partes respectivas (\ldots)  La división de lo sensible muestra quién puede tomar parte en lo común en función de lo que hace, del tiempo y del espacio en los que se ejerce dicha actividad (\ldots)  Esto define el hecho de ser o no visible en un espacio común, estar dotado de una palabra común (\ldots)  Hay por lo tanto, en la base de la política, una \enquote{estética}  (\ldots)  como el sistema de las formas que \emph{a priori} determinan lo que se va a experimentar}. \cite[][15]{@7085-RANCIERE2002}. La verificación política de la igualdad de los no contados motiva a un nuevo reparto de lo sensible, disputando quiénes son los que participan del \emph{logos} comunitario.} La disputa política se refiere siempre a la cuestión \emph{prejudicial}, es decir, a la validez del mundo común y a los actores involucrados en esa interlocución que abren las condiciones de posibilidad para que ciertas prácticas, ciertos argumentos sean considerados partes del \emph{logos} que informa a la comunidad. Siguiendo con el caso de los obreros franceses en 1830, Rancière no solo destaca la performatividad del sujeto político, sino también el hecho de que la lógica política que se verifica en este caso indica cómo el \enquote{escenario de comunidad no existe más que en la relación de un \enquote{nosotros} con un \enquote{ellos}. Y esa relación es asimismo una no relación}. \footcite[][74]{@7064-RANCIERE2010} Esto indica que el conflicto por el carácter válido de las partes abre un espacio en que se acercan y distancian estos dispositivos de identificación: es una \emph{comunidad} y también una \emph{división}. Por eso la política al irrumpir establece comunidades de litigio, porque el conflicto crea nuevos espacios. Es un mundo común precisamente en que se encuentran interlocutores cuyo mismo carácter debe ser dotado de validez, en el marco del enfrentamiento entre ellos.

La práctica política implica así una argumentación que, en un mismo movimiento, obliga a reconfigurar la partición de lo sensible de forma tal de validar sus propias prácticas según el \emph{logos} de los semejantes. No hay así posibilidad de separar el lenguaje que da razones, del lenguaje creativo que desplaza sentidos de lo social. De esta manera, la emergencia del sujeto político supone siempre la performatividad de su práctica, es decir, que no hay identidades preexistentes y estancas sino que estas se constituyen en el desplazamiento propio de toda subjetivación. \enquote{La demostración propia de la política siempre es al mismo tiempo argumentación y apertura del mundo donde la argumentación puede ser recibida y hacer efecto}. \footcite[][76]{@7064-RANCIERE2010} En estos casos, donde la argumentación está unida a la apertura de un espacio para poder ser efectiva, la dimensión creativa, metafórica, adquiere preeminencia. De aquí que a la lógica de la demostración le \enquote{corresponda} una estética de la manifestación: toda racionalidad de esta nueva comunidad de litigio precisa abrirse al juego de una partición \emph{desplazada} de lo sensible. La estética y la política se relacionan así en función de la capacidad performativa del lenguaje, aprovechando la siempre fallida relación entre las palabras y las cosas, el significante y el significado, creando de tal forma una comunidad de litigio donde estén en disputa los sentidos dados a las prácticas.

Así, uno de los corolarios de nuestro énfasis en una ontología de la distorsión que interpretamos en las contribuciones rancièranas, se refiere a la potencialidad del sujeto político para crear mundos imposibles desde la posibilidad que está permeada por el conflicto. Esta capacidad estética y creativa se engarza con lo dado de manera especial, pero no implica dejar radicalmente de lado la realidad en que nos situamos. Este olvido de la dimensión policial, debido a un supuesto excesivo énfasis en la lógica política, está presente en la lectura dicotómica de la que pretendemos alejarnos aquí. Dice Žižek:

\begin{quote}
	(\ldots) cuando Rancière o Badiou menosprecian la política como una policía que se limita a cuidar el aceitado \emph{service des biens}, omiten considerar el hecho de que el orden social no puede reproducirse si se limita a los términos del \emph{service des biens}: debe haber Uno que asuma la responsabilidad final (\ldots)  los abogados de lo político en tanto opuesto a lo policial no toman en cuenta el exceso inherente al amo que sostiene el \emph{service des biens} (\ldots)  En síntesis, no son conscientes de que su demanda incondicional de \emph{egaliberté} no va más allá de una provocación histérica dirigida al Amo.\footcite[][258]{@7063-ZIZEK2005}
\end{quote}

Así, parece que, en primer lugar, la oposición de la policía y la política es tajante; y en segundo lugar que ese carácter de exterioridad entre ambas supone que el reclamo político es insuficiente ya que no apunta a la centralidad del orden policial: el Uno, el Amo siempre excesivo que \emph{sostiene} el orden.

Estas consideraciones habilitan a Žižek a decir algo así como que la provocación histérica en la que caen las propuestas de los autores reseñados surge de suponer una excesiva confianza en \enquote{el poder sustancial del orden positivo del ser {[}en lenguaje rancièrano según Žižek: la policía{]}, pasando por alto el hecho de que el orden del ser nunca es simplemente dado (\ldots)  la brecha del acto {[}político{]} no se introduce posteriormente en el orden del ser: está allí todo el tiempo\ldots}. \footcite[][258]{@7063-ZIZEK2005} Como vemos, sostener, por un lado, el carácter central del tratamiento de la igualdad en la argumentación rancièrana, y por otro, la co-constitutividad de las lógicas de la policía y la política, permite plantear que la propuesta de Rancière efectivamente parte de considerar el \emph{carácter distorsionado de todo orden social}, por lo que nunca es consistente de por sí y siempre está sostenido en la tensión entre la desigualdad de las relaciones sociales y la igualdad de los seres parlantes. Esa distorsión constitutiva es la que abre el juego de la subjetivación política, no como irrupción momentánea en el trasfondo de un orden positivo consistente que engulle al sujeto y vuelve a la tranquilidad de un predominio de la lógica policial, sino como verificación contingente de la igualdad que sostiene a todo ser parlante como tal, que habilita a una práctica performativa y estetizante que reformula los parámetros de inteligibilidad comunitaria {[}\emph{logos}{]} inaugurando así nuevos sentidos y nuevas prácticas del vivir en común.

\section{La subjetivación, entre identificaciones}


Una vez que hemos analizado la emergencia del sujeto político en la institución de una comunidad de litigio, como así también en su capacidad de reconfigurar el reparto de lo sensible, veamos otro aspecto de la específica relación entre policía y política, referido a la dinámica de la subjetivación y la identificación. Toda identidad es tal al interior de una cierta distribución de funciones y modos de ser, hacer y decir, que se \emph{presenta} clausurada. La subjetivación política implica romper con la lógica uno a uno, o unitaria, de la identificación para demostrar que hay un \enquote{más-uno} que abre el juego polémico que disputa los límites de la comunidad. Así toda subjetivación supone una desidentificación, yendo más allá de la distribución policial de posiciones y funciones. El sujeto político, dice Rancière, siempre está entre-dos, en el intervalo de identidades, y verifica la igualdad de cualquiera con cualquiera. Prosiguiendo con la caracterización de los sujetos políticos, estos no pueden ser encarnados en una identidad específica, escapan a la lógica estricta de representación, precisamente, por ubicarse en el intervalo entre las identidades, por exceder siempre la distribución de funciones. Desde ya podemos ver cómo es necesaria la relación paradojal entre identidades al interior de un orden distorsionado, y sujetos que emergen en la actualización de ese daño, que expresa la inconmensurabilidad entre dos lógicas que se constituyen mutuamente.

De esta manera podemos comprender que \enquote{la lógica de la subjetivación es una heterología, una lógica del otro}, \footcite[][3]{@7082-RANCIERE2004} ya que, en primer lugar, nunca un sujeto \emph{afirma simplemente una identidad}, es decir, siempre se sitúa \enquote{más allá} de esa distribución determinada, abriendo un espacio novedoso, un mundo de sentidos donde se disputa la validez de los argumentos presentados. En segundo lugar, la subjetivación es heterológica porque un sujeto siempre demuestra algo a otro, incluso ante el rechazo de este último, instituyendo así una comunidad basada en el litigio. Finalmente, el sujeto siempre se dirige a otro, porque siempre implica una identificación imposible, alude a una posición cuya manifestación rompe la distribución jerárquica precedente.

Continuando con esta dinámica de la subjetivación política, Rancière plantea que los \enquote{sujetos políticos existen en el intervalo entre diferentes \emph{nombres} de sujetos}. \footcite[][86]{@7066-RANCIERE2007} Estos nombres están caracterizados por su carácter intrínsecamente polémico, al estar disponibles a la disputa por los límites de su comprensión, de~allí que puedan ser suplementados políticamente. Esa práctica de suplemento es la que caracteriza al sujeto político. La reivindicación de la pretensión igualitaria que está a la base de la política implica disputar la semantización de los nombres, tales como hombre, ciudadano, proletario, mujer. Lo que estos nombres designen será siempre litigioso, estará atado a la disputa, y por ello a la suplementación política.

El suplemento sirve aquí para enfatizar la polisemia que caracteriza al lenguaje. Nunca un término posee un significado pleno y positivo, sino que este está siempre sujeto a la conflictividad política. Ella lo completa, \emph{sin agregar nada}, más que la inclusión, performativa, en un nuevo mundo de sentido donde las argumentaciones y manifestaciones puedan considerarse válidas. Allí, Rancière retoma el ejemplo de Olympe de Gouges, una escritora muy conocida durante la época de la Revolución Francesa por defender los derechos de las mujeres, y en general de todas las minorías. Ella escribe en su Declaración de los Derechos de la Mujer y la Ciudadana, en 1791 que \enquote{si la mujer tiene el derecho de subir al cadalso, debe tener también igualmente el de subir a la Tribuna}, \footcite[][87]{@7066-RANCIERE2007} enfatizando así la capacidad de participación pública de las mujeres. Aquí Rancière se detiene en la forma lógica de la protesta de las feministas de esta época, donde se puede observar cómo la subjetivación política radica en el \enquote{entre} de los nombres de mujer y ciudadana. Las mujeres en la Francia pos-revolución no podían ejercer los derechos de expresión que la Declaración de los Derechos del Hombre y el Ciudadano les otorgaba, pero al protestar, al exigir la igualdad de tratamiento como miembros válidos participantes ejercían ese mismo derecho que se les negaba. Lo primero que salta a la vista, nuevamente, es la capacidad performativa, creadora de identidad, del acto político: se dota de un sentido novedoso a la mujer, y se ejerce un derecho que era negado en los hechos. Entre la mujer como ser humano y la mujer como ciudadana, en el intervalo de esas identidades, el sujeto político reconfigura \enquote{las distribuciones de lo privado y lo público, de lo universal y lo particular}. \footcite[][89]{@7066-RANCIERE2007}

Los sujetos políticos se encuentran entonces en relación de conflictividad y de capacidad performativa con respecto a las identidades presas de la lógica unitaria del orden social. La relación entre policía y política seguirá carriles similares. La política encuentra su lugar en las heterogéneas posiciones de lo social, en los posibles desplazamientos de las fracturas que señalan las partes que no cuentan como tales. Pueblo, proletario, mujeres, son distintos nombres que permiten verificar esa distorsión que mencionamos. Como ya hemos adelantado, nuestro autor rechaza la posibilidad de que la política tenga que ver con identidades predeterminadas. Lejos de ello el espacio para la emergencia del sujeto político es habilitado por la dinámica entre la identificación y la desidentificación. El escenario en que se verifica el encuentro entre las dos lógicas heterogéneas de la policía y la política, es decir, el espacio de lo político, se abre debido a que el cómputo de las partes es siempre erróneo, no debido a una negación radical de la igualdad, que implicaría la exterioridad entre las lógicas de la policía y la política, sino debido al daño que todo orden social marca en ella, dando forma así a una noción distorsiva de la realidad social. Este espacio es ocupado por los sujetos políticos que actúan sobre el intervalo de los nombres cuyo alcance es polémico. Así los dispositivos subjetivos llevan adelante las prácticas políticas, ya sea que encarnen el exceso que desbarata la distribución comunitaria de partes, el pueblo, o bien por representar la excepción de los sectores no contados en el reparto policial. \enquote{La política en general está hecha de esas cuentas erróneas (\ldots)  que inscriben con el nombre particular de una parte excepcional \emph{o} de un todo de la comunidad (los pobres, el proletariado, el pueblo) la distorsión que separa y reúne dos lógicas heterogéneas de la comunidad}.\footnote{\cite[][56]{@7064-RANCIERE2010} (las cursivas son nuestras).} En ambas circunstancias, la lógica política permite distinguir la distorsión que impide la total identificación entre una distribución de partes, y la plenitud de su construcción simbólica, debido a la inconmensurabilidad de la igualdad de las inteligencias y la desigualdad que habita en las relaciones sociales. La sociedad es, de esta manera, siempre distinta de sí misma.

Precisamente, la ausencia radical de fundamento para el gobierno indica la co-constitutividad de la policía y la política. El carácter de esta relación es indicada por la posición fundante de la distorsión: \enquote{la persistencia de esta es \emph{infinita} porque la verificación de la igualdad es \emph{infinita} y la resistencia de todo orden policial a esa verificación es una \emph{cuestión de principios}}.\footcite[][57]{@7064-RANCIERE2010}[(las cursivas son nuestras)] La imposibilidad de encontrar una medida justa entre ambas dimensiones no inhabilita sin embargo a procesar esa tensión, \enquote{(\ldots)  mediante dispositivos de subjetivación que la hacen consistir como \emph{relación modificable entre partes}, como \emph{modificación incluso del terreno sobre el cual se libra el juego}}.\footcite[][57]{@7064-RANCIERE2010}[(las cursivas son nuestras)] Entre la lógica policial y la lógica igualitaria persiste una tensión constitutiva, lo político \enquote{trata} esa tensión, a través de la subjetivación, el desplazamiento de fronteras que modifican las relaciones del hacer, del ser y del decir. De esta manera, la dinámica del evento igualitario radica en ese espacio abierto, entre la emergencia del sujeto político y la instancia de ordenamiento de los nombres en litigio. La emergencia de la subjetividad política establece \enquote{una \emph{topología}, una \emph{distribución aleatoria} de lugares y casos, de sitios y situaciones que son, en su dispersión misma, otras \emph{tantas ocasiones para un resurgimiento del significante igualitario}, para un nuevo trazado de verificación de la comunidad de iguales}. \footcite[][71]{@7065-RANCIERE2007}

\section{La emancipación y la democracia}


Un comentario nos queda sobre la co-constitutividad de política y policía, a partir de la distorsión ontológica, que tiene que ver con la crítica a una noción de emancipación total, un corolario que se entronca directamente en una de las cuestiones centrales del pensamiento político contemporáneo, la crítica al totalitarismo. Rancière ubica la posibilidad de una emancipación en la práctica de demostración de razones: \enquote{probar que efectivamente pertenecen a la sociedad, que efectivamente se comunican con todos en un espacio común, que no son solo seres de necesidad, de queja o de grito (\ldots)  emanciparse no es escindirse, es afirmarse como copartícipe de un mundo común}. \footcite[][39]{@7086-RANCIERE2007} Sabiendo que todo mundo común está informado por la tensión entre las lógicas igualitaria y desigualitaria, esta emancipación nunca es total, sino más bien precaria y contingente.

De este modo, no es factible plantear la posibilidad de una emancipación absoluta como la capacidad de alcanzar una situación de plenitud. Desde el momento en que se considera factible identificar la emancipación con un sujeto político específico, la práctica política termina oponiéndose a su propio principio, es decir, cae en una verdadera contradicción en sus términos.\footnote{Aquí nos es posible apuntar uno de los callejones sin salida que una visión dicotómica de la obra rancièrana, como la que rastreamos en Žižek, nos provee. El filósofo esloveno apunta correctamente que \enquote{la universalidad de la que estamos hablando no es entonces una universalidad positiva con un contenido determinado, sino una universalidad vacía (\ldots)  que solo existe como la experiencia de la injusticia inflingida al sujeto particular que politiza su situación difícil}. \cite[][245]{@7063-ZIZEK2005}. Sin embargo, más adelante considera que \enquote{{[}p{]}ara Rancière, la subjetivización involucra la afirmación de un \emph{singulier universal}, la parte singular / excesiva del edificio social que \emph{encarna directamente} la dimensión de la universalidad}. \cite[Ver][249]{@7063-ZIZEK2005}. Las cursivas son nuestras. Aquí la noción de universalidad está tomada en dos registros distintos: la universalidad en Rancière es la igualdad de los seres parlantes, que es verificada por el sujeto en un verdadero acto político, pero que no la \emph{encarna}, y nunca podría hacerlo, ya que desde el momento de ocuparla absolutamente para sí, la emancipación se anularía. Precisamente en la tensión que acompaña a todo proceso de emancipación, Rancière centra el núcleo de su reflexión. Y allí es preciso detenerse en la relación que entabla el argumento entre la igualdad y el acto político. Esta igualdad en Rancière no es un objetivo a cumplir, sino un presupuesto a verificar, que se asume como universalidad vacía: todos somos iguales para participar de los asuntos considerados comunes y en ese nombre reivindicamos una solución para las injusticias inflingidas. Pero la política no puede asumir como universal el lugar de su propia verificación. Desde el momento en que lo hace, obtura la emancipación posible, pretendiendo llevar consigo la verdad que justifica el orden social de su propuesta. Es por eso que la igualdad no es en sí política.} Cualquier vanguardia iluminada implica la institucionalización que ordena las diferencias entre aquellos que conocen, y aquellos a los que se les enseña cómo o qué conocer, haciendo imposible de ese modo cualquier emergencia de una subjetividad política. Quizás en este sentido se ubica la misma reflexión de Rancière, escapando de verdades absolutas, sin recetas específicas, sino más bien intentando abrir el juego a la verificación de la igualdad, a la práctica política--democrática: \enquote{Yo no digo nunca qué hay que hacer o cómo hacerlo. Yo intento rediseñar el abanico de lo pensable con el fin de remover las imposibilidades y las prohibiciones que se alojan a menudo en el corazón mismo de los pensamientos que se pretenden subversivos}. \footcite[]{@7087-RANCIERE2007}

Así entonces, la política transita siempre entre precipicios. Por un lado, se encuentra la apropiación por la inflación absoluta de la lógica policial que pretende volver natural la partición de lo sensible que deviene de todo evento igualitario, pero por el otro, se encuentra la erección de la lucha política concreta en verdad irrefutable, universal por sí misma. Así, en las palabras del autor,

\begin{quote}
	(\ldots) toda política actúa también en el borde de su riesgo radical que es la \emph{incorporación policial}, la realización del sujeto como cuerpo social. La acción política se sostiene siempre en el intervalo, entre la figura natural, la figura policial de la incorporación de una sociedad dividida en órganos funcionales y la figura límite de otra incorporación arquipolítica o metapolítica: la transformación del sujeto que sirvió para la desincorporación del cuerpo social natural en un cuerpo glorioso de la \emph{verdad}.\footnote{\cite[][118]{@7064-RANCIERE2010} (las cursivas son nuestras).}
\end{quote}

La política radica en la tensión entre caer en la distribución diferencial del orden establecido, y embarcarse en el proyecto absoluto de sostener que hay una plenitud posible a partir de llevar hasta las últimas consecuencias las condiciones que presenta ese sujeto político. esta es la raíz, pensamos, de la propuesta rancièrana, ya que precisamente, ¿no es a \emph{esto} a lo que se refiere un proyecto de una democracia radical: sentar las bases que permitan el desarrollo de luchas políticas en la tensión entre orden y el conflicto, demostrando performativamente las distorsiones de un orden que se sustenta en la ausencia de fundamento, sin pretender llevar en su seno la simiente de una verdad incontestable?

\section{Algunas notas para concluir}


Pensar ontológicamente a la política supone, por un lado, despegar la reflexión de la búsqueda de aquellas condiciones de posibilidad para el conocimiento de lo que es la política, y así, por el otro, considerar que lo dado se encuentra en una continua tensión de configuraciones y reconfiguraciones, donde el conflicto, político, adquiere así un carácter central. La obra de Rancière se detiene en esta siempre-re-configuración de lo dado, y destaca allí su carácter distorsionado. A lo largo del presente artículo hemos pretendido defender una ontología de la distorsión, que se encuentra presente en nuestra lectura sobre la obra del filósofo argelino, y que permite al mismo tiempo desmarcar a esta última de ciertas lecturas simplistas, dicotómicas, violentas que se realizan sobre ella.

La distorsión se hace presente en dos vertientes principales que se articulan, como esperamos haya quedado claro a partir de lo dicho en los apartados previos. En primer lugar, en la relación de inconmensurabilidad entre la igualdad de las inteligencias, la igualdad del lenguaje, y la desigualdad implícita en toda relación social que supone la jerarquización de cuerpos, funciones y prerrogativas. En segundo lugar, en el carácter co-constitutivo que adquiere la lógica de la policía y la de la política. Esta relación específica es posible de ser caracterizada a partir de la manera en que Rancière desarrolla la noción de sujeto político, y también la idea de la distancia aporética entre la subjetivación y la identificación.

El sujeto político se ubica en la construcción de una comunidad de litigio que cuestiona los principios del \emph{logos} establecido, abriéndose a la construcción de nuevos marcos de sentido donde la demostración de sus razones sea considerada válida. Allí, las partes que no eran tenidas en cuenta se arrogan la capacidad de hablar, retrazando las líneas de lo común, reconfigurando la partición de lo sensible. La performatividad del sujeto político viene unida al carácter litigioso de su emergencia y de su práctica. El conflicto crea nuevos sentidos, activa la posibilidad de reunir significantes con significados fuera de lo ya establecido.

Antes que pensar la diferencia entre policía y política como la dicotomía de un orden ontológico positivo frente a una brecha que imposibilita su clausura, tal como lo observamos en la lectura de Žižek, hemos enfatizado aquí el carácter paradójico \rdm{fundante y abisal} de la igualdad, la distorsión a la base de todo orden social infundado, y la tensión necesaria entre la policía y la política, expresada en la construcción de comunidades de litigio, único espacio posible de emancipación. En cada una de las prácticas políticas, se construye la efectividad de ese principio no político de la igualdad, siendo imposible entonces alcanzar una sutura absoluta de la distorsión constitutiva del orden social. De esta manera, la dinámica política no puede seguir los carriles de una misión emancipadora absoluta, pero tampoco caer en situaciones de distribución plena de diferencias sociales. Entre ambos extremos se abre la posibilidad política de la democracia, no una demanda histérica y marginalista, sino la apuesta por la creación de nuevos mundos de sentido, de los lugares donde lo que hay, la facticidad sin razón inmanente, se presta a retrazados igualitarios, a desplazamientos contingentes.

\section*{Referencias}
\printbibliography[heading=none]   % Sin título automático



%%%%%%%%%%%%%%%%%%%%%%%
\ifPDF
\separata{capitulo4}
\fi

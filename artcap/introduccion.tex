\ifPDF
\chapter[\hspace{1.5pc}¿Por qué ontologías políticas?]{¿Por qué ontologías políticas?}
\chaptermark{Introducción}
\Author{Introducción}
\setcounter{PrimPag}{\theCurrentPage}
	\else
	\ifHTMLEPUB
	\chapter{¿Por qué ontologías políticas?}
	\fi
\fi



\section*{1.}


La apuesta colectiva de este libro surge de un posicionamiento frente a los estudios contemporáneos sobre la política. Con la expresión \enquote{ontología política} planteamos un distanciamiento respecto al privilegio del abordaje gnoseológico, puesto que la teoría política, la filosofía política o el pensamiento impolítico suponen un vínculo singular entre el saber y la política. Tratándose siempre de las posibilidades o imposibilidades abiertas por esa vinculación entre un área del saber y determinadas formas políticas. Esto ha llevado a cuestionamientos recurrentes en el pensamiento contemporáneo a la filosofía o teoría política en tanto las mismas determinaciones de lo filosófico o lo teórico imposibilitarían abordar en toda su complejidad la política sin llevar a su subordinación o eliminación.

Frente a ello, creemos oportuno hablar de ontologías políticas en tanto se trata de formas de pensar la configuración del mundo y no solo elaboraciones teóricas. Esto se debe a que en tal caso se parte, al mismo tiempo, de cierta definición de lo teórico y de un área de la realidad nombrada con el término política. Las diversas ontologías políticas presentadas aquí no surgen de una relación exterior entre conocimiento y realidad, o entre sujeto y objeto, sino que son formas de pensar cómo se constituye el mundo como tal. De un lado, se destituye el privilegio de la teoría en cuanto no se define una forma de conocer, sino una forma de constituir el mundo. De otro lado, la política ya no se considera un área determinada dentro de la realidad, sino el mismo proceso de constitución de \linebreak  lo real (lo que supone un juego infinito entre lo constituyente y lo constituido).

Pretendemos poner en primer plano la cuestión del ser, su interrogación, rompiendo con la sutura epistemológica que sostiene que las teorías solo pueden preguntarse por las condiciones de posibilidad del conocimiento. De modo que la primera aproximación a un posicionamiento ontológico para abordar la política surge del distanciamiento y la subversión de las perspectivas epistemológicas, puesto que no se trata del conocimiento exterior de un objeto determinado, sino del pensamiento del \emph{ser qua ser}, es decir, de las maneras en las que se configura de uno u otro modo.

La apuesta entonces es pasar del conocer al ser.


\section*{2.}

La palabra ontología tal como lo indica su etimología nombra el \enquote{discurso sobre el ser}, es decir, la relación entre el discurso, el lenguaje o la razón con el ser en tanto que ser. Resulta central señalar que no nos referimos al ser en tanto que ser \rdm{el ser en sí mismo} sino al o los \emph{discursos} sobre el ser. La cuestión es pensar de qué modo, entonces, se da el vínculo entre discurso y ser tal como lo pensamos aquí.

Desde lo establecido en el apartado anterior es posible señalar que nos diferenciamos de dos perspectivas al respecto. En primer lugar, nos diferenciamos de una perspectiva que identifica pensamiento y ser, esto es, que parte de la \enquote{identidad} entre ambos. En segundo lugar, nos separamos de aquellas posiciones que la piensan como una relación de exterioridad, esto es, un discurso que se dirige al mundo (en su forma moderna implicaría el esquema de la representación donde el sujeto fundamenta la legitimidad del objeto). Para no pensar en términos de identidad o exterioridad, partimos de la copertenencia entre ser y pensar. Esta implicancia mutua da cuenta de una vinculación necesaria (de ahí la ausencia de identidad) pero sin pensarla como dos dimensiones opuestas (de ahí la  \linebreak ausencia de exterioridad).

El discurso sobre el ser no es un discurso sobre una dimensión u objeto externo, sino un discurso que en la pregunta por el ser abre su misma posibilidad. Dicho en otros términos, la identificación es imposible en tanto existe un distanciamiento propio de la pregunta que abre, pero es el mismo ser quien realiza la pregunta, en tanto no existe un algo más allá del ser que pregunte por el ser. Al preguntar por el ser en tanto que tal surge un pliegue en el mismo ser. Por ello ser y pensar son lo mismo sin ser idénticos.

La forma de los discursos sobre el ser que aquí presentamos es la pregunta o el preguntar. O mejor, los discursos presentados están sobredeterminados por la forma-pregunta. La pregunta por el ser es lo que \emph{abre} el mismo ser: no lo crea, no lo reconoce, no lo experimenta, no lo percibe, sino que es una indagación o una cuestión que abre una grieta en lo existente al preguntar por su modo de ser. En este sentido, la pregunta se dirige de un modo singular a la multiplicidad de lo existente en tanto indaga en esa diversidad por sus maneras de ser.

Los discursos sobre el ser en tanto que apertura (grieta, hiancia, brecha) de lo dado suponen dos cosas. Por un lado, que lo dado al mismo tiempo que es lo único existente no es solo lo dado. Existe una diferencia inherente en cuanto la pregunta permite diferenciar entre lo dado y aquello que lo hace ser como tal. Esto es lo que, siguiendo a Heidegger, podemos llamar diferencia ontológica. La pregunta es así la condición de posibilidad incesantemente renovada de la diferencia entre lo dado como existente y su modo de ser específico. Por otro lado, al introducir una grieta en lo real, lo dado deja de ser evidencia o inmediatez, para convertirse en algo cuya constitución es contingente. La pregunta ontológica abre al ser como pura posibilidad. Esto nos permite señalar que una indagación ontológica es aquella que piensa los modos en que se configura lo dado, o mejor, los procesos contingentes desde los que se estabiliza una forma de lo existente. Lo posible no es algo más allá del mundo, sino su misma condición, y, por lo tanto, la posibilidad de ya no ser como tal.

Por todo esto, los discursos sobre el ser \rdm{las ontologías propuestas aquí} tiene un estatuto cuasi-trascendental. \emph{Trascendental} en tanto abren lo dado más allá de lo dado sin conducir a otro existente. Desde la tradición kantiana lo trascendental indica un estatuto singular: la condición de posibilidad. \emph{Cuasi} en tanto la pregunta supone condiciones de posibilidad y de imposibilidad y, al mismo tiempo, no existe un trascendental puro sino que siempre se encuentra contaminado de facticidad o empiricidad.

Esta última indicación nos permite afirmar que nuestros discursos están producidos de manera situada y por ello se ubican en la tensión entre las discusiones teóricas y los acontecimientos políticos. La situacionalidad no la entendemos como la ubicación en determinado contexto histórico, que otra vez podríamos reconstruir como sujetos cognoscentes, sino como la contaminación irreductible de cualquier pureza teórica con lo que acontece. Esto no significa solo atenerse a lo existente, sino indagando sus condiciones abrir hacia un más allá incierto, la incertidumbre no es coyuntural sino su mismo exceso.

Los discursos sobre el ser son el preguntar que abre lo dado más allá de lo dado: a su modo contingente de configuración. Esto implica pasar de la pregunta por el \enquote{qué} a la pregunta por el \enquote{cómo}. Se trata de pensar el cómo de la multiplicidad de lo dado sin remitir a algo más allá de ello. Se acentúa de este modo el proceso de constitución de lo existente, lo que nos permite señalar que las condiciones de posibilidad son condiciones de existencia.

\section*{3.}

El término ontología tal como lo comprendemos aquí supone una determinada concepción de lo dado que se opone a dos perspectivas o, en otros términos, la singularidad de nuestra propuesta se comprende en el distanciamiento respecto del esencialismo y del constructivismo. Primero, al acentuar la dimensión ontológica como apertura en tanto posibilidad nos oponemos a cualquier posición metafísica que fije lo existente, fundándolo de modo trascendente o inmanente. Esto significa que aquello que existe no tiene una esencia o idea que pueda ser fijada de un modo definitivo, por lo que existe una inestabilidad constitutiva donde se producen estabilizaciones precarias. Segundo, nos distanciamos de cierto constructivismo que desde metáforas arquitectónicas supone un agente, una forma o idea y una materia informe. Este constructivismo parte de que lo dado es construido desde una alteridad respecto de lo dado, sea un sujeto individual, sea la sociedad en su conjunto, sea dios. Lo que, al mismo tiempo que cuestiona el esencialismo, restituye un lugar trascendente respecto del mundo que posibilita su construcción (constructivismo sobredeterminado por una especie de voluntarismo que bajo las formas del lenguaje, la cultura, el sujeto, la racionalidad o la sociedad ubican lo posible en un exterior, es decir, la contingencia como algo exterior a lo existente).

El doble distanciamiento, respecto del esencialismo y del constructivismo, permite comprender el vínculo entre ontología e historia que plantamos aquí. Si desde el esencialismo se afirma la perennidad de una idea o concepto, el constructivismo lo cuestiona señalando que existe una historicidad contextual constitutiva de los lenguajes y las instituciones políticas. Ahora bien, el problema del constructivismo es que la historicidad se ubica en un contexto exterior a aquello que historiza. Un concepto varía, así, porque es ubicado en uno u otro momento histórico. Frente a ello, aquí postulamos un historicismo radical, lo que significa dos cosas: por una parte, que la historicidad no es exterior o contextual sino inherente a un lenguaje o institución; por otra parte, que la indagación ontológica es historial y no histórica, se trata de la diferencia que hace posible la historia misma.

Desde nuestra perspectiva, lo dado al mismo tiempo es y no es lo único existente. Esta paradoja se entiende si afirmamos que no existe algo más allá, un fundamento exterior que dé origen a lo dado (cuya figura histórica por excelencia ha sido la del dios creador), pero al mismo tiempo la pregunta abre lo existente más allá de su existencia: abre una grieta en cuanto se establece la diferencia entre lo existente y su modo de ser (o entre ente y ser para retomar los términos heideggerianos).

Las ontologías como formas de indagación por el cómo de lo dado destituyen la explicación causal que supone una relación exterior entre el efecto y su causa, que también conlleva una temporalidad lineal donde existe una precedencia de la causa sobre el efecto. Aquí partimos de lo dado para indagar \linebreak cómo determinada configuración lo ha constituido como tal, por lo que   nunca se cierra como algo autoconstituido. Debido a que el ente nunca se constituye a sí mismo (imposibilidad de cierre), eso le imprime un carácter ontológico a la indagación, de su inacabamiento a la apertura constitutiva.

Por estos motivos, más allá del esencialismo y del constructivismo, partimos de la ontología como indagación sobre la \emph{constitución} de lo dado.

\section*{4.}

La utilización del calificativo \enquote{políticas} para referirnos a las ontologías presentadas busca mantenerse en la indeterminación. Partimos no solo de la variación histórica del concepto de política, sino de su contingencia, lo que significa que aun cuando fuera posible sistematizar la totalidad de las definiciones dadas de política no sería posible fijar su sentido. Esto nos permite afirmar que no existe concepto o esencia de la política. O, en otros términos, que la política es constitutivamente inadecuada a su concepto. Por lo que el concepto mismo se configura en su imposibilidad de cierre e inacabamiento significativo (esto no quiere decir que no se afirme a través de nuevos términos significantes)

La política es aporética, es decir, al mismo tiempo que está saturada por múltiples definiciones existe una falta que imposibilita esa saturación. Lo aporético se da en el cruce entre exceso y falta que se juega en distintos niveles discursivos y términos electivos (entre ontología y política, lo político y la política, la teoría y la praxis, etc.). Siendo así, la política comienza siempre con una lucha por la definición de la política, por la estabilización precaria de los límites que permiten considerar a algo político. Por esto mismo, la sobredeterminación de una definición de política desde el conflicto, el orden, el acuerdo, la tragedia, es secundaria respecto a su radical inestabilidad. Así, como no existe un núcleo último al cual acceder, solo es posible moverse en una u otra sobredeterminación.

La inestabilidad también se ubica en la política como dispositivo: la precaria fijación de un sentido de política surge en la tensión entre las formas-políticas y las formas-veritativas de una época. Con el término forma-política nos referimos a las instituciones, prácticas, relaciones políticas de una determinada época. Con el término forma-veritativa nos referimos al modo como se configura el saber y la verdad también epocalmente. La política, así, es la tensión entre formas políticas y veritativas en cuanto nunca se corresponden de modo pleno.

Con esto queremos indicar, primero, que no partimos de una definición dada de política, sino que la misma surge en los diferentes textos (apostando así por la indefinición o la definición en acto en cada apuesta de pensamiento); segundo, que la política no tiene un vínculo privilegiado con la ontología, no existe una relación necesaria entre una y la otra, sino que aquí pensamos ontologías que en su pluralidad pueden ser calificadas de políticas. Puede haber otras ontologías (estéticas, matemáticas, etc.) y puede haber otras ontologías políticas. Aquí presentamos distintos discursos que no tratan la política como un área determinada, sino como formas posibles que puede adquirir la configuración del mundo.

\section*{5.}

Atender a la constitución de lo dado es indagar por su modo de ser. Esta indagación, señalábamos, no parte simplemente de la multiplicidad o pluralidad de lo existente, sino que abre hacia su dimensión ontológica. Esto conlleva un doble movimiento: un momento negativo puesto que al indagar por el modo de ser se niega lo existente como tal, el ser de algo no es lo dado, no es lo ente, y así es la nada de lo ente; pero también un momento positivo en tanto allí aparecen los modos de constitución de lo existente como procesos de configuración.

Una perspectiva ontológica como la propuesta aquí de ningún modo legitima el mundo como tal, puesto que parte de su socavamiento. Se trata de una indagación que abre lo existente a su configuración desde un trasfondo de posibilidades. De ahí que rompa con la lógica de la legitimidad que supone la exterioridad del juicio. En otros términos, la apuesta por pensar ontologías políticas supone una redefinición de lo que se entiende por tarea crítica del pensamiento. La pregunta por la legitimidad de lo dado, desde su dependencia del dispositivo político moderno, se dirige a preguntar por el porqué de lo existente. Esto es, supone la estructura del juicio en tanto el tribunal de la razón juzga lo existente desde un criterio y conlleva por ello mismo la fijación de un fundamento que otorgue o niega la legitimación (en términos históricos sería posible mostrar el paso de una fundamentación trascendente a una inmanente).

La ontología como crítica más allá de la crítica no juzga lo real desde la razón, sino que abre lo real a su posibilidad. Por ello mismo se trata de una posición que constituye en sí misma una apuesta política. No se trata de no aceptar lo dado en cuanto no se ajusta a un criterio previamente establecido, sino de mostrar que lo dado no es tal, que su modo de ser es la misma posibilidad.

Nuestra apuesta teórica, nuestra apuesta política: preguntar para abrir lo posible \emph{en} lo dado.

\vspace{8mm}
\begin{flushright}
	\textbf{Los autores.}
\end{flushright}

\ifPDF
\separata{introduccion}
\fi


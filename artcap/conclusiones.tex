\ifPDF
\chapter[\hspace{1.5pc}Conclusiones]{Conclusiones}
\chaptermark{Conclusiones}
\Author{Conclusiones}
\setcounter{PrimPag}{\theCurrentPage}
\fi

\ifBNPDF
\chapter[\hspace{1.5pc}Conclusiones]{Conclusiones}
\chaptermark{Conclusiones}
\Author{Conclusiones}
\fi

\ifHTMLEPUB
\chapter{Conclusiones}
\fi


Hemos tratado de mostrar en este breve recorrido por diversos autores, debates casi olvidados y heterogéneos niveles discursivos, nuestra (im)propia ontología filosófico política, donde el término \emph{nodal} ha venido a resultar ser el efecto mismo de sus respectivos cruces y equivalencias, hasta el punto de producir cierta indistinción relativa de los términos en juego (ontología=política=filosofía) al interior de este marco de pensamiento posfundacional y de su breve genealogía conceptual materialista. Hemos mostrado así, un sujeto filosófico que se constituye como operación implícita y contingente de articulación \emph{entre} discursos heterogéneos e irreductibles. Sujeto que no estaba antes y que tampoco conduce luego a ninguna superación, progreso o síntesis de las perspectivas esbozadas. Nuestro método ha consistido más bien en una suerte de \emph{materialismo discursivo del encuentro}, por el cual los discursos se han aproximado y contorneado desde distintos bordes, encontrando puntos de convergencia y bifurcaciones, tensiones y compatibilidades. Pues, una ontología crítica de nosotros mismos, en sentido foucaultiano, no se escribe aparte, es decir, no se pretende trascendental, fundamental u originaria, sino que se escribe entre las líneas difusas de los discursos circulantes.\footnote{Aurora Romero desarrolla esta perspectiva foucaultiana en el capítulo \enquote{Ontología genealógica} (capítulo~\ref{cap:ontologia}, página~\pageref{cap:ontologia}).}

Pensamos así la ontología filosófico materialista como operación efectiva de articulación de otras prácticas, transvalorando las formas habituales de distinción y clasificación, sus límites y posibilidades; en ese sentido la calificamos de \emph{transpolítica} (más que metapolítica). Al colocarse bajo condición de verdades olvidadas, desestimadas, o valoradas según parámetros rígidos, y al pensarlas en conjunción/disyunción, esta perspectiva materialista \emph{es} política en segundo grado: anuda y composibilita verdades subversivas a los órdenes y saberes instituidos (incluso método-lógico-gramaticales). \emph{Ergo}, no se trata aquí de re-presentar, reflejar o fijar correspondencias, ni siquiera de describir o prescribir lo que acontece, se trata, en cambio, de composibilitar/articular de manera abierta y compleja ámbitos de pensamiento heterogéneos e irreductibles entre sí; de mostrar que no existe relación esencial/fundamental entre ellos, sin que por esa razón sea imposible su pensamiento conjunto, siempre naciente y precario. En definitiva, hemos tratado de circunscribir el ser esencialmente contingente que surge de rigurosos anudamientos discursivos. Esta ha sido \rdm{y continúa siendo} nuestra apuesta ontológico-política.


%%%%%%%%%%%%%%%%%%%%%%%
\ifPDF
\separata{Conclusiones}
\fi

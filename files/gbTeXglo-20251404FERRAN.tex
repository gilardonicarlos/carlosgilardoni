
\newglossaryentry{@glo232-ciecs}{
type = \acronymtype,
name         = {CIECS},
description  = {Centro de Investigaciones y Estudios sobre la Cultura y la Sociedad},
first        = {Centro de Investigaciones y Estudios sobre la Cultura y la Sociedad (CIECS)},
text         = {CIECS},
}
\newglossaryentry{@glo233-cne}{
type = \acronymtype,
name         = {CNE},
description  = {Consejo Nacional de Investigaciones Científicas y Técnicas},
first        = {Consejo Nacional de Investigaciones Científicas y Técnicas (CNE)},
text         = {CNE},
}
\newglossaryentry{@glo234-petp}{
type = \acronymtype,
name         = {PETP},
description  = {Programa de Estudios en Teoría Política},
first        = {Programa de Estudios en Teoría Política (PETP)},
text         = {PETP},
}
\newglossaryentry{@glo235-unc}{
type = \acronymtype,
name         = {UNC},
description  = {Universidad Nacional de Córdoba},
first        = {Universidad Nacional de Córdoba (UNC)},
text         = {UNC},
}
\newglossaryentry{@glo236-diferencia}{
name         = {Diferencia},
description  = {Concepto que refiere a la no-identidad o no-coincidencia entre elementos; en filosofía contemporánea, es un principio constitutivo de lo real y no solo una relación entre entidades preexistentes},
text         = {diferencia},
}
\newglossaryentry{@glo237-multiculturalismo}{
name         = {Multiculturalismo},
description  = {Doctrina política que promueve el reconocimiento y la convivencia de diversas culturas dentro de una misma sociedad},
text         = {multiculturalismo},
}
\newglossaryentry{@glo238-filosofiadiferencia}{
name         = {Filosofia diferencia},
description  = {Corriente filosófica contemporánea que privilegia la diferencia como categoría fundante frente a la identidad, en oposición a la lógica dialéctica tradicional},
text         = {filosofía de la diferencia},
}
\newglossaryentry{@glo239-alteridad}{
name         = {Alteridad},
description  = {Condición de ser otro o diferente; categoría central en debates sobre subjetividad, identidad y reconocimiento en contextos multiculturales},
text         = {alteridad},
}
\newglossaryentry{@glo240-interculturalidad}{
name         = {Interculturalidad},
description  = {Proceso de interacción equitativa entre culturas que busca un diálogo y enriquecimiento mutuo sin imposición},
text         = {Interculturalidad},
}
\newglossaryentry{@glo241-alteridadradical}{
name         = {Alteridad radical},
description  = {Forma extrema de alteridad que no puede ser subsumida ni comprendida completamente desde marcos de referencia propios},
text         = {alteridad radical},
}
\newglossaryentry{@glo242-negatividad}{
name         = {Negatividad},
description  = {En Hegel y Kojève, proceso mediante el cual la razón se realiza negando lo dado y superándolo dialécticamente},
text         = {negatividad},
}